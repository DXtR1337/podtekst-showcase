% ============================================================
% Rozdział 12 — Struktura Danych: Kompletna Referencja
% ============================================================
\chapter{Struktura Danych --- Kompletna Referencja}
\label{ch:typy-danych}

\begin{infobox}[title={Zakres rozdziału}]
Niniejszy rozdział stanowi wyczerpującą referencję wszystkich typów danych
używanych w~\podtekst. Obejmuje typy parserowe (model zunifikowanych wiadomości
i~wyniki analizy ilościowej), typy analizy AI (schematy wyjściowe każdego przebiegu),
typy magazynowania (localStorage) oraz diagramy relacji między typami.
Każde pole jest udokumentowane z~typem, opisem i~ograniczeniami.
\end{infobox}

\begin{warningbox}[title={Konwencja nazewnicza}]
Wszystkie interfejsy zdefiniowane są w~\typescript w~trybie \texttt{strict}.
Klucze JSON (pola interfejsów) zapisane są w~języku angielskim, natomiast
wartości tekstowe generowane przez AI są w~języku polskim (pl-PL).
Pola opcjonalne oznaczone są znakiem \texttt{?} w~definicji TypeScript.
\end{warningbox}


% ============================================================
\section{Typy parserowe}
\label{sec:parser-types}

Wszystkie typy z~tej sekcji zdefiniowane są w~pliku:

\filepath{src/lib/parsers/types.ts}

Stanowią one model zunifikowany --- niezależny od platformy źródłowej.
Parser każdej platformy (Messenger, WhatsApp, Instagram, Telegram, Discord) normalizuje dane
do tych samych interfejsów.


% ────────────────────────────────────────────────────────────
\subsection{Participant}
\label{subsec:participant}

Reprezentuje uczestnika rozmowy.

\begin{lstlisting}[style=podcode, caption={Interfejs Participant}]
export interface Participant {
  name: string;
  platformId?: string;
}
\end{lstlisting}

\begin{table}[H]
\centering
\caption{Pola interfejsu \tstype{Participant}}
\begin{tabularx}{\textwidth}{l l l X}
\toprule
\textbf{Pole} & \textbf{Typ} & \textbf{Wymagane} & \textbf{Opis} \\
\midrule
\texttt{name} & \tstype{string} & tak & Nazwa wyświetlana uczestnika (po dekodowaniu Unicode) \\
\texttt{platformId} & \tstype{string} & nie & Identyfikator platformowy (np. Facebook ID), jeśli dostępny \\
\bottomrule
\end{tabularx}
\end{table}


% ────────────────────────────────────────────────────────────
\subsection{Reaction}
\label{subsec:reaction}

Reakcja na wiadomość (emoji).

\begin{lstlisting}[style=podcode, caption={Interfejs Reaction}]
export interface Reaction {
  emoji: string;
  actor: string;
  timestamp?: number;
}
\end{lstlisting}

\begin{table}[H]
\centering
\caption{Pola interfejsu \tstype{Reaction}}
\begin{tabularx}{\textwidth}{l l l X}
\toprule
\textbf{Pole} & \textbf{Typ} & \textbf{Wymagane} & \textbf{Opis} \\
\midrule
\texttt{emoji} & \tstype{string} & tak & Emoji reakcji (po dekodowaniu Unicode z~Facebooka) \\
\texttt{actor} & \tstype{string} & tak & Kto dodał reakcję \\
\texttt{timestamp} & \tstype{number} & nie & Timestamp w~ms (jeśli dostępny z~platformy) \\
\bottomrule
\end{tabularx}
\end{table}


% ────────────────────────────────────────────────────────────
\subsection{UnifiedMessage}
\label{subsec:unified-message}

Zunifikowany model wiadomości --- rdzeń systemu typów.

\begin{lstlisting}[style=podcode, caption={Interfejs UnifiedMessage}]
export interface UnifiedMessage {
  index: number;
  sender: string;
  content: string;
  timestamp: number;
  type: 'text' | 'media' | 'sticker' | 'link'
      | 'call' | 'system' | 'unsent';
  reactions: Reaction[];
  hasMedia: boolean;
  hasLink: boolean;
  isUnsent: boolean;
}
\end{lstlisting}

\begin{longtable}{l l l p{6.5cm}}
\caption{Pola interfejsu \tstype{UnifiedMessage}} \label{tab:unified-message} \\
\toprule
\textbf{Pole} & \textbf{Typ} & \textbf{Wymagane} & \textbf{Opis} \\
\midrule
\endfirsthead
\toprule
\textbf{Pole} & \textbf{Typ} & \textbf{Wymagane} & \textbf{Opis} \\
\midrule
\endhead
\texttt{index} & \tstype{number} & tak & Sekwencyjny indeks w~rozmowie (0-based) \\
\texttt{sender} & \tstype{string} & tak & Kto wysłał wiadomość \\
\texttt{content} & \tstype{string} & tak & Treść tekstowa. Pusty string dla wiadomości czysto medialnych \\
\texttt{timestamp} & \tstype{number} & tak & Unix timestamp w~milisekundach \\
\texttt{type} & enum & tak & Typ wiadomości: \texttt{text}, \texttt{media}, \texttt{sticker}, \texttt{link}, \texttt{call}, \texttt{system}, \texttt{unsent} \\
\texttt{reactions} & \tstype{Reaction[]} & tak & Lista reakcji na tę wiadomość (może być pusta) \\
\texttt{hasMedia} & \tstype{boolean} & tak & Czy wiadomość zawiera załącznik multimedialny \\
\texttt{hasLink} & \tstype{boolean} & tak & Czy wiadomość zawiera link/udostępnienie \\
\texttt{isUnsent} & \tstype{boolean} & tak & Czy wiadomość została wycofana/usunięta \\
\bottomrule
\end{longtable}


% ────────────────────────────────────────────────────────────
\subsection{ParsedConversation}
\label{subsec:parsed-conversation}

Kontener rozmowy po parsowaniu.

\begin{lstlisting}[style=podcode, caption={Interfejs ParsedConversation}]
export interface ParsedConversation {
  platform: 'messenger' | 'whatsapp'
           | 'instagram' | 'telegram';
  title: string;
  participants: Participant[];
  messages: UnifiedMessage[];
  metadata: {
    totalMessages: number;
    dateRange: {
      start: number; // Unix ms
      end: number;   // Unix ms
    };
    isGroup: boolean;
    durationDays: number;
  };
}
\end{lstlisting}

\begin{longtable}{l l l p{6cm}}
\caption{Pola interfejsu \tstype{ParsedConversation}} \label{tab:parsed-conversation} \\
\toprule
\textbf{Pole} & \textbf{Typ} & \textbf{Wymagane} & \textbf{Opis} \\
\midrule
\endfirsthead
\toprule
\textbf{Pole} & \textbf{Typ} & \textbf{Wymagane} & \textbf{Opis} \\
\midrule
\endhead
\texttt{platform} & enum & tak & Platforma źródłowa: \texttt{messenger}, \texttt{whatsapp}, \texttt{instagram}, \texttt{telegram} \\
\texttt{title} & \tstype{string} & tak & Tytuł rozmowy (z~parsera lub generowany) \\
\texttt{participants} & \tstype{Participant[]} & tak & Lista uczestników \\
\texttt{messages} & \tstype{UnifiedMessage[]} & tak & Wszystkie wiadomości, posortowane chronologicznie (najstarsze pierwsze) \\
\texttt{metadata.totalMessages} & \tstype{number} & tak & Łączna liczba wiadomości \\
\texttt{metadata.dateRange.start} & \tstype{number} & tak & Unix ms --- początek rozmowy \\
\texttt{metadata.dateRange.end} & \tstype{number} & tak & Unix ms --- koniec rozmowy \\
\texttt{metadata.isGroup} & \tstype{boolean} & tak & Czy to czat grupowy (3+ uczestników) \\
\texttt{metadata.durationDays} & \tstype{number} & tak & Czas trwania rozmowy w~dniach \\
\bottomrule
\end{longtable}


% ────────────────────────────────────────────────────────────
\subsection{PersonMetrics}
\label{subsec:person-metrics}

Metryki ilościowe obliczane per uczestnik. Łącznie 22~pola.

\begin{lstlisting}[style=podcode, caption={Interfejs PersonMetrics}]
export interface PersonMetrics {
  totalMessages: number;
  totalWords: number;
  totalCharacters: number;
  averageMessageLength: number;
  averageMessageChars: number;
  longestMessage: {
    content: string; length: number; timestamp: number;
  };
  shortestMessage: {
    content: string; length: number; timestamp: number;
  };
  messagesWithEmoji: number;
  emojiCount: number;
  topEmojis: Array<{ emoji: string; count: number }>;
  questionsAsked: number;
  mediaShared: number;
  linksShared: number;
  reactionsGiven: number;
  reactionsReceived: number;
  topReactionsGiven: Array<{
    emoji: string; count: number;
  }>;
  unsentMessages: number;
  topWords: Array<{ word: string; count: number }>;
  topPhrases: Array<{
    phrase: string; count: number;
  }>;
  uniqueWords: number;
  vocabularyRichness: number;
}
\end{lstlisting}

\begin{longtable}{l l p{6.5cm}}
\caption{Pola interfejsu \tstype{PersonMetrics} (22 pola)} \label{tab:person-metrics} \\
\toprule
\textbf{Pole} & \textbf{Typ} & \textbf{Opis} \\
\midrule
\endfirsthead
\toprule
\textbf{Pole} & \textbf{Typ} & \textbf{Opis} \\
\midrule
\endhead
\texttt{totalMessages} & \tstype{number} & Łączna liczba wiadomości wysłanych przez tę osobę \\
\texttt{totalWords} & \tstype{number} & Łączna liczba słów \\
\texttt{totalCharacters} & \tstype{number} & Łączna liczba znaków \\
\texttt{averageMessageLength} & \tstype{number} & Średnia długość wiadomości w~słowach \\
\texttt{averageMessageChars} & \tstype{number} & Średnia długość wiadomości w~znakach \\
\texttt{longestMessage} & \tstype{object} & Najdłuższa wiadomość: \texttt{content}, \texttt{length} (słowa), \texttt{timestamp} \\
\texttt{shortestMessage} & \tstype{object} & Najkrótsza wiadomość z~treścią: \texttt{content}, \texttt{length}, \texttt{timestamp} \\
\texttt{messagesWithEmoji} & \tstype{number} & Liczba wiadomości zawierających co najmniej jedno emoji \\
\texttt{emojiCount} & \tstype{number} & Łączna liczba emoji użytych \\
\texttt{topEmojis} & \tstype{Array} & Najczęściej używane emoji: \texttt{[\{emoji, count\}]} \\
\texttt{questionsAsked} & \tstype{number} & Wiadomości zawierające znak ,,?'' \\
\texttt{mediaShared} & \tstype{number} & Liczba wiadomości z~załącznikami multimedialnymi \\
\texttt{linksShared} & \tstype{number} & Liczba wiadomości z~linkami \\
\texttt{reactionsGiven} & \tstype{number} & Łączna liczba reakcji wystawionych innym \\
\texttt{reactionsReceived} & \tstype{number} & Łączna liczba reakcji otrzymanych \\
\texttt{topReactionsGiven} & \tstype{Array} & Najczęściej dawane reakcje: \texttt{[\{emoji, count\}]} \\
\texttt{unsentMessages} & \tstype{number} & Liczba wycofanych/usuniętych wiadomości \\
\texttt{topWords} & \tstype{Array} & Top 20 najczęstszych słów (bez stopwords): \texttt{[\{word, count\}]} \\
\texttt{topPhrases} & \tstype{Array} & Top 10 najczęstszych fraz (2--3 słowa): \texttt{[\{phrase, count\}]} \\
\texttt{uniqueWords} & \tstype{number} & Liczba unikalnych słów użytych \\
\texttt{vocabularyRichness} & \tstype{number} & Bogactwo słownictwa: $\frac{\text{uniqueWords}}{\text{totalWords}}$, zakres 0--1 \\
\bottomrule
\end{longtable}


% ────────────────────────────────────────────────────────────
\subsection{TimingMetrics}
\label{subsec:timing-metrics}

Metryki czasowe rozmowy.

\begin{lstlisting}[style=podcode, caption={Interfejs TimingMetrics}]
export interface TimingMetrics {
  perPerson: Record<string, {
    averageResponseTimeMs: number;
    medianResponseTimeMs: number;
    fastestResponseMs: number;
    slowestResponseMs: number;
    responseTimeTrend: number;
  }>;
  conversationInitiations: Record<string, number>;
  conversationEndings: Record<string, number>;
  longestSilence: {
    durationMs: number;
    startTimestamp: number;
    endTimestamp: number;
    lastSender: string;
    nextSender: string;
  };
  lateNightMessages: Record<string, number>;
}
\end{lstlisting}

\begin{longtable}{l l p{6cm}}
\caption{Pola interfejsu \tstype{TimingMetrics}} \label{tab:timing-metrics} \\
\toprule
\textbf{Pole} & \textbf{Typ} & \textbf{Opis} \\
\midrule
\endfirsthead
\toprule
\textbf{Pole} & \textbf{Typ} & \textbf{Opis} \\
\midrule
\endhead
\texttt{perPerson[].averageResponseTimeMs} & \tstype{number} & Średni czas odpowiedzi w~ms \\
\texttt{perPerson[].medianResponseTimeMs} & \tstype{number} & Mediana czasu odpowiedzi w~ms (bardziej miarodajna niż średnia) \\
\texttt{perPerson[].fastestResponseMs} & \tstype{number} & Najszybsza odpowiedź w~ms \\
\texttt{perPerson[].slowestResponseMs} & \tstype{number} & Najwolniejsza odpowiedź w~ms \\
\texttt{perPerson[].responseTimeTrend} & \tstype{number} & Trend czasu odpowiedzi: wartość dodatnia = coraz wolniej \\
\texttt{conversationInitiations} & \tstype{Record} & Ile razy dana osoba zainicjowała rozmowę (pierwsza wiadomość po przerwie >6h) \\
\texttt{conversationEndings} & \tstype{Record} & Ile razy dana osoba zakończyła rozmowę (ostatnia wiadomość przed przerwą >6h) \\
\texttt{longestSilence.durationMs} & \tstype{number} & Czas trwania najdłuższej ciszy w~ms \\
\texttt{longestSilence.startTimestamp} & \tstype{number} & Timestamp ostatniej wiadomości przed ciszą \\
\texttt{longestSilence.endTimestamp} & \tstype{number} & Timestamp pierwszej wiadomości po ciszy \\
\texttt{longestSilence.lastSender} & \tstype{string} & Kto wysłał ostatnią wiadomość przed ciszą \\
\texttt{longestSilence.nextSender} & \tstype{string} & Kto przerwał ciszę \\
\texttt{lateNightMessages} & \tstype{Record} & Wiadomości wysłane między 22:00 a~04:00, per osoba \\
\bottomrule
\end{longtable}


% ────────────────────────────────────────────────────────────
\subsection{EngagementMetrics}
\label{subsec:engagement-metrics}

Metryki zaangażowania w~rozmowę.

\begin{lstlisting}[style=podcode, caption={Interfejs EngagementMetrics}]
export interface EngagementMetrics {
  doubleTexts: Record<string, number>;
  maxConsecutive: Record<string, number>;
  messageRatio: Record<string, number>;
  reactionRate: Record<string, number>;
  avgConversationLength: number;
  totalSessions: number;
}
\end{lstlisting}

\begin{table}[H]
\centering
\caption{Pola interfejsu \tstype{EngagementMetrics} (6 pól)}
\label{tab:engagement-metrics}
\begin{tabularx}{\textwidth}{l l X}
\toprule
\textbf{Pole} & \textbf{Typ} & \textbf{Opis} \\
\midrule
\texttt{doubleTexts} & \tstype{Record} & Liczba ,,double texts'' (2+ wiadomości z~rzędu bez odpowiedzi), per osoba \\
\texttt{maxConsecutive} & \tstype{Record} & Maks. wiadomości z~rzędu bez odpowiedzi, per osoba \\
\texttt{messageRatio} & \tstype{Record} & Proporcja: wiadomości osoby / łącznie. Zakres 0--1 \\
\texttt{reactionRate} & \tstype{Record} & Wskaźnik reakcji: reakcje dane / wiadomości otrzymane \\
\texttt{avgConversationLength} & \tstype{number} & Średnia liczba wiadomości na sesję konwersacyjną \\
\texttt{totalSessions} & \tstype{number} & Łączna liczba sesji konwersacyjnych (rozdzielonych przerwą >6h) \\
\bottomrule
\end{tabularx}
\end{table}


% ────────────────────────────────────────────────────────────
\subsection{PatternMetrics}
\label{subsec:pattern-metrics}

Metryki wzorców aktywności.

\begin{lstlisting}[style=podcode, caption={Interfejs PatternMetrics}]
export interface PatternMetrics {
  monthlyVolume: Array<{
    month: string; // YYYY-MM
    perPerson: Record<string, number>;
    total: number;
  }>;
  weekdayWeekend: {
    weekday: Record<string, number>;
    weekend: Record<string, number>;
  };
  volumeTrend: number;
  bursts: Array<{
    startDate: string;
    endDate: string;
    messageCount: number;
    avgDaily: number;
  }>;
}
\end{lstlisting}

\begin{table}[H]
\centering
\caption{Pola interfejsu \tstype{PatternMetrics} (4 pola)}
\label{tab:pattern-metrics}
\begin{tabularx}{\textwidth}{l l X}
\toprule
\textbf{Pole} & \textbf{Typ} & \textbf{Opis} \\
\midrule
\texttt{monthlyVolume} & \tstype{Array} & Wolumen wiadomości per miesiąc: \texttt{month} (YYYY-MM), \texttt{perPerson}, \texttt{total} \\
\texttt{weekdayWeekend} & \tstype{object} & Podział aktywności: dni robocze vs. weekendy, per osoba \\
\texttt{volumeTrend} & \tstype{number} & Trend wolumenu: $>0.1$ = rosnący, $<-0.1$ = malejący, reszta = stabilny \\
\texttt{bursts} & \tstype{Array} & Wykryte ,,wybuchy'' aktywności ($>3\times$ średnia dzienna): daty, liczba msg, średnia/dzień \\
\bottomrule
\end{tabularx}
\end{table}


% ────────────────────────────────────────────────────────────
\subsection{HeatmapData}
\label{subsec:heatmap-data}

Dane do mapy ciepła aktywności.

\begin{lstlisting}[style=podcode, caption={Interfejs HeatmapData}]
export interface HeatmapData {
  perPerson: Record<string, number[][]>;
  combined: number[][];
}
\end{lstlisting}

\begin{table}[H]
\centering
\caption{Pola interfejsu \tstype{HeatmapData}}
\begin{tabularx}{\textwidth}{l l X}
\toprule
\textbf{Pole} & \textbf{Typ} & \textbf{Opis} \\
\midrule
\texttt{perPerson} & \tstype{Record<string, number[][]>} & Macierz $7 \times 24$ per osoba: \texttt{[dayOfWeek][hourOfDay] = messageCount} \\
\texttt{combined} & \tstype{number[][]} & Macierz $7 \times 24$ zsumowana ze wszystkich uczestników \\
\bottomrule
\end{tabularx}
\end{table}

Indeksy: dzień tygodnia 0 (poniedziałek) -- 6 (niedziela), godzina 0--23.


% ────────────────────────────────────────────────────────────
\subsection{TrendData}
\label{subsec:trend-data}

Dane trendów miesięcznych.

\begin{lstlisting}[style=podcode, caption={Interfejs TrendData}]
export interface TrendData {
  responseTimeTrend: Array<{
    month: string;
    perPerson: Record<string, number>;
  }>;
  messageLengthTrend: Array<{
    month: string;
    perPerson: Record<string, number>;
  }>;
  initiationTrend: Array<{
    month: string;
    perPerson: Record<string, number>;
  }>;
}
\end{lstlisting}

\begin{table}[H]
\centering
\caption{Pola interfejsu \tstype{TrendData}}
\begin{tabularx}{\textwidth}{l l X}
\toprule
\textbf{Pole} & \textbf{Typ} & \textbf{Opis} \\
\midrule
\texttt{responseTimeTrend} & \tstype{Array} & Miesięczny średni czas odpowiedzi per osoba \\
\texttt{messageLengthTrend} & \tstype{Array} & Miesięczna średnia długość wiadomości per osoba \\
\texttt{initiationTrend} & \tstype{Array} & Miesięczna proporcja inicjacji rozmów per osoba \\
\bottomrule
\end{tabularx}
\end{table}


% ────────────────────────────────────────────────────────────
\subsection{ViralScores}
\label{subsec:viral-scores}

Metryki ,,wiralowe'' --- zaprojektowane jako angażujące, udostępnialne wyniki.

\begin{lstlisting}[style=podcode, caption={Interfejsy ViralScores i~GhostRiskData}]
export interface ViralScores {
  compatibilityScore: number;
  interestScores: Record<string, number>;
  ghostRisk: Record<string, GhostRiskData>;
  delusionScore: number;
  delusionHolder?: string;
}

export interface GhostRiskData {
  score: number;
  factors: string[];
}
\end{lstlisting}

\begin{table}[H]
\centering
\caption{Pola interfejsu \tstype{ViralScores}}
\begin{tabularx}{\textwidth}{l l X}
\toprule
\textbf{Pole} & \textbf{Typ/Zakres} & \textbf{Opis} \\
\midrule
\texttt{compatibilityScore} & 0--100 & Wynik kompatybilności na podstawie nakładania aktywności, symetrii odpowiedzi, równowagi zaangażowania \\
\texttt{interestScores} & 0--100, per os. & Wynik zainteresowania na podstawie inicjacji, trendów czasu odpowiedzi, trendów długości wiadomości \\
\texttt{ghostRisk} & per os. & Ryzyko ghostingu: \texttt{score} (0--100, wyższy = większe ryzyko) + \texttt{factors} (czynniki ryzyka) \\
\texttt{delusionScore} & 0--100 & Rozbieżność poziomów zainteresowania między uczestnikami \\
\texttt{delusionHolder} & \tstype{string?} & Kto jest bardziej ,,zludzony'' (ma wyższe zainteresowanie przy niższym drugiej strony) \\
\bottomrule
\end{tabularx}
\end{table}


% ────────────────────────────────────────────────────────────
\subsection{Badge}
\label{subsec:badge}

Odznaki i~osiągnięcia przyznawane automatycznie.

\begin{lstlisting}[style=podcode, caption={Interfejs Badge}]
export interface Badge {
  id: string;
  name: string;
  emoji: string;
  description: string;
  holder: string;
  evidence: string;
}
\end{lstlisting}

\begin{table}[H]
\centering
\caption{Pola interfejsu \tstype{Badge}}
\begin{tabularx}{\textwidth}{l l X}
\toprule
\textbf{Pole} & \textbf{Typ} & \textbf{Opis} \\
\midrule
\texttt{id} & \tstype{string} & Unikalny identyfikator odznaki (np. \texttt{early\_bird}) \\
\texttt{name} & \tstype{string} & Wyświetlana nazwa odznaki \\
\texttt{emoji} & \tstype{string} & Emoji reprezentujące odznakę \\
\texttt{description} & \tstype{string} & Opis warunku przyznania \\
\texttt{holder} & \tstype{string} & Kto zdobył odznakę \\
\texttt{evidence} & \tstype{string} & Wartość statystyczna potwierdzająca przyznanie \\
\bottomrule
\end{tabularx}
\end{table}


% ────────────────────────────────────────────────────────────
\subsection{BestTimeToText}
\label{subsec:best-time}

Najlepszy czas na napisanie wiadomości per osoba.

\begin{lstlisting}[style=podcode, caption={Interfejs BestTimeToText}]
export interface BestTimeToText {
  perPerson: Record<string, {
    bestDay: string;
    bestHour: number;
    bestWindow: string;
    avgResponseMs: number;
  }>;
}
\end{lstlisting}

\begin{table}[H]
\centering
\caption{Pola interfejsu \tstype{BestTimeToText}}
\begin{tabularx}{\textwidth}{l l X}
\toprule
\textbf{Pole} & \textbf{Typ} & \textbf{Opis} \\
\midrule
\texttt{bestDay} & \tstype{string} & Dzień tygodnia z~najszybszymi odpowiedziami \\
\texttt{bestHour} & \tstype{number} & Godzina (0--23) z~najszybszymi odpowiedziami \\
\texttt{bestWindow} & \tstype{string} & Opis okna czasowego (np. ,,wtorek 14:00--16:00'') \\
\texttt{avgResponseMs} & \tstype{number} & Średni czas odpowiedzi w~tym oknie (ms) \\
\bottomrule
\end{tabularx}
\end{table}


% ────────────────────────────────────────────────────────────
\subsection{CatchphraseEntry i~CatchphraseResult}
\label{subsec:catchphrase}

Charakterystyczne frazy (catchphrases) per osoba.

\begin{lstlisting}[style=podcode, caption={Interfejsy CatchphraseEntry i~CatchphraseResult}]
export interface CatchphraseEntry {
  phrase: string;
  count: number;
  uniqueness: number;
}

export interface CatchphraseResult {
  perPerson: Record<string, CatchphraseEntry[]>;
}
\end{lstlisting}

\begin{table}[H]
\centering
\caption{Pola interfejsu \tstype{CatchphraseEntry}}
\begin{tabularx}{\textwidth}{l l X}
\toprule
\textbf{Pole} & \textbf{Typ/Zakres} & \textbf{Opis} \\
\midrule
\texttt{phrase} & \tstype{string} & Fraza (2--4 słowa) \\
\texttt{count} & \tstype{number} & Ile razy użyta \\
\texttt{uniqueness} & 0--1 & Unikalność frazy dla tej osoby vs. inne osoby (1 = tylko ta osoba używa) \\
\bottomrule
\end{tabularx}
\end{table}


% ────────────────────────────────────────────────────────────
\subsection{NetworkNode, NetworkEdge, NetworkMetrics}
\label{subsec:network}

Metryki sieciowe (głównie dla czatów grupowych).

\begin{lstlisting}[style=podcode, caption={Interfejsy sieciowe}]
export interface NetworkNode {
  name: string;
  totalMessages: number;
  centrality: number;
}

export interface NetworkEdge {
  from: string;
  to: string;
  weight: number;
  fromToCount: number;
  toFromCount: number;
}

export interface NetworkMetrics {
  nodes: NetworkNode[];
  edges: NetworkEdge[];
  density: number;
  mostConnected: string;
}
\end{lstlisting}

\begin{longtable}{l l p{6cm}}
\caption{Pola interfejsów sieciowych} \label{tab:network} \\
\toprule
\textbf{Pole} & \textbf{Typ/Zakres} & \textbf{Opis} \\
\midrule
\endfirsthead
\toprule
\textbf{Pole} & \textbf{Typ/Zakres} & \textbf{Opis} \\
\midrule
\endhead
\multicolumn{3}{l}{\textbf{\tstype{NetworkNode}}} \\
\midrule
\texttt{name} & \tstype{string} & Nazwa uczestnika \\
\texttt{totalMessages} & \tstype{number} & Łączna liczba wiadomości \\
\texttt{centrality} & 0--1 & Centralność stopnia (degree centrality) --- jak połączony jest ten węzeł \\
\midrule
\multicolumn{3}{l}{\textbf{\tstype{NetworkEdge}}} \\
\midrule
\texttt{from} & \tstype{string} & Uczestnik źródłowy \\
\texttt{to} & \tstype{string} & Uczestnik docelowy \\
\texttt{weight} & \tstype{number} & Łączna liczba wzajemnych interakcji \\
\texttt{fromToCount} & \tstype{number} & Wiadomości \texttt{from} $\rightarrow$ \texttt{to} (A wysłała po B) \\
\texttt{toFromCount} & \tstype{number} & Wiadomości \texttt{to} $\rightarrow$ \texttt{from} (B wysłała po A) \\
\midrule
\multicolumn{3}{l}{\textbf{\tstype{NetworkMetrics}}} \\
\midrule
\texttt{nodes} & \tstype{NetworkNode[]} & Lista wszystkich węzłów \\
\texttt{edges} & \tstype{NetworkEdge[]} & Lista wszystkich krawędzi \\
\texttt{density} & 0--1 & Gęstość grafu: rzeczywiste krawędzie / możliwe krawędzie \\
\texttt{mostConnected} & \tstype{string} & Osoba o~najwyższej centralności \\
\bottomrule
\end{longtable}


% ────────────────────────────────────────────────────────────
\subsection{ReciprocityIndex}
\label{subsec:reciprocity}

Indeks wzajemności --- kompozytowa metryka mierząca równowagę relacji.

\begin{lstlisting}[style=podcode, caption={Interfejs ReciprocityIndex}]
export interface ReciprocityIndex {
  overall: number;
  messageBalance: number;
  initiationBalance: number;
  responseTimeSymmetry: number;
  reactionBalance: number;
}
\end{lstlisting}

\begin{table}[H]
\centering
\caption{Pola interfejsu \tstype{ReciprocityIndex}}
\begin{tabularx}{\textwidth}{l l X}
\toprule
\textbf{Pole} & \textbf{Zakres} & \textbf{Opis} \\
\midrule
\texttt{overall} & 0--100 & Ogólny wynik wzajemności. 50 = idealna równowaga. 0~lub 100 = całkowita jednostronność \\
\texttt{messageBalance} & 0--100 & Równowaga liczby wiadomości \\
\texttt{initiationBalance} & 0--100 & Kto inicjuje rozmowy --- równowaga \\
\texttt{responseTimeSymmetry} & 0--100 & Symetria czasów odpowiedzi \\
\texttt{reactionBalance} & 0--100 & Równowaga dawania reakcji/emoji \\
\bottomrule
\end{tabularx}
\end{table}


% ────────────────────────────────────────────────────────────
\subsection{QuantitativeAnalysis}
\label{subsec:quantitative-analysis}

Kontener zbierający wszystkie metryki ilościowe.

\begin{lstlisting}[style=podcode, caption={Interfejs QuantitativeAnalysis}]
export interface QuantitativeAnalysis {
  perPerson: Record<string, PersonMetrics>;
  timing: TimingMetrics;
  engagement: EngagementMetrics;
  patterns: PatternMetrics;
  heatmap: HeatmapData;
  trends: TrendData;
  viralScores?: ViralScores;
  badges?: Badge[];
  bestTimeToText?: BestTimeToText;
  catchphrases?: CatchphraseResult;
  networkMetrics?: NetworkMetrics;
  reciprocityIndex?: ReciprocityIndex;
}
\end{lstlisting}

\begin{table}[H]
\centering
\caption{Pola interfejsu \tstype{QuantitativeAnalysis}}
\begin{tabularx}{\textwidth}{l l l X}
\toprule
\textbf{Pole} & \textbf{Typ} & \textbf{Wym.} & \textbf{Opis} \\
\midrule
\texttt{perPerson} & \tstype{Record<PersonMetrics>} & tak & Metryki per uczestnik \\
\texttt{timing} & \tstype{TimingMetrics} & tak & Metryki czasowe \\
\texttt{engagement} & \tstype{EngagementMetrics} & tak & Metryki zaangażowania \\
\texttt{patterns} & \tstype{PatternMetrics} & tak & Wzorce aktywności \\
\texttt{heatmap} & \tstype{HeatmapData} & tak & Dane mapy ciepła \\
\texttt{trends} & \tstype{TrendData} & tak & Trendy miesięczne \\
\texttt{viralScores} & \tstype{ViralScores} & nie & Wyniki ,,wiralowe'' \\
\texttt{badges} & \tstype{Badge[]} & nie & Przyznane odznaki \\
\texttt{bestTimeToText} & \tstype{BestTimeToText} & nie & Najlepszy czas na wiadomość \\
\texttt{catchphrases} & \tstype{CatchphraseResult} & nie & Charakterystyczne frazy \\
\texttt{networkMetrics} & \tstype{NetworkMetrics} & nie & Metryki sieciowe (grupy) \\
\texttt{reciprocityIndex} & \tstype{ReciprocityIndex} & nie & Indeks wzajemności \\
\bottomrule
\end{tabularx}
\end{table}


% ============================================================
\section{Typy analizy AI}
\label{sec:ai-types}

Wszystkie typy z~tej sekcji zdefiniowane są w~plikach:

\filepath{src/lib/analysis/types.ts}

\filepath{src/lib/analysis/communication-patterns.ts}


% ────────────────────────────────────────────────────────────
\subsection{Pass 1: RelationshipType, PersonTone, OverallDynamic}
\label{subsec:pass1-types}

\begin{table}[H]
\centering
\caption{Pola interfejsu \tstype{RelationshipType}}
\begin{tabularx}{\textwidth}{l l X}
\toprule
\textbf{Pole} & \textbf{Typ} & \textbf{Opis} \\
\midrule
\texttt{category} & enum & \texttt{romantic | friendship | family | professional | acquaintance} \\
\texttt{sub\_type} & \tstype{string} & Podtyp relacji (np. ,,wczesne randkowanie'', ,,bliscy przyjaciele'') \\
\texttt{confidence} & 0--100 & Pewność klasyfikacji \\
\bottomrule
\end{tabularx}
\end{table}

\begin{table}[H]
\centering
\caption{Pola interfejsu \tstype{PersonTone}}
\begin{tabularx}{\textwidth}{l l X}
\toprule
\textbf{Pole} & \textbf{Typ/Zakres} & \textbf{Opis} \\
\midrule
\texttt{primary\_tone} & \tstype{string} & Dominujący ton emocjonalny \\
\texttt{secondary\_tones} & \tstype{string[]} & Drugorzędne tony \\
\texttt{formality\_level} & 1--10 & Poziom formalności \\
\texttt{humor\_presence} & 1--10 & Obecność humoru \\
\texttt{humor\_style} & enum & \texttt{self-deprecating | teasing | absurdist | sarcastic | wordplay | absent} \\
\texttt{warmth} & 1--10 & Ciepłota emocjonalna \\
\texttt{confidence} & 0--100 & Pewność oceny \\
\texttt{evidence\_indices} & \tstype{number[]} & Indeksy wiadomości jako dowody \\
\bottomrule
\end{tabularx}
\end{table}

\begin{table}[H]
\centering
\caption{Pola interfejsu \tstype{OverallDynamic}}
\begin{tabularx}{\textwidth}{l l X}
\toprule
\textbf{Pole} & \textbf{Typ} & \textbf{Opis} \\
\midrule
\texttt{description} & \tstype{string} & 2--3 zdania opisujące główną dynamikę \\
\texttt{energy} & enum & \texttt{high | medium | low} \\
\texttt{balance} & enum & \texttt{balanced | person\_a\_dominant | person\_b\_dominant} \\
\texttt{trajectory} & enum & \texttt{warming | stable | cooling | volatile} \\
\texttt{confidence} & 0--100 & Pewność oceny \\
\bottomrule
\end{tabularx}
\end{table}

\begin{lstlisting}[style=podcode, caption={Interfejs Pass1Result (kontener)}]
export interface Pass1Result {
  relationship_type: RelationshipType;
  tone_per_person: Record<string, PersonTone>;
  overall_dynamic: OverallDynamic;
}
\end{lstlisting}


% ────────────────────────────────────────────────────────────
\subsection{Pass 2: Dynamika relacyjna}
\label{subsec:pass2-types}

\subsubsection{PowerDynamics}

\begin{table}[H]
\centering
\caption{Pola interfejsu \tstype{PowerDynamics}}
\begin{tabularx}{\textwidth}{l l X}
\toprule
\textbf{Pole} & \textbf{Typ/Zakres} & \textbf{Opis} \\
\midrule
\texttt{balance\_score} & $-100$ do $+100$ & $-100$: A dominuje. $0$: równowaga. $+100$: B dominuje \\
\texttt{who\_adapts\_more} & \tstype{string} & Nazwa osoby, która dostosowuje się bardziej \\
\texttt{adaptation\_type} & enum & \texttt{linguistic | emotional | topical | scheduling} \\
\texttt{evidence} & \tstype{string[]} & Opisy dowodów \\
\texttt{confidence} & 0--100 & Pewność \\
\bottomrule
\end{tabularx}
\end{table}

\subsubsection{EmotionalLaborPattern i~EmotionalLabor}

\begin{table}[H]
\centering
\caption{Pola interfejsu \tstype{EmotionalLaborPattern}}
\begin{tabularx}{\textwidth}{l l X}
\toprule
\textbf{Pole} & \textbf{Typ} & \textbf{Opis} \\
\midrule
\texttt{type} & enum & \texttt{comforting | checking\_in | remembering\_details | managing\_mood | initiating\_plans | emotional\_support} \\
\texttt{performed\_by} & \tstype{string} & Kto wykonuje tę pracę \\
\texttt{frequency} & enum & \texttt{frequent | occasional | rare} \\
\texttt{evidence\_indices} & \tstype{number[]} & Indeksy wiadomości \\
\bottomrule
\end{tabularx}
\end{table}

\begin{table}[H]
\centering
\caption{Pola interfejsu \tstype{EmotionalLabor}}
\begin{tabularx}{\textwidth}{l l X}
\toprule
\textbf{Pole} & \textbf{Typ/Zakres} & \textbf{Opis} \\
\midrule
\texttt{primary\_caregiver} & \tstype{string} & Główny opiekun emocjonalny lub \texttt{"balanced"} \\
\texttt{patterns} & \tstype{EmotionalLaborPattern[]} & Lista wzorców pracy emocjonalnej \\
\texttt{balance\_score} & $-100$ do $+100$ & Równowaga pracy emocjonalnej \\
\texttt{confidence} & 0--100 & Pewność \\
\bottomrule
\end{tabularx}
\end{table}

\subsubsection{ConflictPatterns}

\begin{table}[H]
\centering
\caption{Pola interfejsu \tstype{ConflictPatterns}}
\begin{tabularx}{\textwidth}{l l X}
\toprule
\textbf{Pole} & \textbf{Typ} & \textbf{Opis} \\
\midrule
\texttt{conflict\_frequency} & enum & \texttt{none\_observed | rare | occasional | frequent} \\
\texttt{typical\_trigger} & \tstype{string | null} & Typowy wyzwalacz konfliktów \\
\texttt{resolution\_style} & \tstype{Record<string, enum>} & Styl rozwiązywania per osoba: \texttt{direct\_confrontation | avoidant | passive\_aggressive | apologetic | deflecting | humor} \\
\texttt{unresolved\_tensions} & \tstype{string[]} & Lista nierozwiązanych napięć \\
\texttt{confidence} & 0--100 & Pewność \\
\bottomrule
\end{tabularx}
\end{table}

\subsubsection{VulnerabilityProfile i~SharedLanguage}

\begin{table}[H]
\centering
\caption{Pola interfejsów \tstype{VulnerabilityProfile} i~\tstype{SharedLanguage}}
\begin{tabularx}{\textwidth}{l l l X}
\toprule
\textbf{Interfejs} & \textbf{Pole} & \textbf{Typ} & \textbf{Opis} \\
\midrule
\multirow{3}{*}{\tstype{VulnerabilityProfile}} & \texttt{score} & 1--10 & Poziom odsłaniania podatności na zranienie \\
 & \texttt{examples} & \tstype{string[]} & Przykłady \\
 & \texttt{trend} & enum & \texttt{increasing | stable | decreasing} \\
\midrule
\multirow{4}{*}{\tstype{SharedLanguage}} & \texttt{inside\_jokes} & 0--10 & Liczba wewnętrznych żartów \\
 & \texttt{pet\_names} & \tstype{boolean} & Czy używają pieszczotliwych imion \\
 & \texttt{unique\_phrases} & \tstype{string[]} & Unikalne frazy \\
 & \texttt{language\_mirroring} & 1--10 & Stopień lustrzanego odbijania języka \\
\bottomrule
\end{tabularx}
\end{table}

\subsubsection{IntimacyMarkers}

\begin{table}[H]
\centering
\caption{Pola interfejsu \tstype{IntimacyMarkers}}
\begin{tabularx}{\textwidth}{l l X}
\toprule
\textbf{Pole} & \textbf{Typ} & \textbf{Opis} \\
\midrule
\texttt{vulnerability\_level} & \tstype{Record<string, VulnerabilityProfile>} & Profile podatności per osoba \\
\texttt{shared\_language} & \tstype{SharedLanguage} & Wspólny język relacji \\
\texttt{confidence} & 0--100 & Pewność \\
\bottomrule
\end{tabularx}
\end{table}

\subsubsection{RedFlag i~GreenFlag}

\begin{table}[H]
\centering
\caption{Pola interfejsów \tstype{RedFlag} i~\tstype{GreenFlag}}
\begin{tabularx}{\textwidth}{l l l X}
\toprule
\textbf{Interfejs} & \textbf{Pole} & \textbf{Typ} & \textbf{Opis} \\
\midrule
\multirow{5}{*}{\tstype{RedFlag}} & \texttt{pattern} & \tstype{string} & Opis wzorca (PL) \\
 & \texttt{severity} & enum & \texttt{mild | moderate | severe} \\
 & \texttt{context\_note} & \tstype{string?} & Kontekst powagi w~danej fazie relacji \\
 & \texttt{evidence\_indices} & \tstype{number[]} & Indeksy wiadomości \\
 & \texttt{confidence} & 0--100 & Pewność \\
\midrule
\multirow{3}{*}{\tstype{GreenFlag}} & \texttt{pattern} & \tstype{string} & Opis wzorca (PL) \\
 & \texttt{evidence\_indices} & \tstype{number[]} & Indeksy wiadomości \\
 & \texttt{confidence} & 0--100 & Pewność \\
\bottomrule
\end{tabularx}
\end{table}

\subsubsection{Pass2Result (kontener)}

\begin{lstlisting}[style=podcode, caption={Interfejs Pass2Result}]
export interface Pass2Result {
  power_dynamics: PowerDynamics;
  emotional_labor: EmotionalLabor;
  conflict_patterns: ConflictPatterns;
  intimacy_markers: IntimacyMarkers;
  relationship_phase?: 'new' | 'developing'
    | 'established' | 'long_term';
  red_flags: RedFlag[];
  green_flags: GreenFlag[];
}
\end{lstlisting}


% ────────────────────────────────────────────────────────────
\subsection{Pass 3: Profile indywidualne}
\label{subsec:pass3-types}

\subsubsection{BigFiveTrait i~BigFiveApproximation}

\begin{table}[H]
\centering
\caption{Pola interfejsów Wielkiej Piątki}
\begin{tabularx}{\textwidth}{l l l X}
\toprule
\textbf{Interfejs} & \textbf{Pole} & \textbf{Typ} & \textbf{Opis} \\
\midrule
\multirow{3}{*}{\tstype{BigFiveTrait}} & \texttt{range} & \tstype{[number, number]} & Zakres estymacji [dolny, górny], każdy 1--10 \\
 & \texttt{evidence} & \tstype{string} & Dowody tekstowe \\
 & \texttt{confidence} & 0--100 & Pewność \\
\midrule
\multirow{5}{*}{\tstype{BigFiveApproximation}} & \texttt{openness} & \tstype{BigFiveTrait} & Otwartość na doświadczenia \\
 & \texttt{conscientiousness} & \tstype{BigFiveTrait} & Sumienność \\
 & \texttt{extraversion} & \tstype{BigFiveTrait} & Ekstrawersja \\
 & \texttt{agreeableness} & \tstype{BigFiveTrait} & Ugodowość \\
 & \texttt{neuroticism} & \tstype{BigFiveTrait} & Neurotyczność \\
\bottomrule
\end{tabularx}
\end{table}

\subsubsection{AttachmentIndicator i~AttachmentIndicators}

\begin{table}[H]
\centering
\caption{Pola interfejsów stylu przywiązania}
\begin{tabularx}{\textwidth}{l l l X}
\toprule
\textbf{Interfejs} & \textbf{Pole} & \textbf{Typ} & \textbf{Opis} \\
\midrule
\multirow{3}{*}{\tstype{AttachmentIndicator}} & \texttt{behavior} & \tstype{string} & Opis obserwowanego zachowania \\
 & \texttt{attachment\_relevance} & \tstype{string} & Co to sugeruje o~przywiązaniu \\
 & \texttt{evidence\_indices} & \tstype{number[]} & Indeksy wiadomości \\
\midrule
\multirow{4}{*}{\tstype{AttachmentIndicators}} & \texttt{primary\_style} & enum & \texttt{secure | anxious | avoidant | disorganized | insufficient\_data} \\
 & \texttt{indicators} & \tstype{AttachmentIndicator[]} & Lista wskaźników \\
 & \texttt{confidence} & 0--65 & \textbf{Maks. 65} --- ograniczenie analizy tekstowej \\
 & \texttt{disclaimer} & \tstype{string?} & Obowiązkowe zastrzeżenie \\
\bottomrule
\end{tabularx}
\end{table}

\subsubsection{CommunicationProfile}

\begin{table}[H]
\centering
\caption{Pola interfejsu \tstype{CommunicationProfile}}
\begin{tabularx}{\textwidth}{l l X}
\toprule
\textbf{Pole} & \textbf{Typ/Zakres} & \textbf{Opis} \\
\midrule
\texttt{style} & enum & \texttt{direct | indirect | mixed} \\
\texttt{assertiveness} & 1--10 & Asertywność \\
\texttt{emotional\_expressiveness} & 1--10 & Ekspresja emocjonalna \\
\texttt{self\_disclosure\_depth} & 1--10 & Głębokość ujawniania siebie \\
\texttt{question\_to\_statement\_ratio} & enum & \texttt{asks\_more | states\_more | balanced} \\
\texttt{typical\_message\_structure} & \tstype{string} & Opis typowej struktury wiadomości \\
\texttt{verbal\_tics} & \tstype{string[]} & Powtarzane frazy, filler words \\
\texttt{emoji\_personality} & \tstype{string} & Opis osobowości emoji \\
\bottomrule
\end{tabularx}
\end{table}

\subsubsection{CommunicationNeeds}

\begin{table}[H]
\centering
\caption{Pola interfejsu \tstype{CommunicationNeeds}}
\begin{tabularx}{\textwidth}{l l X}
\toprule
\textbf{Pole} & \textbf{Typ} & \textbf{Opis} \\
\midrule
\texttt{primary} & enum & \texttt{affirmation | space | consistency | spontaneity | depth | humor | control | freedom} \\
\texttt{secondary} & \tstype{string} & Opis wtórnej potrzeby \\
\texttt{unmet\_needs\_signals} & \tstype{string[]} & Zachowania sygnalizujące niezaspokojone potrzeby \\
\texttt{confidence} & 0--100 & Pewność \\
\bottomrule
\end{tabularx}
\end{table}

\subsubsection{EmotionalPatterns}

\begin{table}[H]
\centering
\caption{Pola interfejsu \tstype{EmotionalPatterns}}
\begin{tabularx}{\textwidth}{l l X}
\toprule
\textbf{Pole} & \textbf{Typ/Zakres} & \textbf{Opis} \\
\midrule
\texttt{emotional\_range} & 1--10 & Zakres emocjonalny (1 = monotonny, 10 = szeroki) \\
\texttt{dominant\_emotions} & \tstype{string[]} & Dominujące emocje \\
\texttt{coping\_mechanisms\_visible} & \tstype{string[]} & Widoczne mechanizmy radzenia sobie \\
\texttt{stress\_indicators} & \tstype{string[]} & Jak stres przejawia się w~wiadomościach \\
\texttt{confidence} & 0--100 & Pewność \\
\bottomrule
\end{tabularx}
\end{table}

\subsubsection{ClinicalObservations}

\begin{longtable}{l l p{6cm}}
\caption{Pola interfejsu \tstype{ClinicalObservations}} \label{tab:clinical-obs-type} \\
\toprule
\textbf{Pole} & \textbf{Typ} & \textbf{Opis} \\
\midrule
\endfirsthead
\toprule
\textbf{Pole} & \textbf{Typ} & \textbf{Opis} \\
\midrule
\endhead
\texttt{anxiety\_markers.present} & \tstype{boolean} & Czy wykryto markery lękowe \\
\texttt{anxiety\_markers.patterns} & \tstype{string[]} & Konkretne wzorce \\
\texttt{anxiety\_markers.severity} & enum & \texttt{none | mild | moderate | significant} \\
\texttt{anxiety\_markers.confidence} & 0--100 & Pewność \\
\midrule
\texttt{avoidance\_markers.present} & \tstype{boolean} & Czy wykryto markery unikania \\
\texttt{avoidance\_markers.patterns} & \tstype{string[]} & Konkretne wzorce \\
\texttt{avoidance\_markers.severity} & enum & \texttt{none | mild | moderate | significant} \\
\texttt{avoidance\_markers.confidence} & 0--100 & Pewność \\
\midrule
\texttt{manipulation\_patterns.present} & \tstype{boolean} & Czy wykryto wzorce manipulacji \\
\texttt{manipulation\_patterns.types} & \tstype{string[]} & Typy manipulacji \\
\texttt{manipulation\_patterns.severity} & enum & \texttt{none | mild | moderate | severe} \\
\texttt{manipulation\_patterns.confidence} & 0--100 & Pewność. Musi być $\geq 70$ aby \texttt{present = true} \\
\midrule
\texttt{boundary\_respect.score} & 1--10 & Szacunek dla granic \\
\texttt{boundary\_respect.examples} & \tstype{string[]} & Przykłady \\
\texttt{boundary\_respect.confidence} & 0--100 & Pewność \\
\midrule
\texttt{codependency\_signals.present} & \tstype{boolean} & Czy wykryto sygnały współzależności \\
\texttt{codependency\_signals.indicators} & \tstype{string[]} & Wskaźniki \\
\texttt{codependency\_signals.confidence} & 0--100 & Pewność \\
\midrule
\texttt{disclaimer} & \tstype{string} & Obowiązkowe zastrzeżenie prawne \\
\bottomrule
\end{longtable}

\subsubsection{ConflictResolution}

\begin{table}[H]
\centering
\caption{Pola interfejsu \tstype{ConflictResolution}}
\begin{tabularx}{\textwidth}{l l X}
\toprule
\textbf{Pole} & \textbf{Typ} & \textbf{Opis} \\
\midrule
\texttt{primary\_style} & enum & \texttt{direct\_confrontation | avoidant | explosive | passive\_aggressive | collaborative | humor\_deflection} \\
\texttt{triggers} & \tstype{string[]} & Tematy/sytuacje wyzwalające konflikt \\
\texttt{recovery\_speed} & enum & \texttt{fast | moderate | slow | unresolved} \\
\texttt{de\_escalation\_skills} & 1--10 & Umiejętności deeskalacji \\
\texttt{confidence} & 0--100 & Pewność \\
\bottomrule
\end{tabularx}
\end{table}

\subsubsection{EmotionalIntelligence}

\begin{table}[H]
\centering
\caption{Pola interfejsu \tstype{EmotionalIntelligence}}
\begin{tabularx}{\textwidth}{l l X}
\toprule
\textbf{Pole} & \textbf{Typ/Zakres} & \textbf{Opis} \\
\midrule
\texttt{empathy.score} & 1--10 & Empatia \\
\texttt{empathy.evidence} & \tstype{string} & Dowody \\
\texttt{self\_awareness.score} & 1--10 & Samoświadomość \\
\texttt{self\_awareness.evidence} & \tstype{string} & Dowody \\
\texttt{emotional\_regulation.score} & 1--10 & Regulacja emocjonalna \\
\texttt{emotional\_regulation.evidence} & \tstype{string} & Dowody \\
\texttt{social\_skills.score} & 1--10 & Umiejętności społeczne \\
\texttt{social\_skills.evidence} & \tstype{string} & Dowody \\
\texttt{overall} & 1--10 & Ogólny wynik EQ \\
\texttt{confidence} & 0--100 & Pewność \\
\bottomrule
\end{tabularx}
\end{table}

\subsubsection{MBTIResult}

\begin{table}[H]
\centering
\caption{Pola interfejsu \tstype{MBTIResult}}
\begin{tabularx}{\textwidth}{l l X}
\toprule
\textbf{Pole} & \textbf{Typ} & \textbf{Opis} \\
\midrule
\texttt{type} & \tstype{string} & 4-literowy typ MBTI (np. ,,INFJ'', ,,ENTP'') \\
\texttt{confidence} & 0--100 & Pewność ogólna \\
\texttt{reasoning.ie.letter} & \tstype{'I' | 'E'} & Introwersja vs. Ekstrawersja \\
\texttt{reasoning.ie.evidence} & \tstype{string} & Dowody \\
\texttt{reasoning.ie.confidence} & 0--100 & Pewność wymiaru \\
\texttt{reasoning.sn.letter} & \tstype{'S' | 'N'} & Odczuwanie vs. Intuicja \\
\texttt{reasoning.sn.evidence} & \tstype{string} & Dowody \\
\texttt{reasoning.sn.confidence} & 0--100 & Pewność wymiaru \\
\texttt{reasoning.tf.letter} & \tstype{'T' | 'F'} & Myślenie vs. Odczuwanie \\
\texttt{reasoning.tf.evidence} & \tstype{string} & Dowody \\
\texttt{reasoning.tf.confidence} & 0--100 & Pewność wymiaru \\
\texttt{reasoning.jp.letter} & \tstype{'J' | 'P'} & Osądzanie vs. Percepcja \\
\texttt{reasoning.jp.evidence} & \tstype{string} & Dowody \\
\texttt{reasoning.jp.confidence} & 0--100 & Pewność wymiaru \\
\bottomrule
\end{tabularx}
\end{table}

\subsubsection{LoveLanguageResult}

\begin{table}[H]
\centering
\caption{Pola interfejsu \tstype{LoveLanguageResult}}
\begin{tabularx}{\textwidth}{l l X}
\toprule
\textbf{Pole} & \textbf{Typ/Zakres} & \textbf{Opis} \\
\midrule
\texttt{primary} & enum & Główny język miłości: \texttt{words\_of\_affirmation | quality\_time | acts\_of\_service | gifts\_pebbling | physical\_touch} \\
\texttt{secondary} & enum & Drugorzędny język miłości \\
\texttt{scores.words\_of\_affirmation} & 0--100 & Wynik: Słowa potwierdzenia \\
\texttt{scores.quality\_time} & 0--100 & Wynik: Czas jakościowy \\
\texttt{scores.acts\_of\_service} & 0--100 & Wynik: Akty służby \\
\texttt{scores.gifts\_pebbling} & 0--100 & Wynik: Prezenty / pebbling \\
\texttt{scores.physical\_touch} & 0--100 & Wynik: Dotyk fizyczny \\
\texttt{evidence} & \tstype{string} & Dowody z~wiadomości \\
\texttt{confidence} & 0--100 & Pewność \\
\bottomrule
\end{tabularx}
\end{table}

\subsubsection{PersonProfile (kontener Pass 3)}

\begin{lstlisting}[style=podcode, caption={Interfejs PersonProfile}]
export interface PersonProfile {
  big_five_approximation: BigFiveApproximation;
  attachment_indicators: AttachmentIndicators;
  communication_profile: CommunicationProfile;
  communication_needs: CommunicationNeeds;
  emotional_patterns: EmotionalPatterns;
  clinical_observations: ClinicalObservations;
  conflict_resolution: ConflictResolution;
  emotional_intelligence: EmotionalIntelligence;
  mbti?: MBTIResult;
  love_language?: LoveLanguageResult;
}
\end{lstlisting}


% ────────────────────────────────────────────────────────────
\subsection{Pass 4: Synteza}
\label{subsec:pass4-types}

\subsubsection{HealthScoreComponents i~HealthScore}

\begin{table}[H]
\centering
\caption{Pola interfejsów Health Score}
\begin{tabularx}{\textwidth}{l l l X}
\toprule
\textbf{Interfejs} & \textbf{Pole} & \textbf{Zakres} & \textbf{Opis (waga)} \\
\midrule
\multirow{5}{*}{\tstype{HealthScoreComponents}} & \texttt{balance} & 0--100 & Równowaga wkładu (25\%) \\
 & \texttt{reciprocity} & 0--100 & Wzajemność (20\%) \\
 & \texttt{response\_pattern} & 0--100 & Wzorce odpowiedzi (20\%) \\
 & \texttt{emotional\_safety} & 0--100 & Bezpieczeństwo emocjonalne (20\%) \\
 & \texttt{growth\_trajectory} & 0--100 & Trajektoria wzrostu (15\%) \\
\midrule
\multirow{3}{*}{\tstype{HealthScore}} & \texttt{overall} & 0--100 & Wynik ważony (wzór \ref{eq:health-score} z~rozdz.~6) \\
 & \texttt{components} & \tstype{HealthScoreComponents} & Poszczególne komponenty \\
 & \texttt{explanation} & \tstype{string} & Co napędza wynik w~górę/dół \\
\bottomrule
\end{tabularx}
\end{table}

Wzór na wynik ogólny (powtórzony z~rozdziału~\ref{ch:analiza-ai}):
\begin{equation}
\text{overall} = 0.25b + 0.20r + 0.20p + 0.20s + 0.15g
\tag{12.1}
\end{equation}
gdzie $b$ = balance, $r$ = reciprocity, $p$ = response\_pattern, $s$ = emotional\_safety, $g$ = growth\_trajectory.

\subsubsection{KeyFinding, InflectionPoint, RelationshipTrajectory}

\begin{table}[H]
\centering
\caption{Pola interfejsów Pass 4 --- obserwacje i~trajektoria}
\begin{tabularx}{\textwidth}{l l l X}
\toprule
\textbf{Interfejs} & \textbf{Pole} & \textbf{Typ} & \textbf{Opis} \\
\midrule
\multirow{3}{*}{\tstype{KeyFinding}} & \texttt{finding} & \tstype{string} & Jedno zdanie \\
 & \texttt{significance} & enum & \texttt{positive | neutral | concerning} \\
 & \texttt{detail} & \tstype{string} & 2--3 zdania kontekstu \\
\midrule
\multirow{3}{*}{\tstype{InflectionPoint}} & \texttt{approximate\_date} & \tstype{string} & Format YYYY-MM \\
 & \texttt{description} & \tstype{string} & Co się zmieniło i~dlaczego \\
 & \texttt{evidence} & \tstype{string} & Dowody \\
\midrule
\multirow{3}{*}{\tstype{RelationshipTrajectory}} & \texttt{current\_phase} & \tstype{string} & Obecna faza relacji \\
 & \texttt{direction} & enum & \texttt{strengthening | stable | weakening | volatile} \\
 & \texttt{inflection\_points} & \tstype{InflectionPoint[]} & Lista punktów przełomowych \\
\bottomrule
\end{tabularx}
\end{table}

\subsubsection{Insight}

\begin{table}[H]
\centering
\caption{Pola interfejsu \tstype{Insight}}
\begin{tabularx}{\textwidth}{l l X}
\toprule
\textbf{Pole} & \textbf{Typ} & \textbf{Opis} \\
\midrule
\texttt{for} & \tstype{string} & Dla kogo: nazwa osoby lub \texttt{"both"} \\
\texttt{insight} & \tstype{string} & Konkretna, wykonalna obserwacja \\
\texttt{priority} & enum & \texttt{high | medium | low} \\
\bottomrule
\end{tabularx}
\end{table}

\subsubsection{ConversationPersonality}

\begin{table}[H]
\centering
\caption{Pola interfejsu \tstype{ConversationPersonality}}
\begin{tabularx}{\textwidth}{l l X}
\toprule
\textbf{Pole} & \textbf{Typ} & \textbf{Opis} \\
\midrule
\texttt{if\_this\_conversation\_were\_a.movie\_genre} & \tstype{string} & Gatunek filmowy \\
\texttt{if\_this\_conversation\_were\_a.weather} & \tstype{string} & Pogoda \\
\texttt{if\_this\_conversation\_were\_a.one\_word} & \tstype{string} & Jedno słowo \\
\bottomrule
\end{tabularx}
\end{table}

\subsubsection{Pass4Result (kontener)}

\begin{lstlisting}[style=podcode, caption={Interfejs Pass4Result}]
export interface Pass4Result {
  executive_summary: string;
  health_score: HealthScore;
  key_findings: KeyFinding[];
  relationship_trajectory: RelationshipTrajectory;
  insights: Insight[];
  conversation_personality: ConversationPersonality;
}
\end{lstlisting}


% ────────────────────────────────────────────────────────────
\subsection{RoastResult}
\label{subsec:roast-type}

\begin{lstlisting}[style=podcode, caption={Interfejs RoastResult}]
export interface RoastResult {
  roasts_per_person: Record<string, string[]>;
  relationship_roast: string;
  superlatives: Array<{
    title: string;
    holder: string;
    roast: string;
  }>;
  verdict: string;
}
\end{lstlisting}

\begin{table}[H]
\centering
\caption{Pola interfejsu \tstype{RoastResult}}
\begin{tabularx}{\textwidth}{l l X}
\toprule
\textbf{Pole} & \textbf{Typ} & \textbf{Opis} \\
\midrule
\texttt{roasts\_per\_person} & \tstype{Record<string, string[]>} & Nazwa osoby $\rightarrow$ tablica roastów (4--6 per osoba) \\
\texttt{relationship\_roast} & \tstype{string} & Akapit roastujący dynamikę relacji (3--4 zdania) \\
\texttt{superlatives} & \tstype{Array} & Zabawne tytuły/odznaki: \texttt{title}, \texttt{holder}, \texttt{roast} \\
\texttt{verdict} & \tstype{string} & Jedno zdanie --- brutalny werdykt \\
\bottomrule
\end{tabularx}
\end{table}


% ────────────────────────────────────────────────────────────
\subsection{StandUpAct i~StandUpRoastResult}
\label{subsec:standup-type}

\begin{lstlisting}[style=podcode, caption={Interfejsy Stand-Up Roast}]
export interface StandUpAct {
  number: number;
  title: string;
  emoji: string;
  lines: string[];
  callback?: string;
  gradientColors: [string, string];
}

export interface StandUpRoastResult {
  showTitle: string;
  acts: StandUpAct[];
  closingLine: string;
  audienceRating: string;
}
\end{lstlisting}

\begin{table}[H]
\centering
\caption{Pola interfejsów Stand-Up Roast}
\begin{tabularx}{\textwidth}{l l l X}
\toprule
\textbf{Interfejs} & \textbf{Pole} & \textbf{Typ} & \textbf{Opis} \\
\midrule
\multirow{6}{*}{\tstype{StandUpAct}} & \texttt{number} & \tstype{number} & Numer aktu (1--7) \\
 & \texttt{title} & \tstype{string} & Tytuł aktu (np. ,,Otwarcie'') \\
 & \texttt{emoji} & \tstype{string} & Emoji dla aktu \\
 & \texttt{lines} & \tstype{string[]} & 4--6 punchline'ów \\
 & \texttt{callback} & \tstype{string?} & Nawiązanie do wcześniejszego aktu \\
 & \texttt{gradientColors} & \tstype{[string, string]} & Para kolorów hex gradientu \\
\midrule
\multirow{4}{*}{\tstype{StandUpRoastResult}} & \texttt{showTitle} & \tstype{string} & Kreatywny tytuł występu \\
 & \texttt{acts} & \tstype{StandUpAct[]} & 7 aktów \\
 & \texttt{closingLine} & \tstype{string} & Nokautująca kwestia końcowa \\
 & \texttt{audienceRating} & \tstype{string} & Zabawna ocena publiczności \\
\bottomrule
\end{tabularx}
\end{table}


% ────────────────────────────────────────────────────────────
\subsection{Typy CPS (Communication Pattern Screening)}
\label{subsec:cps-types}

Zdefiniowane w~pliku: \filepath{src/lib/analysis/communication-patterns.ts}

\subsubsection{CPSQuestion i~CPSPattern}

\begin{table}[H]
\centering
\caption{Pola interfejsów \tstype{CPSQuestion} i~\tstype{CPSPattern}}
\begin{tabularx}{\textwidth}{l l l X}
\toprule
\textbf{Interfejs} & \textbf{Pole} & \textbf{Typ} & \textbf{Opis} \\
\midrule
\multirow{4}{*}{\tstype{CPSQuestion}} & \texttt{id} & \tstype{number} & Numer pytania (1--63) \\
 & \texttt{text} & \tstype{string} & Treść pytania (po polsku, oryginalna) \\
 & \texttt{pattern} & \tstype{string} & Klucz wzorca, do którego należy \\
 & \texttt{messageSignals} & \tstype{string} & Sygnały do szukania w~wiadomościach (EN, wewnętrzne) \\
\midrule
\multirow{7}{*}{\tstype{CPSPattern}} & \texttt{key} & \tstype{string} & Klucz wzorca (np. \texttt{intimacy\_avoidance}) \\
 & \texttt{name} & \tstype{string} & Nazwa po polsku (opisowa, nie kliniczna) \\
 & \texttt{nameEn} & \tstype{string} & Nazwa po angielsku \\
 & \texttt{description} & \tstype{string} & 1-zdaniowy opis \\
 & \texttt{color} & \tstype{string} & Kolor hex do wizualizacji \\
 & \texttt{threshold} & \tstype{number} & Minimum odpowiedzi ,,tak'' do przekroczenia progu \\
 & \texttt{recommendation} & \tstype{string} & Rekomendacja (PL) jeśli próg przekroczony \\
\bottomrule
\end{tabularx}
\end{table}

\subsubsection{CPSAnswer, CPSPatternResult, CPSResult}

\begin{table}[H]
\centering
\caption{Pola interfejsów wynikowych CPS}
\begin{tabularx}{\textwidth}{l l l X}
\toprule
\textbf{Interfejs} & \textbf{Pole} & \textbf{Typ} & \textbf{Opis} \\
\midrule
\multirow{3}{*}{\tstype{CPSAnswer}} & \texttt{answer} & \tstype{boolean | null} & Odpowiedź: \texttt{true}, \texttt{false}, lub \texttt{null} (nieocenialne) \\
 & \texttt{confidence} & 0--100 & Pewność oceny \\
 & \texttt{evidence} & \tstype{string[]} & Cytaty lub opisy wzorców (maks. 5) \\
\midrule
\multirow{7}{*}{\tstype{CPSPatternResult}} & \texttt{yesCount} & \tstype{number} & Liczba odpowiedzi ,,tak'' \\
 & \texttt{total} & \tstype{number} & Łączna liczba ocenionych pytań \\
 & \texttt{threshold} & \tstype{number} & Próg dla tego wzorca \\
 & \texttt{meetsThreshold} & \tstype{boolean} & yesCount $\geq$ threshold \\
 & \texttt{percentage} & 0--100 & $\min(100, \lfloor\frac{\text{yesCount}}{\text{threshold}} \times 100\rfloor)$ \\
 & \texttt{confidence} & 0--100 & Średnia pewność odpowiedzi wzorca \\
 & \texttt{answers} & \tstype{Record<number, CPSAnswer>} & Odpowiedzi na pytania tego wzorca \\
\midrule
\multirow{6}{*}{\tstype{CPSResult}} & \texttt{answers} & \tstype{Record<number, CPSAnswer>} & Wszystkie 63 odpowiedzi \\
 & \texttt{patterns} & \tstype{Record<string, CPSPatternResult>} & Wyniki 10 wzorców \\
 & \texttt{overallConfidence} & 0--100 & Średnia pewność wszystkich odpowiedzi \\
 & \texttt{disclaimer} & \tstype{string} & Zastrzeżenie prawne \\
 & \texttt{analyzedAt} & \tstype{number} & Unix timestamp ms \\
 & \texttt{participantName} & \tstype{string} & Czyj profil analizowano \\
\bottomrule
\end{tabularx}
\end{table}


% ============================================================
\section{Typy magazynowania}
\label{sec:storage-types}

Typy definiujące strukturę danych zapisywanych w~\texttt{localStorage} przeglądarki.

% ────────────────────────────────────────────────────────────
\subsection{QualitativeAnalysis}
\label{subsec:qualitative-analysis}

Kontener na wszystkie wyniki analizy AI, z~informacją o~statusie wykonania.

\begin{lstlisting}[style=podcode, caption={Interfejs QualitativeAnalysis}]
export interface QualitativeAnalysis {
  status: 'pending' | 'running' | 'complete'
        | 'partial' | 'error';
  error?: string;
  currentPass?: number;
  pass1?: Pass1Result;
  pass2?: Pass2Result;
  pass3?: Record<string, PersonProfile>;
  pass4?: Pass4Result;
  roast?: RoastResult;
  cps?: CPSResult;
  standupRoast?: StandUpRoastResult;
  completedAt?: number;
}
\end{lstlisting}

\begin{longtable}{l l l p{5.5cm}}
\caption{Pola interfejsu \tstype{QualitativeAnalysis}} \label{tab:qualitative-analysis} \\
\toprule
\textbf{Pole} & \textbf{Typ} & \textbf{Wym.} & \textbf{Opis} \\
\midrule
\endfirsthead
\toprule
\textbf{Pole} & \textbf{Typ} & \textbf{Wym.} & \textbf{Opis} \\
\midrule
\endhead
\texttt{status} & enum & tak & Stan wykonania: \texttt{pending} (oczekuje), \texttt{running} (w~trakcie), \texttt{complete} (ukończono), \texttt{partial} (częściowy sukces), \texttt{error} (porażka) \\
\texttt{error} & \tstype{string} & nie & Komunikat błędu (jeśli \texttt{status = 'error' | 'partial'}) \\
\texttt{currentPass} & \tstype{number} & nie & Numer aktualnie wykonywanego przebiegu (1--4) \\
\texttt{pass1} & \tstype{Pass1Result} & nie & Wynik Pass~1 (ton, styl, typ relacji) \\
\texttt{pass2} & \tstype{Pass2Result} & nie & Wynik Pass~2 (dynamika, konflikt, bliskość) \\
\texttt{pass3} & \tstype{Record<PersonProfile>} & nie & Wyniki Pass~3, kluczowane po nazwie uczestnika \\
\texttt{pass4} & \tstype{Pass4Result} & nie & Wynik Pass~4 (synteza, health score, insights) \\
\texttt{roast} & \tstype{RoastResult} & nie & Wynik trybu Roast \\
\texttt{cps} & \tstype{CPSResult} & nie & Wynik Pass~5 CPS \\
\texttt{standupRoast} & \tstype{StandUpRoastResult} & nie & Wynik Stand-Up Roast \\
\texttt{completedAt} & \tstype{number} & nie & Unix timestamp ms ukończenia analizy \\
\bottomrule
\end{longtable}


% ────────────────────────────────────────────────────────────
\subsection{StoredAnalysis}
\label{subsec:stored-analysis}

Pełny rekord analizy utrwalany w~\texttt{localStorage}.

\begin{lstlisting}[style=podcode, caption={Interfejs StoredAnalysis}]
export interface StoredAnalysis {
  id: string;
  title: string;
  createdAt: number;
  relationshipContext?: RelationshipContext;
  conversation: ParsedConversation;
  quantitative: QuantitativeAnalysis;
  qualitative?: QualitativeAnalysis;
}
\end{lstlisting}

\begin{table}[H]
\centering
\caption{Pola interfejsu \tstype{StoredAnalysis}}
\begin{tabularx}{\textwidth}{l l l X}
\toprule
\textbf{Pole} & \textbf{Typ} & \textbf{Wym.} & \textbf{Opis} \\
\midrule
\texttt{id} & \tstype{string} & tak & Unikalny identyfikator (UUID) \\
\texttt{title} & \tstype{string} & tak & Tytuł rozmowy \\
\texttt{createdAt} & \tstype{number} & tak & Unix timestamp ms utworzenia analizy \\
\texttt{relationshipContext} & \tstype{RelationshipContext} & nie & Zadeklarowany typ relacji: \texttt{romantic | friendship | colleague | professional | family | other} \\
\texttt{conversation} & \tstype{ParsedConversation} & tak & Sparsowana rozmowa (pełna) \\
\texttt{quantitative} & \tstype{QuantitativeAnalysis} & tak & Wyniki analizy ilościowej \\
\texttt{qualitative} & \tstype{QualitativeAnalysis} & nie & Wyniki analizy AI (jeśli uruchomiono) \\
\bottomrule
\end{tabularx}
\end{table}


% ────────────────────────────────────────────────────────────
\subsection{AnalysisIndexEntry}
\label{subsec:index-entry}

Lekki wpis do listy analiz na dashboardzie.

\begin{lstlisting}[style=podcode, caption={Interfejs AnalysisIndexEntry}]
export interface AnalysisIndexEntry {
  id: string;
  title: string;
  createdAt: number;
  messageCount: number;
  participants: string[];
  hasQualitative: boolean;
  healthScore?: number;
}
\end{lstlisting}

\begin{table}[H]
\centering
\caption{Pola interfejsu \tstype{AnalysisIndexEntry}}
\begin{tabularx}{\textwidth}{l l l X}
\toprule
\textbf{Pole} & \textbf{Typ} & \textbf{Wym.} & \textbf{Opis} \\
\midrule
\texttt{id} & \tstype{string} & tak & UUID analizy \\
\texttt{title} & \tstype{string} & tak & Tytuł rozmowy \\
\texttt{createdAt} & \tstype{number} & tak & Unix timestamp ms \\
\texttt{messageCount} & \tstype{number} & tak & Łączna liczba wiadomości \\
\texttt{participants} & \tstype{string[]} & tak & Nazwy uczestników \\
\texttt{hasQualitative} & \tstype{boolean} & tak & Czy przeprowadzono analizę AI \\
\texttt{healthScore} & \tstype{number} & nie & Health Score 0--100 (jeśli dostępny) \\
\bottomrule
\end{tabularx}
\end{table}


% ============================================================
\section{Diagramy relacji typów}
\label{sec:type-diagrams}

Poniższe diagramy UML ilustrują zależności kompozycji między kluczowymi typami \podtekst.

\subsection{Diagram główny: StoredAnalysis}

\begin{figure}[H]
\centering
\begin{tikzpicture}[
  node distance=1.2cm and 2.5cm,
  every node/.style={font=\small},
  classbox/.style={
    rectangle,
    draw=PodBlue!50,
    fill=PodBlue!5,
    rounded corners=3pt,
    minimum width=3.8cm,
    minimum height=0.9cm,
    align=center,
    font=\small\bfseries,
    text=PodBlueDark,
    inner sep=6pt,
  },
  classbox purple/.style={
    classbox,
    draw=PodPurple!50,
    fill=PodPurple!5,
    text=PodPurpleDark,
  },
  classbox green/.style={
    classbox,
    draw=PodSuccess!50,
    fill=PodSuccess!5,
    text=PodSuccess!80!black,
  },
  composition/.style={
    ->,
    >=open diamond,
    thick,
    color=PodBlue!60,
  },
  optional/.style={
    ->,
    >=open diamond,
    thick,
    dashed,
    color=PodPurple!50,
  },
]
  % ── Root ──
  \node[classbox, minimum width=5cm, fill=PodBlue!12] (stored) {StoredAnalysis};

  % ── Level 1 ──
  \node[classbox, below left=1.5cm and 1cm of stored] (conv) {ParsedConversation};
  \node[classbox, below=1.5cm of stored] (quant) {QuantitativeAnalysis};
  \node[classbox purple, below right=1.5cm and 1cm of stored] (qual) {QualitativeAnalysis};

  % ── Level 2 — Quantitative ──
  \node[classbox, below left=1.5cm and -0.5cm of quant] (person) {PersonMetrics};
  \node[classbox, below=1.5cm of quant] (timing) {TimingMetrics};
  \node[classbox, below right=1.5cm and -1.3cm of quant] (engage) {EngagementMetrics};

  \node[classbox, below left=2.8cm and 0.5cm of quant] (pattern) {PatternMetrics};
  \node[classbox, below=2.8cm of quant] (heat) {HeatmapData};
  \node[classbox, below right=2.8cm and 0.5cm of quant] (trend) {TrendData};

  \node[classbox green, below=4cm of quant] (viral) {ViralScores};

  % ── Level 2 — Qualitative ──
  \node[classbox purple, below left=1.5cm and -0.5cm of qual] (p1) {Pass1Result};
  \node[classbox purple, below=1.5cm of qual] (p2) {Pass2Result};
  \node[classbox purple, below right=1.5cm and -0.5cm of qual] (p3) {PersonProfile};

  \node[classbox purple, below=2.8cm of qual] (p4) {Pass4Result};
  \node[classbox green, below left=4cm and -0.5cm of qual] (roast) {RoastResult};
  \node[classbox green, below right=4cm and -0.5cm of qual] (cps) {CPSResult};

  % ── Level 1 arrows ──
  \draw[composition] (stored) -- (conv);
  \draw[composition] (stored) -- (quant);
  \draw[optional] (stored) -- (qual);

  % ── Level 2 — Quant arrows ──
  \draw[composition] (quant) -- (person);
  \draw[composition] (quant) -- (timing);
  \draw[composition] (quant) -- (engage);
  \draw[composition] (quant) -- (pattern);
  \draw[composition] (quant) -- (heat);
  \draw[composition] (quant) -- (trend);
  \draw[optional] (quant) -- (viral);

  % ── Level 2 — Qual arrows ──
  \draw[optional] (qual) -- (p1);
  \draw[optional] (qual) -- (p2);
  \draw[optional] (qual) -- (p3);
  \draw[optional] (qual) -- (p4);
  \draw[optional] (qual) -- (roast);
  \draw[optional] (qual) -- (cps);

  % ── Legend ──
  \node[anchor=north west, font=\scriptsize, text=PodTextMuted] at (-6.5, -7.5) {
    \begin{tabular}{ll}
    \tikz\draw[composition] (0,0) -- (0.8,0); & Kompozycja (wymagane) \\
    \tikz\draw[optional] (0,0) -- (0.8,0); & Kompozycja (opcjonalne) \\
    \end{tabular}
  };

\end{tikzpicture}
\caption{Diagram kompozycji \tstype{StoredAnalysis} --- główny kontener danych.
Linie ciągłe oznaczają wymagane składniki, przerywane --- opcjonalne.}
\label{fig:stored-analysis-uml}
\end{figure}


\subsection{Diagram: QuantitativeAnalysis --- szczegóły}

\begin{figure}[H]
\centering
\begin{tikzpicture}[
  node distance=0.8cm and 1.5cm,
  every node/.style={font=\scriptsize},
  qbox/.style={
    rectangle,
    draw=PodBlue!40,
    fill=PodBlue!5,
    rounded corners=2pt,
    minimum width=3.2cm,
    minimum height=0.7cm,
    align=center,
    font=\scriptsize\bfseries,
    text=PodBlueDark,
    inner sep=4pt,
  },
  obox/.style={
    qbox,
    draw=PodSuccess!40,
    fill=PodSuccess!5,
    text=PodSuccess!80!black,
  },
  arr/.style={->, >=stealth, thick, PodBlue!50},
  oarr/.style={->, >=stealth, dashed, PodSuccess!50},
]
  \node[qbox, minimum width=4.5cm, fill=PodBlue!10] (root) {QuantitativeAnalysis};

  % Required
  \node[qbox, below left=1.0cm and 2.5cm of root] (pm) {Record<PersonMetrics>};
  \node[qbox, below left=1.0cm and 0cm of root] (tm) {TimingMetrics};
  \node[qbox, below right=1.0cm and 0cm of root] (em) {EngagementMetrics};
  \node[qbox, below right=1.0cm and 2.5cm of root] (ptm) {PatternMetrics};

  \node[qbox, below=2.0cm of root, xshift=-2cm] (hm) {HeatmapData};
  \node[qbox, below=2.0cm of root, xshift=2cm] (td) {TrendData};

  % Optional
  \node[obox, below=3.0cm of root, xshift=-4cm] (vs) {ViralScores?};
  \node[obox, below=3.0cm of root, xshift=-1.5cm] (bd) {Badge[]?};
  \node[obox, below=3.0cm of root, xshift=1cm] (btt) {BestTimeToText?};
  \node[obox, below=3.0cm of root, xshift=3.5cm] (cp) {CatchphraseResult?};

  \node[obox, below=4.0cm of root, xshift=-2cm] (nm) {NetworkMetrics?};
  \node[obox, below=4.0cm of root, xshift=2cm] (ri) {ReciprocityIndex?};

  % Arrows
  \draw[arr] (root) -- (pm);
  \draw[arr] (root) -- (tm);
  \draw[arr] (root) -- (em);
  \draw[arr] (root) -- (ptm);
  \draw[arr] (root) -- (hm);
  \draw[arr] (root) -- (td);
  \draw[oarr] (root) -- (vs);
  \draw[oarr] (root) -- (bd);
  \draw[oarr] (root) -- (btt);
  \draw[oarr] (root) -- (cp);
  \draw[oarr] (root) -- (nm);
  \draw[oarr] (root) -- (ri);

\end{tikzpicture}
\caption{Szczegółowy diagram \tstype{QuantitativeAnalysis}.
Pola wymagane (linia ciągła) i~opcjonalne (linia przerywana, oznaczone \texttt{?}).}
\label{fig:quant-uml}
\end{figure}


\subsection{Diagram: QualitativeAnalysis --- szczegóły}

\begin{figure}[H]
\centering
\begin{tikzpicture}[
  node distance=0.8cm and 1.5cm,
  every node/.style={font=\scriptsize},
  pbox/.style={
    rectangle,
    draw=PodPurple!40,
    fill=PodPurple!5,
    rounded corners=2pt,
    minimum width=3.2cm,
    minimum height=0.7cm,
    align=center,
    font=\scriptsize\bfseries,
    text=PodPurpleDark,
    inner sep=4pt,
  },
  arr/.style={->, >=stealth, dashed, PodPurple!50},
]
  \node[pbox, minimum width=4.5cm, fill=PodPurple!10] (root) {QualitativeAnalysis};

  % Pass results
  \node[pbox, below left=1.0cm and 3cm of root] (p1) {Pass1Result?};
  \node[pbox, below left=1.0cm and 0.5cm of root] (p2) {Pass2Result?};
  \node[pbox, below right=1.0cm and 0.5cm of root] (p3) {Record<PersonProfile>?};
  \node[pbox, below right=1.0cm and 3cm of root] (p4) {Pass4Result?};

  % Optional extras
  \node[pbox, below=2.2cm of root, xshift=-3cm] (roast) {RoastResult?};
  \node[pbox, below=2.2cm of root] (standup) {StandUpRoastResult?};
  \node[pbox, below=2.2cm of root, xshift=3cm] (cps) {CPSResult?};

  % Sub-types of Pass2
  \node[pbox, below=1.8cm of p2, xshift=-1cm] (pd) {PowerDynamics};
  \node[pbox, below=1.8cm of p2, xshift=1.5cm] (el) {EmotionalLabor};
  \node[pbox, below=2.8cm of p2, xshift=-1cm] (cp2) {ConflictPatterns};
  \node[pbox, below=2.8cm of p2, xshift=1.5cm] (im) {IntimacyMarkers};

  % Sub-types of PersonProfile
  \node[pbox, below=1.8cm of p3, xshift=-1cm] (bf) {BigFiveApproximation};
  \node[pbox, below=1.8cm of p3, xshift=1.5cm] (ai) {AttachmentIndicators};
  \node[pbox, below=2.8cm of p3, xshift=-1cm] (cpr) {CommunicationProfile};
  \node[pbox, below=2.8cm of p3, xshift=1.5cm] (ei) {EmotionalIntelligence};

  % Arrows from root
  \draw[arr] (root) -- (p1);
  \draw[arr] (root) -- (p2);
  \draw[arr] (root) -- (p3);
  \draw[arr] (root) -- (p4);
  \draw[arr] (root) -- (roast);
  \draw[arr] (root) -- (standup);
  \draw[arr] (root) -- (cps);

  % Sub arrows
  \draw[arr] (p2) -- (pd);
  \draw[arr] (p2) -- (el);
  \draw[arr] (p2) -- (cp2);
  \draw[arr] (p2) -- (im);

  \draw[arr] (p3) -- (bf);
  \draw[arr] (p3) -- (ai);
  \draw[arr] (p3) -- (cpr);
  \draw[arr] (p3) -- (ei);

\end{tikzpicture}
\caption{Szczegółowy diagram \tstype{QualitativeAnalysis} z~kluczowymi pod-typami
Pass~2 i~Pass~3. Wszystkie pola opcjonalne (przerywane strzałki).}
\label{fig:qual-uml}
\end{figure}


\subsection{Diagram: PersonProfile --- pełne drzewo}

\begin{figure}[H]
\centering
\begin{tikzpicture}[
  node distance=0.6cm and 1.2cm,
  every node/.style={font=\scriptsize},
  leaf/.style={
    rectangle,
    draw=PodBlue!30,
    fill=white,
    rounded corners=2pt,
    minimum width=2.8cm,
    minimum height=0.5cm,
    align=center,
    font=\scriptsize,
    text=PodBlueDark,
    inner sep=3pt,
  },
  root/.style={
    leaf,
    draw=PodPurple!50,
    fill=PodPurple!8,
    font=\scriptsize\bfseries,
    text=PodPurpleDark,
    minimum width=3.5cm,
  },
  arr/.style={->, >=stealth, thin, PodBlue!40},
]
  \node[root] (pp) {PersonProfile};

  % Left column
  \node[leaf, below left=0.8cm and 2.5cm of pp] (bf) {BigFiveApproximation};
  \node[leaf, below=0.5cm of bf] (att) {AttachmentIndicators};
  \node[leaf, below=0.5cm of att] (comm) {CommunicationProfile};
  \node[leaf, below=0.5cm of comm] (needs) {CommunicationNeeds};
  \node[leaf, below=0.5cm of needs] (emot) {EmotionalPatterns};

  % Right column
  \node[leaf, below right=0.8cm and 2.5cm of pp] (clin) {ClinicalObservations};
  \node[leaf, below=0.5cm of clin] (conf) {ConflictResolution};
  \node[leaf, below=0.5cm of conf] (ei) {EmotionalIntelligence};
  \node[leaf, below=0.5cm of ei] (mbti) {MBTIResult?};
  \node[leaf, below=0.5cm of mbti] (love) {LoveLanguageResult?};

  % Arrows
  \draw[arr] (pp) -- (bf);
  \draw[arr] (pp) -- (att);
  \draw[arr] (pp) -- (comm);
  \draw[arr] (pp) -- (needs);
  \draw[arr] (pp) -- (emot);
  \draw[arr] (pp) -- (clin);
  \draw[arr] (pp) -- (conf);
  \draw[arr] (pp) -- (ei);
  \draw[arr] (pp) -- (mbti);
  \draw[arr] (pp) -- (love);

\end{tikzpicture}
\caption{Drzewo kompozycji \tstype{PersonProfile} --- 10~sekcji tematycznych.
\texttt{MBTIResult} i~\texttt{LoveLanguageResult} są opcjonalne.}
\label{fig:person-profile-tree}
\end{figure}


\subsection{Podsumowanie zależności}

\begin{table}[H]
\centering
\caption{Mapa zależności typów --- ile interfejsów zawiera każdy kontener}
\label{tab:type-dependency-map}
\begin{tabularx}{\textwidth}{l c X}
\toprule
\textbf{Kontener} & \textbf{Podtypów} & \textbf{Kluczowe składniki} \\
\midrule
\tstype{StoredAnalysis} & 3 & ParsedConversation, QuantitativeAnalysis, QualitativeAnalysis? \\
\tstype{QuantitativeAnalysis} & 12 & PersonMetrics, TimingMetrics, EngagementMetrics, PatternMetrics, HeatmapData, TrendData, ViralScores?, Badge[]?, BestTimeToText?, CatchphraseResult?, NetworkMetrics?, ReciprocityIndex? \\
\tstype{QualitativeAnalysis} & 7 & Pass1Result?, Pass2Result?, Record<PersonProfile>?, Pass4Result?, RoastResult?, StandUpRoastResult?, CPSResult? \\
\tstype{PersonProfile} & 10 & BigFiveApproximation, AttachmentIndicators, CommunicationProfile, CommunicationNeeds, EmotionalPatterns, ClinicalObservations, ConflictResolution, EmotionalIntelligence, MBTIResult?, LoveLanguageResult? \\
\tstype{Pass2Result} & 7 & PowerDynamics, EmotionalLabor, ConflictPatterns, IntimacyMarkers, RedFlag[], GreenFlag[], relationship\_phase? \\
\tstype{CPSResult} & 4 & CPSAnswer (63$\times$), CPSPatternResult (10$\times$), disclaimer, participantName \\
\bottomrule
\end{tabularx}
\end{table}

Łączna liczba zdefiniowanych interfejsów w~\podtekst: \textbf{47}, w~tym:
\begin{itemize}
  \item 15 interfejsów parserowych/ilościowych
  \item 25 interfejsów analizy AI
  \item 4 interfejsy CPS
  \item 3 interfejsy magazynowania
\end{itemize}

% ============================================================
\section{Typy Dekodera Podtekstów}
\label{sec:subtext-types}
% ============================================================

Typy z~tej sekcji zdefiniowane są w~pliku:

\filepath{src/lib/analysis/subtext.ts}

Dekoder Podtekstów (Subtext Decoder) stanowi osobny moduł analizy AI, który interpretuje ukryte znaczenia w~wybranych wiadomościach. Wykorzystuje system okien kontekstowych (\tstype{ExchangeWindow}) do ekstrakcji fragmentów rozmowy, które następnie są przetwarzane wsadowo przez model \gemini.


% ────────────────────────────────────────────────────────────
\subsection{SubtextCategory}
\label{subsec:subtext-category}

Typ unii literałowej definiujący 12~kategorii ukrytych podtekstów.

\begin{lstlisting}[style=podcode, caption={Typ SubtextCategory}]
type SubtextCategory =
  | 'deflection'         // unikanie tematu
  | 'hidden_anger'       // ukryty gniew
  | 'seeking_validation' // szukanie potwierdzenia
  | 'power_move'         // gra o wladze
  | 'genuine'            // szczere
  | 'testing'            // testowanie partnera
  | 'guilt_trip'         // wzbudzanie poczucia winy
  | 'passive_aggressive' // bierna agresja
  | 'love_signal'        // ukryty sygnal milosci
  | 'insecurity'         // niepewnosc
  | 'distancing'         // dystansowanie sie
  | 'humor_shield';      // humor jako tarcza
\end{lstlisting}

\begin{table}[H]
\centering
\caption{Wartości \tstype{SubtextCategory} z~opisami}
\label{tab:subtext-categories-types}
\begin{tabularx}{\textwidth}{l l X}
\toprule
\textbf{Wartość} & \textbf{Pol. etykieta} & \textbf{Opis} \\
\midrule
\texttt{deflection} & Unikanie tematu & Zmiana tematu lub unik, gdy rozmowa staje się niewygodna \\
\texttt{hidden\_anger} & Ukryty gniew & Kontrolowany gniew maskowany neutralnym tonem \\
\texttt{seeking\_validation} & Szukanie potwierdzenia & Subtelne próby uzyskania aprobaty lub zapewnienia \\
\texttt{power\_move} & Gra o~władzę & Próba przejęcia kontroli nad dynamiką rozmowy \\
\texttt{genuine} & Szczere & Brak ukrytego podtekstu --- wiadomość jest autentyczna \\
\texttt{testing} & Testowanie partnera & Prowokacja lub próba sprawdzenia reakcji drugiej strony \\
\texttt{guilt\_trip} & Wzbudzanie winy & Manipulacja poprzez wywoływanie poczucia winy \\
\texttt{passive\_aggressive} & Bierna agresja & Agresja wyrażona w~sposób pośredni lub pozornie neutralny \\
\texttt{love\_signal} & Ukryty sygnał miłości & Uczucia wyrażone nie wprost, zamaskowane humorem lub nonszalancją \\
\texttt{insecurity} & Niepewność & Lęk lub brak pewności siebie ujawniający się w~stylu komunikacji \\
\texttt{distancing} & Dystansowanie się & Emocjonalne wycofywanie się z~relacji lub tematu \\
\texttt{humor\_shield} & Humor jako tarcza & Używanie żartów do unikania wrażliwych tematów \\
\bottomrule
\end{tabularx}
\end{table}


% ────────────────────────────────────────────────────────────
\subsection{SubtextItem}
\label{subsec:subtext-item}

Pojedynczy zdekodowany podtekst --- centralny element wyników dekodera.

\begin{lstlisting}[style=podcode, caption={Interfejs SubtextItem}]
interface SubtextItem {
  originalMessage: string;
  sender: string;
  timestamp: number;
  subtext: string;
  emotion: string;
  confidence: number;      // 0-100
  category: SubtextCategory;
  isHighlight: boolean;
  exchangeContext: string;
  windowId: number;
  surroundingMessages: Array<{
    sender: string;
    content: string;
    timestamp: number;
  }>;
}
\end{lstlisting}

\begin{table}[H]
\centering
\caption{Pola interfejsu \tstype{SubtextItem}}
\label{tab:subtext-item}
\begin{tabularx}{\textwidth}{l l l X}
\toprule
\textbf{Pole} & \textbf{Typ} & \textbf{Wym.} & \textbf{Opis} \\
\midrule
\texttt{originalMessage} & \tstype{string} & tak & Oryginalna treść wiadomości (po dekodowaniu Unicode) \\
\texttt{sender} & \tstype{string} & tak & Nazwa nadawcy wiadomości \\
\texttt{timestamp} & \tstype{number} & tak & Unix timestamp ms oryginalnej wiadomości \\
\texttt{subtext} & \tstype{string} & tak & Zdekodowany podtekst --- co naprawdę miała na myśli osoba \\
\texttt{emotion} & \tstype{string} & tak & Dominująca emocja ukryta w~wiadomości (np.~,,frustracja'', ,,tęsknota'') \\
\texttt{confidence} & 0--100 & tak & Pewność AI co do poprawności dekodowania podtekstu \\
\texttt{category} & \tstype{SubtextCategory} & tak & Jedna z~12 kategorii podtekstu \\
\texttt{isHighlight} & \tstype{boolean} & tak & Czy wiadomość jest szczególnie odkrywcza (top 20\%) \\
\texttt{exchangeContext} & \tstype{string} & tak & Krótki opis kontekstu wymiany zdań \\
\texttt{windowId} & \tstype{number} & tak & Identyfikator okna kontekstowego (\tstype{ExchangeWindow}) \\
\texttt{surroundingMessages} & \tstype{Array} & tak & Wiadomości otaczające --- kontekst dla interpretacji (3--5 wiadomości) \\
\bottomrule
\end{tabularx}
\end{table}

\begin{figure}[H]
\centering
\begin{tikzpicture}[
  node distance=0.6cm and 1.4cm,
  every node/.style={font=\scriptsize},
  leaf/.style={
    rectangle,
    draw=PodPurple!30,
    fill=white,
    rounded corners=2pt,
    minimum width=3cm,
    minimum height=0.5cm,
    align=center,
    font=\scriptsize,
    text=PodPurpleDark,
    inner sep=3pt,
  },
  root/.style={
    leaf,
    draw=PodPurple!50,
    fill=PodPurple!8,
    font=\scriptsize\bfseries,
    text=PodPurpleDark,
    minimum width=3.5cm,
  },
  typelabel/.style={
    font=\tiny,
    text=PodTextMuted,
    anchor=west,
  },
  arr/.style={->, >=stealth, thin, PodPurple!40},
]
  \node[root] (si) {SubtextItem};

  % Left column
  \node[leaf, below left=0.8cm and 2.5cm of si] (orig) {originalMessage};
  \node[leaf, below=0.5cm of orig] (sender) {sender};
  \node[leaf, below=0.5cm of sender] (ts) {timestamp};
  \node[leaf, below=0.5cm of ts] (sub) {subtext};
  \node[leaf, below=0.5cm of sub] (emo) {emotion};
  \node[leaf, below=0.5cm of emo] (conf) {confidence};

  % Right column
  \node[leaf, below right=0.8cm and 2.5cm of si] (cat) {category};
  \node[leaf, below=0.5cm of cat] (hl) {isHighlight};
  \node[leaf, below=0.5cm of hl] (ctx) {exchangeContext};
  \node[leaf, below=0.5cm of ctx] (wid) {windowId};
  \node[leaf, below=0.5cm of wid] (surr) {surroundingMessages};

  % Type labels
  \node[typelabel] at (orig.east) {\hspace{4pt}\texttt{string}};
  \node[typelabel] at (sender.east) {\hspace{4pt}\texttt{string}};
  \node[typelabel] at (ts.east) {\hspace{4pt}\texttt{number}};
  \node[typelabel] at (sub.east) {\hspace{4pt}\texttt{string}};
  \node[typelabel] at (emo.east) {\hspace{4pt}\texttt{string}};
  \node[typelabel] at (conf.east) {\hspace{4pt}\texttt{0--100}};
  \node[typelabel] at (cat.east) {\hspace{4pt}\texttt{SubtextCategory}};
  \node[typelabel] at (hl.east) {\hspace{4pt}\texttt{boolean}};
  \node[typelabel] at (ctx.east) {\hspace{4pt}\texttt{string}};
  \node[typelabel] at (wid.east) {\hspace{4pt}\texttt{number}};
  \node[typelabel] at (surr.east) {\hspace{4pt}\texttt{Array}};

  % Arrows
  \draw[arr] (si) -- (orig);
  \draw[arr] (si) -- (sender);
  \draw[arr] (si) -- (ts);
  \draw[arr] (si) -- (sub);
  \draw[arr] (si) -- (emo);
  \draw[arr] (si) -- (conf);
  \draw[arr] (si) -- (cat);
  \draw[arr] (si) -- (hl);
  \draw[arr] (si) -- (ctx);
  \draw[arr] (si) -- (wid);
  \draw[arr] (si) -- (surr);

\end{tikzpicture}
\caption{Drzewo pól \tstype{SubtextItem} --- 11~pól, z~czego \texttt{category} jest typem unii (\tstype{SubtextCategory}), a~\texttt{surroundingMessages} zawiera tablicę wiadomości kontekstowych.}
\label{fig:subtext-item-tree}
\end{figure}


% ────────────────────────────────────────────────────────────
\subsection{SubtextSummary}
\label{subsec:subtext-summary}

Zagregowane podsumowanie wyników dekodera podtekstów.

\begin{lstlisting}[style=podcode, caption={Interfejs SubtextSummary}]
interface SubtextSummary {
  hiddenEmotionBalance: Record<string, number>;
  mostDeceptivePerson: string;
  deceptionScore: Record<string, number>;
  topCategories: Array<{
    category: SubtextCategory;
    count: number;
  }>;
  biggestReveal: SubtextItem;
}
\end{lstlisting}

\begin{table}[H]
\centering
\caption{Pola interfejsu \tstype{SubtextSummary}}
\label{tab:subtext-summary}
\begin{tabularx}{\textwidth}{l l X}
\toprule
\textbf{Pole} & \textbf{Typ} & \textbf{Opis} \\
\midrule
\texttt{hiddenEmotionBalance} & \tstype{Record<string, number>} & Bilans ukrytych emocji per osoba --- stosunek emocji negatywnych do pozytywnych \\
\texttt{mostDeceptivePerson} & \tstype{string} & Osoba z~największą liczbą podtekstów niezgodnych z~dosłownym przekazem \\
\texttt{deceptionScore} & \tstype{Record<string, number>} & Wynik ,,zwodniczości'' per osoba (0--100, wyższy = więcej ukrytych intencji) \\
\texttt{topCategories} & \tstype{Array} & Ranking najczęstszych kategorii podtekstów z~liczbą wystąpień \\
\texttt{biggestReveal} & \tstype{SubtextItem} & Wiadomość z~największą rozbieżnością między dosłownym a~ukrytym przekazem \\
\bottomrule
\end{tabularx}
\end{table}


% ────────────────────────────────────────────────────────────
\subsection{SubtextResult}
\label{subsec:subtext-result}

Kontener wyników --- główny interfejs zwracany przez dekoder.

\begin{lstlisting}[style=podcode, caption={Interfejs SubtextResult}]
interface SubtextResult {
  items: SubtextItem[];
  summary: SubtextSummary;
  disclaimer: string;
  analyzedAt: number;
}
\end{lstlisting}

\begin{table}[H]
\centering
\caption{Pola interfejsu \tstype{SubtextResult}}
\label{tab:subtext-result}
\begin{tabularx}{\textwidth}{l l l X}
\toprule
\textbf{Pole} & \textbf{Typ} & \textbf{Wym.} & \textbf{Opis} \\
\midrule
\texttt{items} & \tstype{SubtextItem[]} & tak & Tablica zdekodowanych podtekstów (typowo 20--60 elementów) \\
\texttt{summary} & \tstype{SubtextSummary} & tak & Zagregowane statystyki i~podsumowanie \\
\texttt{disclaimer} & \tstype{string} & tak & Klauzula prawna: analiza ma charakter rozrywkowy, nie diagnostyczny \\
\texttt{analyzedAt} & \tstype{number} & tak & Unix timestamp ms zakończenia analizy \\
\bottomrule
\end{tabularx}
\end{table}


% ────────────────────────────────────────────────────────────
\subsection{Typy pomocnicze: SimplifiedMsg i~ExchangeWindow}
\label{subsec:exchange-window}

Wewnętrzne typy pipeline'u dekodera --- ekstrakcja okien kontekstowych z~rozmowy.

\begin{lstlisting}[style=podcode, caption={Interfejsy SimplifiedMsg i~ExchangeWindow}]
interface SimplifiedMsg {
  sender: string;
  content: string;
  timestamp: number;
  index: number;
}

interface ExchangeWindow {
  windowId: number;
  messages: SimplifiedMsg[];
  targetIndices: number[];
  context: string;
}
\end{lstlisting}

\begin{table}[H]
\centering
\caption{Pola interfejsu \tstype{SimplifiedMsg}}
\begin{tabularx}{\textwidth}{l l X}
\toprule
\textbf{Pole} & \textbf{Typ} & \textbf{Opis} \\
\midrule
\texttt{sender} & \tstype{string} & Nazwa nadawcy \\
\texttt{content} & \tstype{string} & Treść wiadomości (uproszczona, bez mediów) \\
\texttt{timestamp} & \tstype{number} & Unix timestamp ms \\
\texttt{index} & \tstype{number} & Indeks wiadomości w~oryginalnej tablicy \\
\bottomrule
\end{tabularx}
\end{table}

\begin{table}[H]
\centering
\caption{Pola interfejsu \tstype{ExchangeWindow}}
\begin{tabularx}{\textwidth}{l l X}
\toprule
\textbf{Pole} & \textbf{Typ} & \textbf{Opis} \\
\midrule
\texttt{windowId} & \tstype{number} & Unikalny identyfikator okna kontekstowego \\
\texttt{messages} & \tstype{SimplifiedMsg[]} & Wiadomości w~oknie (typowo 5--15 wiadomości) \\
\texttt{targetIndices} & \tstype{number[]} & Indeksy wiadomości wybranych do dekodowania \\
\texttt{context} & \tstype{string} & Krótki opis kontekstu okna (generowany automatycznie) \\
\bottomrule
\end{tabularx}
\end{table}


% ============================================================
\section{Typy Procesu Sądowego (Chat Court)}
\label{sec:court-types}
% ============================================================

Typy trybu ,,Twój Chat w~Sądzie'' --- satyrycznego procesu sądowego generowanego przez AI na podstawie wzorców komunikacyjnych. Zdefiniowane w~typach analizy jakościowej.


% ────────────────────────────────────────────────────────────
\subsection{CourtCharge}
\label{subsec:court-charge}

Pojedynczy zarzut w~procesie sądowym.

\begin{lstlisting}[style=podcode, caption={Interfejs CourtCharge}]
interface CourtCharge {
  id: string;
  charge: string;
  article: string;
  severity: 'wykroczenie' | 'wystepek' | 'zbrodnia';
  evidence: string[];
  defendant: string;
}
\end{lstlisting}

\begin{table}[H]
\centering
\caption{Pola interfejsu \tstype{CourtCharge}}
\label{tab:court-charge}
\begin{tabularx}{\textwidth}{l l l X}
\toprule
\textbf{Pole} & \textbf{Typ} & \textbf{Wym.} & \textbf{Opis} \\
\midrule
\texttt{id} & \tstype{string} & tak & Unikalny identyfikator zarzutu (np.~\texttt{"Z-001"}) \\
\texttt{charge} & \tstype{string} & tak & Treść zarzutu w~języku prawniczym (satyrycznym) \\
\texttt{article} & \tstype{string} & tak & Fikcyjny artykuł ,,Kodeksu Komunikacji'' \\
\texttt{severity} & \tstype{union} & tak & Powaga: \texttt{'wykroczenie'} $|$ \texttt{'występek'} $|$ \texttt{'zbrodnia'} \\
\texttt{evidence} & \tstype{string[]} & tak & Dowody --- parafrazy wzorców komunikacyjnych (bez cytatów) \\
\texttt{defendant} & \tstype{string} & tak & Imię oskarżonego uczestnika \\
\bottomrule
\end{tabularx}
\end{table}


% ────────────────────────────────────────────────────────────
\subsection{PersonVerdict}
\label{subsec:person-verdict}

Wyrok indywidualny dla jednego uczestnika.

\begin{lstlisting}[style=podcode, caption={Interfejs PersonVerdict}]
interface PersonVerdict {
  name: string;
  verdict: 'winny' | 'niewinny' | 'warunkowo';
  mainCharge: string;
  sentence: string;
  mugshotLabel: string;
  funFact: string;
}
\end{lstlisting}

\begin{table}[H]
\centering
\caption{Pola interfejsu \tstype{PersonVerdict}}
\label{tab:person-verdict}
\begin{tabularx}{\textwidth}{l l l X}
\toprule
\textbf{Pole} & \textbf{Typ} & \textbf{Wym.} & \textbf{Opis} \\
\midrule
\texttt{name} & \tstype{string} & tak & Imię osądzonego uczestnika \\
\texttt{verdict} & \tstype{union} & tak & Wyrok: \texttt{'winny'} $|$ \texttt{'niewinny'} $|$ \texttt{'warunkowo'} \\
\texttt{mainCharge} & \tstype{string} & tak & Główny zarzut, za który wydano wyrok \\
\texttt{sentence} & \tstype{string} & tak & Kara (satyryczna, np.~,,500h przymusowego słuchania partnera'') \\
\texttt{mugshotLabel} & \tstype{string} & tak & Etykieta do generowanego ,,mugshota'' (karty oskarżonego) \\
\texttt{funFact} & \tstype{string} & tak & Zabawny fakt o~komunikacji oskarżonego \\
\bottomrule
\end{tabularx}
\end{table}


% ────────────────────────────────────────────────────────────
\subsection{CourtResult}
\label{subsec:court-result}

Główny kontener wyników procesu sądowego.

\begin{lstlisting}[style=podcode, caption={Interfejs CourtResult}]
interface CourtResult {
  caseNumber: string;
  courtName: string;
  charges: CourtCharge[];
  prosecution: string;
  defense: string;
  verdict: CourtVerdict;
  perPerson: Record<string, PersonVerdict>;
}
\end{lstlisting}

\begin{table}[H]
\centering
\caption{Pola interfejsu \tstype{CourtResult}}
\label{tab:court-result}
\begin{tabularx}{\textwidth}{l l l X}
\toprule
\textbf{Pole} & \textbf{Typ} & \textbf{Wym.} & \textbf{Opis} \\
\midrule
\texttt{caseNumber} & \tstype{string} & tak & Sygnatura akt (generowana, np.~\texttt{"PT/2026/001"}) \\
\texttt{courtName} & \tstype{string} & tak & Nazwa sądu (satyryczna, np.~,,Sąd Najwyższy ds.~Komunikacji'') \\
\texttt{charges} & \tstype{CourtCharge[]} & tak & Lista zarzutów (typowo 3--8) \\
\texttt{prosecution} & \tstype{string} & tak & Mowa oskarżyciela --- humorystyczne podsumowanie zarzutów \\
\texttt{defense} & \tstype{string} & tak & Mowa obrońcy --- próba usprawiedliwienia wzorców \\
\texttt{verdict} & \tstype{CourtVerdict} & tak & Wyrok ogólny sądu z~uzasadnieniem \\
\texttt{perPerson} & \tstype{Record<string, PersonVerdict>} & tak & Wyroki indywidualne, kluczowane po imieniu uczestnika \\
\bottomrule
\end{tabularx}
\end{table}


% ============================================================
\section{Typy Profilu Randkowego}
\label{sec:dating-profile-types}
% ============================================================

Typy generatora ,,Szczery Profil Randkowy'' --- funkcjonalno\'{s}ci tworz\k{a}cej profil Tinder/Hinge na podstawie rzeczywistych danych komunikacyjnych. Zdefiniowane w~pliku:

\filepath{src/lib/analysis/dating-profile-prompts.ts} (253 LOC).

\begin{infobox}[title={Podej\'{s}cie do generowania profilu}]
Profil randkowy jest generowany wy\l{}\k{a}cznie po stronie serwera przez Gemini API. Dane wej\'{s}ciowe obejmuj\k{a} pr\'{o}bki wiadomo\'{s}ci, metryki ilo\'{s}ciowe oraz opcjonalnie wyniki analizy psychologicznej (Pass~1 i~3). Ka\.{z}dy profil zawiera bio napisane \textbf{w~stylu pisania} danej osoby --- jej s\l{}ownictwem, interpunkcj\k{a} i~d\l{}ugo\'{s}ci\k{a} wiadomo\'{s}ci.
\end{infobox}


% ────────────────────────────────────────────────────────────
\subsection{DatingProfileStat}
\label{subsec:dating-profile-stat}

Pojedyncza statystyka w~profilu randkowym --- \l{}\k{a}czy metryki z~roast-style komentarzem.

\begin{lstlisting}[style=podcode, caption={Interfejs DatingProfileStat}]
interface DatingProfileStat {
  label: string;
  value: string;
  emoji: string;
}
\end{lstlisting}


% ────────────────────────────────────────────────────────────
\subsection{DatingProfilePrompt}
\label{subsec:dating-profile-prompt}

Prompt w~stylu Hinge --- pytanie i~brutalnie szczera odpowied\'{z}.

\begin{lstlisting}[style=podcode, caption={Interfejs DatingProfilePrompt}]
interface DatingProfilePrompt {
  prompt: string;
  answer: string;
}
\end{lstlisting}


% ────────────────────────────────────────────────────────────
\subsection{PersonDatingProfile}
\label{subsec:person-dating-profile}

Pe\l{}ny profil randkowy jednego uczestnika rozmowy.

\begin{lstlisting}[style=podcode, caption={Interfejs PersonDatingProfile}]
interface PersonDatingProfile {
  name: string;
  age_vibe: string;
  bio: string;
  stats: DatingProfileStat[];
  prompts: DatingProfilePrompt[];
  red_flags: string[];
  green_flags: string[];
  match_prediction: string;
  dealbreaker: string;
  overall_rating: string;
}
\end{lstlisting}

\begin{table}[H]
\centering
\caption{Pola interfejsu \tstype{PersonDatingProfile}}
\label{tab:person-dating-profile}
\begin{tabularx}{\textwidth}{l l l X}
\toprule
\textbf{Pole} & \textbf{Typ} & \textbf{Wym.} & \textbf{Opis} \\
\midrule
\texttt{name} & \tstype{string} & tak & Imi\k{e} uczestnika \\
\texttt{age\_vibe} & \tstype{string} & tak & ,,Energia wiekowa'' --- sarkastyczna diagnoza, nie prawdziwy wiek \\
\texttt{bio} & \tstype{string} & tak & 2--3 zdania \textbf{w~stylu pisania} tej osoby (ich s\l{}ownictwo, interpunkcja, emoji) \\
\texttt{stats} & \tstype{DatingProfileStat[]} & tak & 5--6 statystyk z~konkretnymi liczbami (np.~czas odpowiedzi, inicjatywa) \\
\texttt{prompts} & \tstype{DatingProfilePrompt[]} & tak & 3 prompty w~stylu Hinge z~brutalnymi odpowiedziami \\
\texttt{red\_flags} & \tstype{string[]} & tak & Czerwone flagi oparte na danych (ghosting, double texting itp.) \\
\texttt{green\_flags} & \tstype{string[]} & tak & Zielone flagi z~konkretnym wzorcem i~liczb\k{a} \\
\texttt{match\_prediction} & \tstype{string} & tak & Prognoza dopasowania na podstawie stylu komunikacji \\
\texttt{dealbreaker} & \tstype{string} & tak & Jeden konkretny pattern z~danych z~liczb\k{a} \\
\texttt{overall\_rating} & \tstype{string} & tak & Ocena gwiazdkowa (1--5) + kr\'{o}tki komentarz \\
\bottomrule
\end{tabularx}
\end{table}


% ────────────────────────────────────────────────────────────
\subsection{DatingProfileResult}
\label{subsec:dating-profile-result}

G\l{}\'{o}wny kontener wynik\'{o}w --- mapuje imiona uczestnik\'{o}w na ich profile randkowe.

\begin{lstlisting}[style=podcode, caption={Interfejs DatingProfileResult}]
interface DatingProfileResult {
  profiles: Record<string, PersonDatingProfile>;
}
\end{lstlisting}

\begin{table}[H]
\centering
\caption{Pola interfejsu \tstype{DatingProfileResult}}
\label{tab:dating-profile-result}
\begin{tabularx}{\textwidth}{l l l X}
\toprule
\textbf{Pole} & \textbf{Typ} & \textbf{Wym.} & \textbf{Opis} \\
\midrule
\texttt{profiles} & \tstype{Record<string, PersonDatingProfile>} & tak & Mapa profili, kluczowana po imieniu uczestnika \\
\bottomrule
\end{tabularx}
\end{table}


% ============================================================
\section{Typy Quizu Samo\'{s}wiadomo\'{s}ci (Delusion Quiz)}
\label{sec:delusion-types}
% ============================================================

Typy modu\l{}u ,,Stawiam Zak\l{}ad'' --- quizu samo\'{s}wiadomo\'{s}ci, kt\'{o}ry por\'{o}wnuje subiektywne odpowiedzi u\.{z}ytkownika z~rzeczywistymi danymi ilo\'{s}ciowymi. Ca\l{}o\'{s}\'{c} dzia\l{}a \textbf{po stronie klienta} --- bez AI. Zdefiniowane w~pliku:

\filepath{src/lib/analysis/delusion-quiz.ts} (569 LOC).

\begin{infobox}[title={Wzorzec callback\'{o}w w~DelusionQuestion}]
Interfejs \tstype{DelusionQuestion} u\.{z}ywa funkcji zwrotnych (\tsfunc{getCorrectAnswer}, \tsfunc{getRevealText}) zamiast statycznych p\'{o}l danych. Pozwala to ka\.{z}demu pytaniu dynamicznie oblicza\'{c} poprawn\k{a} odpowied\'{z} na podstawie bie\.{z}\k{a}cych metryk ilo\'{s}ciowych. Jest to wyj\k{a}tek od typowego wzorca ,,p\l{}askich danych'' --- uzasadniony konieczno\'{s}ci\k{a} por\'{o}wnywania odpowiedzi u\.{z}ytkownika z~rzeczywistymi warto\'{s}ciami.
\end{infobox}


% ────────────────────────────────────────────────────────────
\subsection{DelusionQuestion}
\label{subsec:delusion-question}

Definicja pojedynczego pytania quizu --- zawiera tre\'{s}\'{c}, opcje odpowiedzi oraz callbacki obliczeniowe.

\begin{lstlisting}[style=podcode, caption={Interfejs DelusionQuestion}]
interface DelusionQuestion {
  id: string;
  question: string;
  options: Array<{ label: string; value: string }>;
  getCorrectAnswer: (
    quantitative: QuantitativeAnalysis,
    conversation: ParsedConversation,
  ) => string;
  getRevealText: (
    correct: string,
    userAnswer: string,
    quantitative: QuantitativeAnalysis,
    conversation: ParsedConversation,
  ) => string;
}
\end{lstlisting}

\begin{table}[H]
\centering
\caption{Pola interfejsu \tstype{DelusionQuestion}}
\label{tab:delusion-question}
\begin{tabularx}{\textwidth}{l l l X}
\toprule
\textbf{Pole} & \textbf{Typ} & \textbf{Wym.} & \textbf{Opis} \\
\midrule
\texttt{id} & \tstype{string} & tak & Unikalny identyfikator pytania (np.~\texttt{"q1\_more\_messages"}) \\
\texttt{question} & \tstype{string} & tak & Tre\'{s}\'{c} pytania wy\'{s}wietlanego u\.{z}ytkownikowi \\
\texttt{options} & \tstype{Array<\{label, value\}>} & tak & Opcje odpowiedzi --- tablica lub pusta (dynamicznie wype\l{}niana imionami) \\
\texttt{getCorrectAnswer} & \tstype{(quant, conv) => string} & tak & Callback obliczaj\k{a}cy poprawn\k{a} odpowied\'{z} z~danych ilo\'{s}ciowych \\
\texttt{getRevealText} & \tstype{(correct, user, quant, conv) => string} & tak & Callback generuj\k{a}cy tekst ods\l{}ony z~konkretnymi liczbami \\
\bottomrule
\end{tabularx}
\end{table}


% ────────────────────────────────────────────────────────────
\subsection{DelusionAnswer}
\label{subsec:delusion-answer}

Wynik pojedynczego pytania --- por\'{o}wnanie odpowiedzi u\.{z}ytkownika z~rzeczywisto\'{s}ci\k{a}.

\begin{lstlisting}[style=podcode, caption={Interfejs DelusionAnswer}]
interface DelusionAnswer {
  questionId: string;
  userAnswer: string;
  correctAnswer: string;
  isCorrect: boolean;
  revealText: string;
}
\end{lstlisting}

\begin{table}[H]
\centering
\caption{Pola interfejsu \tstype{DelusionAnswer}}
\label{tab:delusion-answer}
\begin{tabularx}{\textwidth}{l l l X}
\toprule
\textbf{Pole} & \textbf{Typ} & \textbf{Wym.} & \textbf{Opis} \\
\midrule
\texttt{questionId} & \tstype{string} & tak & Referencja do \texttt{DelusionQuestion.id} \\
\texttt{userAnswer} & \tstype{string} & tak & Odpowied\'{z} wybrana przez u\.{z}ytkownika \\
\texttt{correctAnswer} & \tstype{string} & tak & Poprawna odpowied\'{z} obliczona z~danych ilo\'{s}ciowych \\
\texttt{isCorrect} & \tstype{boolean} & tak & Czy odpowied\'{z} u\.{z}ytkownika by\l{}a poprawna \\
\texttt{revealText} & \tstype{string} & tak & Tekst ods\l{}ony z~konkretnymi liczbami i~danymi \\
\bottomrule
\end{tabularx}
\end{table}


% ────────────────────────────────────────────────────────────
\subsection{DelusionQuizResult}
\label{subsec:delusion-quiz-result}

Kontener wyniku ca\l{}ego quizu --- agreguje odpowiedzi i~oblicza Delusion Index.

\begin{lstlisting}[style=podcode, caption={Interfejs DelusionQuizResult}]
interface DelusionQuizResult {
  answers: DelusionAnswer[];
  score: number;
  delusionIndex: number;
  label: string;
}
\end{lstlisting}

\begin{table}[H]
\centering
\caption{Pola interfejsu \tstype{DelusionQuizResult}}
\label{tab:delusion-quiz-result}
\begin{tabularx}{\textwidth}{l l l X}
\toprule
\textbf{Pole} & \textbf{Typ} & \textbf{Wym.} & \textbf{Opis} \\
\midrule
\texttt{answers} & \tstype{DelusionAnswer[]} & tak & Tablica wynik\'{o}w 15~pyta\'{n} \\
\texttt{score} & \tstype{number} & tak & Liczba poprawnych odpowiedzi (0--15) \\
\texttt{delusionIndex} & \tstype{number} & tak & Indeks z\l{}udze\'{n} (0--100), wa\.{z}ony --- pytania o~sobie licz\k{a} si\k{e}~2$\times$ \\
\texttt{label} & \tstype{string} & tak & Etykieta: \texttt{BAZOWANY} $|$ \texttt{REALISTA} $|$ \texttt{LEKKO ODJECHANY} $|$ \texttt{TOTAL DELULU} $|$ \texttt{POZA RZECZYWISTO\'{S}CI\k{A}} \\
\bottomrule
\end{tabularx}
\end{table}


% ============================================================
\section{Typy Symulatora Odpowiedzi}
\label{sec:simulator-types}
% ============================================================

Typy modu\l{}u ,,Reply Simulator'' --- funkcjonalno\'{s}ci symuluj\k{a}cej odpowied\'{z} konkretnej osoby na podstawie jej rzeczywistych wzorc\'{o}w komunikacyjnych. Generowanie po stronie serwera (Gemini API). Zdefiniowane w~pliku:

\filepath{src/lib/analysis/simulator-prompts.ts} (359 LOC).


% ────────────────────────────────────────────────────────────
\subsection{SimulationParams}
\label{subsec:simulation-params}

Parametry wej\'{s}ciowe symulacji --- 14~p\'{o}l opisuj\k{a}cych styl i~kontekst docelowej osoby.

\begin{lstlisting}[style=podcode, caption={Interfejs SimulationParams}]
interface SimulationParams {
  userMessage: string;
  targetPerson: string;
  participants: string[];
  quantitativeContext: string;
  topWords: Array<{ word: string; count: number }>;
  topPhrases: Array<{ phrase: string; count: number }>;
  avgMessageLengthWords: number;
  avgMessageLengthChars: number;
  emojiFrequency: number;
  topEmojis: Array<{ emoji: string; count: number }>;
  medianResponseTimeMs: number;
  exampleMessages: string[];
  previousExchanges: Array<{ role: 'user' | 'target'; message: string }>;
  personalityProfile?: PersonProfile;
  toneAnalysis?: Pass1Result;
  dynamicsAnalysis?: Pass2Result;
}
\end{lstlisting}

\begin{table}[H]
\centering
\caption{Pola interfejsu \tstype{SimulationParams} (14~p\'{o}l + 3~opcjonalne)}
\label{tab:simulation-params}
\begin{tabularx}{\textwidth}{l l l X}
\toprule
\textbf{Pole} & \textbf{Typ} & \textbf{Wym.} & \textbf{Opis} \\
\midrule
\texttt{userMessage} & \tstype{string} & tak & Wiadomo\'{s}\'{c} wpisana przez u\.{z}ytkownika (maks.~200 znak\'{o}w) \\
\texttt{targetPerson} & \tstype{string} & tak & Imi\k{e} osoby, kt\'{o}rej odpowied\'{z} jest symulowana \\
\texttt{participants} & \tstype{string[]} & tak & Lista wszystkich uczestnik\'{o}w rozmowy \\
\texttt{quantitativeContext} & \tstype{string} & tak & Zserializowany kontekst metryk ilo\'{s}ciowych \\
\texttt{topWords} & \tstype{Array<\{word, count\}>} & tak & Najcz\k{e}\'{s}ciej u\.{z}ywane s\l{}owa przez docelow\k{a} osob\k{e} \\
\texttt{topPhrases} & \tstype{Array<\{phrase, count\}>} & tak & Najcz\k{e}stsze frazy docelowej osoby \\
\texttt{avgMessageLengthWords} & \tstype{number} & tak & \'{S}rednia d\l{}ugo\'{s}\'{c} wiadomo\'{s}ci w~s\l{}owach \\
\texttt{avgMessageLengthChars} & \tstype{number} & tak & \'{S}rednia d\l{}ugo\'{s}\'{c} wiadomo\'{s}ci w~znakach \\
\texttt{emojiFrequency} & \tstype{number} & tak & Cz\k{e}stotliwo\'{s}\'{c} u\.{z}ycia emoji (0--1+) \\
\texttt{topEmojis} & \tstype{Array<\{emoji, count\}>} & tak & Najcz\k{e}\'{s}ciej u\.{z}ywane emoji z~licznikami \\
\texttt{medianResponseTimeMs} & \tstype{number} & tak & Mediana czasu odpowiedzi w~milisekundach \\
\texttt{exampleMessages} & \tstype{string[]} & tak & 20--30 rzeczywistych wiadomo\'{s}ci od docelowej osoby \\
\texttt{previousExchanges} & \tstype{Array<\{role, message\}>} & tak & Poprzednie wymiany w~bie\.{z}\k{a}cej sesji symulacji \\
\texttt{personalityProfile} & \tstype{PersonProfile} & nie & Profil osobowo\'{s}ci z~Pass~3 (je\'{s}li dost\k{e}pny) \\
\texttt{toneAnalysis} & \tstype{Pass1Result} & nie & Analiza tonu i~dynamiki z~Pass~1 \\
\texttt{dynamicsAnalysis} & \tstype{Pass2Result} & nie & Analiza konflikt\'{o}w, pracy emocjonalnej, flag z~Pass~2 \\
\bottomrule
\end{tabularx}
\end{table}


% ────────────────────────────────────────────────────────────
\subsection{SimulationResponse}
\label{subsec:simulation-response}

Wynik symulacji --- wygenerowana odpowied\'{z} z~ocen\k{a} pewno\'{s}ci.

\begin{lstlisting}[style=podcode, caption={Interfejs SimulationResponse}]
interface SimulationResponse {
  reply: string;
  confidence: number;
  styleNotes: string;
}
\end{lstlisting}

\begin{table}[H]
\centering
\caption{Pola interfejsu \tstype{SimulationResponse}}
\label{tab:simulation-response}
\begin{tabularx}{\textwidth}{l l l X}
\toprule
\textbf{Pole} & \textbf{Typ} & \textbf{Wym.} & \textbf{Opis} \\
\midrule
\texttt{reply} & \tstype{string} & tak & Wygenerowana wiadomo\'{s}\'{c} w~stylu docelowej osoby \\
\texttt{confidence} & \tstype{number} & tak & Pewno\'{s}\'{c} symulacji (0--100) --- jak prawdopodobne, \.{z}e osoba odpowiedzia\l{}aby podobnie \\
\texttt{styleNotes} & \tstype{string} & tak & Kr\'{o}tki opis odwzorowanych element\'{o}w stylu (osobowo\'{s}\'{c}, ton, nawyki) \\
\bottomrule
\end{tabularx}
\end{table}

% ============================================================
\section{Typy modu\l{}\'{o}w ilościowych (Fazy~24--28)}
\label{sec:quant-module-types}
% ============================================================

Poni\.{z}ej zestawiono nowe interfejsy TypeScript wprowadzone w~Fazach~24--28 dla modu\l{}\'{o}w analizy ilościowej. Wszystkie pola opcjonalne w~\tstype{QuantitativeAnalysis} u\.{z}ywaj\k{a} tych typ\'{o}w.

\subsection{Pursuit-Withdrawal (Faza~24)}
\label{subsec:types-pursuit-withdrawal}

\begin{lstlisting}[style=podcode, caption={Interfejsy PursuitWithdrawalCycle i PursuitWithdrawalResult}]
interface PursuitWithdrawalCycle {
  pursuer: string;             // imie osoby inicjujacej poscius
  withdrawer: string;          // imie osoby milczacej
  pursuerMessages: number;     // liczba wiadomosci w serii poscius
  withdrawerMessages: number;  // liczba wiadomosci withdrawer w tym oknie
  gapMs: number;               // czas milczenia w milisekundach
}

interface PursuitWithdrawalResult {
  cycles: PursuitWithdrawalCycle[];
  dominantPursuer: string | null;  // osoba czesciej inicjujaca
  severity: 'none' | 'mild' | 'moderate' | 'significant';
}
\end{lstlisting}

\subsection{Language Style Matching (Faza~27)}
\label{subsec:types-lsm}

\begin{lstlisting}[style=podcode, caption={Interfejs LSMResult}]
interface LSMResult {
  score: number;                        // ogolny wynik LSM (0-1)
  perCategory: Record<string, number>;  // wynik per kategoria (9 kat.)
  adaptationAsymmetry: {
    adapter: string;                    // osoba bardziej sie dostosowujaca
    delta: number;                      // roznica w stopniu adaptacji
  } | null;
  interpretation: 'very_high' | 'high' | 'moderate' | 'low';
}
\end{lstlisting}

\subsection{Pronoun Analysis (Faza~27)}
\label{subsec:types-pronouns}

\begin{lstlisting}[style=podcode, caption={Interfejsy PronounAnalysisResult i PersonPronounStats}]
interface PersonPronounStats {
  iRate: number;      // % uzycia zaimkow "ja" (deklinacja PL)
  weRate: number;     // % uzycia zaimkow "my"
  youRate: number;    // % uzycia zaimkow "ty"
  orientation: 'self-focused' | 'other-focused' | 'collective' | 'balanced';
}

interface PronounAnalysisResult {
  perPerson: Record<string, PersonPronounStats>;
}
\end{lstlisting}

\subsection{Chronotype Compatibility (Faza~28)}
\label{subsec:types-chronotype}

\begin{lstlisting}[style=podcode, caption={Interfejsy PersonChronotype i ChronotypeCompatibility}]
interface PersonChronotype {
  peakHour: number;                  // godzina szczytowej aktywnosci (0-23)
  midpoint: number;                  // angular mean (atan2-based)
  category: ChronotypeCategory;      // 'early_bird'|'intermediate'|'night_owl'
  label: string;                     // czytelna etykieta
  emoji: string;                     // 🌅 / 😊 / 🦉
  hourlyDistribution: number[];      // 24-elementowa tablica liczby wiad.
}

interface ChronotypeCompatibility {
  persons: Record<string, PersonChronotype>;
  deltaHours: number;                // roznica circular miedzy osobami
  matchScore: number;                // % zgodnosci (10-95)
  interpretation: string;            // opis slowny
  isCompatible: boolean;             // delta <= 2h
}
\end{lstlisting}

\subsection{Emotional Granularity (Faza~28)}
\label{subsec:types-eg}

\begin{lstlisting}[style=podcode, caption={Interfejsy EmotionalGranularityResult i PersonEmotionalGranularity}]
interface PersonEmotionalGranularity {
  distinctCategories: number;          // liczba rozroznionych kategorii (0-12)
  emotionalWordCount: number;          // liczba slow emocjonalnych
  categoryCounts: Record<string, number>; // liczba slow per kategoria
  granularityScore: number;            // wynik 0-100
  dominantCategory: string | null;     // najczesciej uzywana kategoria
}

interface EmotionalGranularityResult {
  perPerson: Record<string, PersonEmotionalGranularity>;
  higherGranularity: string | null;    // osoba z wyzszym wynikiem
}
\end{lstlisting}

\subsection{Bid-Response (Faza~28)}
\label{subsec:types-bid-response}

\begin{lstlisting}[style=podcode, caption={Interfejsy BidResponseResult i PersonBidResponse}]
interface PersonBidResponse {
  bidsMade: number;             // liczba sygnalów wysłanych przez te osobe
  turnedToward: number;         // odpowiedzi "toward" (zwrot ku partnerowi)
  turnedAway: number;           // odpowiedzi "away" (odwrocenie)
  bidsReceived: number;         // liczba bidsow otrzymanych
  bidsRespondedTo: number;      // liczba bidsow na ktore odpowiedziano
  bidSuccessRate: number;       // % bidsow ktore otrzymaly odpowiedz
  responseRate: number;         // % odpowiedzi "toward" na otrzymane bidy
}

interface BidResponseResult {
  perPerson: Record<string, PersonBidResponse>;
  overallResponseRate: number;  // srednia dla calej rozmowy
  gottmanBenchmark: 86;         // sta la: benchmark Gottmana (%)
  interpretation: 'high' | 'moderate' | 'low';
}
\end{lstlisting}

\subsection{Shift-Support CNI (Faza~28)}
\label{subsec:types-cni}

\begin{lstlisting}[style=podcode, caption={Interfejsy ShiftSupportResult i PersonShiftSupport}]
interface PersonShiftSupport {
  shiftCount: number;     // liczba shift-responses
  supportCount: number;   // liczba support-responses
  shiftRatio: number;     // shiftCount / (shift + support)
  cni: number;            // Conversational Narcissism Index (0-100)
}

interface ShiftSupportResult {
  perPerson: Record<string, PersonShiftSupport>;
  higherCNI: string | null;  // osoba z wyzszym CNI
  cniGap: number;            // roznica CNI miedzy osobami
}
\end{lstlisting}

\subsection{Zaktualizowane typy (Faza~26--27)}
\label{subsec:types-updated}

\textbf{DamageReportResult} --- zmiana pola \texttt{therapyNeeded}:

\begin{lstlisting}[style=podcode, caption={Zmiana therapyNeeded -> therapyBenefit (Faza 26)}]
// POPRZEDNI (Faza <= 25):
therapyNeeded: boolean;

// AKTUALNY (Faza 26+):
therapyBenefit: 'WYSOKA' | 'UMIARKOWANA' | 'NISKA';
\end{lstlisting}

\textbf{ClinicalObservation} --- zmiana etykiet nasilenia (Faza~27):

\begin{lstlisting}[style=podcode, caption={Zmiana severity labels -> frequency labels (Faza 27)}]
// POPRZEDNI (Faza <= 26):
severity: 'none' | 'mild' | 'moderate' | 'significant';

// AKTUALNY (Faza 27+):
frequency: 'not_observed' | 'occasional' | 'recurring' | 'pervasive';
// (backward compat: stare wartosci sa mapowane automatycznie)
\end{lstlisting}

% ============================================================
\section{Typy walidacji Zod}
\label{sec:zod-types}
% ============================================================

Walidacja runtime'owa za pomocą biblioteki Zod zapewnia typobezpieczeństwo na granicy API --- waliduje dane wejściowe od klienta oraz wyjścia z~modelu AI. Schematy zdefiniowane są w~pliku:

\filepath{src/lib/analysis/schemas.ts}

\begin{infobox}[title={Rola schematów Zod}]
Schematy Zod zamykają wektor prompt injection: nawet jeśli model AI zwróci nieoczekiwane pola, schemat Zod odetnie je zanim trafią do frontendu. Każdy z~9~endpointów API posiada dedykowany schemat wejściowy.
\end{infobox}

Kluczowe eksportowane typy (inferowane z~schematów \tsfunc{z.infer<>}):

\begin{table}[H]
\centering
\caption{Typy Zod eksportowane ze \filepath{schemas.ts}}
\label{tab:zod-types}
\begin{tabularx}{\textwidth}{l l X}
\toprule
\textbf{Typ / Schemat} & \textbf{Endpoint} & \textbf{Walidowane pola} \\
\midrule
\tstype{AnalyzeInput} & \texttt{/api/analyze} & messages (limit), participants, relationshipContext (enum) \\
\tstype{SubtextInput} & \texttt{/api/analyze/subtext} & messages, participants, windowCount (1--20) \\
\tstype{CourtInput} & \texttt{/api/analyze/court} & messages, participants, relationshipContext \\
\tstype{CPSInput} & \texttt{/api/analyze/cps} & messages, participants, participantName \\
\tstype{StandupInput} & \texttt{/api/analyze/standup} & messages, participants, targetPerson \\
\tstype{RoastInput} & \texttt{/api/analyze/roast} & messages, participants \\
\tstype{ImageInput} & \texttt{/api/analyze/image} & prompt, style (enum), aspect (enum) \\
\tstype{DatingProfileInput} & \texttt{/api/analyze/dating-profile} & samples, participants, quantitativeContext, existingAnalysis? (pass1?, pass3?) \\
\tstype{SimulateInput} & \texttt{/api/analyze/simulate} & userMessage (max 200), targetPerson, participants, topWords?, topPhrases?, topEmojis?, previousExchanges?, personalityProfile?, toneAnalysis?, dynamicsAnalysis? \\
\bottomrule
\end{tabularx}
\end{table}

Schemat \tstype{relationshipContext} jest ograniczony do wartości wyliczeniowych: \texttt{romantic}, \texttt{friendship}, \texttt{colleague}, \texttt{professional}, \texttt{family}, \texttt{other} --- co eliminuje dowolne ciągi tekstowe jako wektor ataku.


% ────────────────────────────────────────────────────────────
\subsection{Zaktualizowane podsumowanie zależności}
\label{subsec:updated-dependency-summary}

\begin{table}[H]
\centering
\caption{Zaktualizowana mapa zależności typów --- Faza 19--20}
\label{tab:type-dependency-map-v2}
\begin{tabularx}{\textwidth}{l c X}
\toprule
\textbf{Kontener} & \textbf{Podtypów} & \textbf{Kluczowe składniki} \\
\midrule
\tstype{StoredAnalysis} & 3 & ParsedConversation, QuantitativeAnalysis, QualitativeAnalysis? \\
\tstype{QuantitativeAnalysis} & 12 & PersonMetrics, TimingMetrics, EngagementMetrics, PatternMetrics, HeatmapData, TrendData, ViralScores?, Badge[]?, BestTimeToText?, CatchphraseResult?, NetworkMetrics?, ReciprocityIndex? \\
\tstype{QualitativeAnalysis} & 9 & Pass1--4?, RoastResult?, StandUpRoastResult?, CPSResult?, \textbf{SubtextResult?}, \textbf{CourtResult?} \\
\tstype{PersonProfile} & 10 & BigFiveApproximation, AttachmentIndicators, CommunicationProfile, CommunicationNeeds, EmotionalPatterns, ClinicalObservations, ConflictResolution, EmotionalIntelligence, MBTIResult?, LoveLanguageResult? \\
\tstype{Pass2Result} & 7 & PowerDynamics, EmotionalLabor, ConflictPatterns, IntimacyMarkers, RedFlag[], GreenFlag[], relationship\_phase? \\
\tstype{CPSResult} & 4 & CPSAnswer (63$\times$), CPSPatternResult (10$\times$), disclaimer, participantName \\
\tstype{SubtextResult} & 3 & SubtextItem[], SubtextSummary, disclaimer \\
\tstype{CourtResult} & 4 & CourtCharge[], CourtVerdict, Record<PersonVerdict>, strings \\
\tstype{DatingProfileResult} & 3 & DatingProfileStat, DatingProfilePrompt, PersonDatingProfile \\
\tstype{DelusionQuizResult} & 1 & DelusionAnswer \\
\tstype{SimulationResponse} & 0 & samodzielny (standalone) \\
\bottomrule
\end{tabularx}
\end{table}

Łączna liczba zdefiniowanych interfejsów w~\podtekst: \textbf{$\sim$65}, w~tym:
\begin{itemize}
  \item 15 interfejsów parserowych/ilościowych
  \item 25 interfejsów analizy AI (Pass 1--5)
  \item 6 interfejsów Dekodera Podtekstów (SubtextCategory, SubtextItem, SubtextSummary, SubtextResult, SimplifiedMsg, ExchangeWindow)
  \item 4 interfejsy Procesu S\k{a}dowego (CourtResult, CourtCharge, PersonVerdict, CourtVerdict)
  \item 4 interfejsy Profilu Randkowego (DatingProfileResult, PersonDatingProfile, DatingProfileStat, DatingProfilePrompt)
  \item 3 interfejsy Quizu Samo\'{s}wiadomo\'{s}ci (DelusionQuizResult, DelusionAnswer, DelusionQuestion)
  \item 2 interfejsy Symulatora Odpowiedzi (SimulationParams, SimulationResponse)
  \item 4 interfejsy CPS
  \item 3 interfejsy magazynowania
  \item 9 schemat\'{o}w walidacji Zod (inferowane typy)
\end{itemize}
