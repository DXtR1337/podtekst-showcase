% ============================================================
% Rozdział 5 — Silnik Analizy Ilościowej
% ============================================================

\chapter{Silnik Analizy Ilościowej}
\label{ch:analiza-ilosciowa}

\begin{center}
\Large\itshape\color{PodBlue}
,,Dane nie kłamią --- ale musisz wiedzieć, co z~nich wycisnąć.''
\end{center}

\vspace{8pt}

Silnik analizy ilościowej stanowi fundament całego systemu \podtekst. Zanim jakikolwiek model AI zobaczy choćby fragment rozmowy, silnik ilościowy przetwarza \emph{każdą} wiadomość i~wydobywa z~niej ponad 60~metryk statystycznych. To czysta matematyka: zero wywołań API, zero kosztów, zero halucynacji.

Niniejszy rozdział dokumentuje kompletną implementację silnika, od architektury jednoprzebiegowej po algorytmy wykrywania wzorców aktywności. Każda metryka jest opisana pod kątem: \emph{co mierzy}, \emph{jak jest obliczana} i~\emph{dlaczego ma znaczenie} dla analizy relacji.

\begin{infobox}[title=Pliki źródłowe]
Główna implementacja silnika znajduje się w~następujących plikach:
\begin{itemize}
  \item \filepath{src/lib/analysis/quantitative.ts} --- główna funkcja \tsfunc{computeQuantitativeAnalysis()}
  \item \filepath{src/lib/analysis/constants.ts} --- stałe i~funkcje współdzielone (stopwords, regresja liniowa)
  \item \filepath{src/lib/analysis/viral-scores.ts} --- wyniki wiralne (kompatybilność, zainteresowanie, ghosting)
  \item \filepath{src/lib/analysis/badges.ts} --- system odznak i~osiągnięć
  \item \filepath{src/lib/analysis/catchphrases.ts} --- frazy charakterystyczne i~najlepszy czas na wiadomość
  \item \filepath{src/lib/analysis/network.ts} --- metryki sieci (czaty grupowe)
  \item \filepath{src/lib/parsers/types.ts} --- definicje typów wynikowych
  \item \filepath{src/lib/analysis/quant/} --- submoduły refaktoryzacji:
  \begin{itemize}
    \item \filepath{helpers.ts} --- funkcje pomocnicze (extractEmojis, countWords, tokenizeWords, median, percentile, topN)
    \item \filepath{types.ts} --- interfejs \tstype{PersonAccumulator} i~factory \tsfunc{createPersonAccumulator()}
    \item \filepath{bursts.ts} --- detekcja burstów aktywności (\tsfunc{detectBursts()})
    \item \filepath{trends.ts} --- obliczanie trendów miesięcznych (\tsfunc{computeTrends()})
    \item \filepath{reciprocity.ts} --- indeks wzajemności (\tsfunc{computeReciprocityIndex()})
    \item \filepath{sentiment.ts} --- analiza sentymentu (\tsfunc{computeSentimentScore()})
    \item \filepath{conflicts.ts} --- detekcja konfliktów (\tsfunc{detectConflicts()})
    \item \filepath{intimacy.ts} --- progresja intymności (\tsfunc{computeIntimacyProgression()})
    \item \filepath{index.ts} --- barrel export
  \end{itemize}
\end{itemize}
\end{infobox}

% ============================================================
\section{Przegląd architektury silnika}
\label{sec:architektura-silnika}
\index{silnik ilościowy!architektura}

Centralnym punktem silnika jest funkcja:

\begin{lstlisting}[style=podcode, caption={Sygnatura głównej funkcji silnika ilościowego}]
export function computeQuantitativeAnalysis(
  conversation: ParsedConversation,
): QuantitativeAnalysis
\end{lstlisting}

Przyjmuje ona obiekt \tstype{ParsedConversation} (wynik parsowania --- patrz Rozdział~4) i~zwraca kompletny obiekt \tstype{QuantitativeAnalysis} zawierający wszystkie metryki.

\subsection{Projekt jednoprzebiegowy --- O(n)}
\index{silnik ilościowy!złożoność O(n)}

Silnik został zaprojektowany z~myślą o~wydajności. Główna pętla przechodzi przez tablicę wiadomości \textbf{dokładnie raz} (złożoność $O(n)$, gdzie $n$ = liczba wiadomości), akumulując dane w~strukturach per-osoba (\emph{accumulators}).

\begin{warningbox}[title=Cel wydajnościowy]
Target: przetworzenie 50\,000 wiadomości w~$< 200$~ms. Cała analiza ilościowa odbywa się po stronie klienta (w~przeglądarce), więc wydajność jest krytyczna dla UX.
\end{warningbox}

Architektura silnika dzieli się na trzy fazy:

\begin{enumerate}
  \item \textbf{Inicjalizacja akumulatorów} --- utworzenie struktur danych per-osoba (mapy częstotliwości, liczniki, tablice czasów odpowiedzi) oraz globalnych akumulatorów (heatmapa, wolumen miesięczny, dzienne zliczenia).
  \item \textbf{Główny przebieg} --- jednokrotna iteracja po \tskey{messages[]} z~aktualizacją wszystkich akumulatorów w~każdej iteracji.
  \item \textbf{Przetwarzanie końcowe} --- obliczenie metryk pochodnych z~akumulatorów: mediany, średnie, regresje liniowe, wykrywanie burstów, indeks wzajemności, wyniki wiralne, odznaki.
\end{enumerate}

\begin{figure}[H]
\centering
\begin{tikzpicture}[node distance=0.8cm]
  % Phase 1
  \node[pipeline] (init) {Faza 1\\Inicjalizacja\\akumulatorów};

  % Phase 2
  \node[pipeline active, right=2cm of init] (main) {Faza 2\\Główny przebieg\\$O(n)$};

  % Phase 3
  \node[pipeline, right=2cm of main] (post) {Faza 3\\Przetwarzanie\\końcowe};

  % Input
  \node[startstop, left=1.5cm of init] (input) {\tstype{ParsedConversation}};

  % Output
  \node[startstop, right=1.5cm of post] (output) {\tstype{QuantitativeAnalysis}};

  % Arrows
  \draw[dataarrow] (input) -- (init);
  \draw[dataarrow] (init) -- (main);
  \draw[dataarrow] (main) -- (post);
  \draw[dataarrow] (post) -- (output);

  % Sub-modules below Phase 3
  \node[podbox blue, below=1.2cm of post, minimum width=2cm, font=\scriptsize\bfseries] (viral) {Viral Scores};
  \node[podbox purple, left=0.3cm of viral, minimum width=1.5cm, font=\scriptsize\bfseries] (badges) {Badges};
  \node[podbox blue, right=0.3cm of viral, minimum width=1.5cm, font=\scriptsize\bfseries] (catch) {Catchphrases};
  \node[podbox purple, below=0.5cm of viral, minimum width=2cm, font=\scriptsize\bfseries] (network) {Network Metrics};

  \draw[podarrow] (post) -- (viral);
  \draw[podarrow] (post) -- (badges);
  \draw[podarrow] (post) -- (catch);
  \draw[podarrow] (viral) -- (network);
\end{tikzpicture}
\caption{Trójfazowa architektura silnika analizy ilościowej z~modułami przetwarzania końcowego.}
\label{fig:engine-architecture}
\end{figure}

\subsection{Akumulatory per-osoba}

Wewnętrznie silnik używa interfejsu \tstype{PersonAccumulator} (niepublikowanego, wewnętrznego), który gromadzi surowe dane podczas głównego przebiegu:

\begin{lstlisting}[style=podcode, caption={Wewnętrzny interfejs akumulatora per-osoba}]
interface PersonAccumulator {
  totalMessages: number;
  totalWords: number;
  totalCharacters: number;
  longestMessage: { content: string; length: number; timestamp: number };
  shortestMessage: { content: string; length: number; timestamp: number };
  messagesWithEmoji: number;
  emojiCount: number;
  emojiFreq: Map<string, number>;
  questionsAsked: number;
  mediaShared: number;
  linksShared: number;
  reactionsGiven: number;
  reactionsReceived: number;
  reactionsGivenFreq: Map<string, number>;
  unsentMessages: number;
  responseTimes: number[];
  monthlyResponseTimes: Map<string, number[]>;
  monthlyWordCounts: Map<string, number[]>;
  messagesReceived: number;
  wordFreq: Map<string, number>;
  phraseFreq: Map<string, number>;
}
\end{lstlisting}

Po zakończeniu głównego przebiegu akumulatory są transformowane do publicznego interfejsu \tstype{PersonMetrics} (sekcja~\ref{sec:person-metrics}), a~dane czasowe do \tstype{TimingMetrics} (sekcja~\ref{sec:timing-metrics}).


% ============================================================
\section{Stałe i~konfiguracja}
\label{sec:stale-konfiguracja}
\index{silnik ilościowy!stałe}

\subsection{SESSION\_GAP\_MS}
\label{subsec:session-gap}

\begin{metricbox}
\textbf{Wartość:} \texttt{6 * 60 * 60 * 1000 = 21\,600\,000 ms} (6~godzin)
\end{metricbox}

Stała \texttt{SESSION\_GAP\_MS} definiuje granicę sesji konwersacyjnej. Jeśli przerwa między dwiema kolejnymi wiadomościami przekracza 6~godzin, silnik traktuje je jako koniec jednej sesji i~początek następnej. Wartość ta wpływa na:

\begin{itemize}
  \item \textbf{Inicjacje rozmów} (\metric{conversationInitiations}) --- pierwsza wiadomość po przerwie $\geq$ 6h jest liczona jako inicjacja nowej rozmowy.
  \item \textbf{Zakończenia rozmów} (\metric{conversationEndings}) --- ostatnia wiadomość przed przerwą $\geq$ 6h zamyka sesję.
  \item \textbf{Czas odpowiedzi} (\metric{responseTime}) --- odpowiedzi po przerwie $\geq$ 6h \emph{nie} są liczone do czasu odpowiedzi (nie chcemy zawyżać średnich przerwy nocne).
  \item \textbf{Sesje ogółem} (\metric{totalSessions}) --- liczba odrębnych sesji konwersacyjnych.
  \item \textbf{Sieć interakcji} (\metric{networkMetrics}) --- interakcje sekwencyjne po przerwie $\geq$ 6h nie tworzą krawędzi grafu.
\end{itemize}

\begin{infobox}[title=Dlaczego 6 godzin?]
Próg 6~godzin został dobrany empirycznie. Przerwa nocna trwa zazwyczaj 6--8~godzin, więc wartość ta pozwala odróżnić ,,pójście spać'' od ,,nowej rozmowy następnego dnia''. Przerwa w~ciągu dnia trwająca $>6$h zazwyczaj oznacza, że temat się wyczerpał, a~kolejna wiadomość to nowy wątek.
\end{infobox}

\subsection{STOPWORDS}
\label{subsec:stopwords}
\index{stopwords}

Lista \texttt{STOPWORDS} to zbiór 130+ najczęstszych słów w~języku polskim i~angielskim, wykluczonych z~analizy częstotliwości słów. Implementacja używa struktury \tstype{Set<string>} dla wyszukiwania w~$O(1)$.

Plik: \filepath{src/lib/analysis/constants.ts}

\begin{lstlisting}[style=podcode, caption={Fragment listy stopwords (polski + angielski)}]
export const STOPWORDS = new Set([
  // English
  'i', 'me', 'my', 'myself', 'we', 'our', 'you', 'your',
  'he', 'him', 'she', 'her', 'it', 'they', 'them',
  'what', 'which', 'who', 'this', 'that', 'am', 'is', 'are',
  'was', 'were', 'be', 'been', 'have', 'has', 'had',
  'do', 'does', 'did', 'a', 'an', 'the', 'and', 'but',
  'if', 'or', 'of', 'at', 'by', 'for', 'with', 'to',
  'from', 'in', 'out', 'on', 'off', 'ok', 'yes', 'yeah',
  'lol', 'haha', 'hahaha', 'xd', 'xdd',
  // Polish
  'i', 'w', 'z', 'na', 'do', 'to', 'je', 'nie', 'co',
  'tak', 'za', 'ale', 'od', 'po', 'jak', 'mi', 'ty', 'ja',
  'jest', 'bo', 'ze', 'sobie', 'tylko', 'jeszcze',
  'bardzo', 'teraz', 'ok', 'dobra', 'xd', 'xdd',
  // ... 130+ pozycji
]);
\end{lstlisting}

\begin{warningbox}[title=Dwujęzyczność]
Lista zawiera słowa z~\textbf{obu} języków, ponieważ polskie rozmowy na Messengerze bardzo często mieszają polski z~angielskim (np.\ ,,lol'', ,,ok'', ,,haha''). Słowa takie jak ,,xd'' i~,,xdd'' to specyficznie polskie markery emocjonalne, ubiquitous w~komunikacji młodych Polaków.
\end{warningbox}

\subsection{linearRegressionSlope()}
\label{subsec:linear-regression}
\index{regresja liniowa}

Funkcja \tsfunc{linearRegressionSlope(values)} oblicza nachylenie prostej regresji liniowej dla ciągu wartości liczbowych. Służy do detekcji trendów (rosnący/malejący) w~danych miesięcznych.

\begin{lstlisting}[style=podcode, caption={Implementacja regresji liniowej}]
export function linearRegressionSlope(values: number[]): number {
  const clean = values.filter((v) => Number.isFinite(v));
  const n = clean.length;
  if (n < 2) return 0;
  const xMean = (n - 1) / 2;
  const yMean = clean.reduce((a, b) => a + b, 0) / n;
  const numerator = clean.reduce(
    (sum, y, x) => sum + (x - xMean) * (y - yMean), 0,
  );
  const denominator = clean.reduce(
    (sum, _, x) => sum + (x - xMean) ** 2, 0,
  );
  if (denominator === 0) return 0;
  const slope = numerator / denominator;
  return Number.isFinite(slope) ? slope : 0;
}
\end{lstlisting}

Matematycznie, dla ciągu wartości $y_0, y_1, \ldots, y_{n-1}$ nachylenie prostej metodą najmniejszych kwadratów wynosi:

\begin{equation}
\label{eq:linear-regression}
\beta = \frac{\sum_{i=0}^{n-1} (x_i - \bar{x})(y_i - \bar{y})}{\sum_{i=0}^{n-1} (x_i - \bar{x})^2}
\end{equation}

gdzie $x_i = i$ (indeks kolejnego punktu), $\bar{x} = \frac{n-1}{2}$, a~$\bar{y}$ to średnia arytmetyczna wartości $y$.

\textbf{Interpretacja:}
\begin{itemize}
  \item $\beta > 0$ --- trend rosnący (np.\ więcej wiadomości każdego miesiąca)
  \item $\beta < 0$ --- trend malejący (np.\ spadek aktywności)
  \item $\beta \approx 0$ --- brak wyraźnego trendu
\end{itemize}

Funkcja automatycznie filtruje wartości \texttt{NaN} i~\texttt{Infinity} oraz zwraca 0, gdy mniej niż 2~punkty danych są dostępne.


% ============================================================
\section{Funkcje pomocnicze}
\label{sec:funkcje-pomocnicze}
\index{silnik ilościowy!funkcje pomocnicze}

Poniżej opisane są wszystkie funkcje pomocnicze używane przez główny silnik. Każda z~nich jest czystą funkcją bez efektów ubocznych.

\subsection{extractEmojis(text)}
\label{subsec:extract-emojis}
\index{emoji!ekstrakcja}

\begin{lstlisting}[style=podcode, caption={Ekstrakcja emoji z~tekstu}]
function extractEmojis(text: string): string[] {
  const emojiRegex = /\p{Emoji_Presentation}|\p{Extended_Pictographic}/gu;
  return text.match(emojiRegex) ?? [];
}
\end{lstlisting}

Używa kategorii Unicode \texttt{Emoji\_Presentation} i~\texttt{Extended\_Pictographic} z~flagą \texttt{u} (Unicode mode). Zwraca tablicę znalezionych emoji. Pusta tablica oznacza brak emoji w~wiadomości.

\subsection{countWords(text)}
\label{subsec:count-words}

\begin{lstlisting}[style=podcode]
function countWords(text: string): number {
  if (!text.trim()) return 0;
  return text.trim().split(/\s+/).length;
}
\end{lstlisting}

Dzieli tekst po białych znakach (\texttt{\textbackslash s+}) i~zwraca liczbę tokenów. Pusty tekst daje 0.

\subsection{median(arr)}
\label{subsec:median}

\begin{lstlisting}[style=podcode]
function median(values: number[]): number {
  if (values.length === 0) return 0;
  const sorted = [...values].sort((a, b) => a - b);
  const mid = Math.floor(sorted.length / 2);
  return sorted.length % 2 === 0
    ? (sorted[mid - 1] + sorted[mid]) / 2
    : sorted[mid];
}
\end{lstlisting}

Oblicza medianę zbioru wartości liczbowych. Tworzy kopię tablicy (nie modyfikuje oryginału), sortuje rosnąco, zwraca element środkowy (lub średnią dwóch środkowych dla parzystej liczby elementów).

\begin{infobox}[title=Dlaczego mediana, nie średnia?]
Dla czasu odpowiedzi mediana jest znacznie bardziej miarodajna niż średnia arytmetyczna. Pojedyncza przerwa 8-godzinna zawyży średnią drastycznie, podczas gdy mediana pozostaje stabilna. Mediana lepiej odpowiada na pytanie: ,,jak szybko ta osoba \emph{zwykle} odpowiada?''
\end{infobox}

\subsection{tokenizeWords(text)}
\label{subsec:tokenize-words}

\begin{lstlisting}[style=podcode, caption={Tokenizacja tekstu do słów}]
function tokenizeWords(text: string): string[] {
  return text
    .toLowerCase()
    .replace(/[\p{Emoji_Presentation}\p{Extended_Pictographic}]/gu, '')
    .split(/[\s.,!?;:()\[\]{}"'\-\/\\<>@#$%^&*+=|~`]+/)
    .filter(w => w.length >= 2 && !STOPWORDS.has(w));
}
\end{lstlisting}

Transformacje zastosowane kolejno:
\begin{enumerate}
  \item Konwersja do małych liter (\tsfunc{toLowerCase()})
  \item Usunięcie emoji (regex Unicode)
  \item Podział po znakach interpunkcyjnych i~białych
  \item Filtracja: minimum 2 znaki, nie należy do \texttt{STOPWORDS}
\end{enumerate}

\subsection{topN(map, n) / topNWords(map, n) / topNPhrases(map, n)}
\label{subsec:top-n}

Trzy warianty tej samej operacji: sortowanie mapy częstotliwości malejąco i~zwrócenie $n$~najczęstszych wpisów. Różnią się jedynie nazwą pola w~obiekcie wynikowym (\texttt{emoji}, \texttt{word}, \texttt{phrase}).

\begin{lstlisting}[style=podcode]
function topN(
  map: Map<string, number>, n: number,
): Array<{ emoji: string; count: number }> {
  return [...map.entries()]
    .sort((a, b) => b[1] - a[1])
    .slice(0, n)
    .map(([emoji, count]) => ({ emoji, count }));
}
\end{lstlisting}

Złożoność: $O(k \log k)$ gdzie $k$ = rozmiar mapy.

\subsection{isLateNight(hour) / isWeekend(day)}
\label{subsec:late-night-weekend}

\begin{lstlisting}[style=podcode]
function isLateNight(timestamp: number): boolean {
  const hour = new Date(timestamp).getHours();
  return hour >= 22 || hour < 4;
}

function isWeekend(timestamp: number): boolean {
  const day = new Date(timestamp).getDay();
  return day === 0 || day === 6; // Sunday or Saturday
}
\end{lstlisting}

\textbf{Late night:} godzina $\geq 22$ lub godzina $< 4$. Pokrywa przedział 22:00--03:59.

\textbf{Weekend:} \texttt{getDay()} zwraca 0 (niedziela) lub 6 (sobota).

\subsection{getMonthKey(timestamp) / getDayKey(timestamp)}
\label{subsec:date-keys}

\begin{lstlisting}[style=podcode]
function getMonthKey(timestamp: number): string {
  return new Date(timestamp).toISOString().slice(0, 7); // "YYYY-MM"
}

function getDayKey(timestamp: number): string {
  return new Date(timestamp).toISOString().slice(0, 10); // "YYYY-MM-DD"
}
\end{lstlisting}

Tworzą klucze grupujące wiadomości po miesiącach lub dniach. Format ISO zapewnia poprawne sortowanie leksykograficzne.


% ============================================================
\section{Metryki per osoba (PersonMetrics)}
\label{sec:person-metrics}
\index{PersonMetrics}
\index{metryki!per osoba}

Interfejs \tstype{PersonMetrics} zawiera 22~pola opisujące aktywność komunikacyjną każdego uczestnika rozmowy. Metryki te są obliczane dla \emph{każdej} osoby niezależnie.

\begin{lstlisting}[style=podcode, caption={Interfejs PersonMetrics --- definicja typu}]
export interface PersonMetrics {
  totalMessages: number;
  totalWords: number;
  totalCharacters: number;
  averageMessageLength: number;    // words
  averageMessageChars: number;     // characters
  longestMessage: { content: string; length: number; timestamp: number };
  shortestMessage: { content: string; length: number; timestamp: number };
  messagesWithEmoji: number;
  emojiCount: number;
  topEmojis: Array<{ emoji: string; count: number }>;
  questionsAsked: number;
  mediaShared: number;
  linksShared: number;
  reactionsGiven: number;
  reactionsReceived: number;
  topReactionsGiven: Array<{ emoji: string; count: number }>;
  unsentMessages: number;
  topWords: Array<{ word: string; count: number }>;
  topPhrases: Array<{ phrase: string; count: number }>;
  uniqueWords: number;
  vocabularyRichness: number;
}
\end{lstlisting}

Poniżej szczegółowy opis każdego pola.

\subsection{Metryki wolumenu}

\begin{description}[style=nextline]

\item[\metric{totalMessages}]
\textbf{Typ:} \tstype{number} \quad \textbf{Co mierzy:} Całkowita liczba wiadomości wysłanych przez osobę.\\
\textbf{Obliczanie:} Inkrementacja licznika \texttt{acc.totalMessages++} dla każdej wiadomości.\\
\textbf{Znaczenie:} Podstawowy wskaźnik aktywności. Duża dysproporcja (np.\ 3:1) sugeruje nierównowagę w~zaangażowaniu.

\item[\metric{totalWords}]
\textbf{Typ:} \tstype{number} \quad \textbf{Co mierzy:} Suma słów we wszystkich wiadomościach osoby.\\
\textbf{Obliczanie:} \texttt{acc.totalWords += countWords(msg.content)}.\\
\textbf{Znaczenie:} Bardziej miarodajny niż \texttt{totalMessages} --- osoba pisząca krótkie ,,ok'' i~,,haha'' ma wysoki totalMessages, ale niski totalWords.

\item[\metric{totalCharacters}]
\textbf{Typ:} \tstype{number} \quad \textbf{Co mierzy:} Suma znaków (łącznie z~białymi) we wszystkich wiadomościach.\\
\textbf{Obliczanie:} \texttt{acc.totalCharacters += msg.content.length}.\\
\textbf{Znaczenie:} Uzupełnia totalWords --- uwzględnia różnicę między polskimi (dłuższymi) a~angielskimi (krótszymi) słowami.

\item[\metric{averageMessageLength}]
\textbf{Typ:} \tstype{number} \quad \textbf{Jednostka:} słowa/wiadomość\\
\textbf{Obliczanie:}
\begin{equation}
  \text{averageMessageLength} = \frac{\text{totalWords}}{\text{totalMessages}}
\end{equation}
\textbf{Znaczenie:} Wskaźnik stylu komunikacji. Osoby piszące średnio 2--3 słowa/wiadomość komunikują się w~stylu ,,szybkich strzałów''; $> 15$ słów/wiadomość to narracyjny, refleksyjny styl.

\item[\metric{averageMessageChars}]
\textbf{Typ:} \tstype{number} \quad \textbf{Jednostka:} znaki/wiadomość\\
\textbf{Obliczanie:}
\begin{equation}
  \text{averageMessageChars} = \frac{\text{totalCharacters}}{\text{totalMessages}}
\end{equation}
\textbf{Znaczenie:} Alternatywna miara długości, bardziej precyzyjna dla języka polskiego (gdzie słowa są dłuższe z~powodu deklinacji i~koniugacji).

\end{description}

\subsection{Metryki skrajne}

\begin{description}[style=nextline]

\item[\metric{longestMessage}]
\textbf{Typ:} \tstype{\{content: string, length: number, timestamp: number\}}\\
\textbf{Co mierzy:} Najdłuższa wiadomość osoby pod względem liczby słów.\\
\textbf{Obliczanie:} Podczas przejścia, jeśli \texttt{wordCount > acc.longestMessage.length}, następuje aktualizacja. Puste wiadomości są pomijane.\\
\textbf{Znaczenie:} Najdłuższa wiadomość często jest emocjonalnie naładowana --- wyznania, tłumaczenia się, przeprosiny. Pole \texttt{timestamp} pozwala zlokalizować ją na osi czasu relacji.

\item[\metric{shortestMessage}]
\textbf{Typ:} \tstype{\{content: string, length: number, timestamp: number\}}\\
\textbf{Co mierzy:} Najkrótsza niepusta wiadomość (minimum 1~słowo).\\
\textbf{Obliczanie:} Inicjalizacja z~\texttt{length: Infinity}. Aktualizacja gdy \texttt{wordCount > 0 \&\& wordCount < acc.shortestMessage.length}. Po przebiegu, jeśli length = Infinity, ustawiane na \texttt{\{content: '', length: 0, timestamp: 0\}}.\\
\textbf{Znaczenie:} Ujawnia typowe jednowyrazowe reakcje osoby (np.\ ,,ok'', ,,hmm'', ,,co'').

\end{description}

\subsection{Metryki emoji}
\index{emoji!metryki}

\begin{description}[style=nextline]

\item[\metric{messagesWithEmoji}]
\textbf{Typ:} \tstype{number} \quad \textbf{Co mierzy:} Liczba wiadomości zawierających przynajmniej jedno emoji.\\
\textbf{Obliczanie:} \texttt{if (emojis.length > 0) acc.messagesWithEmoji++}.\\
\textbf{Znaczenie:} Informuje o~stylu ekspresji --- niektóre osoby dodają emoji do prawie każdej wiadomości, inne nie używają ich wcale.

\item[\metric{emojiCount}]
\textbf{Typ:} \tstype{number} \quad \textbf{Co mierzy:} Łączna liczba emoji we wszystkich wiadomościach.\\
\textbf{Obliczanie:} \texttt{acc.emojiCount += emojis.length}.\\
\textbf{Znaczenie:} Metryka intensywności wyrażania emocji. Stosunek emojiCount/totalMessages daje wskaźnik ,,gęstości emoji''.

\item[\metric{topEmojis}]
\textbf{Typ:} \tstype{Array<\{emoji: string, count: number\}>} \quad \textbf{Rozmiar:} top 10\\
\textbf{Co mierzy:} 10 najczęściej używanych emoji przez osobę.\\
\textbf{Obliczanie:} Zliczanie w~mapie \texttt{emojiFreq}, potem \tsfunc{topN(emojiFreq, 10)}.\\
\textbf{Znaczenie:} Profil emocjonalny: dominacja serduszek ($\heartsuit$) vs.\ emoji śmiechowych sugeruje różny charakter komunikacji.

\end{description}

\subsection{Metryki treści}

\begin{description}[style=nextline]

\item[\metric{questionsAsked}]
\textbf{Typ:} \tstype{number} \quad \textbf{Co mierzy:} Liczba wiadomości zawierających znak zapytania (po wykluczeniu URL-i).\\
\textbf{Obliczanie:}
\begin{lstlisting}[style=podcode]
const contentWithoutUrls = msg.content.replace(/https?:\/\/\S+/g, '');
if (contentWithoutUrls.includes('?')) acc.questionsAsked++;
\end{lstlisting}
\textbf{Znaczenie:} Osoba zadająca więcej pytań wykazuje ciekawość i~zainteresowanie drugą osobą. Duża dysproporcja (jedna osoba pyta, druga nie) może wskazywać na nierówną dynamikę relacji.

\begin{warningbox}[title=Wykluczanie URL-i]
URL-e (np.\ \texttt{https://example.com/search?q=test}) często zawierają znaki zapytania jako parametry query string. Bez filtrowania zafałszowałoby to statystykę pytań.
\end{warningbox}

\item[\metric{mediaShared}]
\textbf{Typ:} \tstype{number} \quad \textbf{Co mierzy:} Liczba wiadomości z~załączonym mediem (zdjęcia, filmy, pliki audio).\\
\textbf{Obliczanie:} \texttt{if (msg.hasMedia) acc.mediaShared++}.

\item[\metric{linksShared}]
\textbf{Typ:} \tstype{number} \quad \textbf{Co mierzy:} Liczba wiadomości zawierających linki.\\
\textbf{Obliczanie:} \texttt{if (msg.hasLink) acc.linksShared++}.\\
\textbf{Znaczenie:} Dzielenie się linkami to forma ekspresji zainteresowań --- osoba wysyłająca wiele linków aktywnie buduje wspólne doświadczenie.

\end{description}

\subsection{Metryki reakcji}
\index{reakcje!metryki}

\begin{description}[style=nextline]

\item[\metric{reactionsGiven}]
\textbf{Typ:} \tstype{number} \quad \textbf{Co mierzy:} Ile razy osoba zareagowała na wiadomości innych.\\
\textbf{Obliczanie:} Inkrementacja \texttt{actorAcc.reactionsGiven++} w~pętli po \texttt{msg.reactions}, gdzie actor to osoba reagująca.

\item[\metric{reactionsReceived}]
\textbf{Typ:} \tstype{number} \quad \textbf{Co mierzy:} Ile reakcji otrzymały wiadomości osoby.\\
\textbf{Obliczanie:} Inkrementacja \texttt{acc.reactionsReceived++} (akumulator nadawcy wiadomości) dla każdej reakcji na wiadomość.

\item[\metric{topReactionsGiven}]
\textbf{Typ:} \tstype{Array<\{emoji: string, count: number\}>} \quad \textbf{Rozmiar:} top 5\\
\textbf{Co mierzy:} 5 najczęściej używanych emoji reakcji dawanych przez osobę.\\
\textbf{Obliczanie:} Zliczanie w~mapie \texttt{reactionsGivenFreq}, potem \tsfunc{topN(reactionsGivenFreq, 5)}.

\end{description}

\subsection{Metryki wycofanych wiadomości}

\begin{description}[style=nextline]

\item[\metric{unsentMessages}]
\textbf{Typ:} \tstype{number} \quad \textbf{Co mierzy:} Liczba wiadomości wycofanych/usuniętych przez osobę.\\
\textbf{Obliczanie:} \texttt{if (msg.isUnsent) acc.unsentMessages++}.\\
\textbf{Znaczenie:} Częste wycofywanie wiadomości może wskazywać na niepewność, impulsywność lub lęk przed oceną.

\end{description}

\subsection{Metryki słownictwa}
\index{słownictwo!metryki}

\begin{description}[style=nextline]

\item[\metric{topWords}]
\textbf{Typ:} \tstype{Array<\{word: string, count: number\}>} \quad \textbf{Rozmiar:} top 20\\
\textbf{Co mierzy:} 20 najczęściej używanych słów (po odrzuceniu stopwords).\\
\textbf{Obliczanie:} Tokenizacja przez \tsfunc{tokenizeWords()}, zliczanie w~mapie \texttt{wordFreq}, \tsfunc{topNWords(wordFreq, 20)}.

\item[\metric{topPhrases}]
\textbf{Typ:} \tstype{Array<\{phrase: string, count: number\}>} \quad \textbf{Rozmiar:} top 10\\
\textbf{Co mierzy:} 10 najczęstszych bigramów (par kolejnych słów).\\
\textbf{Obliczanie:} Bigramy tworzone z~tokenów: \texttt{tokens[j] + " " + tokens[j+1]}, zliczanie w~\texttt{phraseFreq}.

\item[\metric{uniqueWords}]
\textbf{Typ:} \tstype{number} \quad \textbf{Co mierzy:} Liczba unikalnych słów użytych przez osobę (po filtracji stopwords).\\
\textbf{Obliczanie:} \texttt{acc.wordFreq.size} --- rozmiar mapy częstotliwości.

\item[\metric{vocabularyRichness}]
\textbf{Typ:} \tstype{number} \quad \textbf{Zakres:} $[0, 1]$\\
\textbf{Co mierzy:} Bogactwo słownictwa --- stosunek słów unikalnych do wszystkich słów.
\begin{equation}
  \text{vocabularyRichness} = \frac{\text{uniqueWords}}{\text{totalWords}}
\end{equation}
\textbf{Znaczenie:} Wartość bliska 1 oznacza duże zróżnicowanie słownictwa (osoby czytające, wykształcone). Wartość bliska 0 oznacza powtarzanie tych samych słów. Typowy zakres: 0.15--0.5.

\end{description}

\begin{table}[H]
\centering
\caption{Podsumowanie 22 pól interfejsu \tstype{PersonMetrics}}
\label{tab:person-metrics-summary}
\rowcolors{2}{white}{PodBlue!3}
\small
\begin{tabularx}{\textwidth}{L{3.5cm} C{1.8cm} X}
\toprule
\textbf{Pole} & \textbf{Typ} & \textbf{Opis} \\
\midrule
totalMessages & number & Łączna liczba wysłanych wiadomości \\
totalWords & number & Suma słów we wszystkich wiadomościach \\
totalCharacters & number & Suma znaków we wszystkich wiadomościach \\
averageMessageLength & number & Średnia długość wiadomości (słowa) \\
averageMessageChars & number & Średnia długość wiadomości (znaki) \\
longestMessage & object & Najdłuższa wiadomość (treść, długość, timestamp) \\
shortestMessage & object & Najkrótsza niepusta wiadomość \\
messagesWithEmoji & number & Wiadomości zawierające emoji \\
emojiCount & number & Łączna liczba emoji \\
topEmojis & array & Top 10 najczęstszych emoji \\
questionsAsked & number & Wiadomości z~pytajnikiem (bez URL) \\
mediaShared & number & Wiadomości z~mediami \\
linksShared & number & Wiadomości z~linkami \\
reactionsGiven & number & Dane reakcje na wiadomości innych \\
reactionsReceived & number & Otrzymane reakcje na własne wiadomości \\
topReactionsGiven & array & Top 5 najczęstszych reakcji dawanych \\
unsentMessages & number & Wycofane/usunięte wiadomości \\
topWords & array & Top 20 najczęstszych słów \\
topPhrases & array & Top 10 najczęstszych bigramów \\
uniqueWords & number & Liczba unikalnych słów \\
vocabularyRichness & number & Stosunek unikalnych do wszystkich słów \\
\bottomrule
\end{tabularx}
\end{table}


% ============================================================
\section{Metryki czasowe (TimingMetrics)}
\label{sec:timing-metrics}
\index{TimingMetrics}
\index{metryki!czasowe}

Metryki czasowe analizują \emph{kiedy} i~\emph{jak szybko} uczestnicy komunikują się. Interfejs \tstype{TimingMetrics} dzieli się na dwie części: metryki per-osoba i~metryki globalne.

\begin{lstlisting}[style=podcode, caption={Interfejs TimingMetrics}]
export interface TimingMetrics {
  perPerson: Record<string, {
    averageResponseTimeMs: number;
    medianResponseTimeMs: number;
    fastestResponseMs: number;
    slowestResponseMs: number;
    responseTimeTrend: number;
  }>;
  conversationInitiations: Record<string, number>;
  conversationEndings: Record<string, number>;
  longestSilence: {
    durationMs: number;
    startTimestamp: number;
    endTimestamp: number;
    lastSender: string;
    nextSender: string;
  };
  lateNightMessages: Record<string, number>;
}
\end{lstlisting}

\subsection{Metryki czasu odpowiedzi (per osoba)}

\begin{description}[style=nextline]

\item[\metric{averageResponseTimeMs}]
Średnia arytmetyczna czasów odpowiedzi. Czas odpowiedzi osoby A = czas między wiadomością osoby B a~następną wiadomością osoby A, pod warunkiem, że przerwa $<$ 6h.

\begin{equation}
  \text{averageResponseTimeMs} = \frac{1}{|\mathcal{R}|} \sum_{r \in \mathcal{R}} r
\end{equation}

gdzie $\mathcal{R}$ = zbiór czasów odpowiedzi osoby w~milisekundach.

\item[\metric{medianResponseTimeMs}]
Mediana zbioru czasów odpowiedzi. Bardziej miarodajna niż średnia (odporna na wartości odstające --- patrz sekcja~\ref{subsec:median}).

\item[\metric{fastestResponseMs}]
Najszybsza odpowiedź (minimum z~$\mathcal{R}$). Często poniżej 10~sekund --- sygnalizuje aktywną rozmowę w~czasie rzeczywistym.

\item[\metric{slowestResponseMs}]
Najwolniejsza odpowiedź (maksimum z~$\mathcal{R}$, ale $< 6$h bo powyżej to nowa sesja). Może wskazywać na ,,zamyślenie się'' lub brak zaangażowania w~danym momencie.

\item[\metric{responseTimeTrend}]
Nachylenie regresji liniowej (\ref{eq:linear-regression}) obliczonej na miesięcznych średnich czasach odpowiedzi.

\textbf{Interpretacja:}
\begin{itemize}
  \item Wartość \textbf{dodatnia}: czas odpowiedzi \emph{rośnie} z~miesiąca na miesiąc --- potencjalny spadek zainteresowania.
  \item Wartość \textbf{ujemna}: czas odpowiedzi \emph{maleje} --- rosnące zaangażowanie.
  \item Bliska \textbf{zeru}: stabilny wzorzec odpowiedzi.
\end{itemize}

\end{description}

\begin{warningbox}[title=Warunek zaliczenia odpowiedzi]
Czas odpowiedzi jest rejestrowany \textbf{tylko} gdy:
\begin{enumerate}
  \item Nadawca aktualnej wiadomości $\neq$ nadawca poprzedniej (zmiana rozmówcy)
  \item Przerwa $<$ \texttt{SESSION\_GAP\_MS} (6h) --- inaczej to nowa sesja, nie odpowiedź
\end{enumerate}
Te warunki eliminują zarówno ,,double texting'' (ta sama osoba pisze kilka razy z~rzędu), jak i~przerwy nocne/długie.
\end{warningbox}

\subsection{Metryki sesji (globalne)}

\begin{description}[style=nextline]

\item[\metric{conversationInitiations}]
\textbf{Typ:} \tstype{Record<string, number>} \quad \textbf{Co mierzy:} Ile razy każda osoba rozpoczęła nową sesję konwersacyjną.\\
\textbf{Algorytm:} Pierwsza wiadomość po przerwie $\geq$ 6h jest liczona jako inicjacja. Pierwsza wiadomość w~całej rozmowie również.\\
\textbf{Znaczenie:} Kto inicjuje kontakt? Duża dysproporcja (np.\ 80/20) jest jednym z~najsilniejszych wskaźników nierównego zainteresowania.

\item[\metric{conversationEndings}]
\textbf{Typ:} \tstype{Record<string, number>} \quad \textbf{Co mierzy:} Ile razy każda osoba wysłała ostatnią wiadomość przed przerwą $\geq$ 6h.\\
\textbf{Algorytm:} Ostatnia wiadomość przed przerwą $\geq$ 6h jest liczona jako zakończenie sesji. Ostatnia wiadomość w~całej rozmowie również.

\item[\metric{longestSilence}]
\textbf{Typ:} obiekt z~polami \texttt{durationMs}, \texttt{startTimestamp}, \texttt{endTimestamp}, \texttt{lastSender}, \texttt{nextSender}.\\
\textbf{Co mierzy:} Najdłuższa przerwa (cisza) między jakimikolwiek dwiema kolejnymi wiadomościami w~rozmowie.\\
\textbf{Algorytm:} Porównanie \texttt{gap} z~bieżącym rekordem w~każdej iteracji.\\
\textbf{Znaczenie:} Cisza może oznaczać kłótnię, okres bez kontaktu, podróż, lub po prostu koniec relacji na jakiś czas. Pola \texttt{lastSender} i~\texttt{nextSender} pozwalają określić, kto ,,zgasił'' rozmowę i~kto ją wznowił.

\item[\metric{lateNightMessages}]
\textbf{Typ:} \tstype{Record<string, number>} \quad \textbf{Co mierzy:} Liczba wiadomości wysłanych między 22:00 a~03:59 per osoba.\\
\textbf{Znaczenie:} Nocne wiadomości mają szczególny charakter emocjonalny --- często bardziej intymne, impulsywne lub wyrażające tęsknotę.

\end{description}

\subsection{Próg 6 godzin --- uzasadnienie}
\label{subsec:prog-6h}

\begin{figure}[H]
\centering
\begin{tikzpicture}[xscale=0.5, yscale=0.7]
  % Time axis
  \draw[->, thick, PodTextMuted] (0,0) -- (30,0) node[right] {\small czas};

  % Session 1
  \fill[PodBlue!30, rounded corners=2pt] (1,0.3) rectangle (8,1.5);
  \node[font=\scriptsize\bfseries, PodBlueDark] at (4.5, 0.9) {Sesja 1};

  % Messages in session 1
  \foreach \x/\c in {1.5/PodBlue, 2.2/PodPurple, 3.0/PodBlue, 4.5/PodPurple, 5.5/PodBlue, 7/PodPurple, 7.5/PodBlue} {
    \fill[\c] (\x, 0.1) circle (3pt);
  }

  % Gap
  \draw[<->, PodDanger, thick] (8, -0.5) -- (16, -0.5)
    node[midway, below, font=\scriptsize\color{PodDanger}] {$\geq$ 6h (nowa sesja)};

  % Session 2
  \fill[PodPurple!20, rounded corners=2pt] (16,0.3) rectangle (27,1.5);
  \node[font=\scriptsize\bfseries, PodPurpleDark] at (21.5, 0.9) {Sesja 2};

  % Messages in session 2
  \foreach \x/\c in {16.5/PodPurple, 17.5/PodBlue, 18.5/PodPurple, 20/PodBlue, 22/PodPurple, 24/PodBlue, 26/PodPurple} {
    \fill[\c] (\x, 0.1) circle (3pt);
  }

  % Labels
  \node[font=\scriptsize, PodBlue, above] at (1.5, 1.5) {Inicjacja};
  \node[font=\scriptsize, PodBlue, above] at (7.5, 1.5) {Zakończenie};
  \node[font=\scriptsize, PodPurple, above] at (16.5, 1.5) {Inicjacja};

  % Legend
  \fill[PodBlue] (1, -1.5) circle (3pt);
  \node[font=\scriptsize, anchor=west] at (1.4, -1.5) {Osoba A};
  \fill[PodPurple] (6, -1.5) circle (3pt);
  \node[font=\scriptsize, anchor=west] at (6.4, -1.5) {Osoba B};
\end{tikzpicture}
\caption{Wizualizacja podziału na sesje konwersacyjne z~progiem 6~godzin.}
\label{fig:session-gap}
\end{figure}


% ============================================================
\section{Metryki zaangażowania (EngagementMetrics)}
\label{sec:engagement-metrics}
\index{EngagementMetrics}
\index{metryki!zaangażowania}

\begin{lstlisting}[style=podcode, caption={Interfejs EngagementMetrics}]
export interface EngagementMetrics {
  doubleTexts: Record<string, number>;
  maxConsecutive: Record<string, number>;
  messageRatio: Record<string, number>;
  reactionRate: Record<string, number>;
  avgConversationLength: number;
  totalSessions: number;
}
\end{lstlisting}

\begin{description}[style=nextline]

\item[\metric{doubleTexts}]
\textbf{Co mierzy:} Ile razy osoba wysłała 2 lub więcej wiadomości z~rzędu \emph{bez odpowiedzi} drugiej strony.\\
\textbf{Algorytm:} Śledzenie zmiennych \texttt{consecutiveSender} i~\texttt{consecutiveCount}. Gdy nadawca zmienia się i~\texttt{consecutiveCount >= 2}, inkrementacja \texttt{doubleTexts[consecutiveSender]++}.\\
\textbf{Znaczenie:} Częste ,,double texting'' może wskazywać na niepokój, desperację kontaktu lub po prostu entuzjazm.

\item[\metric{maxConsecutive}]
\textbf{Co mierzy:} Maksymalna liczba wiadomości z~rzędu wysłanych przez osobę bez odpowiedzi.\\
\textbf{Algorytm:} \texttt{Math.max(maxConsecutive[sender], consecutiveCount)} przy każdej zmianie nadawcy.\\
\textbf{Znaczenie:} Wartość 15+ wiadomości z~rzędu to potencjalny sygnał lęku przed porzuceniem lub bombardowania emocjonalnego.

\item[\metric{messageRatio}]
\textbf{Co mierzy:} Proporcja wiadomości osoby do ogółu wiadomości.
\begin{equation}
  \text{messageRatio}[p] = \frac{\text{totalMessages}[p]}{\sum_i \text{totalMessages}[i]}
\end{equation}
\textbf{Zakres:} $[0, 1]$. Ideał dla dwóch osób: 0.5 (równy podział).\\
\textbf{Znaczenie:} Podstawowy wskaźnik równowagi konwersacji.

\item[\metric{reactionRate}]
\textbf{Co mierzy:} Wskaźnik reaktywności --- stosunek danych reakcji do otrzymanych wiadomości od innych.
\begin{equation}
  \text{reactionRate}[p] = \frac{\text{reactionsGiven}[p]}{\text{messagesReceived}[p]}
\end{equation}
\textbf{Znaczenie:} Osoba o~wysokim reactionRate aktywnie angażuje się emocjonalnie w~wiadomości innych. Zerowy reactionRate sugeruje pasywny odbiór.

\item[\metric{avgConversationLength}]
\textbf{Co mierzy:} Średnia liczba wiadomości w~jednej sesji konwersacyjnej.
\begin{equation}
  \text{avgConversationLength} = \frac{\text{totalMessages}}{\text{totalSessions}}
\end{equation}
\textbf{Znaczenie:} Krótkie sesje (5--10 wiadomości) = komunikacja transakcyjna. Długie sesje (100+) = głębokie, angażujące rozmowy.

\item[\metric{totalSessions}]
\textbf{Co mierzy:} Łączna liczba odrębnych sesji konwersacyjnych (rozdzielonych przerwami $\geq$ 6h).\\
\textbf{Znaczenie:} W~połączeniu z~zakresem dat pozwala obliczyć częstotliwość kontaktu (np.\ 2.3~sesji/dzień).

\end{description}


% ============================================================
\section{Metryki wzorców (PatternMetrics)}
\label{sec:pattern-metrics}
\index{PatternMetrics}
\index{metryki!wzorców}

\begin{lstlisting}[style=podcode, caption={Interfejs PatternMetrics}]
export interface PatternMetrics {
  monthlyVolume: Array<{
    month: string;
    perPerson: Record<string, number>;
    total: number;
  }>;
  weekdayWeekend: {
    weekday: Record<string, number>;
    weekend: Record<string, number>;
  };
  volumeTrend: number;
  bursts: Array<{
    startDate: string;
    endDate: string;
    messageCount: number;
    avgDaily: number;
  }>;
}
\end{lstlisting}

\subsection{monthlyVolume}

Tablica obiektów, jeden na każdy miesiąc kalendarzowy obecny w~rozmowie, posortowana chronologicznie. Każdy obiekt zawiera:
\begin{itemize}
  \item \texttt{month} --- klucz w~formacie \texttt{"YYYY-MM"} (np.\ \texttt{"2024-06"})
  \item \texttt{perPerson} --- mapa \texttt{nazwa -> liczba wiadomości}
  \item \texttt{total} --- suma wiadomości w~tym miesiącu
\end{itemize}

\subsection{weekdayWeekend}

Podział aktywności na dni robocze (poniedziałek--piątek) i~weekend (sobota--niedziela) per osoba.

\subsection{volumeTrend}

Nachylenie regresji liniowej na miesięcznych sumach wiadomości (\texttt{monthlyTotals}). Wartość dodatnia = rozmowa się rozwija; ujemna = zamiera.

\subsection{Wykrywanie burstów}
\label{subsec:burst-detection}
\index{burst detection}

\textbf{Burst} to okres nadzwyczaj intensywnej komunikacji --- dzień, w~którym liczba wiadomości przekracza $3\times$ średnią ruchomą z~7~dni.

\begin{lstlisting}[style=podcode, caption={Algorytm wykrywania burstów}]
function detectBursts(dailyCounts: Map<string, number>): Burst[] {
  const sortedDays = [...dailyCounts.keys()].sort();
  if (sortedDays.length < 8) return [];

  // Compute rolling 7-day average
  for (let i = 0; i < dayValues.length; i++) {
    let rollingAvg;
    if (i < 7) {
      rollingAvg = overallAvg; // fallback for first 7 days
    } else {
      rollingAvg = sum(dayValues[i-7..i]) / 7;
    }

    if (dayValues[i].count > 3 * rollingAvg && rollingAvg > 0) {
      burstDays.push(dayValues[i]);
    }
  }

  // Merge consecutive burst days into periods
  // ...
}
\end{lstlisting}

\textbf{Algorytm krok po kroku:}

\begin{enumerate}
  \item Posortuj dni chronologicznie.
  \item Dla każdego dnia oblicz 7-dniową średnią ruchomą (dla pierwszych 7 dni użyj średniej ogólnej jako baseline).
  \item Jeśli liczba wiadomości w~danym dniu $> 3 \times$ średnia ruchoma, oznacz jako dzień burstowy.
  \item Scal kolejne dni burstowe w~ciągłe okresy: dwa dni są ,,kolejne'' jeśli różnica $\leq 1$ dzień kalendarzowy.
  \item Dla każdego okresu burstowego oblicz \texttt{avgDaily = messageCount / days}.
\end{enumerate}

\textbf{Interpretacja burstów:} Nagłe skoki aktywności często korelują z~ważnymi wydarzeniami relacyjnymi --- zakochanie, kłótnia, pojednanie, kryzys, wyjazd.


% ============================================================
\section{Mapa cieplna (HeatmapData)}
\label{sec:heatmap}
\index{heatmap}
\index{mapa cieplna}

\begin{lstlisting}[style=podcode, caption={Interfejs HeatmapData}]
export interface HeatmapData {
  perPerson: Record<string, number[][]>;  // 7x24 matrix
  combined: number[][];                    // 7x24 matrix
}
\end{lstlisting}

Mapa cieplna to macierz $7 \times 24$ (dzień tygodnia $\times$ godzina dnia), gdzie każda komórka zawiera liczbę wiadomości wysłanych w~danym przedziale. Indeksowanie:
\begin{itemize}
  \item Wiersz 0 = niedziela, 1 = poniedziałek, ..., 6 = sobota (konwencja JavaScript \tsfunc{getDay()})
  \item Kolumna 0 = 00:00--00:59, 1 = 01:00--01:59, ..., 23 = 23:00--23:59
\end{itemize}

Generowane są dwie wersje: \texttt{perPerson} (osobna heatmapa dla każdego uczestnika) i~\texttt{combined} (zsumowana).

\begin{figure}[H]
\centering
\begin{tikzpicture}[scale=0.48]
  % Title
  \node[font=\small\bfseries, PodBlueDark] at (12, 8.5) {Przykładowa mapa cieplna --- aktywność łączna};

  % Day labels (left)
  \node[font=\tiny, anchor=east] at (-0.3, 7) {Nd};
  \node[font=\tiny, anchor=east] at (-0.3, 6) {Pn};
  \node[font=\tiny, anchor=east] at (-0.3, 5) {Wt};
  \node[font=\tiny, anchor=east] at (-0.3, 4) {Sr};
  \node[font=\tiny, anchor=east] at (-0.3, 3) {Cz};
  \node[font=\tiny, anchor=east] at (-0.3, 2) {Pt};
  \node[font=\tiny, anchor=east] at (-0.3, 1) {Sb};

  % Hour labels (bottom)
  \foreach \h in {0,2,...,22} {
    \node[font=\tiny, anchor=north] at (\h+0.5, 0.4) {\h};
  }
  \node[font=\scriptsize, PodTextSecondary] at (12, -0.2) {Godzina dnia};

  % Heatmap cells — example data pattern
  % Night hours (0-7): low activity
  % Morning (8-11): moderate
  % Afternoon (12-17): moderate-high
  % Evening (18-22): high peak
  % Late night (23): moderate

  \foreach \day in {0,...,6} {
    \foreach \hour in {0,...,23} {
      % Generate pseudo-random intensity based on typical messaging patterns
      \pgfmathsetmacro{\baseval}{
        (\hour >= 0 && \hour <= 6) ? 5 :
        (\hour >= 7 && \hour <= 9) ? 25 :
        (\hour >= 10 && \hour <= 14) ? 40 :
        (\hour >= 15 && \hour <= 17) ? 55 :
        (\hour >= 18 && \hour <= 21) ? 80 :
        (\hour == 22) ? 60 :
        30
      }
      % Weekend boost for late hours
      \pgfmathsetmacro{\weekendboost}{
        (\day == 0 || \day == 6) ? (
          (\hour >= 22 || \hour <= 2) ? 20 : 5
        ) : 0
      }
      \pgfmathsetmacro{\intensity}{min(\baseval + \weekendboost, 100)}
      \pgfmathsetmacro{\colorval}{\intensity/100}
      \fill[PodBlue!\intensity!white] (\hour, {7-\day}) rectangle ++(1, 1);
      \draw[white, line width=0.3pt] (\hour, {7-\day}) rectangle ++(1, 1);
    }
  }

  % Color scale legend
  \node[font=\tiny, anchor=east] at (26.5, 7) {Więcej};
  \node[font=\tiny, anchor=east] at (26.5, 1) {Mniej};
  \foreach \i in {0,...,5} {
    \pgfmathsetmacro{\pct}{\i*20}
    \fill[PodBlue!\pct!white] (27, {1+\i}) rectangle ++(0.7, 1);
    \draw[white, line width=0.3pt] (27, {1+\i}) rectangle ++(0.7, 1);
  }
\end{tikzpicture}
\caption{Przykładowa mapa cieplna aktywności wiadomości (7 dni $\times$ 24~godziny). Intensywność koloru odpowiada liczbie wiadomości w~danym przedziale.}
\label{fig:heatmap-example}
\end{figure}


% ============================================================
\section{Dane trendów (TrendData)}
\label{sec:trend-data}
\index{TrendData}
\index{metryki!trendów}

\begin{lstlisting}[style=podcode, caption={Interfejs TrendData}]
export interface TrendData {
  responseTimeTrend: Array<{
    month: string;
    perPerson: Record<string, number>;
  }>;
  messageLengthTrend: Array<{
    month: string;
    perPerson: Record<string, number>;
  }>;
  initiationTrend: Array<{
    month: string;
    perPerson: Record<string, number>;
  }>;
}
\end{lstlisting}

Obiekt \tstype{TrendData} zawiera trzy tablice trendów miesięcznych:

\begin{description}[style=nextline]

\item[\metric{responseTimeTrend}]
Średni czas odpowiedzi (w~ms) per osoba w~każdym miesiącu. Obliczany z~akumulatora \texttt{monthlyResponseTimes}. Rosnący trend = malejące zainteresowanie; malejący = rosnące zaangażowanie.

\item[\metric{messageLengthTrend}]
Średnia długość wiadomości (w~słowach) per osoba w~każdym miesiącu. Obliczana z~akumulatora \texttt{monthlyWordCounts}. Malejący trend = coraz krótsze odpowiedzi, możliwe znudzenie.

\item[\metric{initiationTrend}]
Liczba inicjacji rozmów per osoba w~każdym miesiącu. Obliczana z~mapy \texttt{monthlyInitiations}. Trend ujawnia zmiany w~tym, kto częściej szuka kontaktu.

\end{description}

Wszystkie trzy trendy są kluczowymi danymi wejściowymi dla wyników wiralnych (sekcja~\ref{sec:viral-scores}) --- w~szczególności \metric{interestScores} i~\metric{ghostRisk}.


% ============================================================
\section{Indeks wzajemności (ReciprocityIndex)}
\label{sec:reciprocity-index}
\index{ReciprocityIndex}
\index{indeks wzajemności}

\begin{lstlisting}[style=podcode, caption={Interfejs ReciprocityIndex}]
export interface ReciprocityIndex {
  overall: number;          // 0-100, 50 = perfect balance
  messageBalance: number;   // 0-100
  initiationBalance: number;// 0-100
  responseTimeSymmetry: number; // 0-100
  reactionBalance: number;  // 0-100
}
\end{lstlisting}

Indeks wzajemności to metryka kompozytowa mierząca \textbf{równowagę} między uczestnikami. Składa się z~4~komponentów, każdy w~skali 0--100, gdzie 50 oznacza idealną równowagę (dla rozmów 1:1).

\subsection{Komponenty}

\begin{description}[style=nextline]

\item[\metric{messageBalance}]
Jak blisko podziału 50/50 jest proporcja wiadomości.
\begin{equation}
  \text{messageBalance} = \lfloor 100 \cdot (1 - 2 \cdot |\text{ratioA} - 0.5|) \rceil
\end{equation}
Wartości: 100 gdy ratio = 0.5 (idealnie); 0 gdy ratio = 0 lub 1 (jednostronna).

\item[\metric{initiationBalance}]
Jak równomiernie rozkładają się inicjacje rozmów.
\begin{equation}
  \text{initiationBalance} = \lfloor 100 \cdot \left(1 - 2 \cdot \left|\frac{\text{initA}}{\text{initA} + \text{initB}} - 0.5\right|\right) \rceil
\end{equation}

\item[\metric{responseTimeSymmetry}]
Jak podobne są mediany czasu odpowiedzi obu osób.
\begin{equation}
  \text{responseTimeSymmetry} = \lfloor \frac{\min(\text{rtA}, \text{rtB})}{\max(\text{rtA}, \text{rtB})} \cdot 100 \rceil
\end{equation}
Wartość 100 gdy obie osoby odpowiadają tak samo szybko; spada gdy jedna jest znacząco wolniejsza.

\item[\metric{reactionBalance}]
Jak równomiernie obie strony reagują na wiadomości.
\begin{equation}
  \text{reactionBalance} = \lfloor 100 \cdot \left(1 - 2 \cdot \left|\frac{\text{reactA}}{\text{reactA} + \text{reactB}} - 0.5\right|\right) \rceil
\end{equation}

\end{description}

\subsection{Wynik ogólny}

\begin{equation}
\label{eq:reciprocity-overall}
\text{overall} = \lfloor \frac{\text{messageBalance} + \text{initiationBalance} + \text{responseTimeSymmetry} + \text{reactionBalance}}{4} \rceil
\end{equation}

Wszystkie 4~komponenty mają jednakową wagę. Wynik \texttt{overall} jest następnie używany jako komponent \metric{Reciprocity} w~wyniku zdrowia relacji (Rozdział~\ref{ch:health-score}).

\begin{infobox}[title=Ograniczenie do rozmów 1:1]
Indeks wzajemności jest najbardziej miarodajny dla rozmów dwuosobowych. Dla rozmów grupowych ($> 2$ uczestników) silnik używa pierwszych dwóch uczestników z~listy, co daje wynik przybliżony. Pełna analiza wzajemności w~grupach wymaga analizy każdej pary osobno.
\end{infobox}


% ============================================================
\section{Wyniki wiralne (ViralScores)}
\label{sec:viral-scores}
\index{ViralScores}
\index{wyniki wiralne}

Moduł wiralny oblicza ,,zabawne'' metryki zaprojektowane z~myślą o~udostępnianiu w~mediach społecznościowych. Wszystkie wyniki to czysta matematyka, bez AI.

Plik: \filepath{src/lib/analysis/viral-scores.ts}

\begin{lstlisting}[style=podcode, caption={Interfejs ViralScores}]
export interface ViralScores {
  compatibilityScore: number;       // 0-100
  interestScores: Record<string, number>;  // 0-100 per person
  ghostRisk: Record<string, GhostRiskData>; // per person
  delusionScore: number;            // 0-100
  delusionHolder?: string;
}
\end{lstlisting}

\subsection{compatibilityScore (0--100)}
\label{subsec:compatibility}
\index{compatibility score}

Wynik kompatybilności oparty na 5~równoważnych pod-metrykach:

\begin{equation}
\text{compatibilityScore} = \text{clamp}\left(\lfloor\frac{S_1 + S_2 + S_3 + S_4 + S_5}{5}\rceil, 0, 100\right)
\end{equation}

\begin{table}[H]
\centering
\caption{Pod-metryki wyniku kompatybilności}
\label{tab:compatibility-submetrics}
\rowcolors{2}{white}{PodPurple!5}
\small
\begin{tabularx}{\textwidth}{L{3.5cm} X}
\toprule
\textbf{Pod-metryka} & \textbf{Opis i~algorytm} \\
\midrule
$S_1$: Activity Overlap &
Nakładanie się rozkładów godzinowych aktywności. Sumuje rozkłady procentowe na 24~godziny i~oblicza overlap jako $\sum_{h=0}^{23} \min(\text{pct}_A[h], \text{pct}_B[h])$. Wzorzec Szymkiewicza-Simpsona. \\

$S_2$: Response Symmetry &
Symetria medianowych czasów odpowiedzi: $100 - \frac{|\text{med}_A - \text{med}_B|}{\max(\text{med}_A, \text{med}_B)} \cdot 100$. \\

$S_3$: Message Balance &
Równowaga ilości wiadomości: $100 - |\text{ratio}_A - 0.5| \cdot 200$. \\

$S_4$: Engagement Balance &
Symetria wskaźników reaktywności: $100 - |\text{rate}_A - \text{rate}_B| \cdot 500$. Współczynnik 500 penalizuje nawet niewielkie różnice. \\

$S_5$: Length Match &
Podobieństwo średnich długości wiadomości: $100 - \frac{|\text{avg}_A - \text{avg}_B|}{\max(\text{avg}_A, \text{avg}_B)} \cdot 100$. \\
\bottomrule
\end{tabularx}
\end{table}

\subsection{interestScores (0--100, per osoba)}
\label{subsec:interest-scores}
\index{interest score}

Wynik zainteresowania mierzy, jak bardzo dana osoba jest zaangażowana w~relację. Obliczany z~6~ważonych czynników:

\begin{equation}
\label{eq:interest-score}
\text{interest} = 0.25 \cdot I_1 + 0.20 \cdot I_2 + 0.15 \cdot I_3 + 0.20 \cdot I_4 + 0.10 \cdot I_5 + 0.10 \cdot I_6
\end{equation}

\begin{table}[H]
\centering
\caption{Czynniki wyniku zainteresowania z~wagami}
\label{tab:interest-factors}
\rowcolors{2}{white}{PodBlue!3}
\small
\begin{tabularx}{\textwidth}{C{1cm} L{3cm} C{1.2cm} X}
\toprule
\textbf{Nr} & \textbf{Czynnik} & \textbf{Waga} & \textbf{Algorytm} \\
\midrule
$I_1$ & Częstotliwość inicjacji & 25\% & $\frac{\text{init}[p]}{\sum \text{init}} \cdot 100$ --- wyżej = bardziej aktywnie szuka kontaktu \\
$I_2$ & Trend czasu odpowiedzi & 20\% & $50 - \frac{\text{slope}}{1200}$ --- ujemny slope (szybsze odpowiedzi) = wyższy wynik \\
$I_3$ & Trend długości wiadomości & 15\% & $50 + \text{slope} \cdot 25$ --- rosnąca długość = większe zaangażowanie \\
$I_4$ & Częstotliwość reakcji & 20\% & $\text{reactionRate} \cdot 500$ --- wiecej reakcji = więcej uwagi \\
$I_5$ & Double texting (odwrotność) & 10\% & $\frac{\text{dt} \cdot 1000}{\text{total}} \cdot 2$ --- więcej double textów = wyższe zaangażowanie \\
$I_6$ & Aktywność nocna & 10\% & $\frac{\text{lateNight}}{\text{totalMessages}} \cdot 1000$ --- nocne wiadomości = silniejsze emocje \\
\bottomrule
\end{tabularx}
\end{table}

\subsection{ghostRisk (0--100, per osoba)}
\label{subsec:ghost-risk}
\index{ghost risk}

Ryzyko ghostingu mierzy prawdopodobieństwo, że osoba odejdzie z~rozmowy. Opiera się na porównaniu \textbf{ostatnich 3~miesięcy} z~\textbf{wcześniejszym okresem}. Wymaga minimum 6~miesięcy danych.

\begin{equation}
\label{eq:ghost-risk}
\text{ghostRisk} = \text{clamp}\left(0.30 \cdot G_1 + 0.25 \cdot G_2 + 0.25 \cdot G_3 + 0.20 \cdot G_4, \; 0, 100\right)
\end{equation}

\begin{table}[H]
\centering
\caption{4 trendy analizowane przez algorytm ghostRisk}
\label{tab:ghost-risk-trends}
\rowcolors{2}{white}{PodDanger!5}
\small
\begin{tabularx}{\textwidth}{C{1cm} L{3.2cm} C{1.2cm} X}
\toprule
\textbf{Nr} & \textbf{Trend} & \textbf{Waga} & \textbf{Negatywny sygnał} \\
\midrule
$G_1$ & Czas odpowiedzi & 30\% & Rośnie --- osoba odpowiada coraz wolniej \\
$G_2$ & Długość wiadomości & 25\% & Maleje --- osoba pisze coraz krócej \\
$G_3$ & Inicjacje rozmów & 25\% & Maleje --- osoba rzadziej inicjuje kontakt \\
$G_4$ & Wolumen wiadomości & 20\% & Maleje --- osoba pisze mniej wiadomości \\
\bottomrule
\end{tabularx}
\end{table}

Każdy pod-wynik $G_i$ jest obliczany jako procentowy wzrost/spadek między okresami:
\begin{equation}
  G_i = \text{clamp}\left(\frac{|\text{earlier}_i - \text{recent}_i|}{\text{earlier}_i} \cdot 100, \; 0, 100\right)
\end{equation}

Gdy $G_i > 30$, do tablicy \texttt{factors[]} dodawany jest opis w~języku polskim (np.\ ,,Czas odpowiedzi rośnie'').

\subsection{delusionScore (0--100) i delusionHolder}
\label{subsec:delusion-score}
\index{delusion score}

Wynik ,,urojeniowy'' mierzy \textbf{asymetrię zainteresowania} --- jak bardzo różnią się interest scores obu osób.

\begin{equation}
  \text{delusionScore} = |\text{interest}_A - \text{interest}_B|
\end{equation}

\begin{itemize}
  \item \texttt{delusionHolder} = osoba z~\emph{wyższym} interestem (ta, która ,,bardziej się stara'' podczas gdy druga strona jest mniej zaangażowana)
  \item Jeśli różnica $< 5$ punktów, \texttt{delusionHolder = undefined} (nie ma ,,urojenia'')
  \item Wynik 0 = idealna symetria; 100 = skrajnie jednostronne zainteresowanie
\end{itemize}

\begin{infobox}[title=Kontekst wiralowy]
Nazwa ,,delusion score'' jest celowo prowokacyjna --- ma zachęcać do udostępniania wyników w~social media. W~interfejsie UI jest prezentowana z~humorem i~odpowiednim disclaimerem.
\end{infobox}


% ============================================================
\section{System odznak (Badges)}
\label{sec:badges}
\index{badges}
\index{odznaki}

System odznak przyznaje zabawne osiągnięcia uczestnikom na podstawie ich wzorców komunikacji. Każda odznaka ma unikalnego zwycięzcę --- osobę z~najwyższym wynikiem w~danej kategorii.

Plik: \filepath{src/lib/analysis/badges.ts}

\begin{lstlisting}[style=podcode, caption={Interfejs Badge}]
export interface Badge {
  id: string;
  name: string;
  emoji: string;
  description: string;
  holder: string;      // kto zdobył odznakę
  evidence: string;    // dowód (wartość liczbowa)
}
\end{lstlisting}

\begin{table}[H]
\centering
\caption{Kompletna lista 12 odznak systemu \podtekst}
\label{tab:badges}
\rowcolors{2}{white}{PodPurple!4}
\small
\begin{tabularx}{\textwidth}{C{0.5cm} C{0.5cm} L{2.3cm} L{3.2cm} X}
\toprule
\textbf{\#} & \textbf{} & \textbf{Nazwa} & \textbf{ID} & \textbf{Kryterium przyznania} \\
\midrule
1 & \raisebox{-1pt}{\Large $\bigcirc$} & Nocny Marek & \texttt{night-owl} & Najwyższy \% wiadomości wysłanych 22:00--3:59 \\
2 & \raisebox{-1pt}{\Large $\bigcirc$} & Ranny Ptaszek & \texttt{early-bird} & Najwyższa łączna liczba wiadomości przed 8:00 \\
3 & \raisebox{-1pt}{\Large $\bigcirc$} & Ghosting Champion & \texttt{ghost-champion} & Wysłał(a) ostatnią wiadomość przed najdłuższą ciszą \\
4 & \raisebox{-1pt}{\Large $\bigcirc$} & Double Texter & \texttt{double-texter} & Najczęściej pisał(a) 2+ wiadomości bez odpowiedzi \\
5 & \raisebox{-1pt}{\Large $\bigcirc$} & Powieściopisarz & \texttt{novelist} & Najwyższa średnia długość wiadomości (słowa) \\
6 & \raisebox{-1pt}{\Large $\bigcirc$} & Speed Demon & \texttt{speed-demon} & Najszybsza mediana czasu odpowiedzi (\tsfunc{findLowest}) \\
7 & \raisebox{-1pt}{\Large $\bigcirc$} & Emoji King/Queen & \texttt{emoji-monarch} & Najwyższy stosunek emoji na wiadomość \\
8 & \raisebox{-1pt}{\Large $\bigcirc$} & Inicjator & \texttt{initiator} & Najczęstsze rozpoczynanie rozmów \\
9 & \raisebox{-1pt}{\Large $\bigcirc$} & Heart Bomber & \texttt{heart-bomber} & Najwięcej reakcji serduszkowych (wszystkie warianty $\heartsuit$) \\
10 & \raisebox{-1pt}{\Large $\bigcirc$} & Link Lord & \texttt{link-lord} & Najwięcej udostępnionych linków \\
11 & \raisebox{-1pt}{\Large $\bigcirc$} & Streak Master & \texttt{streak-master} & Najdłuższa seria kolejnych dni z~wiadomościami \\
12 & \raisebox{-1pt}{\Large $\bigcirc$} & Detektyw & \texttt{question-master} & Najwięcej zadanych pytań \\
\bottomrule
\end{tabularx}
\end{table}

\subsection{Algorytm computeStreaks()}
\label{subsec:compute-streaks}
\index{streak!algorytm}

Odznaka \emph{Streak Master} wymaga obliczenia najdłuższej serii kolejnych dni, w~których osoba wysłała przynajmniej jedną wiadomość.

\begin{lstlisting}[style=podcode, caption={Algorytm obliczania streak-ów}]
function computeStreaks(
  conversation: ParsedConversation,
): Record<string, number> {
  // 1. Zbierz unikalne dni per osoba
  const daysPerPerson = new Map<string, Set<string>>();
  for (const msg of conversation.messages) {
    const dayKey = new Date(msg.timestamp).toISOString().slice(0, 10);
    if (!daysPerPerson.has(msg.sender))
      daysPerPerson.set(msg.sender, new Set());
    daysPerPerson.get(msg.sender)!.add(dayKey);
  }

  // 2. Dla kazdej osoby: posortuj dni, znajdz najdluzszy ciag
  for (const [name, daySet] of daysPerPerson) {
    const sortedDays = [...daySet].sort();
    let maxStreak = 1, currentStreak = 1;

    for (let i = 1; i < sortedDays.length; i++) {
      const prevDate = new Date(sortedDays[i - 1] + 'T00:00:00Z');
      const currDate = new Date(sortedDays[i] + 'T00:00:00Z');
      const diffDays = (currDate - prevDate) / (1000 * 60 * 60 * 24);

      if (Math.round(diffDays) === 1) {
        currentStreak++;
        maxStreak = Math.max(maxStreak, currentStreak);
      } else {
        currentStreak = 1;
      }
    }
    streaks[name] = maxStreak;
  }
  return streaks;
}
\end{lstlisting}

\textbf{Krok po kroku:}
\begin{enumerate}
  \item Dla każdej wiadomości oblicz klucz dnia (format ISO \texttt{YYYY-MM-DD}) i~dodaj do zbioru \tstype{Set<string>} danej osoby.
  \item Posortuj unikalne dni leksykograficznie (format ISO gwarantuje poprawne sortowanie).
  \item Iteruj po parach kolejnych dni: jeśli różnica wynosi dokładnie 1~dzień, inkrementuj \texttt{currentStreak}; w~przeciwnym razie resetuj do 1.
  \item Śledź \texttt{maxStreak} --- najdłuższa znaleziona seria.
\end{enumerate}

\textbf{Uwaga:} Odznaka jest przyznawana tylko gdy \texttt{maxStreak > 1} (seria minimum 2 dni).

\subsection{Mechanizm Heart Bomber}
\label{subsec:heart-bomber}

Odznaka Heart Bomber wymaga identyfikacji reakcji serduszkowych we wszystkich ich wariantach Unicode. Implementacja używa wyrażenia regularnego dopasowującego 15~wariantów:

\begin{lstlisting}[style=podcode]
// Match all heart emoji variants
if (/\u2764|\u{1F493}|\u{1F496}|\u{1F497}|\u{1F498}|
     \u{1F499}|\u{1F49A}|\u{1F49B}|\u{1F49C}|\u{1F5A4}|
     \u{1F90D}|\u{1F90E}|\u{1FA77}|\u{2763}|\u{1F9E1}/u
    .test(reaction.emoji)) {
  hearts += reaction.count;
}
\end{lstlisting}

Warianty obejmują: czerwone serce ($\heartsuit$), pulsujące, rosnące, ze strzałą, niebieskie, zielone, żółte, fioletowe, czarne, białe, brązowe, różowe, wykrzyknikowe i~ogniste.


% ============================================================
\section{Najlepszy czas na wiadomość (BestTimeToText)}
\label{sec:best-time-to-text}
\index{BestTimeToText}
\index{najlepszy czas}

\begin{lstlisting}[style=podcode, caption={Interfejs BestTimeToText}]
export interface BestTimeToText {
  perPerson: Record<string, {
    bestDay: string;        // Polska nazwa dnia (np. "Poniedziałek")
    bestHour: number;       // 0-23
    bestWindow: string;     // np. "Poniedziałki 14:00-16:00"
    avgResponseMs: number;  // mediana czasu odpowiedzi
  }>;
}
\end{lstlisting}

Algorytm iteruje po macierzy heatmapy danej osoby ($7 \times 24$) i~znajduje komórkę \texttt{(day, hour)} z~najwyższą liczbą wiadomości. Wynik prezentowany jest w~formie czytelnego okna czasowego.

\textbf{Algorytm:}
\begin{enumerate}
  \item Dla każdej osoby: przeszukaj macierz heatmapy, znajdź \texttt{(bestDay, bestHour)} z~maksymalnym zliczeniem.
  \item Utwórz 2-godzinne okno zaczynające się od \texttt{bestHour}.
  \item Sformatuj wynik: polska nazwa dnia w~liczbie mnogiej + zakres godzin (np.\ ,,Piątki 20:00--22:00'').
  \item Dodaj medianę czasu odpowiedzi jako informację uzupełniającą.
\end{enumerate}

Polskie nazwy dni są mapowane za pomocą stałych:

\begin{lstlisting}[style=podcode]
const POLISH_DAYS: Record<number, string> = {
  0: 'Niedziela', 1: 'Poniedzialek', 2: 'Wtorek',
  3: 'Sroda',     4: 'Czwartek',     5: 'Piatek',
  6: 'Sobota',
};
const POLISH_DAYS_PLURAL: Record<number, string> = {
  0: 'Niedziele', 1: 'Poniedzialki', 2: 'Wtorki',
  3: 'Srody',     4: 'Czwartki',     5: 'Piatki',
  6: 'Soboty',
};
\end{lstlisting}


% ============================================================
\section{Frazy charakterystyczne (Catchphrases)}
\label{sec:catchphrases}
\index{catchphrases}
\index{frazy charakterystyczne}

Moduł \texttt{catchphrases} identyfikuje unikalne frazy (bigramy i~trigramy), które dana osoba używa \emph{znacznie częściej} niż inni uczestnicy rozmowy.

Plik: \filepath{src/lib/analysis/catchphrases.ts}

\begin{lstlisting}[style=podcode, caption={Interfejs CatchphraseEntry}]
export interface CatchphraseEntry {
  phrase: string;     // 2-3 slowa, np. "no dobra"
  count: number;      // minimum 3
  uniqueness: number; // 0-1, minimum 0.6
}
\end{lstlisting}

\subsection{Algorytm}

\begin{enumerate}
  \item \textbf{Ekstrakcja n-gramów:} Dla każdej wiadomości każdej osoby:
  \begin{itemize}
    \item Tokenizacja (lowercase, bez emoji, bez stopwords, min.\ 2~znaki)
    \item Generacja bigramów: \texttt{tokens[j] + " " + tokens[j+1]}
    \item Generacja trigramów: \texttt{tokens[j] + " " + tokens[j+1] + " " + tokens[j+2]}
  \end{itemize}

  \item \textbf{Zliczanie globalne:} Sumuj wystąpienia każdej frazy \emph{po wszystkich} osobach.

  \item \textbf{Obliczenie unikalności:} Dla każdej frazy osoby:
  \begin{equation}
    \text{uniqueness} = \frac{\text{count}_{person}}{\text{count}_{global}}
  \end{equation}
  Wartość 1.0 oznacza, że \emph{tylko} ta osoba używa tej frazy. Wartość 0.5 oznacza, że dwie osoby używają jej równo.

  \item \textbf{Filtracja:}
  \begin{itemize}
    \item \texttt{count $\geq$ 3} --- fraza musi pojawić się przynajmniej 3~razy
    \item \texttt{uniqueness $\geq$ 0.6} --- fraza musi być używana w~$\geq 60\%$ przez tę osobę
  \end{itemize}

  \item \textbf{Ranking:} Sortuj malejąco po $\text{count} \times \text{uniqueness}$. Weź top 8 per osoba.
\end{enumerate}

\begin{infobox}[title=Przykłady catchphrases]
Typowe wyniki:
\begin{itemize}
  \item Anna: ,,no dobra'' (count: 47, uniqueness: 0.92), ,,wiesz co'' (count: 31, uniqueness: 0.78)
  \item Michał: ,,dawaj jutro'' (count: 22, uniqueness: 0.85), ,,spoko luzik'' (count: 15, uniqueness: 1.0)
\end{itemize}
\end{infobox}


% ============================================================
\section{Metryki sieci (NetworkMetrics)}
\label{sec:network-metrics}
\index{NetworkMetrics}
\index{metryki sieci}
\index{czat grupowy}

Moduł sieciowy jest aktywowany \textbf{wyłącznie} dla czatów grupowych (\texttt{conversation.metadata.isGroup === true}). Buduje ważony graf interakcji między uczestnikami.

Plik: \filepath{src/lib/analysis/network.ts}

\begin{lstlisting}[style=podcode, caption={Interfejsy NetworkNode, NetworkEdge, NetworkMetrics}]
export interface NetworkNode {
  name: string;
  totalMessages: number;
  centrality: number;    // 0-1, degree centrality
}

export interface NetworkEdge {
  from: string;
  to: string;
  weight: number;        // total mutual interaction count
  fromToCount: number;   // messages from->to
  toFromCount: number;   // messages to->from
}

export interface NetworkMetrics {
  nodes: NetworkNode[];
  edges: NetworkEdge[];
  density: number;       // actual edges / possible edges
  mostConnected: string; // highest centrality
}
\end{lstlisting}

\subsection{Budowa grafu interakcji}

\textbf{Definicja interakcji:} Gdy osoba A wysyła wiadomość bezpośrednio po osobie B (w~tej samej sesji, $< 6$h), tworzona jest krawędź B$\rightarrow$A (B ,,rozpoczął'' interakcję, A ,,odpowiedział''). Wiadomości tej samej osoby z~rzędu nie tworzą krawędzi.

\textbf{Algorytm:}
\begin{enumerate}
  \item Inicjalizuj macierz interakcji $n \times n$ (zerami), gdzie $n$ = liczba uczestników.
  \item Iteruj po wiadomościach: jeśli \texttt{prev.sender $\neq$ curr.sender} i~\texttt{gap $<$ 6h}, inkrementuj \texttt{interactions[prev.sender][curr.sender]}.
  \item Scal krawędzie w~nieskierowane: \texttt{weight = fromToCount + toFromCount}.
  \item Oblicz centralność stopniową: $\text{centrality}[p] = \frac{|\text{connections}[p]|}{n - 1}$.
  \item Oblicz gęstość grafu: $\text{density} = \frac{\text{actual edges}}{\binom{n}{2}}$.
\end{enumerate}

\subsection{Centralność stopniowa (Degree Centrality)}
\index{centralność stopniowa}

\begin{equation}
  C_D(v) = \frac{\deg(v)}{n - 1}
\end{equation}

gdzie $\deg(v)$ = liczba unikalnych osób, z~którymi $v$ wymienia wiadomości, a~$n$ = łączna liczba uczestników. Wartość 1.0 oznacza, że osoba wchodzi w~interakcje z~\emph{każdym} uczestnikiem.

\subsection{Gęstość grafu}

\begin{equation}
  D = \frac{|E|}{\frac{n(n-1)}{2}}
\end{equation}

gdzie $|E|$ = liczba krawędzi z~wagą $> 0$. Gęstość 1.0 oznacza, że każda para uczestników wymienia wiadomości.

\textbf{Interpretacja:}
\begin{itemize}
  \item $D > 0.8$ --- intensywna, wielostronna rozmowa grupowa
  \item $0.4 < D < 0.8$ --- typowy czat grupowy z~podgrupami
  \item $D < 0.4$ --- fragmentaryczna komunikacja; kilka osób dominuje
\end{itemize}

\begin{figure}[H]
\centering
\begin{tikzpicture}
  % Nodes
  \node[badge, fill=PodBlue!15, draw=PodBlue] (a) at (0, 2) {\small A};
  \node[badge, fill=PodPurple!15, draw=PodPurple] (b) at (3, 3.5) {\small B};
  \node[badge, fill=PodSuccess!15, draw=PodSuccess] (c) at (5, 1) {\small C};
  \node[badge, fill=PodWarning!15, draw=PodWarning] (d) at (2, -0.5) {\small D};

  % Edges with varying thickness
  \draw[PodBlue, line width=3pt, opacity=0.5] (a) -- (b) node[midway, above left, font=\scriptsize] {147};
  \draw[PodPurple, line width=2pt, opacity=0.5] (b) -- (c) node[midway, above right, font=\scriptsize] {89};
  \draw[PodSuccess, line width=1.5pt, opacity=0.5] (a) -- (c) node[midway, below, font=\scriptsize] {52};
  \draw[PodWarning, line width=1pt, opacity=0.5] (a) -- (d) node[midway, left, font=\scriptsize] {31};
  \draw[PodBlue!50, line width=2.5pt, opacity=0.5] (b) -- (d) node[midway, right, font=\scriptsize] {103};
  \draw[PodPurple!50, line width=0.5pt, opacity=0.5] (c) -- (d) node[midway, below right, font=\scriptsize] {12};

  % Centrality labels
  \node[font=\scriptsize\color{PodTextMuted}, below=3pt of a] {$C_D=1.0$};
  \node[font=\scriptsize\color{PodTextMuted}, above=3pt of b] {$C_D=1.0$};
  \node[font=\scriptsize\color{PodTextMuted}, right=3pt of c] {$C_D=1.0$};
  \node[font=\scriptsize\color{PodTextMuted}, below=3pt of d] {$C_D=1.0$};

  % Density label
  \node[font=\small\bfseries\color{PodBlueDark}, anchor=west] at (6.5, 2) {$D = \frac{6}{6} = 1.0$};
  \node[font=\scriptsize\color{PodTextSecondary}, anchor=west] at (6.5, 1.3) {Pełne połączenie};
\end{tikzpicture}
\caption{Przykładowy graf sieci dla czatu 4-osobowego. Grubość krawędzi odpowiada wadze (liczbie interakcji). Liczby na krawędziach = łączna liczba sekwencyjnych wymian.}
\label{fig:network-graph}
\end{figure}


% ============================================================
\subsection{Refaktoryzacja modularna (TIER 3.3)}
\label{subsec:quant-refactoring}
\index{silnik ilościowy!refaktoryzacja}

W~ramach audytu technicznego TIER~3.3 przeprowadzono refaktoryzację monolitycznego pliku \filepath{quantitative.ts} (629~LOC) na orkiestrator (490~LOC) delegujący obliczenia do 9~wyspecjalizowanych submodułów w~katalogu \filepath{src/lib/analysis/quant/}.

Główna funkcja \tsfunc{computeQuantitativeAnalysis()} pozostaje jedynym publicznym punktem wejścia --- jej sygnatura i~zwracany typ \tstype{QuantitativeAnalysis} nie uległy zmianie. Wewnętrznie deleguje ona obliczenia do submodułów, co poprawia czytelność, testowalność i~możliwość niezależnego rozwoju poszczególnych algorytmów.

\begin{table}[H]
\centering
\caption{Submoduły katalogu \filepath{src/lib/analysis/quant/}}
\label{tab:quant-submodules}
\begin{tabularx}{\textwidth}{L{2.8cm}L{5.5cm}X}
\toprule
\textbf{Moduł} & \textbf{Eksportowane funkcje} & \textbf{Opis} \\
\midrule
\filepath{helpers.ts} & \tsfunc{extractEmojis}, \tsfunc{countWords}, \tsfunc{tokenizeWords}, \tsfunc{median}, \tsfunc{percentile}, \tsfunc{topN} & Funkcje narzędziowe wielokrotnego użytku \\
\filepath{types.ts} & \tstype{PersonAccumulator}, \tsfunc{createPersonAccumulator} & Typ akumulatora per osoba i~jego factory \\
\filepath{bursts.ts} & \tsfunc{detectBursts} & Detekcja serii wiadomości (burst activity) \\
\filepath{trends.ts} & \tsfunc{computeTrends} & Obliczanie trendów miesięcznych \\
\filepath{reciprocity.ts} & \tsfunc{computeReciprocityIndex} & Indeks wzajemności komunikacji \\
\filepath{sentiment.ts} & \tsfunc{computeSentimentScore} & Wynik sentymentu wiadomości \\
\filepath{conflicts.ts} & \tsfunc{detectConflicts} & Detekcja konfliktów w~konwersacji \\
\filepath{intimacy.ts} & \tsfunc{computeIntimacyProgression} & Progresja intymności w~czasie \\
\filepath{index.ts} & (re-eksporty) & Barrel export --- agreguje eksporty wszystkich submodułów \\
\bottomrule
\end{tabularx}
\end{table}

\begin{infobox}[title=Korzyści refaktoryzacji]
\begin{itemize}
  \item \textbf{Testowalność} --- każdy submoduł może być testowany w~izolacji (patrz TIER~3.1: Vitest test suite)
  \item \textbf{Czytelność} --- orkiestrator (\filepath{quantitative.ts}) zawiera wyłącznie logikę kompozycji, a~nie szczegóły implementacyjne poszczególnych algorytmów
  \item \textbf{Rozszerzalność} --- dodanie nowej metryki wymaga utworzenia nowego submodułu i~jednolinijkowego importu w~orkiestratorze
  \item \textbf{Kompatybilność} --- publiczny interfejs (\tsfunc{computeQuantitativeAnalysis()} $\to$ \tstype{QuantitativeAnalysis}) pozostał niezmieniony --- żaden konsument nie wymagał modyfikacji
\end{itemize}
\end{infobox}


% ============================================================
\section{Kompletny przepływ danych}
\label{sec:data-flow-complete}

Na zakończenie rozdziału przedstawiamy kompletny schemat przepływu danych przez silnik analizy ilościowej --- od wejścia \tstype{ParsedConversation} do wyjścia \tstype{QuantitativeAnalysis}.

\begin{figure}[H]
\centering
\begin{tikzpicture}[node distance=0.6cm, font=\small]
  % Input
  \node[startstop, minimum width=4cm] (input) {ParsedConversation};

  % Main function
  \node[pipeline active, below=0.8cm of input, minimum width=5cm] (main) {computeQuantitativeAnalysis()};

  % Phase 2 outputs
  \node[podbox blue, below left=1cm and -0.5cm of main, minimum width=2.8cm] (pm) {PersonMetrics};
  \node[podbox blue, below=1cm of main, minimum width=2.8cm] (tm) {TimingMetrics};
  \node[podbox blue, below right=1cm and -0.5cm of main, minimum width=2.8cm] (em) {EngagementMetrics};

  % Phase 3 outputs
  \node[podbox purple, below left=0.8cm and -0.5cm of pm, minimum width=2.2cm] (pat) {PatternMetrics};
  \node[podbox purple, below=0.8cm of tm, minimum width=2.2cm] (heat) {HeatmapData};
  \node[podbox purple, below right=0.8cm and -0.5cm of em, minimum width=2.2cm] (trend) {TrendData};

  % Sub-modules
  \node[podbox green, below left=0.8cm and 0.3cm of heat, minimum width=2.2cm] (recip) {ReciprocityIndex};
  \node[podbox green, below=0.8cm of heat, minimum width=2.2cm] (viral) {ViralScores};
  \node[podbox green, below right=0.8cm and 0.3cm of heat, minimum width=2.2cm] (badge) {Badges};

  \node[podbox amber, below left=0.5cm and -0.2cm of viral, minimum width=2.2cm] (best) {BestTimeToText};
  \node[podbox amber, below right=0.5cm and -0.2cm of viral, minimum width=2.2cm] (catch) {Catchphrases};
  \node[podbox amber, below=1.2cm of viral, minimum width=2.2cm] (net) {NetworkMetrics};

  % Output
  \node[startstop, below=1cm of net, minimum width=4cm] (output) {QuantitativeAnalysis};

  % Arrows
  \draw[dataarrow] (input) -- (main);
  \draw[podarrow] (main) -- (pm);
  \draw[podarrow] (main) -- (tm);
  \draw[podarrow] (main) -- (em);
  \draw[podarrow] (pm) -- (pat);
  \draw[podarrow] (tm) -- (heat);
  \draw[podarrow] (em) -- (trend);
  \draw[podarrow] (heat) -- (recip);
  \draw[podarrow] (heat) -- (viral);
  \draw[podarrow] (heat) -- (badge);
  \draw[podarrow] (viral) -- (best);
  \draw[podarrow] (viral) -- (catch);
  \draw[podarrow] (viral) -- (net);
  \draw[dataarrow] (net) -- (output);
\end{tikzpicture}
\caption{Kompletny schemat przepływu danych w~silniku analizy ilościowej. Kolory: \textcolor{PodBlue}{niebieski} = metryki fazy 2, \textcolor{PodPurple}{fioletowy} = metryki fazy 3, \textcolor{PodSuccess}{zielony} = metryki kompozytowe, \textcolor{PodWarning}{pomarańczowy} = moduły specjalistyczne.}
\label{fig:complete-data-flow}
\end{figure}

Wynikowy obiekt \tstype{QuantitativeAnalysis} zawiera 12~grup metryk:

\begin{lstlisting}[style=podcode, caption={Kompletny interfejs QuantitativeAnalysis}]
export interface QuantitativeAnalysis {
  perPerson: Record<string, PersonMetrics>;
  timing: TimingMetrics;
  engagement: EngagementMetrics;
  patterns: PatternMetrics;
  heatmap: HeatmapData;
  trends: TrendData;
  viralScores?: ViralScores;
  badges?: Badge[];
  bestTimeToText?: BestTimeToText;
  catchphrases?: CatchphraseResult;
  networkMetrics?: NetworkMetrics;
  reciprocityIndex?: ReciprocityIndex;
  // Faza 24-27: Deep Analytics + Validation
  sentimentScore?: SentimentScore;
  conflicts?: ConflictEvent[];
  intimacyProgression?: IntimacyPoint[];
  pursuitWithdrawal?: PursuitWithdrawalResult;
  responseTimeDistribution?: ResponseTimeDistribution;
  yearMilestones?: YearMilestones;
  rankingPercentiles?: RankingPercentiles;
  lsm?: LSMResult;
  pronounAnalysis?: PronounAnalysisResult;
  // Faza 28: Nowe moduły psychologiczne
  chronotypeCompatibility?: ChronotypeCompatibility;
  emotionalGranularity?: EmotionalGranularityResult;
  bidResponse?: BidResponseResult;
  shiftSupport?: ShiftSupportResult;
}
\end{lstlisting}

Pola oznaczone \texttt{?} (opcjonalne) są obecne po fazie 3 przetwarzania końcowego. \texttt{networkMetrics} jest dostępny wyłącznie dla czatów grupowych. Pola od \tstype{pursuitWithdrawal} wzwyż wprowadzono w~Fazach 24--28.

% ============================================================
\section{Wykrywanie wzorca pościg--wycofanie}
\label{sec:pursuit-withdrawal}
% ============================================================

\textbf{Plik:} \filepath{src/lib/analysis/quant/pursuit-withdrawal.ts}\\
\textbf{Podstawa naukowa:} Christensen \& Heavey (1990) --- wzorzec Demand--Withdraw.

Moduł wykrywa naprzemienne serie wiadomości, w~których jedna osoba wysyła sekwencję co~najmniej 4~kolejnych wiadomości (faza \emph{pościgu}), po której następuje przerwa milczenia trwająca co~najmniej 4~godziny (faza \emph{wycofania}). Progi podwyższono w~Fazie~26 z~wartości 3~wiadomości / 2~godziny, aby ograniczyć liczbę fałszywych alarmów.

\begin{lstlisting}[style=podcode, caption={Interfejs modułu pursuit-withdrawal}]
interface PursuitWithdrawalCycle {
  pursuer: string;           // nadawca inicjujacy poscius
  withdrawer: string;        // osoba milczaca
  pursuerMessages: number;   // liczba wiadomosci w serii
  withdrawerMessages: number;
  gapMs: number;             // czas milczenia w ms
}

interface PursuitWithdrawalResult {
  cycles: PursuitWithdrawalCycle[];
  dominantPursuer: string | null;  // osoba czesciej inicjujaca
  severity: 'none' | 'mild' | 'moderate' | 'significant';
}
\end{lstlisting}

Wynik \tstype{severity} jest wyliczany na~podstawie liczby wykrytych cykli w~stosunku do~łącznej liczby wiadomości w~konwersacji. Komponent UI: \filepath{src/components/analysis/PursuitWithdrawalCard.tsx}.

% ============================================================
\section{Dopasowanie stylu językowego (LSM)}
\label{sec:lsm}
% ============================================================

\textbf{Plik:} \filepath{src/lib/analysis/quant/lsm.ts}\\
\textbf{Podstawa naukowa:} Ireland \& Pennebaker (2010), \emph{Journal of Personality and Social Psychology}.

Language Style Matching (LSM) mierzy stopień, w~jakim dwie osoby nieświadomie dostosowują swój styl językowy do~siebie nawzajem. Moduł analizuje 9~kategorii \emph{function words} (słów funkcyjnych) w~językach polskim i~angielskim:

\begin{itemize}
  \item artykuły (\emph{the, a, an} / \emph{ten, ta, to, \ldots})
  \item zaimki osobowe (\emph{I, you, he} / \emph{ja, ty, on, \ldots})
  \item zaimki względne (\emph{who, which, that} / \emph{który, która, \ldots})
  \item negacje (\emph{no, not, never} / \emph{nie, nigdy, bez, \ldots})
  \item przyimki (\emph{in, on, at} / \emph{w, na, przy, \ldots})
  \item spójniki (\emph{and, but, or} / \emph{i, ale, lub, \ldots})
  \item kwantyfikatory (\emph{all, few, some} / \emph{wszystko, kilka, trochę, \ldots})
  \item przysłówki (\emph{very, quite} / \emph{bardzo, dość, \ldots})
  \item pomocnicze (\emph{is, was, have} / \emph{jest, był, ma, \ldots})
\end{itemize}

Wynik LSM na~kategorię: $\text{LSM}_k = 1 - \frac{|r_A - r_B|}{r_A + r_B + 0{,}0001}$, gdzie $r_A$, $r_B$ to~odsetki użycia słów z~kategorii $k$ przez osoby A i~B. Ogólny wynik LSM to~ważona średnia 9~kategorii.

Dodatkowa metryka \emph{asymetrii adaptacji} wskazuje, która osoba bardziej dostosowuje swój styl do~rozmówcy --- efekt ,,kameleona komunikacyjnego''.

\begin{table}[H]
\centering
\caption{Interpretacja wyniku LSM}
\begin{tabularx}{\textwidth}{lX}
\toprule
\textbf{Wynik LSM} & \textbf{Interpretacja} \\
\midrule
$\geq 0{,}90$ & Bardzo wysokie dopasowanie stylistyczne \\
$\geq 0{,}80$ & Wysokie dopasowanie \\
$\geq 0{,}70$ & Umiarkowane dopasowanie \\
$< 0{,}70$ & Niskie dopasowanie (wyraźna rozbieżność stylu) \\
\bottomrule
\end{tabularx}
\label{tab:lsm-interpretation}
\end{table}

Komponent UI: \filepath{src/components/analysis/LSMCard.tsx}.

% ============================================================
\section{Analiza zaimków (Pronoun Analysis)}
\label{sec:pronouns}
% ============================================================

\textbf{Plik:} \filepath{src/lib/analysis/quant/pronouns.ts}\\
\textbf{Podstawa naukowa:} Pennebaker (2011), \emph{The Secret Life of Pronouns}.

Moduł zlicza użycie zaimków pierwszoosobowych (\emph{ja}/\emph{I}), inkluzywnych (\emph{my}/\emph{we}) i~drugoosobowych (\emph{ty}/\emph{you}) z~uwzględnieniem pełnej polskiej deklinacji:
\begin{itemize}
  \item \textbf{ja:} \emph{ja, mnie, mi, mną, mój, moja, moje, moi, moją}
  \item \textbf{my:} \emph{my, nas, nam, nami, nasz, nasza, nasze, nasi}
  \item \textbf{ty:} \emph{ty, cię, ci, tobie, twój, twoja, twoje, twoi}
\end{itemize}

Dla każdej osoby obliczane są odsetki: \texttt{iRate}, \texttt{weRate}, \texttt{youRate} (jako \% wszystkich słów). Na~ich podstawie wyznaczana jest orientacja relacyjna:

\begin{description}
  \item[\texttt{self-focused}] dominuje \emph{ja} --- wysoka uwaga na~siebie
  \item[\texttt{other-focused}] dominuje \emph{ty} --- wysoka uwaga na~partnera
  \item[\texttt{collective}] dominuje \emph{my} --- silne poczucie jedności pary
  \item[\texttt{balanced}] żaden zaimek nie dominuje
\end{description}

Komponent UI: \filepath{src/components/analysis/PronounCard.tsx}.

% ============================================================
\section{Rozkład czasu odpowiedzi}
\label{sec:response-time-dist}
% ============================================================

\textbf{Plik:} \filepath{src/lib/analysis/quant/response-time-distribution.ts}

Moduł buduje histogram czasów odpowiedzi per osoba w~6~przedziałach:

\begin{table}[H]
\centering
\caption{Przedziały histogramu czasu odpowiedzi}
\begin{tabularx}{\textwidth}{lX}
\toprule
\textbf{Bucket} & \textbf{Zakres} \\
\midrule
Błyskawiczna & $< 1$~minuta \\
Szybka & $1$--$5$ minut \\
Normalna & $5$--$30$ minut \\
Wolna & $30$ minut -- $2$ godziny \\
Bardzo wolna & $2$--$8$ godzin \\
Ghosting & $> 8$ godzin \\
\bottomrule
\end{tabularx}
\label{tab:response-buckets}
\end{table}

Oprócz histogramu moduł oblicza medianę oraz percentyle 75. i~90. dla każdej osoby. Dane są wizualizowane przez komponent \filepath{src/components/analysis/ResponseTimeHistogram.tsx} jako pogrupowany wykres słupkowy.

% ============================================================
\section{Analiza chronotypu (Faza~28)}
\label{sec:chronotype}
% ============================================================

\textbf{Plik:} \filepath{src/lib/analysis/quant/chronotype.ts}\\
\textbf{Podstawa naukowa:} Aledavood et~al.\ 2018 ($N=400$); Randler et~al.\ 2017 (Chronobiology International, 34(10), 1407--1416).

Moduł wyznacza chronotyp każdej osoby (ranny ptaszek / pośredni / nocna sowa) na~podstawie godzinowego rozkładu wysyłanych wiadomości. Punkt szczytowy (\emph{peak hour}) obliczany jest metodą okrągłej średniej (\emph{angular mean}, opartej na~$\text{atan}2$), która poprawnie obsługuje przejście przez północ.

\begin{lstlisting}[style=podcode, caption={Kategorie chronotypu}]
type ChronotypeCategory = 'early_bird' | 'intermediate' | 'night_owl';
// early_bird:   peakHour < 10  (emoji: 🌅)
// night_owl:    peakHour >= 20 (emoji: 🦉)
// intermediate: pozostale godziny
\end{lstlisting}

Wynik zgodności pary (\texttt{matchScore}) jest obliczany na~podstawie różnicy kołowej między punktami szczytowymi obojga rozmówców:

\begin{table}[H]
\centering
\caption{Przelicznik delta-godzin na matchScore}
\begin{tabularx}{\textwidth}{lX}
\toprule
\textbf{Różnica (deltaHours)} & \textbf{matchScore} \\
\midrule
$\leq 1$~h & 95\% \\
$\leq 2$~h & 80\% \\
$\leq 3$~h & 60\% \\
$\leq 4$~h & 40\% \\
$\leq 5$~h & 20\% \\
$> 5$~h & 10\% \\
\bottomrule
\end{tabularx}
\label{tab:chronotype-match}
\end{table}

Minimalne wymagania: 20~wiadomości per osoba. Komponent UI: \filepath{src/components/analysis/ChronotypePair.tsx}.

% ============================================================
\section{Granularność emocjonalna (Faza~28)}
\label{sec:emotional-granularity}
% ============================================================

\textbf{Plik:} \filepath{src/lib/analysis/quant/emotional-granularity.ts}\\
\textbf{Podstawa naukowa:} Vishnubhotla et~al.\ 2024 (EMNLP); Suvak et~al.\ 2011; Kashdan et~al.\ 2015.

Granularność emocjonalna (\emph{emotional granularity}) mierzy zdolność osoby do~precyzyjnego różnicowania stanów emocjonalnych. Moduł analizuje użycie słów z~12~kategorii emocjonalnych (łącznie $\sim$160+ słów w~językach polskim i~angielskim):

\begin{itemize}
  \item radość, smutek, złość, strach, zaskoczenie, wstręt
  \item oczekiwanie, zaufanie, frustracja, czułość, samotność, duma
\end{itemize}

Wynik granularności: $G = \frac{d}{12} \times 70 + \min(30,\, \frac{e}{w} \times 300)$, gdzie $d$ = liczba rozróżnionych kategorii, $e$ = liczba słów emocjonalnych, $w$ = łączna liczba słów.

Minimalne wymagania: 200~słów per osoba. Komponent UI: \filepath{src/components/analysis/EmotionalGranularityCard.tsx}.

% ============================================================
\section{Wskaźnik bid-response --- ,,zwrot ku partnerowi'' (Faza~28)}
\label{sec:bid-response}
% ============================================================

\textbf{Plik:} \filepath{src/lib/analysis/quant/bid-response.ts}\\
\textbf{Podstawa naukowa:} Gottman \& Silver (1999); Driver \& Gottman (2004). Benchmark: $\geq 86\%$ = pary stabilne (Gottman).

Bid-response ratio mierzy, jak często rozmówca odpowiada na~sygnały nawiązania kontaktu (\emph{bids for connection}) wysyłane przez partnera.

\textbf{Definicja bidu:} wiadomość zawierająca \texttt{?}, zaczynająca się od~frazy nawiązania (PL: ,,słuchaj'', ,,wiesz co''; EN: ,,listen'') lub zawierająca URL.

\textbf{Klasyfikacja odpowiedzi:}
\begin{description}
  \item[\emph{toward} (zwrot ku partnerowi)] zawiera pytanie, overlap słów $\geq 1$ z~bidem lub jest dłuższa niż 5~znaków i~nie jest odrzuceniem
  \item[\emph{away} (wycofanie)] brak odpowiedzi $> 4$~godziny lub zawiera tokeny odrzucenia (PL: ,,spoko'', ,,zapomnij''; EN: ,,whatever'')
\end{description}

\begin{table}[H]
\centering
\caption{Interpretacja bid-response rate}
\begin{tabularx}{\textwidth}{lX}
\toprule
\textbf{Wynik} & \textbf{Interpretacja} \\
\midrule
$\geq 80\%$ & Wysoka responsywność \\
$\geq 60\%$ & Umiarkowana responsywność \\
$< 60\%$ & Niska responsywność \\
$\geq 86\%$ & Poziom stabilnych par (benchmark Gottmana) \\
\bottomrule
\end{tabularx}
\label{tab:bid-response}
\end{table}

Minimalne wymagania: 10~bidów łącznie.

% ============================================================
\section{Narcyzm konwersacyjny --- wskaźnik CNI (Faza~28)}
\label{sec:cni}
% ============================================================

\textbf{Plik:} \filepath{src/lib/analysis/quant/shift-support.ts}\\
\textbf{Podstawa naukowa:} Derber (1979); Vangelisti, Knapp \& Daly (1990), \emph{Communication Monographs}.

Conversational Narcissism Index (CNI) operacjonalizuje teorię Derbera o~tendencji do~przejmowania centrum uwagi w~rozmowie przez nadmierne używanie \emph{shift-response} (odpowiedzi przekierowujących temat na~siebie) zamiast \emph{support-response} (odpowiedzi wspierających temat rozmówcy).

\textbf{Klasyfikacja odpowiedzi:}
\begin{description}
  \item[\emph{support-response}] zawiera \texttt{?}, zaczyna się od~pytajnika lub overlap słów z~poprzednią wiadomością $\geq 2$
  \item[\emph{shift-response}] zaczyna się od~tokenu self-referential (np.\ ,,ja'', ,,mnie'', ,,I'') i~overlap = 0
\end{description}

$$\text{CNI} = \frac{\text{shiftCount}}{\text{shiftCount} + \text{supportCount}} \times 100$$

Minimalne wymagania: 10~odpowiedzi per osoba. Interpretacja: CNI $\geq 70\%$ = wysoki narcyzm konwersacyjny; $\geq 45\%$ = umiarkowany; $< 45\%$ = niski. Komponent UI: \filepath{src/components/analysis/ConversationalNarcissismCard.tsx}.

\vfill
\begin{center}
\small\color{PodTextMuted}
\rule{4cm}{0.5pt}\\[4pt]
Koniec Rozdziału~\thechapter. Następny: Rozdział~6 --- Silnik Analizy AI.
\end{center}
