% ============================================================
% Rozdział 1: Marka PodTeksT
% ============================================================

\chapter{Marka PodTeksT}
\label{ch:marka}

\begin{center}
\Large\itshape\color{PodBlue}
,,Odkryj to, co kryje się między wierszami.''
\end{center}

\vspace{8pt}

Każda marka zaczyna się od nazwy. Każda nazwa kryje w~sobie obietnicę. W~przypadku \podtekst ta obietnica jest podwójna --- dosłowna i~metaforyczna, lingwistyczna i~emocjonalna. Ten rozdział przedstawia kompletną tożsamość marki: od genezy nazwy, przez misję i~wizję, aż po szczegółową specyfikację wizualną, która definiuje każdy piksel interfejsu.

% ============================================================
\section{Geneza nazwy}
\label{sec:geneza-nazwy}
% ============================================================

Nazwa \podtekst to wielowarstwowa gra słowna, zaprojektowana tak, by działała na trzech poziomach jednocześnie:

\subsection{Poziom pierwszy: ,,Pod tekstem''}

W~języku polskim wyrażenie ,,pod tekstem'' oznacza to, co kryje się pod powierzchnią komunikatu --- ukryte intencje, niewypowiedziane emocje, nieświadome wzorce. To dosłowne tłumaczenie angielskiego ,,subtext'', pojęcia głęboko zakorzenionego w~psychologii komunikacji, analizie literackiej i~terapii par.

Każda rozmowa na Messengerze czy WhatsAppie ma swój \emph{tekst} --- to, co zostało napisane. Ale ma też swój \emph{podtekst} --- to, co naprawdę zostało powiedziane. \podtekst obiecuje użytkownikowi dostęp do tego drugiego poziomu: algorytmiczną i~psychologiczną dekonstrukcję tego, co kryje się między wierszami.

\begin{infobox}[title=Etymologia]
Słowo ,,podtekst'' (ang.~\emph{subtext}) pochodzi od łacińskiego \emph{sub} (pod) i~\emph{textus} (tkanka, splot). W~literaturoznawstwie oznacza warstwę znaczeniową, która nie jest wyrażona wprost, lecz wynika z~kontekstu, tonu i~tego, czego \emph{nie} powiedziano. Idealnie oddaje istotę naszego produktu --- analizujemy nie tylko to, co użytkownik napisał, ale przede wszystkim to, co kryje się pod spodem.
\end{infobox}

\subsection{Poziom drugi: ,,eks-t'' --- rozmowy z~byłymi}

W~nazwie ,,Pod\emph{TeksT}'' kryje się subtelna fonetyczna aluzja: sylaba ,,tekst'' brzmi jak ,,eks-t'' (od angielskiego \emph{ex}). To nieprzypadkowe. Jednym z~najsilniejszych przypadków użycia \podtekst jest analiza rozmów z~byłymi partnerami --- zrozumienie, co poszło nie tak, kiedy relacja zaczęła się zmieniać, jakie wzorce komunikacyjne prowadziły do rozpadu.

Generacja Z~i~Millenialsów --- nasza grupa docelowa --- regularnie wraca do starych rozmów. Przeglądają archiwalne wiadomości z~eksami, analizują je mentalnie, zastanawiają się ,,co on/ona \emph{naprawdę} miał(a) na myśli''. \podtekst zamienia tę naturalną ludzką potrzebę w~usystematyzowaną, opartą na danych analizę.

To dlatego zapis nazwy jest celowy: \textbf{\textcolor{PodBlue}{Pod}\textcolor{PodPurple}{TeksT}} --- z~wielkim ,,T'' na początku i~na końcu, tworząc wizualną ramkę wokół słowa ,,eks''.

\subsection{Poziom trzeci: podtekst konwersacji}

Na najbardziej ogólnym poziomie \podtekst to narzędzie do odkrywania podtekstu \emph{każdej} konwersacji --- nie tylko romantycznej. Rozmowy z~przyjaciółmi, rodziną, współpracownikami, grupowe czaty --- wszędzie tam, gdzie ludzie komunikują się tekstowo, istnieje warstwa znaczeniowa niedostępna gołym okiem. Częstotliwość odpowiedzi, godziny aktywności, stosunek pytań do odpowiedzi, rozkład emoji, długość wiadomości --- to wszystko to dane, które opowiadają historię relacji.

\subsection{Tagline}

Oficjalny tagline marki brzmi:

\begin{center}
\Large\itshape\color{PodPurple}
,,Odkryj to, co kryje się między wierszami.''
\end{center}

Tagline działa na dwóch poziomach:
\begin{itemize}
  \item \textbf{Dosłownie:} analizujemy to, co jest ,,między wierszami'' (wiadomościami) --- przerwy, czasy odpowiedzi, wzorce aktywności, niedopowiedzenia.
  \item \textbf{Metaforycznie:} czytanie ,,między wierszami'' to idiom oznaczający rozumienie ukrytego znaczenia --- dokładnie to, co robi nasz silnik AI.
\end{itemize}

Drugorzędny tagline, używany w~kontekstach marketingowych:

\begin{center}
\large\itshape\color{PodBlue}
,,Twoje rozmowy mówią więcej niż myślisz.''
\end{center}

% ============================================================
\section{Misja i~wizja}
\label{sec:misja-wizja}
% ============================================================

\subsection{Misja}

\begin{featurebox}[title=Misja PodTeksT]
Demokratyzacja rozumienia relacji międzyludzkich przez dane. Dajemy każdemu człowiekowi narzędzia, które dotąd były dostępne wyłącznie terapeutom i~psychologom --- obiektywną, opartą na danych analizę wzorców komunikacyjnych, dynamiki relacji i~profili osobowości, wydobytą z~codziennych rozmów tekstowych.
\end{featurebox}

Misja \podtekst opiera się na trzech filarach:

\begin{enumerate}
  \item \textbf{Dostępność.} Sesja terapii par kosztuje 200--500~zł. Pełna analiza \podtekst w~planie Pro kosztuje mniej niż 10~zł miesięcznie. Nie zastępujemy terapii --- ale dajemy punkt wyjścia, który pozwala użytkownikowi zrozumieć, o~czym warto rozmawiać z~terapeutą.

  \item \textbf{Obiektywność.} Ludzie są stronniczy w~ocenie własnych relacji. Zapamiętujemy selektywnie, wyolbrzymiamy lub minimalizujemy konflikty. Dane nie kłamią: jeśli czas odpowiedzi partnera wzrósł o~340\% w~ciągu trzech miesięcy, to jest fakt, nie interpretacja.

  \item \textbf{Samoświadomość.} Najlepsza terapia to ta, której nie potrzebujesz, bo rozumiesz siebie. \podtekst to lustro --- pokazuje użytkownikowi jego własne wzorce komunikacyjne: czy jest osobą, która double-textuje, czy unika konfliktów, czy dominuje w~rozmowie, czy szuka ciągłego potwierdzenia.
\end{enumerate}

\subsection{Wizja}

\begin{featurebox}[title=Wizja PodTeksT]
Świat, w~którym każda znacząca rozmowa tekstowa jest przeanalizowana --- gdzie ludzie rozumieją dynamikę swoich relacji tak dobrze, jak rozumieją swoje finanse czy statystyki zdrowotne. Chcemy być ,,Stravą dla relacji'' --- aplikacją, którą ludzie otwierają po każdej ważnej rozmowie, by zobaczyć, jak ich relacja ,,biega''.
\end{featurebox}

Wizja długoterminowa zakłada:
\begin{itemize}
  \item Obsługę wszystkich głównych platform komunikacyjnych (Messenger, WhatsApp, Instagram, Telegram, Discord, iMessage, Teams).
  \item Ciągłe monitorowanie rozmów (za zgodą obu stron) z~alertami o~zmieniającej się dynamice.
  \item Porównania międzyrelacyjne --- jak Twoja komunikacja z~partnerem wypada w~porównaniu z~rozmową z~najlepszym przyjacielem.
  \item Wspólne analizy par --- oboje partnerzy analizują tę samą rozmowę i~widzą, jak ich perspektywy się różnią.
  \item API dla terapeutów i~coachów relacyjnych, integrujące dane z~\podtekst z~procesem terapeutycznym.
\end{itemize}

% ============================================================
\section{Grupa docelowa}
\label{sec:grupa-docelowa}
% ============================================================

\subsection{Profil demograficzny}

\begin{table}[H]
\centering
\caption{Profil demograficzny grupy docelowej \podtekst}
\label{tab:demografia}
\begin{tabularx}{\textwidth}{L{3.5cm}X}
\toprule
\textbf{Cecha} & \textbf{Opis} \\
\midrule
Wiek & 18--35 lat (rdzeń: 20--28) \\
Pokolenie & Gen Z i~Millennials \\
Język & Polski (rynek pierwotny), angielski (ekspansja) \\
Płeć & 60--65\% kobiety, 35--40\% mężczyźni \\
Lokalizacja & Polska (miasta 100k+), diaspora polska \\
Wykształcenie & Studenci, absolwenci uczelni wyższych \\
Status relacji & W~związku, po rozstaniu, komplikowane \\
\bottomrule
\end{tabularx}
\end{table}

\subsection{Profil psychograficzny}

Nasz idealny użytkownik to osoba, która:

\begin{itemize}
  \item \textbf{Jest ,,relationship-curious''} --- aktywnie myśli o~dynamice swoich relacji, czyta artykuły o~stylach przywiązania, ogląda TikToki o~psychologii związków, zna pojęcia takie jak ,,love bombing'', ,,gaslighting'', ,,anxious attachment''.

  \item \textbf{Jest native w~social media} --- żyje w~komunikatorach, prowadzi 5--15 aktywnych konwersacji jednocześnie, traktuje tekst jako podstawowy kanał komunikacji emocjonalnej.

  \item \textbf{Jest data-curious} --- lubi Spotify Wrapped, sprawdza Screen Time, ma nałóg śledzenia statystyk. Chce widzieć swoje życie przez dane.

  \item \textbf{Jest introspektywna} --- potrafi spojrzeć na siebie z~dystansu, jest gotowa na (czasem bolesne) obserwacje o~własnych wzorcach komunikacyjnych.

  \item \textbf{Dzieli się odkryciami} --- naturalnie udostępnia ciekawe wyniki na Instagramie, TikToku, w~grupowych czatach. Viralowy potencjał jest kluczowy.
\end{itemize}

\subsection{Scenariusze użycia}

\begin{description}
  \item[,,Po rozstaniu''] Użytkowniczka wraca do rozmów z~byłym partnerem. Chce zrozumieć, kiedy relacja zaczęła się psuć, kto wycofywał się emocjonalnie pierwszy, jakie wzorce komunikacyjne prowadziły do konfliktów.

  \item[,,Sprawdzanie partnera''] Użytkownik w~aktywnym związku chce obiektywnie ocenić, czy relacja jest zdrowa. Czy partner odpowiada coraz wolniej? Czy rozmowy są jednostronne? Czy ktoś unika pewnych tematów?

  \item[,,Ciekawość''] Użytkowniczka wrzuca rozmowę z~najlepszą przyjaciółką ,,dla beki'', by zobaczyć, jakie odznaki dostaną i~jaki jest ich ,,Relationship Health Score''. Wynik trafia na Instagram Stories.

  \item[,,Grupowy czat''] Grupa znajomych analizuje wspólny czat grupowy, by zobaczyć, kto pisze najwięcej, kto jest ,,duchem'' czatu, kto dostaje ,,nagrodę'' za najdłuższe monologi.

  \item[,,Autoanaliza''] Użytkownik analizuje kilka swoich rozmów z~różnymi osobami, by zrozumieć własne wzorce --- czy zawsze on inicjuje, czy jest pasywno-agresywny, czy stosuje ,,double texting'' nawet w~rozmowach z~rodziną.
\end{description}

% ============================================================
\section{Propozycja wartości}
\label{sec:propozycja-wartosci}
% ============================================================

\begin{center}
\LARGE\bfseries\color{PodBlue}
,,Zobacz swoje relacje przez dane.''
\end{center}

\vspace{6pt}

Propozycja wartości \podtekst opiera się na połączeniu dwóch światów, które dotąd istniały oddzielnie:

\begin{table}[H]
\centering
\caption{Dwa filary propozycji wartości \podtekst}
\label{tab:value-prop}
\begin{tabularx}{\textwidth}{C{2cm}XC{2cm}X}
\toprule
\multicolumn{2}{c}{\textbf{\personA{Analiza ilościowa}}} &
\multicolumn{2}{c}{\textbf{\personB{Analiza jakościowa AI}}} \\
\midrule
\personA{28+} & metryk obliczanych bez AI &
\personB{5} & przejść analizy AI (Gemini) \\
\personA{Gratis} & zerowy koszt API &
\personB{Głęboka} & analiza psychologiczna \\
\personA{Sekundy} & natychmiastowe wyniki &
\personB{Minuty} & streaming w~czasie rzeczywistym \\
\personA{Fakty} & czyste liczby i~wzorce &
\personB{Wnioski} & interpretacja i~kontekst \\
\bottomrule
\end{tabularx}
\end{table}

\subsection{Co odróżnia PodTeksT od konkurencji}

\begin{enumerate}
  \item \textbf{Głębokość analizy.} Konkurenci (np.~,,Chat Stats for Messenger'') dają podstawowe statystyki: ile wiadomości, kto pisze więcej, top emoji. \podtekst idzie o~kilka wymiarów dalej --- profile osobowości Big~Five, style przywiązania, analiza dynamiki władzy, praca emocjonalna, wynik zdrowia relacji.

  \item \textbf{Analiza AI.} Żaden konkurent nie oferuje wieloprzejściowej analizy AI, która generuje profile psychologiczne, identyfikuje punkty zwrotne relacji i~tworzy spersonalizowane porady.

  \item \textbf{Viralowość.} Tryb Roast, odznaki, karty do udostępniania, tryb Story --- to mechaniki zaprojektowane specjalnie pod viralowy potencjał w~social media. Analiza to produkt; udostępnienie wyniku to dystrybucja.

  \item \textbf{Estetyka.} Ciemna, redakcyjna, data-dense estetyka odróżnia nas od pastelowych, ,,przyjaznych'' aplikacji do wellness. \podtekst wygląda jak Bloomberg Terminal spotkał się ze Spotify Wrapped i~raportem klinicznym.

  \item \textbf{Polskie roots.} Pierwsza aplikacja tego typu na rynku polskim, natywnie obsługująca polskie znaki diakrytyczne, rozumiejąca polskie zwroty i~kulturowy kontekst komunikacji.
\end{enumerate}

% ============================================================
\section{Identyfikacja wizualna}
\label{sec:identyfikacja-wizualna}
% ============================================================

\subsection{Logo}

Logo \podtekst składa się z~dwóch elementów typograficznych, tworzących spójną, ale wizualnie rozdzieloną całość:

\begin{center}
\begin{tikzpicture}
  % Background card
  \fill[PodBg, rounded corners=8pt] (-5, -1.5) rectangle (5, 1.8);
  \draw[PodBorder, rounded corners=8pt] (-5, -1.5) rectangle (5, 1.8);

  % Logo text
  \node[anchor=center] at (0, 0.5) {
    {\fontsize{48}{56}\selectfont
      \textbf{\textcolor{PodBlue}{Pod}\textcolor{PodPurple}{TeksT}}}
  };

  % Tagline
  \node[anchor=center, font=\small\itshape, text=PodTextSecondary] at (0, -0.6) {
    odkryj to, co kryje się między wierszami
  };
\end{tikzpicture}
\end{center}

\begin{description}
  \item[,,Pod''] --- zapisywane w~kolorze \personA{PodBlue (\#3B82F6)}, grubą czcionką. Reprezentuje fundament, bazę, punkt wyjścia. Kojarzy się z~\personA{Osobą A} w~analizie --- nadawcą, inicjatorem.

  \item[,,TeksT''] --- zapisywane w~kolorze \personB{PodPurple (\#A855F7)}, grubą czcionką. Wielkie ,,T'' na obu końcach tworzy wizualną ramkę. Kojarzy się z~\personB{Osobą B} --- odbiorcą, odpowiadającym.
\end{description}

\subsection{Zasady użycia logo}

\begin{itemize}
  \item Logo zawsze zapisywane jest jako jedno słowo: \podtekst (nie ,,Pod TeksT'', nie ,,Podtekst'', nie ,,PODTEKST'').
  \item W~kontekstach monochromatycznych (np.~favicon) dozwolone jest użycie samego ,,PT'' w~kolorze PodBlue.
  \item Minimalna wielkość logo to 24px wysokości w~kontekstach cyfrowych.
  \item Logo nie jest otaczane ramką, cieniem ani żadnymi dodatkowymi ozdobnikami.
  \item Strefa ochronna wokół logo wynosi minimum 50\% wysokości litery ,,P'' z~każdej strony.
\end{itemize}

\subsection{Warianty logo}

\begin{table}[H]
\centering
\caption{Warianty logo \podtekst}
\label{tab:logo-warianty}
\begin{tabularx}{\textwidth}{L{3.5cm}X}
\toprule
\textbf{Wariant} & \textbf{Zastosowanie} \\
\midrule
Pełne kolorowe & Nagłówek strony, ekran powitalny, materiały brandowe \\
Monochromatyczne białe & Na ciemnych tłach, gdy kolory nie są dostępne \\
Favicon ,,PT'' & Zakładka przeglądarki, ikona aplikacji \\
Pełne + tagline & Strona główna, materiały marketingowe \\
Minimalny ,,P'' & Ikona mobilna, avatar w~social media \\
\bottomrule
\end{tabularx}
\end{table}

% ============================================================
\section{Paleta kolorów}
\label{sec:paleta-kolorow}
% ============================================================

Paleta kolorów \podtekst jest ciemna, precyzyjna i~funkcjonalna. Każdy kolor ma przypisaną rolę semantyczną --- nie jest dekoracyjny, lecz komunikacyjny.

\subsection{Kolory podstawowe}

Dwa kolory główne definiują dualną naturę produktu --- dwie osoby, dwa perspektywy, dwa strony każdej rozmowy:

\begin{figure}[H]
\centering
\begin{tikzpicture}
  % PodBlue swatch
  \fill[PodBlue, rounded corners=4pt] (0, 0) rectangle (5.5, 2);
  \node[anchor=north west, font=\bfseries\large, text=white] at (0.3, 1.8) {PodBlue};
  \node[anchor=south west, font=\ttfamily\small, text=white, opacity=0.85] at (0.3, 0.15) {\#3B82F6};
  \node[anchor=south east, font=\small\itshape, text=white, opacity=0.7] at (5.2, 0.15) {Osoba A, akcje, linki};

  % PodPurple swatch
  \fill[PodPurple, rounded corners=4pt] (6.5, 0) rectangle (12, 2);
  \node[anchor=north west, font=\bfseries\large, text=white] at (6.8, 1.8) {PodPurple};
  \node[anchor=south west, font=\ttfamily\small, text=white, opacity=0.85] at (6.8, 0.15) {\#A855F7};
  \node[anchor=south east, font=\small\itshape, text=white, opacity=0.7] at (11.7, 0.15) {Osoba B, sekcje, podtytuły};
\end{tikzpicture}
\caption{Kolory podstawowe --- dualność uczestników rozmowy}
\label{fig:kolory-podstawowe}
\end{figure}

\subsection{Tła i~powierzchnie}

Hierarchia ciemnych tło tworzy głębię i~strukturę interfejsu bez użycia cieni:

\begin{figure}[H]
\centering
\begin{tikzpicture}
  % Background
  \fill[PodBg, rounded corners=3pt] (0, 3.6) rectangle (12, 4.8);
  \node[anchor=west, font=\bfseries\small, text=PodText] at (0.3, 4.2) {Background};
  \node[anchor=east, font=\ttfamily\small, text=PodTextSecondary] at (11.7, 4.2) {\#050505};

  % BgSecondary
  \fill[PodBgSecondary, rounded corners=3pt] (0, 2.4) rectangle (12, 3.4);
  \node[anchor=west, font=\bfseries\small, text=PodText] at (0.3, 2.9) {Background Secondary};
  \node[anchor=east, font=\ttfamily\small, text=PodTextSecondary] at (11.7, 2.9) {\#0A0A0A};

  % Card
  \fill[PodCard, rounded corners=3pt] (0, 1.2) rectangle (12, 2.2);
  \node[anchor=west, font=\bfseries\small, text=PodText] at (0.3, 1.7) {Card};
  \node[anchor=east, font=\ttfamily\small, text=PodTextSecondary] at (11.7, 1.7) {\#111111};

  % Border
  \fill[PodBorder, rounded corners=3pt] (0, 0) rectangle (12, 1.0);
  \node[anchor=west, font=\bfseries\small, text=PodText] at (0.3, 0.5) {Border};
  \node[anchor=east, font=\ttfamily\small, text=PodTextSecondary] at (11.7, 0.5) {\#1A1A1A};
\end{tikzpicture}
\caption{Hierarchia ciemnych tło}
\label{fig:tla}
\end{figure}

\subsection{Typografia kolorowa}

\begin{figure}[H]
\centering
\begin{tikzpicture}
  % Text Primary on dark bg
  \fill[PodBg, rounded corners=3pt] (0, 2.4) rectangle (12, 3.4);
  \node[anchor=west, font=\bfseries\small, text=PodText] at (0.3, 2.9) {Text Primary \#FAFAFA};
  \node[anchor=east, font=\small\itshape, text=PodTextSecondary] at (11.7, 2.9) {nagłówki, treść główna};

  % Text Secondary
  \fill[PodBg, rounded corners=3pt] (0, 1.2) rectangle (12, 2.2);
  \node[anchor=west, font=\bfseries\small, text=PodTextSecondary] at (0.3, 1.7) {Text Secondary \#888888};
  \node[anchor=east, font=\small\itshape, text=PodTextSecondary] at (11.7, 1.7) {opisy, etykiety, metadane};

  % Text Muted
  \fill[PodBg, rounded corners=3pt] (0, 0) rectangle (12, 1.0);
  \node[anchor=west, font=\bfseries\small, text=PodTextMuted] at (0.3, 0.5) {Text Muted \#555555};
  \node[anchor=east, font=\small\itshape, text=PodTextSecondary] at (11.7, 0.5) {drugorzędne info, dezaktywowane};
\end{tikzpicture}
\caption{Kolory tekstu na ciemnym tle}
\label{fig:tekst-kolory}
\end{figure}

\subsection{Kolory semantyczne}

Kolory semantyczne komunikują stan i~wartość --- nie wymagają odczytywania tekstu, by zrozumieć, czy wynik jest pozytywny, neutralny czy alarmujący:

\begin{figure}[H]
\centering
\begin{tikzpicture}
  % Success
  \fill[PodSuccess, rounded corners=4pt] (0, 0) rectangle (3.5, 1.5);
  \node[anchor=center, font=\bfseries, text=white] at (1.75, 1.0) {Success};
  \node[anchor=center, font=\ttfamily\small, text=white, opacity=0.85] at (1.75, 0.35) {\#10B981};

  % Warning
  \fill[PodWarning, rounded corners=4pt] (4.25, 0) rectangle (7.75, 1.5);
  \node[anchor=center, font=\bfseries, text=white] at (6.0, 1.0) {Warning};
  \node[anchor=center, font=\ttfamily\small, text=white, opacity=0.85] at (6.0, 0.35) {\#F59E0B};

  % Danger
  \fill[PodDanger, rounded corners=4pt] (8.5, 0) rectangle (12, 1.5);
  \node[anchor=center, font=\bfseries, text=white] at (10.25, 1.0) {Danger};
  \node[anchor=center, font=\ttfamily\small, text=white, opacity=0.85] at (10.25, 0.35) {\#EF4444};
\end{tikzpicture}
\caption{Kolory semantyczne --- stan pozytywny, ostrzegawczy i~negatywny}
\label{fig:semantyczne}
\end{figure}

\begin{table}[H]
\centering
\caption{Zastosowanie kolorów semantycznych w~interfejsie}
\label{tab:semantyczne-uzycie}
\begin{tabularx}{\textwidth}{L{2.5cm}L{2.5cm}X}
\toprule
\textbf{Kolor} & \textbf{Hex} & \textbf{Zastosowanie} \\
\midrule
\score{Success} & \texttt{\#10B981} & Wysokie wyniki CPS (80--100), zdrowe wzorce, pozytywne trendy, wskaźnik ,,w~normie'' \\
\warn{Warning} & \texttt{\#F59E0B} & Średnie wyniki CPS (40--79), wzorce wymagające uwagi, trendy neutralne, potencjalne ryzyka \\
\danger{Danger} & \texttt{\#EF4444} & Niskie wyniki CPS (0--39), toksyczne wzorce, red flagi, alarmy o~manipulacji \\
\bottomrule
\end{tabularx}
\end{table}

\subsection{Kolor dodatkowy}

\begin{figure}[H]
\centering
\begin{tikzpicture}
  \fill[PodCyan, rounded corners=4pt] (0, 0) rectangle (5.5, 1.5);
  \node[anchor=center, font=\bfseries, text=white] at (2.75, 1.0) {Cyan};
  \node[anchor=center, font=\ttfamily\small, text=white, opacity=0.85] at (2.75, 0.35) {\#06B6D4};
\end{tikzpicture}
\caption{Kolor dodatkowy --- akcenty informacyjne i~wyróżnienia}
\label{fig:cyan}
\end{figure}

Kolor \textcolor{PodCyan}{Cyan (\#06B6D4)} jest używany sporadycznie jako trzeci kolor akcentowy --- w~tooltipach, linkach informacyjnych, ikonach pomocy i~elementach nawigacyjnych, które nie są akcjami pierwotnymi.

\subsection{Paleta wykresów}

Wykresy w~\podtekst używają spójnej palety pięciu kolorów, zoptymalizowanej pod czytelność na ciemnym tle i~rozróżnialność dla osób z~zaburzeniami widzenia kolorów:

\begin{figure}[H]
\centering
\begin{tikzpicture}
  % Chart color 1 - Blue
  \fill[PodBlue, rounded corners=3pt] (0, 0) rectangle (2.2, 1.2);
  \node[font=\ttfamily\footnotesize, text=white] at (1.1, 0.6) {\#3B82F6};

  % Chart color 2 - Purple
  \fill[PodPurple, rounded corners=3pt] (2.5, 0) rectangle (4.7, 1.2);
  \node[font=\ttfamily\footnotesize, text=white] at (3.6, 0.6) {\#A855F7};

  % Chart color 3 - Green
  \fill[PodSuccess, rounded corners=3pt] (5.0, 0) rectangle (7.2, 1.2);
  \node[font=\ttfamily\footnotesize, text=white] at (6.1, 0.6) {\#10B981};

  % Chart color 4 - Amber
  \fill[PodWarning, rounded corners=3pt] (7.5, 0) rectangle (9.7, 1.2);
  \node[font=\ttfamily\footnotesize, text=white] at (8.6, 0.6) {\#F59E0B};

  % Chart color 5 - Red
  \fill[PodDanger, rounded corners=3pt] (10.0, 0) rectangle (12.2, 1.2);
  \node[font=\ttfamily\footnotesize, text=white] at (11.1, 0.6) {\#EF4444};

  % Labels
  \node[font=\scriptsize, text=PodTextSecondary, anchor=north] at (1.1, -0.15) {Osoba A};
  \node[font=\scriptsize, text=PodTextSecondary, anchor=north] at (3.6, -0.15) {Osoba B};
  \node[font=\scriptsize, text=PodTextSecondary, anchor=north] at (6.1, -0.15) {Osoba C};
  \node[font=\scriptsize, text=PodTextSecondary, anchor=north] at (8.6, -0.15) {Osoba D};
  \node[font=\scriptsize, text=PodTextSecondary, anchor=north] at (11.1, -0.15) {Osoba E};
\end{tikzpicture}
\caption{Paleta pięciu kolorów do wykresów wieloosobowych}
\label{fig:paleta-wykresy}
\end{figure}

W~rozmowach dwuosobowych (najczęstszy przypadek) używane są wyłącznie dwa pierwsze kolory: \personA{PodBlue} i~\personB{PodPurple}. Pozostałe wchodzą do gry tylko w~czatach grupowych z~3+ uczestnikami.

\subsection{Kompletna tabela kolorów}

\begin{table}[H]
\centering
\caption{Kompletna paleta kolorów \podtekst z~wartościami CSS}
\label{tab:paleta-kompletna}
\small
\begin{tabularx}{\textwidth}{L{3.5cm}C{2cm}L{2.5cm}X}
\toprule
\textbf{Nazwa} & \textbf{Hex} & \textbf{Zmienna CSS} & \textbf{Rola} \\
\midrule
\personA{PodBlue} & \texttt{\#3B82F6} & \texttt{-{}-accent} & Kolor główny, Osoba A, CTA, linki \\
\personB{PodPurple} & \texttt{\#A855F7} & \texttt{-{}-chart-2} & Kolor drugorzędny, Osoba B, sekcje \\
Background & \texttt{\#050505} & \texttt{-{}-bg-primary} & Tło główne strony \\
Card & \texttt{\#111111} & \texttt{-{}-bg-card} & Tło kart i~paneli \\
Border & \texttt{\#1A1A1A} & \texttt{-{}-border} & Obramowania, separatory \\
Text Primary & \texttt{\#FAFAFA} & \texttt{-{}-text-primary} & Nagłówki, treść główna \\
Text Secondary & \texttt{\#888888} & \texttt{-{}-text-secondary} & Opisy, etykiety \\
Text Muted & \texttt{\#555555} & \texttt{-{}-text-muted} & Info drugorzędne \\
\score{Success} & \texttt{\#10B981} & \texttt{-{}-success} & Pozytywne wyniki, zielone światło \\
\warn{Warning} & \texttt{\#F59E0B} & \texttt{-{}-warning} & Ostrzeżenia, uwagi \\
\danger{Danger} & \texttt{\#EF4444} & \texttt{-{}-danger} & Alarmy, czerwone flagi \\
\textcolor{PodCyan}{Cyan} & \texttt{\#06B6D4} & \texttt{-{}-cyan} & Akcenty informacyjne \\
\bottomrule
\end{tabularx}
\end{table}

% ============================================================
\section{Typografia}
\label{sec:typografia}
% ============================================================

System typograficzny \podtekst łączy pięć rodzin fontów, z~których każda ma ściśle zdefiniowaną rolę. Celem jest stworzenie wizualnej hierarchii, która jednocześnie komunikuje precyzję danych i~ciepło narracji.

\subsection{Geist Sans --- treść główna}

\begin{featurebox}[title=Geist Sans]
\textbf{Rodzaj:} sans-serif, neo-groteskowy\\
\textbf{Autor:} Vercel\\
\textbf{Rola:} Treść główna, opisy, etykiety, nawigacja, formularze\\
\textbf{Rozmiary:} 14--16px (body), 12px (etykiety), 18--20px (wprowadzenia)\\
\textbf{Wagi:} Regular (400), Medium (500), SemiBold (600), Bold (700)
\end{featurebox}

Geist Sans to font zaprojektowany przez Vercel specjalnie do interfejsów webowych. Jego neo-groteskowy charakter łączy czytelność Helvetiki z~nowoczesnością Inter. Używamy go wszędzie tam, gdzie użytkownik czyta dłuższe teksty: opisy analiz, wyjaśnienia metryk, etykiety formularzy, komunikaty błędów.

Geist Sans jest fontem domyślnym --- jeśli element nie ma przypisanej innej rodziny fontów, używa Geist Sans.

\subsection{Geist Mono --- dane liczbowe}

\begin{featurebox}[title=Geist Mono]
\textbf{Rodzaj:} monospaced\\
\textbf{Autor:} Vercel\\
\textbf{Rola:} Wartości liczbowe, procenty, wyniki, identyfikatory, kod\\
\textbf{Rozmiary:} 12--14px (inline), 24--48px (KPI), 11px (tabele)\\
\textbf{Wagi:} Regular (400), Bold (700)
\end{featurebox}

Geist Mono jest kluczowy dla ,,data-dense'' estetyki \podtekst. Każda wartość liczbowa w~interfejsie --- wynik CPS, procenty, czasy odpowiedzi, liczniki wiadomości --- jest wyświetlana w~Geist Mono. Monospaced font zapewnia idealne wyrównanie kolumn w~tabelach i~daje wrażenie precyzji naukowej.

Cechy tabularnych cyfr (\emph{tabular figures}) w~Geist Mono sprawiają, że liczby mają stałą szerokość, co pozwala na animowane liczniki bez ,,skakania'' layoutu.

\subsection{JetBrains Mono --- nagłówki i~display}

\begin{featurebox}[title=JetBrains Mono]
\textbf{Rodzaj:} monospaced, display\\
\textbf{Autor:} JetBrains\\
\textbf{Rola:} Nagłówki sekcji, tytuły kart, nazwy metryk, elementy wyróżnione\\
\textbf{Rozmiary:} 18--32px (nagłówki), 14--16px (tytuły kart)\\
\textbf{Wagi:} Medium (500), Bold (700), ExtraBold (800)
\end{featurebox}

JetBrains Mono nadaje nagłówkom techniczny, terminalowy charakter, spójny z~estetyką ,,Bloomberg Terminal meets clinical report''. Jego wyraziste ligatury programistyczne (choć wyłączone w~nagłówkach) i~zwiększona wysokość x-height zapewniają doskonałą czytelność nawet w~małych rozmiarach.

Używany wyłącznie do nagłówków i~elementów display --- nigdy do treści ciągłej, co zapewnia wyraźny kontrast typograficzny z~Geist Sans.

\subsection{Syne --- tryb Story}

\begin{featurebox}[title=Syne]
\textbf{Rodzaj:} display, artystyczny\\
\textbf{Autor:} Bonjour Monde / Lucas Descroix\\
\textbf{Rola:} Nagłówki i~tytuły w~trybie Story\\
\textbf{Rozmiary:} 28--64px\\
\textbf{Wagi:} Bold (700), ExtraBold (800)
\end{featurebox}

Syne to font o~wyrazistym, artystycznym charakterze, używany wyłącznie w~trybie Story --- narracyjnej prezentacji wyników analizy, inspirowanej estetyką Spotify Wrapped. Jego geometryczne, nieco eksperymentalne kształty liter nadają trybu Story odrębną tożsamość wizualną w~ramach ekosystemu \podtekst.

\subsection{Space Grotesk --- treść Story}

\begin{featurebox}[title=Space Grotesk]
\textbf{Rodzaj:} sans-serif, geometryczny\\
\textbf{Autor:} Florian Karsten\\
\textbf{Rola:} Treść ciągła w~trybie Story\\
\textbf{Rozmiary:} 16--20px\\
\textbf{Wagi:} Regular (400), Medium (500), Bold (700)
\end{featurebox}

Space Grotesk dopełnia Syne w~trybie Story --- tam, gdzie Syne jest ekspresyjny i~artystyczny (nagłówki), Space Grotesk jest czytelny i~nowoczesny (treść). Jego geometryczne proporcje harmonizują z~Syne, tworząc spójną, ale zróżnicowaną parę typograficzną.

\subsection{Hierarchia typograficzna --- podsumowanie}

\begin{table}[H]
\centering
\caption{Hierarchia typograficzna \podtekst}
\label{tab:typografia-hierarchia}
\begin{tabularx}{\textwidth}{L{3cm}L{2.5cm}C{2cm}X}
\toprule
\textbf{Element} & \textbf{Font} & \textbf{Rozmiar} & \textbf{Przykład} \\
\midrule
Tytuł sekcji & JetBrains Mono & 24--32px & ,,Profil osobowości'' \\
Tytuł karty & JetBrains Mono & 16--18px & ,,Czas odpowiedzi'' \\
Treść główna & Geist Sans & 14--16px & Opis metryki \\
Etykiety & Geist Sans & 12px & ,,Średnia:'' \\
Wartość KPI & Geist Mono & 36--48px & ,,87.4'' \\
Wartość w~tabeli & Geist Mono & 13px & ,,2m 34s'' \\
Nagłówek Story & Syne & 48--64px & ,,Wasza historia'' \\
Treść Story & Space Grotesk & 18px & Narracja wyników \\
Kod źródłowy & Geist Mono & 13px & \texttt{parseMessages()} \\
\bottomrule
\end{tabularx}
\end{table}

% ============================================================
\section{Ton komunikacji}
\label{sec:ton-komunikacji}
% ============================================================

Ton komunikacji \podtekst jest starannie wyważony między czterema biegunami:

\subsection{Ciemny i~pewny siebie}

\podtekst nie przeprasza za swoje wnioski. Nie używa hedgingów typu ,,być może'', ,,to tylko sugestia'', ,,trudno powiedzieć''. Zamiast tego: ,,Osoba A wykazuje wzorzec przywiązania lękowego z~pewnością 78\%''. Dane mówią --- my je przekazujemy.

To nie arogancja, lecz precyzja. Każdy wniosek jest opatrzony poziomem pewności (0--100) i~cytatami z~rozmowy jako dowodami. Użytkownik wie, \emph{skąd} bierze się dany wniosek, i~może się z~nim nie zgodzić --- ale wniosek jest jasny i~jednoznaczny.

\subsection{Oparty na danych}

Każde twierdzenie w~interfejsie \podtekst jest podparte liczbą. Nie ,,dużo piszecie'' --- ale ,,16 432 wiadomości w~14 miesięcy, średnio 39 dziennie''. Nie ,,odpowiada szybko'' --- ale ,,mediana czasu odpowiedzi: 2m 14s, o~47\% szybciej niż średnia''. Dane przed interpretacją, zawsze.

\subsection{Lekko prowokujący}

\podtekst nie jest neutralny emocjonalnie. Tryb Roast jawnie ,,hejtuje'' wzorce komunikacyjne użytkownika. Odznaki mają nazwy jak ,,Duch Czatu'' (dla kogoś, kto prawie nie pisze) czy ,,Bombardier Miłosny'' (dla love-bombingu). Wynik ,,Ghost Forecast'' przewiduje prawdopodobieństwo ghostingu.

Ta prowokacja jest kontrolowana i~humorystyczna --- nigdy złośliwa, nigdy osobista. Cel: wzbudzić reakcję emocjonalną, która prowadzi do udostępnienia wyniku w~social media.

\subsection{Nie słodki, nie pastelowy}

\podtekst celowo unika estetyki wellness-app: brak pastelowych kolorów, brak okrągłych kształtów, brak emotikonów w~interfejsie (ironicznie --- analizujemy emoji, ale ich nie używamy w~UI), brak ,,przyjaznych'' komunikatów typu ,,Świetnie ci idzie!''.

\begin{infobox}[title=Manifesto tonu]
\textbf{Jesteśmy:} precyzyjni, pewni siebie, trochę złośliwi, zawsze oparci na danych.\\
\textbf{Nie jesteśmy:} miluscy, ogólnikowi, moralizujący, paternalistyczni.\\
\textbf{Brzmimy jak:} przyjaciel, który jest psychologiem i~mówi ci prawdę prosto w~twarz.\\
\textbf{Nie brzmimy jak:} chatbot wellness, coach motywacyjny, pasywno-agresywna apka randkowa.
\end{infobox}

\subsection{Estetyka referencyjna}

Najlepsze odniesienie do estetyki \podtekst to przecięcie trzech światów:

\begin{enumerate}
  \item \textbf{Bloomberg Terminal} --- gęstość danych, ciemne tło, monospaced font na wartościach, zero przestrzeni marnowanej na dekoracje. Informacja jest dekoracją.

  \item \textbf{Spotify Wrapped} --- narracyjność, personalizacja, viralowość, efekt ,,wow'' z~poznania własnych danych. Emocjonalne zaangażowanie w~statystyki.

  \item \textbf{Raport kliniczny} --- struktura, precyzja, brak emocji w~prezentacji (emocje są w~\emph{treści}, nie w~\emph{formie}), profesjonalizm, wiarygodność. Pewność diagnostyczna z~zastrzeżeniami.
\end{enumerate}

% ============================================================
\section{Elementy wizualne}
\label{sec:elementy-wizualne}
% ============================================================

\subsection{Tekstura ziarnista (grain overlay)}

Jednym z~najbardziej subtelnych, ale istotnych elementów wizualnych \podtekst jest nakładka z~teksturą ziarnistą (grain/noise overlay), aplikowana na tła stron i~kart.

\begin{description}
  \item[Format] SVG \texttt{<filter>} z~elementem \texttt{<feTurbulence>}, generujący proceduralny szum Perlina.
  \item[Parametry] \texttt{baseFrequency="0.65"}, \texttt{numOctaves="4"}, typ \texttt{fractalNoise}.
  \item[Opacity] 2--3\% --- ledwo wyczuwalny, ale nadający głębię ciemnym powierzchniom.
  \item[Zastosowanie] Tło główne strony, tło kart, tło modali, tło trybu Story.
  \item[Cel] Przełamanie ,,cyfrowej płaskości'' ciemnego interfejsu. Analogowe, filmowe odczucie, jakby dane wyświetlały się na ekranie kineskopowym, nie na LCD.
\end{description}

\begin{warningbox}[title=Uwaga wydajnościowa]
Filtr SVG \texttt{feTurbulence} jest kosztowny obliczeniowo. Na urządzeniach mobilnych z~niską mocą GPU stosujemy fallback: statyczny PNG z~szumem, powtarzany jako \texttt{background-image}. Rozmiar kafelka: 200\(\times\)200px, waga: <5KB.
\end{warningbox}

\subsection{Ciemne tła i~kontrast}

Interfejs \podtekst jest \emph{wyłącznie} ciemny --- nie oferujemy trybu jasnego. Ta decyzja jest celowa:

\begin{itemize}
  \item \textbf{Spójność marki.} Ciemna estetyka jest fundamentem tożsamości wizualnej. Tryb jasny wymagałby kompletnego redesignu każdego komponentu.
  \item \textbf{Skupienie na danych.} Ciemne tło sprawia, że kolorowe elementy (wykresy, wskaźniki, wartości) ,,wyskakują'' silniej. Dane są gwiazdami --- interfejs jest tłem.
  \item \textbf{Atmosfera.} Analizowanie rozmów z~eksem o~2 w~nocy wymaga ciemnego interfejsu. Pół żartem, pół serio --- nasz użytkownik dosłownie robi to w~ciemności.
  \item \textbf{Wyróżnienie na rynku.} W~świecie pastelowych wellness-appów ciemny interfejs natychmiast komunikuje: ,,to jest coś innego''.
\end{itemize}

\subsection{Layouty gęste w~dane}

Filozofia layoutu \podtekst to ,,data density'' --- maksimum informacji na jednostkę powierzchni ekranu, bez utraty czytelności.

\begin{itemize}
  \item \textbf{Karty KPI:} 4--6 kart w~jednym rzędzie, każda z~wartością, etykietą, sparkline i~trendem. Bez paddingu większego niż 12px.
  \item \textbf{Wykresy:} Pełna szerokość kontenera, legendy inline (nie zewnętrzne), tooltips on hover z~dokładnymi wartościami.
  \item \textbf{Tabele:} Kompaktowe, z~kolorowaniem wierszy kontekstowym, bez zebra-striping (zbyt ,,corporate'').
  \item \textbf{Sekcje:} Separowane cienkimi liniami (\texttt{1px solid \#1A1A1A}), nie białą przestrzenią. Każdy piksel pracuje.
\end{itemize}

\subsection{Animacje i~mikro-interakcje}

\begin{table}[H]
\centering
\caption{System animacji \podtekst}
\label{tab:animacje}
\begin{tabularx}{\textwidth}{L{3.5cm}L{2.5cm}X}
\toprule
\textbf{Element} & \textbf{Typ} & \textbf{Opis} \\
\midrule
Przejście strony & Fade + slide up & 300ms ease-out, opacity 0$\to$1 + translateY 20px$\to$0 \\
Wejście kart & Staggered reveal & IntersectionObserver, każda karta z~opóźnieniem 50ms \\
Liczniki KPI & Count up & Animacja od 0 do wartości, 1200ms ease-out \\
Rysowanie wykresu & Line draw & SVG \texttt{stroke-dashoffset} animowany od długości do 0 \\
Hover na karcie & Scale + border & \texttt{scale(1.02)}, border-color shift do \texttt{\#2A2A2A} \\
Loading & Skeleton pulse & Pulsujące szare prostokąty, 1500ms ease-in-out infinite \\
Progres analizy AI & Step progress & 5 kroków z~animowanym paskiem, streaming tekstu \\
Odznaki & Pop-in & \texttt{scale(0)$\to$1} z~bounce easing, staggered 100ms \\
\bottomrule
\end{tabularx}
\end{table}

Wszystkie animacje używają biblioteki Framer Motion i~szanują preferencję \texttt{prefers-reduced-motion} --- gdy użytkownik ma wyłączone animacje w~systemie, wszystkie przejścia są natychmiastowe.

\subsection{Ikonografia}

\podtekst nie używa dedykowanego zestawu ikon. Zamiast tego wykorzystuje:

\begin{itemize}
  \item \textbf{Lucide Icons} --- open-source zestaw ikon SVG, spójny wizualnie, 24$\times$24px domyślnie. Używany w~nawigacji, przyciskach, kartach metryk.
  \item \textbf{Emoji natywne} --- w~sekcjach analizy emoji (np.~,,Top Reactions''), wyświetlamy prawdziwe emoji systemowe, nie ikony zastępcze.
  \item \textbf{Brak ilustracji} --- \podtekst nie używa ilustracji, maskotek ani grafik dekoracyjnych. Dane i~wykresy \emph{są} wizualizacją. Jedynym wyjątkiem jest animacja 3D Spline na stronie głównej.
\end{itemize}
