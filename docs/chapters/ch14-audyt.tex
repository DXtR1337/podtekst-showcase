% ============================================================
% Rozdział 14: Audyt wieloagentowy
% ============================================================

\chapter{Audyt wieloagentowy}
\label{ch:audyt}

\begin{center}
\Large\itshape\color{PodBlue}
,,Mierz dwa razy, tnij raz. Albo lepiej --- mierz dziewięć razy, bo każdy agent widzi co innego.''
\end{center}

\vspace{8pt}

Niniejszy rozdział dokumentuje wyniki kompleksowego audytu przeprowadzonego w~	extbf{dziewięciu równoległych osiach}: 	extbf{audyt UX/designu} strony analizy, 	extbf{audyt monetyzacji} z~planem cenowym, 	extbf{audyt optymalizacji} wydajności, 	extbf{unit economics} kosztów AI, 	extbf{audyt bezpieczeństwa}, 	extbf{audyt mobile UX}, 	extbf{audyt onboardingu}, 	extbf{audyt jakości kodu} oraz 	extbf{audyt SEO}. Każda oś została zrealizowana przez niezależnego agenta AI (Claude Opus~4.6), który przeanalizował kod źródłowy, architekturę komponentów i~wzorce użytkowania.

\begin{warningbox}[title={\textbf{Kontekst audytu}}]
Audyt przeprowadzono na stanie kodu z~lutego 2026 (po Fazie~22). Analiza opiera się na faktycznej zawartości plików źródłowych, nie na założeniach. Kluczowe statystyki wejściowe:
\begin{itemize}
  \item \filepath{src/app/(dashboard)/analysis/[id]/page.tsx} --- \textbf{1322 linii} kodu, \textbf{50+ komponentów} importowanych
  \item \textbf{37} instancji \tstype{IntersectionObserver} (via Framer Motion \tsfunc{whileInView})
  \item \textbf{9} handlerów \tsfunc{useCallback} w~jednym pliku
  \item \filepath{src/lib/rate-limit.ts} --- rate limiting \danger{wyłączony} (zwraca \tstype{\{allowed: true\}} zawsze)
  \item Zero infrastruktury monetyzacji --- brak auth, brak płatności, brak tierów
\end{itemize}
\end{warningbox}


% ============================================================
\section{Audyt UX/designu strony analizy}
\label{sec:audyt-design}
% ============================================================

\subsection{Diagnoza: monolit na jednej stronie}

Strona wyników analizy (\filepath{src/app/(dashboard)/analysis/[id]/page.tsx}) renderuje \textbf{wszystkie} dane na jednej stronie --- od metryk ilościowych, przez wykresy, viral scores, odznaki, share cards, aż po pełną analizę AI i~funkcje rozrywkowe. Skutkuje to:

\begin{table}[H]
\centering
\caption{Metryki strony analizy --- stan obecny}
\label{tab:audyt-strona-metryki}
\begin{tabularx}{\textwidth}{L{5cm}C{3cm}X}
\toprule
\textbf{Metryka} & \textbf{Wartość} & \textbf{Problem} \\
\midrule
Wysokość strony (2-osobowa) & 4000--6000\,px & 15--20 ekranów mobilnych \\
Wysokość strony (5+ osób) & 5000--7000\,px & Dodatkowe sekcje serwera \\
Importowane komponenty & 50+ & Pojedynczy \tstype{page.tsx} \\
Instancje IntersectionObserver & 37 & Każde \tsfunc{whileInView} tworzy osobną instancję \\
Sekcje nawigacji (SectionNavigator) & 6--9 & Niewystarczające pokrycie treści \\
Odległość AI Analysis od foldu & $\sim$3000\,px & Najcenniejsza treść pogrzebana pod metrykami \\
\bottomrule
\end{tabularx}
\end{table}

\subsubsection{Mapa sekcji strony}

Aktualna kolejność renderowania w~\tstype{page.tsx}:

\begin{enumerate}[label=\textcolor{PodBlue}{\arabic*.}]
  \item \textbf{Hero Zone} --- \tstype{AnalysisHeader}, \tstype{ParticipantStrip}, linki Story/Wrapped
  \item \textbf{Kluczowe metryki} --- \tstype{KPICards}, \tstype{StatsGrid}
  \item \textbf{Longitudinal Delta} --- porównanie z~poprzednią analizą (opcjonalne)
  \item \textbf{Aktywność i~czas} --- \tstype{TimelineChart}, \tstype{EmojiReactions}, \tstype{HeatmapChart}, \tstype{ResponseTimeChart}
  \item \textbf{Wzorce komunikacji} --- \tstype{MessageLengthSection}, \tstype{WeekdayWeekendCard}, \tstype{BurstActivity}, \tstype{TopWordsCard}, \tstype{SentimentChart}, \tstype{IntimacyChart}, \tstype{ConflictTimeline}
  \item \textbf{Viral Scores} --- \tstype{ViralScoresSection}, \tstype{BestTimeToTextCard}, \tstype{CatchphraseCard}
  \item \textbf{Ghost Forecast} --- \tstype{GhostForecast}
  \item \textbf{Osiągnięcia} --- \tstype{BadgesGrid}
  \item \textbf{Delusion Quiz} --- wynik quizu (quiz jest gate screen)
  \item \textbf{Group Chat Awards} --- \tstype{GroupChatAwards} (grupowe)
  \item \textbf{Sieć interakcji} --- \tstype{NetworkGraph} (grupowe)
  \item \textbf{Udostępnij wyniki} --- \tstype{ShareCardGallery}, PDF export, captiony
  \item \textbf{Analiza AI} --- \tstype{AIAnalysisButton}, Roast, Attachment, Tone, Love Language, Turning Points, Personality, CPS, Subtext, Court, Dating Profile, Reply Simulator
\end{enumerate}

\begin{warningbox}[title={\textbf{Kluczowy problem}}]
Sekcja \textbf{Analiza AI} (pozycja 13) --- czyli najbardziej wartościowa treść z~perspektywy monetyzacji i~zaangażowania --- znajduje się na samym \emph{dole} strony, za 12 sekcjami danych ilościowych. Użytkownik musi przewinąć $\sim$3000\,px, zanim zobaczy wyniki AI.
\end{warningbox}

\subsection{Nawigacja: SectionNavigator}

Komponent \tstype{SectionNavigator} (\filepath{src/components/analysis/SectionNavigator.tsx}) zapewnia nawigację po sekcjach:

\begin{itemize}
  \item \textbf{Desktop:} sticky sidebar z~6--9 przyciskami sekcji
  \item \textbf{Mobile:} bottom pill bar z~horizontalnym scrollem
  \item Tworzy osobny \tstype{IntersectionObserver} per sekcję (6--9 instancji)
  \item Oddzielny scroll listener dla progress baru i~przycisku ,,wróć na górę''
\end{itemize}

Problem: przy 15+ sekcjach treści, 6--9 przycisków nawigacji jest niewystarczające. Użytkownik nie wie, gdzie jest na stronie, bo wiele sekcji nie ma swojego przycisku w~nawigatorze.

\subsection{Proponowane rozwiązanie: architektura tabowa}
\label{sec:architektura-tabowa}

Zamiast jednej długiej strony z~nawigacją sekcji, proponujemy \textbf{5 zakładek (tabów)} z~lazy-loadingiem:

\begin{table}[H]
\centering
\caption{Proponowana architektura tabowa strony analizy}
\label{tab:audyt-taby}
\begin{tabularx}{\textwidth}{C{0.8cm}L{2.2cm}X}
\toprule
\textbf{Nr} & \textbf{Tab} & \textbf{Zawartość} \\
\midrule
1 & \textbf{Przegląd} & AnalysisHeader, ParticipantStrip, KPICards, StatsGrid, HealthScore, LongitudinalDelta \\
2 & \textbf{Metryki} & Timeline, Heatmap, ResponseTime, Emoji, MessageLength, Weekday/Weekend, Burst, TopWords, Sentiment, Intimacy, Conflicts, Network \\
3 & \textbf{AI Insights} & AIAnalysisButton, Roast, AttachmentStyle, CommunicationStyle, ToneRadar, LoveLanguage, TurningPoints, PersonalityDeepDive, CPS \\
4 & \textbf{Rozrywka} & CourtTrial, DatingProfile, ReplySimulator, DelusionQuiz, GhostForecast, EnhancedRoast, StandUp \\
5 & \textbf{Udostępnij} & ShareCardGallery, ExportPDF, StandUpPDF, CaptionModal, PhotoUpload, ViralScores, Badges \\
\bottomrule
\end{tabularx}
\end{table}

\subsubsection{Korzyści architekturalne}

\begin{itemize}
  \item \textbf{Redukcja DOM:} z~50+ komponentów jednocześnie do $\sim$10 per tab (lazy-loading)
  \item \textbf{IntersectionObserver:} z~37 instancji do $\sim$8 (tylko aktywny tab)
  \item \textbf{AI Insights na pozycji~3} zamiast 13 --- użytkownik trafia do AI w~2 kliknięciach
  \item \textbf{URL hash sync:} \texttt{\#overview}, \texttt{\#metrics}, \texttt{\#ai}, \texttt{\#entertainment}, \texttt{\#share} --- deep-linkowanie
  \item \textbf{Code splitting:} \tsfunc{React.lazy()} per tab --- Entertainment tab (najcięższy) ładowany tylko gdy kliknięty
\end{itemize}

\subsubsection{Proponowana struktura plików}

\begin{infobox}[title={\textbf{Nowe pliki komponentów}}]
\begin{itemize}
  \item \filepath{src/components/analysis/AnalysisTabs.tsx} --- kontener tabów z~URL hash sync
  \item \filepath{src/components/analysis/tabs/OverviewTab.tsx} --- tab Przegląd
  \item \filepath{src/components/analysis/tabs/MetricsTab.tsx} --- tab Metryki
  \item \filepath{src/components/analysis/tabs/AIInsightsTab.tsx} --- tab AI Insights
  \item \filepath{src/components/analysis/tabs/EntertainmentTab.tsx} --- tab Rozrywka
  \item \filepath{src/components/analysis/tabs/ShareTab.tsx} --- tab Udostępnij
\end{itemize}
Efekt: \filepath{page.tsx} redukuje się z~1322~LOC do $\sim$200~LOC (state, data loading, tab routing).
\end{infobox}

\subsubsection{Server View (5+ uczestników)}

Dla grup z~5+ uczestnikami dodatkowy tab \textbf{,,Serwer''} zastępuje inline sekcje (ServerOverview, TeamRoles, CommunityMap, PersonNavigator, PersonProfile, ServerLeaderboard, PairwiseComparison). Układ tabów zmienia się z~5 na 6.

\subsubsection{Diagram architektury tabowej}

\begin{figure}[H]
\centering
\begin{tikzpicture}[
  tabbox/.style={draw=PodBlue!60, fill=PodBlue!8, rounded corners=4pt,
    minimum width=2.4cm, minimum height=1.0cm, align=center,
    font=\small\bfseries},
  activeTab/.style={tabbox, draw=PodBlue, fill=PodBlue!20, line width=1.2pt},
  contentBox/.style={draw=PodBorder, fill=PodCard!30, rounded corners=4pt,
    minimum width=13cm, minimum height=3cm, align=center},
  compBox/.style={draw=PodPurple!40, fill=PodPurple!5, rounded corners=2pt,
    minimum width=2.2cm, minimum height=0.6cm, align=center, font=\tiny},
  >=Stealth
]

% Tab bar
\node[tabbox] (t1) at (0,0) {Przegląd};
\node[tabbox] (t2) at (2.8,0) {Metryki};
\node[activeTab] (t3) at (5.6,0) {AI Insights};
\node[tabbox] (t4) at (8.4,0) {Rozrywka};
\node[tabbox] (t5) at (11.2,0) {Udostępnij};

% Active indicator
\draw[PodBlue, line width=2pt] ([yshift=-1pt]t3.south west) -- ([yshift=-1pt]t3.south east);

% Content area
\node[contentBox] (content) at (5.6,-2.5) {};

% Components inside active tab
\node[compBox] at (2.2,-1.6) {AttachmentStyle};
\node[compBox] at (4.6,-1.6) {ToneRadar};
\node[compBox] at (7.0,-1.6) {LoveLanguage};
\node[compBox] at (9.4,-1.6) {CPS};
\node[compBox] at (2.2,-2.6) {PersonalityDeepDive};
\node[compBox] at (4.6,-2.6) {TurningPoints};
\node[compBox] at (7.0,-2.6) {RelationshipBalance};
\node[compBox] at (9.4,-2.6) {Subtext};
\node[compBox] at (3.4,-3.4) {AIAnalysisButton};
\node[compBox] at (5.8,-3.4) {RoastSection};
\node[compBox] at (8.2,-3.4) {EnhancedRoast};

\end{tikzpicture}
\caption{Architektura tabowa --- przykładowy widok zakładki ,,AI Insights''}
\label{fig:audyt-tab-arch}
\end{figure}


% ============================================================
\section{Audyt monetyzacji}
\label{sec:audyt-monetyzacja}
% ============================================================

\subsection{Stan obecny: 100\% za darmo}

Na dzień audytu \podtekst nie posiada żadnej infrastruktury monetyzacji:

\begin{itemize}
  \item Brak systemu autentykacji (Supabase Auth --- planowany, niezaimplementowany)
  \item Brak integracji płatności (Stripe --- planowany, niezaimplementowany)
  \item Brak rozróżnienia tierów (Free/Pro/Unlimited)
  \item Brak limitów użycia (analizy/miesiąc, karty share, PDF export)
  \item Rate limiting \danger{wyłączony w~produkcji} --- każdy może wysyłać nieograniczoną liczbę requestów AI
\end{itemize}

\subsection{Proponowany model: Freemium 3-Tier}
\label{sec:model-freemium}

\begin{table}[H]
\centering
\caption{Model cenowy --- porównanie tierów}
\label{tab:audyt-tiers}
\begin{tabularx}{\textwidth}{L{4.5cm}C{2.8cm}C{2.8cm}C{2.8cm}}
\toprule
\textbf{Funkcja} & \textbf{Free (0~PLN)} & \textbf{Pro (29,99~PLN/mies)} & \textbf{Unlimited (49,99~PLN/mies)} \\
\midrule
Metryki ilościowe & Wszystkie & Wszystkie & Wszystkie \\
AI Pass 1 (Ton) & \score{Tak} & \score{Tak} & \score{Tak} \\
AI Pass 2--4 & \danger{Nie} & \score{Tak} & \score{Tak} \\
Roast (podstawowy) & \score{Tak} & \score{Tak} & \score{Tak} \\
Enhanced Roast & \danger{Nie} & \score{Tak} & \score{Tak} \\
Rozrywka (Court, Dating, Simulator) & \danger{Nie} & \score{Tak} & \score{Tak} \\
Share Cards & 5 podstawowych & Wszystkie & Wszystkie \\
Export PDF & \danger{Nie} & \score{Tak} & \score{Tak} \\
StandUp PDF & \danger{Nie} & \score{Tak} & \score{Tak} \\
CPS + Subtext & \danger{Nie} & \score{Tak} & \score{Tak} \\
Analizy / miesiąc & 3 & 15 & Bez limitu \\
Platformy & Wszystkie & Wszystkie & Wszystkie \\
\bottomrule
\end{tabularx}
\end{table}

\subsubsection{Logika Free Tier}

Free Tier zapewnia \textbf{pełne metryki ilościowe} (client-side, zero kosztów serwera) plus ograniczoną analizę AI:

\begin{itemize}
  \item \textbf{Pass 1 (Ton i~styl)} --- darmowy, daje smak AI analizy
  \item \textbf{Roast (podstawowy)} --- darmowy, wiralowy element zachęcający do udostępniania
  \item \textbf{5 kart share} --- wystarczające do wiralowego rozprzestrzeniania (z~watermarkiem ,,PodTeksT.app'')
  \item \textbf{3 analizy/miesiąc} --- wystarczające do testowania, niewystarczające do regularnego użycia
\end{itemize}

\subsubsection{Logika paywall}

Paywall aktywuje się w~\textbf{momencie największego zaangażowania}:

\begin{enumerate}
  \item Użytkownik uploaduje rozmowę --- \score{bezpłatne}
  \item Widzi pełne metryki ilościowe + KPI --- \score{bezpłatne}
  \item Uruchamia analizę AI --- Pass~1 + Roast \score{bezpłatne}
  \item Klika tab ,,AI Insights'' --- widzi wyniki Pass~1, ale Pass~2--4 za \textbf{blurred overlay z~CTA ,,Odblokuj pełną analizę''}
  \item Klika tab ,,Rozrywka'' --- każdy przycisk opakowany w~\textbf{PaywallGate}
  \item Klika tab ,,Udostępnij'' --- 5 kart dostępnych, reszta z~CTA ,,Odblokuj 20+ kart''
\end{enumerate}

\subsection{Implementacja techniczna}
\label{sec:monetyzacja-implementacja}

\subsubsection{Faza 1 --- lokalna (bez Stripe)}

Minimalna infrastruktura do testowania konwersji i~UX paywallu:

\begin{infobox}[title={\textbf{Nowe pliki}}]
\begin{itemize}
  \item \filepath{src/contexts/TierContext.tsx} --- \tstype{TierProvider} z~hookiem \tsfunc{useTier()}
  \item \filepath{src/components/shared/PaywallGate.tsx} --- komponent paywall z~blurred preview
\end{itemize}
\end{infobox}

\paragraph{TierContext}
\begin{itemize}
  \item Przechowuje tier w~\tstype{localStorage}: klucz \texttt{podtekst-tier}, wartości: \texttt{free} | \texttt{pro} | \texttt{unlimited}
  \item Hook \tsfunc{useTier()} zwraca: \texttt{\{ tier, canAccess(feature), upgrade() \}}
  \item Funkcja \tsfunc{canAccess(feature: string)} sprawdza mapę uprawnień:
\end{itemize}

\begin{lstlisting}[style=podcode, caption={Mapa uprawnień funkcji}, label={lst:feature-map}]
const FEATURE_MAP: Record<string, Tier> = {
  'ai-pass-2': 'pro',
  'ai-pass-3': 'pro',
  'ai-pass-4': 'pro',
  'enhanced-roast': 'pro',
  'court-trial': 'pro',
  'dating-profile': 'pro',
  'reply-simulator': 'pro',
  'cps-screener': 'pro',
  'subtext-decoder': 'pro',
  'pdf-export': 'pro',
  'standup-pdf': 'pro',
  'all-share-cards': 'pro',
  'unlimited-analyses': 'unlimited',
};
\end{lstlisting}

\paragraph{PaywallGate}

Komponent opakowujący treść pro/unlimited:

\begin{itemize}
  \item Props: \tstype{requiredTier: 'pro' | 'unlimited'}, \tstype{children}, opcjonalnie \tstype{preview} (blurred preview)
  \item Renderuje: jeśli tier wystarczający --- \tstype{children}; jeśli nie --- blurred overlay z~gradientem, ikoną zamka i~CTA
  \item Dev override: \texttt{?tier=pro} w~URL pozwala na testowanie bez płatności
\end{itemize}

\subsubsection{Faza 2 --- produkcyjna (Stripe + Supabase)}

Pełna infrastruktura płatności (planowana, niezaimplementowana):

\begin{itemize}
  \item \textbf{Supabase Auth} --- rejestracja email + Google OAuth
  \item \textbf{Stripe Checkout} --- sesje płatności, subskrypcje miesięczne
  \item \textbf{Stripe Webhooks} --- \filepath{/api/webhooks/stripe} do obsługi zdarzeń płatności
  \item \textbf{Server-side validation} --- tier sprawdzany w~API routes przed procesowaniem AI
  \item \textbf{Usage tracking} --- licznik analiz/miesiąc per user w~Supabase
\end{itemize}

\subsection{Share Cards jako viral loop}

Share Cards stanowią kluczowy mechanizm wiralowego wzrostu:

\begin{table}[H]
\centering
\caption{Strategia wiralowa --- karty share}
\label{tab:audyt-viral-cards}
\begin{tabularx}{\textwidth}{L{3.5cm}L{4cm}X}
\toprule
\textbf{Tier} & \textbf{Dostępne karty} & \textbf{Watermark} \\
\midrule
Free & PersonalityCard, VersusCard, StatsCard, ReceiptCard, BadgesCard & ,,PodTeksT.app'' (dolny róg) \\
Pro & Wszystkie 20+ kart & Bez watermarku \\
\bottomrule
\end{tabularx}
\end{table}

Watermark na Free-tier kartach służy jako \textbf{darmowa reklama} --- każda karta udostępniona na social media kieruje nowych użytkowników do aplikacji.

\subsection{Projekcja przychodów}

\begin{warningbox}[title={\danger{Założenia modelowe --- niewalidowane}}]
Poniższe liczby MAU (500 / 2\,000 / 5\,000) to \textbf{założenia modelowe}, nie walidowane prognozy. Na dzień audytu brak:
\begin{itemize}
  \item strategii akwizycji użytkowników (paid ads, influencerzy, SEO content),
  \item budżetu marketingowego,
  \item danych historycznych o~ruchu,
  \item walidowanego kanału wzrostu organicznego.
\end{itemize}
Rzeczywiste MAU może być \textbf{10$\times$ niższe} bez inwestycji w~kanały wzrostu. Projekcja służy wyłącznie do modelowania unit economics --- nie jest prognozą biznesową.
\end{warningbox}

\begin{table}[H]
\centering
\caption{Projekcja przychodów (12 miesięcy)}
\label{tab:audyt-projekcja}
\begin{tabularx}{\textwidth}{L{3.5cm}R{2.5cm}R{2.5cm}R{2.5cm}}
\toprule
\textbf{Metryka} & \textbf{Miesiąc 3} & \textbf{Miesiąc 6} & \textbf{Miesiąc 12} \\
\midrule
MAU (Monthly Active Users) & 500 & 2\,000 & 5\,000 \\
Konwersja Free$\rightarrow$Pro & 3\% & 5\% & 5\% \\
Konwersja Pro$\rightarrow$Unlimited & 10\% & 15\% & 20\% \\
Płacący Pro & 15 & 100 & 250 \\
Płacący Unlimited & 2 & 15 & 50 \\
MRR (Monthly Recurring Revenue) & 550~PLN & 3\,749~PLN & 9\,997~PLN \\
ARR (Annual Recurring Revenue) & 6\,600~PLN & 44\,988~PLN & 119\,964~PLN \\
\bottomrule
\end{tabularx}
\end{table}

\begin{warningbox}[title={\textbf{Założenia modelowe (niewalidowane)}}]
Konwersja 3--5\% (benchmark freemium SaaS B2C), churn 5\%/mies, ARPU: Pro 29,99~PLN, Unlimited 49,99~PLN. Wzrost zakładany jako organiczny (viral share cards + SEO + word-of-mouth). \textbf{Żadne z~tych założeń nie zostało zwalidowane danymi z~\podtekst.} Projekcja służy wyłącznie do modelowania struktury kosztów AI --- nie jest prognozą biznesową. Poniżej (sekcja~\ref{sec:strategia-akwizycji}) przedstawiamy prognozę opartą na konkretnej strategii akwizycji.
\end{warningbox}


\subsection{Strategia akwizycji i~prognoza biznesowa}
\label{sec:strategia-akwizycji}

Poprzednie projekcje (tabela~\ref{tab:audyt-projekcja}) operowały na arbitralnych liczbach MAU bez planu ich osiągnięcia. Niniejsza sekcja przedstawia \textbf{konkretną strategię akwizycji} z~kanałami wzrostu, budżetami i~realistycznymi prognozami opartymi na benchmarkach polskiego rynku consumer SaaS.

\subsubsection{Kanały akwizycji}

\begin{table}[H]
\centering
\caption{Kanały akwizycji --- charakterystyka i~szacunkowy CAC}
\label{tab:audyt-strategia-kanaly}
\renewcommand{\arraystretch}{1.3}
\begin{tabularx}{\textwidth}{L{3cm}C{1.5cm}X R{2.5cm}}
\toprule
\textbf{Kanał} & \textbf{Typ} & \textbf{Mechanizm} & \textbf{CAC (PLN)} \\
\midrule
\rowcolor{PodSuccess!5}
Viral share cards & Organic & Karty Free z~watermarkiem ,,PodTeksT.app'' udostępniane na IG/TikTok --- każda karta to darmowa reklama & $\sim$0 \\
\rowcolor{PodSuccess!5}
SEO / blog content & Organic & Artykuły: ,,jak wyeksportować rozmowę z~Messengera'', ,,co znaczy ghosting'', ,,analiza rozmowy z~eksem'' & $\sim$0 (czas) \\
\rowcolor{PodBlue!5}
TikTok/Reels content & Content & Format: ,,Przeanalizowałem rozmowę z~moim eksem'' --- viralowy format dramatyczny, 9:16 & 5--15 \\
\rowcolor{PodPurple!5}
Mikro-influencerzy & Influencer & Polscy twórcy relationship/drama (10--50k followers), 500--2\,000~PLN/kolaboracja & 10--30 \\
\rowcolor{PodPurple!5}
Makro-influencerzy & Influencer & 100k+ followers, 3\,000--10\,000~PLN/kolaboracja, większy zasięg ale mniejsze zaangażowanie & 15--40 \\
\rowcolor{PodWarning!5}
Meta Ads (IG/FB) & Paid & Targeting 18--35, zainteresowania: związki, randki, psychologia, dramaty & 15--40 \\
\rowcolor{PodWarning!5}
Google Ads & Paid & Keywords: ,,analiza rozmowy'', ,,kto pisze pierwszy'', ,,ghosting sprawdzić'' & 20--50 \\
\bottomrule
\end{tabularx}
\end{table}

\begin{infobox}[title={\textbf{Viral share cards --- kluczowy kanał organiczny}}]
\podtekst posiada 20+ typów kart share (PersonalityCard, VersusCard, ReceiptCard, MugshotCard, etc.). Karty Free tier zawierają watermark ,,PodTeksT.app'' w~dolnym rogu. Każda karta udostępniona na social media to \textbf{darmowa reklama} z~wbudowanym CTA. Benchmarki virality-driven SaaS (Wrapped, Spotify Blend):
\begin{itemize}
  \item 1 karta udostępniona $\rightarrow$ średnio 20--50 wyświetleń (stories), 100--500 (post/reel)
  \item Konwersja wyświetlenie$\rightarrow$wizyta: 2--5\% (stories), 0,5--2\% (feed)
  \item \textbf{Przy 100 kartach udostępnionych dziennie:} 200--1\,000 nowych wizyt/dzień organicznie
\end{itemize}
\end{infobox}


\subsubsection{Scenariusze budżetowe}

\begin{table}[H]
\centering
\caption{Trzy scenariusze wzrostu --- budżet, kanały, prognoza MAU}
\label{tab:audyt-scenariusze}
\renewcommand{\arraystretch}{1.3}
\begin{tabularx}{\textwidth}{L{2.5cm}R{2cm}X R{1.5cm}R{1.5cm}R{1.5cm}}
\toprule
\textbf{Scenariusz} & \textbf{Budżet/mies} & \textbf{Kanały} & \textbf{M3} & \textbf{M6} & \textbf{M12} \\
\midrule
\rowcolor{PodSuccess!5}
A --- Bootstrap & 0--500~PLN & Viral cards, SEO, własny TikTok & 100 & 400 & 1\,200 \\
\rowcolor{PodBlue!5}
B --- Lean & 2\,000--5\,000~PLN & Organic + mikro-influencerzy + mały Meta Ads & 300 & 1\,000 & 3\,000 \\
\rowcolor{PodPurple!5}
C --- Growth & 10\,000--20\,000~PLN & Pełny mix: organic + influencerzy + paid & 800 & 3\,000 & 10\,000 \\
\bottomrule
\end{tabularx}
\end{table}

\begin{metricbox}
\textbf{Rekomendacja: Scenariusz B (Lean Startup).} Pozwala na walidację kanałów akwizycji przy umiarkowanym ryzyku finansowym. Jeśli CAC z~influencerów i~Meta Ads wyniesie $<$30~PLN przy konwersji $>$2\%, przejście na scenariusz C jest uzasadnione. Scenariusz A jest bezpieczny, ale wzrost może być zbyt wolny, aby zwalidować model biznesowy przed wyczerpaniem motywacji.
\end{metricbox}


\subsubsection{Prognoza finansowa --- scenariusz B (Lean Startup)}

\begin{table}[H]
\centering
\caption{Prognoza biznesowa --- scenariusz B (Lean Startup, 2\,000--5\,000~PLN/mies)}
\label{tab:audyt-prognoza-lean}
\renewcommand{\arraystretch}{1.3}
\begin{tabularx}{\textwidth}{L{4.5cm}R{2.5cm}R{2.5cm}R{2.5cm}}
\toprule
\textbf{Metryka} & \textbf{Miesiąc 3} & \textbf{Miesiąc 6} & \textbf{Miesiąc 12} \\
\midrule
\multicolumn{4}{l}{\textit{Akwizycja}} \\
MAU & 300 & 1\,000 & 3\,000 \\
Nowi użytkownicy (organic) & $\sim$100/mies & $\sim$300/mies & $\sim$500/mies \\
Nowi użytkownicy (paid) & $\sim$100/mies & $\sim$200/mies & $\sim$250/mies \\
Budżet marketing/mies & 2\,000~PLN & 3\,500~PLN & 5\,000~PLN \\
CAC blended & $\sim$20~PLN & $\sim$18~PLN & $\sim$15~PLN \\
\midrule
\multicolumn{4}{l}{\textit{Konwersja}} \\
Konwersja Free$\rightarrow$Pro & 2\% & 3\% & 4\% \\
Płacący Pro & 6 & 30 & 120 \\
Płacący Unlimited & 1 & 5 & 20 \\
Churn miesięczny & 8\% & 6\% & 5\% \\
\midrule
\multicolumn{4}{l}{\textit{Przychody i~koszty}} \\
MRR (przychód) & 230~PLN & 1\,150~PLN & 4\,600~PLN \\
Koszt AI/mies & $\sim$20~PLN & $\sim$110~PLN & $\sim$460~PLN \\
Koszt marketing/mies & 2\,000~PLN & 3\,500~PLN & 5\,000~PLN \\
Infrastruktura (Cloud Run) & $\sim$50~PLN & $\sim$100~PLN & $\sim$200~PLN \\
\midrule
\rowcolor{PodDanger!5}
\textbf{Wynik netto/mies} & \danger{$-$1\,840~PLN} & \danger{$-$2\,560~PLN} & \danger{$-$1\,060~PLN} \\
\rowcolor{PodWarning!5}
\textbf{Skumulowana strata} & $-$5\,520~PLN & $-$18\,720~PLN & $-$30\,000~PLN \\
\midrule
\textbf{Break-even (szacunek)} & \multicolumn{3}{r}{$\sim$Miesiąc 15--18 (przy utrzymaniu wzrostu)} \\
\textbf{ARR w~miesiącu 12} & \multicolumn{3}{r}{\textbf{$\sim$55\,000~PLN}} \\
\bottomrule
\end{tabularx}
\end{table}

\begin{infobox}[title={\textbf{Kluczowe wskaźniki rentowności}}]
\begin{itemize}
  \item \textbf{CAC payback period:} Przy CAC = 20~PLN i~ARPU = 29,99~PLN/mies $\rightarrow$ payback $<$1 miesiąc. Przy uwzględnieniu churnu (LTV $\approx$ 5--8 mies $\times$ 29,99~PLN = 150--240~PLN) $\rightarrow$ \textbf{LTV/CAC = 7,5--12$\times$} (zdrowy stosunek $>$3$\times$).
  \item \textbf{Dlaczego mimo dobrego LTV/CAC firma jest na stracie?} Inwestycja w~akwizycję wyprzedza przychody --- płacący użytkownicy z~miesiąca~1 generują przychód przez 5--8 miesięcy, ale koszt ich pozyskania jest natychmiastowy.
  \item \textbf{Break-even:} $\sim$miesiąc 15--18, gdy skumulowany MRR pokryje skumulowane koszty marketingowe. Po break-even: margines netto $>$50\% (AI kosztuje $<$10\% MRR).
  \item \textbf{Wymagana inwestycja:} $\sim$30\,000~PLN do break-even (skumulowana strata).
\end{itemize}
\end{infobox}


\subsubsection{Porównanie z~poprzednią projekcją (modelową)}

\begin{table}[H]
\centering
\caption{Porównanie: projekcja modelowa vs prognoza ze strategią}
\label{tab:audyt-porownanie-prognoz}
\begin{tabularx}{\textwidth}{L{4cm}R{3cm}R{3cm}C{2cm}}
\toprule
\textbf{Metryka (M12)} & \textbf{Projekcja modelowa} & \textbf{Prognoza Lean} & \textbf{Różnica} \\
\midrule
MAU & 5\,000 & 3\,000 & $-$40\% \\
Płacący łącznie & 300 & 140 & $-$53\% \\
MRR & 9\,997~PLN & 4\,600~PLN & $-$54\% \\
ARR & 120\,000~PLN & 55\,000~PLN & $-$54\% \\
Koszt marketingu & 0~PLN & 5\,000~PLN/mies & $+$60\,000~PLN/rok \\
\bottomrule
\end{tabularx}
\end{table}

\begin{warningbox}[title={\textbf{Wniosek: projekcja modelowa była $\sim$2$\times$ zawyżona}}]
Projekcja modelowa zakładała 5\,000~MAU i~5\% konwersji \textbf{bez} żadnego budżetu marketingowego. Realistyczna prognoza ze strategią (scenariusz B) daje $\sim$3\,000~MAU i~4\% konwersji \textbf{przy} inwestycji $\sim$60\,000~PLN/rok w~marketing. ARR w~miesiącu~12: 55\,000~PLN vs modelowe 120\,000~PLN --- \textbf{różnica 2,2$\times$}. Projekcja modelowa jest użyteczna wyłącznie jako ilustracja unit economics, nie jako plan biznesowy.
\end{warningbox}


\subsubsection{Ryzyka strategii akwizycji}

\begin{table}[H]
\centering
\caption{Macierz ryzyk strategii akwizycji}
\label{tab:audyt-ryzyka-strategia}
\begin{tabularx}{\textwidth}{L{3.5cm}C{2cm}C{2cm}X}
\toprule
\textbf{Ryzyko} & \textbf{Prawdop.} & \textbf{Wpływ} & \textbf{Mitigacja} \\
\midrule
\rowcolor{PodDanger!5}
Viral cards nie chwycą & \warn{Średnie} & \danger{Wysoki} & A/B testy formatów kart, 9:16 pionowe dla Stories/Reels \\
\rowcolor{PodWarning!5}
CAC $>$40~PLN na paid & \warn{Średnie} & \warn{Średni} & Limit budżetu paid, pivot na influencerów \\
\rowcolor{PodWarning!5}
Churn $>$10\%/mies & \warn{Średnie} & \warn{Średni} & Roczne plany z~rabatem, retention emails, nowe funkcje \\
\rowcolor{PodWarning!5}
Sezonowość (lato) & \score{Niskie} & \warn{Średni} & Content ,,wakacyjny'' (analiza rozmów z~vacation fling) \\
\rowcolor{PodDanger!5}
Konwersja $<$1,5\% & \warn{Średnie} & \danger{Wysoki} & Pay-per-analysis (4,99~PLN) jako alternatywa, obniżka Pro do 14,99~PLN \\
\bottomrule
\end{tabularx}
\end{table}


\subsubsection{Rekomendacje --- fazy wdrożenia}

\begin{featurebox}
\textbf{4 fazy strategii akwizycji:}
\begin{description}
  \item[Faza 0 --- przygotowanie (teraz, 0~PLN):] Zoptymalizować 3--5 kart share pod TikTok/IG (format 9:16, duży tekst, watermark). Dodać export guide per platforma (Messenger, WhatsApp) --- to jednocześnie SEO content i~redukcja drop-off.
  \item[Faza 1 --- walidacja (miesiąc 1--3, $\sim$2\,000~PLN/mies):] Własny content TikTok (3--5 filmów/tydzień, format ,,analizuję rozmowę z~eksem''). Kolaboracja z~2--3 mikro-influencerami. Cel: \textbf{walidacja CAC} --- czy $<$30~PLN? Jeśli tak --- przejście do Fazy~2.
  \item[Faza 2 --- skalowanie (miesiąc 4--6, $\sim$3\,500~PLN/mies):] Jeśli CAC z~Fazy~1 $<$30~PLN: uruchomić Meta Ads (1\,000--2\,000~PLN/mies) + więcej influencerów. Jeśli CAC $>$40~PLN: pivot na pure organic + content, obniżyć cenę Pro do 14,99~PLN.
  \item[Faza 3 --- optymalizacja LTV (miesiąc 7--12, $\sim$5\,000~PLN/mies):] Roczne plany z~rabatem ($\sim$199~PLN/rok = 16,60~PLN/mies), referral program (polecaj $\rightarrow$ 1 miesiąc gratis), retention emails (,,masz nową rozmowę do przeanalizowania?''). Cel: obniżyć churn z~8\% do 4\%, wydłużyć LTV z~5 do 10+ miesięcy.
\end{description}
\end{featurebox}


% ============================================================
\section{Audyt optymalizacji wydajności}
\label{sec:audyt-optymalizacja}
% ============================================================

\subsection{Problemy krytyczne (P0)}

\subsubsection{Rate Limiting wyłączony}
\label{sec:audyt-rate-limit}

\begin{warningbox}[title={\danger{KRYTYCZNE --- ryzyko kosztowe i~bezpieczeństwa}}]
Plik \filepath{src/lib/rate-limit.ts} zawiera \textbf{celowo wyłączony} rate limiting:

\begin{lstlisting}[style=podcode, caption={Rate limiting --- stan aktualny}, label={lst:rate-limit-broken}]
export function rateLimit(_limit: number, _windowMs: number) {
  // TODO: re-enable rate limiting before production
  return function checkRateLimit(_ip: string) {
    return { allowed: true };  // <-- ZAWSZE true
  };
}
\end{lstlisting}

\textbf{Konsekwencje:}
\begin{itemize}
  \item Każdy użytkownik może wysyłać \textbf{nieograniczoną} liczbę requestów do API Gemini
  \item Brak ochrony przed atakami typu abuse/DoS na endpointy AI
  \item Koszty API Gemini mogą rosnąć niekontrolowanie
  \item Plik zawiera poprawną implementację \tstype{rateLimitMap} (linie 1--13), ale funkcja zwracająca nigdy z~niej nie korzysta
\end{itemize}
\end{warningbox}

\paragraph{Naprawa:} Przywrócić logikę rate limitingu --- plik już zawiera \tstype{Map<string, \{count, resetTime\}>} z~automatycznym czyszczeniem co 5 minut. Wystarczy odkomentować sprawdzanie w~\tsfunc{checkRateLimit()}.

\subsubsection{Nadmiar IntersectionObserver (37 instancji)}
\label{sec:audyt-intersection-observers}

Framer Motion \tsfunc{whileInView} tworzy osobny \tstype{IntersectionObserver} dla każdego elementu. Na stronie analizy:

\begin{table}[H]
\centering
\caption{IntersectionObserver --- rozkład instancji}
\label{tab:audyt-observers}
\begin{tabularx}{\textwidth}{L{5cm}C{3cm}X}
\toprule
\textbf{Źródło} & \textbf{Instancje} & \textbf{Uwagi} \\
\midrule
\tsfunc{whileInView} w~page.tsx & 37 & Każdy \tstype{motion.div} z~animacją fade-in \\
SectionNavigator & 6--9 & Osobny observer per sekcję nawigacji \\
Scroll listener (progress) & 1 & \tstype{window.addEventListener('scroll')} \\
\midrule
\textbf{Razem} & \textbf{44--47} & Na jednej stronie \\
\bottomrule
\end{tabularx}
\end{table}

\paragraph{Wpływ na wydajność:}
\begin{itemize}
  \item Każdy \tstype{IntersectionObserver} utrzymuje referencje do obserwowanych elementów i~callbacków
  \item 44+ instancji = szacunkowo 10--20\,MB dodatkowego zużycia pamięci
  \item Browser musi przeliczać intersection ratio przy każdym scroll event dla wszystkich obserwatorów
  \item Na urządzeniach mobilnych z~4\,GB RAM to odczuwalne spowolnienie
\end{itemize}

\paragraph{Naprawa:} Architektura tabowa naturalnie redukuje liczbę do $\sim$8--10 per tab. Dodatkowo, pojedynczy shared \tstype{IntersectionObserver} z~\texttt{threshold: [0, 0.5, 1]} zamiast wielu osobnych.

\subsection{Problemy ważne (P1)}

\subsubsection{Brak lazy-loadingu treści tabów}

Choć 18 komponentów używa \tsfunc{dynamic()} (Next.js lazy import), wszystkie ładują się natychmiast po wejściu na stronę, ponieważ leżą w~jednym drzewie renderowania. Z~architekturą tabową:

\begin{itemize}
  \item \tsfunc{React.lazy()} per tab --- JS Entertainment i~Share tabów ładowany dopiero po kliknięciu
  \item Szacowana redukcja initial JS bundle: \textbf{40--60\%} (Entertainment tab jest najcięższy: Gemini API calls, chat simulator, quiz engine)
\end{itemize}

\subsubsection{Brakujące React.memo na ciężkich komponentach}

Następujące komponenty re-renderują się przy \textbf{każdej} zmianie state'u w~page.tsx (9 handlerów useCallback = częste re-rendery):

\begin{table}[H]
\centering
\caption{Komponenty wymagające React.memo}
\label{tab:audyt-memo}
\begin{tabularx}{\textwidth}{L{4.5cm}C{2cm}X}
\toprule
\textbf{Komponent} & \textbf{Ciężkość} & \textbf{Powód re-renderów} \\
\midrule
\tstype{TimelineChart} & Wysoka & Recharts render per re-render \\
\tstype{HeatmapChart} & Wysoka & Duża siatka SVG \\
\tstype{ResponseTimeChart} & Wysoka & Recharts render per re-render \\
\tstype{ShareCardGallery} & Bardzo wysoka & 20+ kart, canvas rendering \\
\tstype{PersonalityDeepDive} & Wysoka & Wiele sub-komponentów \\
\tstype{NetworkGraph} & Wysoka & Obliczenia grafu + SVG \\
\bottomrule
\end{tabularx}
\end{table}

\paragraph{Naprawa:} Dodać \tsfunc{React.memo()} z~custom comparator na powyższe komponenty. Przenieść handlery \tsfunc{useCallback} do odpowiednich tab komponentów, aby zmiana state w~jednym tabie nie powodowała re-renderów w~innym.

\subsubsection{Stabilizacja callbacków}

Plik \filepath{page.tsx} zawiera 9 handlerów \tsfunc{useCallback}:

\begin{enumerate}
  \item \tsfunc{handleAIComplete}
  \item \tsfunc{handleRoastComplete}
  \item \tsfunc{handleCPSComplete}
  \item \tsfunc{handleSubtextComplete}
  \item \tsfunc{handleDelusionComplete}
  \item \tsfunc{handleCourtComplete}
  \item \tsfunc{handleDatingProfileComplete}
  \item \tsfunc{handlePhotoUpload} / \tsfunc{handlePhotoRemove}
  \item \tsfunc{handleImageSaved}
\end{enumerate}

Wszystkie mają \tstype{[analysis]} w~tablicy zależności --- więc każda zmiana \tstype{analysis} (np. zakończenie dowolnej analizy AI) powoduje re-tworzenie \textbf{wszystkich} callbacków i~re-render \textbf{wszystkich} komponentów, które je przyjmują.

\paragraph{Naprawa:} Przenieść handlery do odpowiednich tab komponentów. Tab AI Insights posiada tylko handlery AI, tab Entertainment --- handlery rozrywkowe. Zmiana w~jednym tabie nie wpływa na callbacki w~drugim.

\subsection{Problemy dodatkowe (P2)}

\subsubsection{Web Worker dla parsowania}

Parsery (\filepath{src/lib/parsers/}) wykonują się na main thread:

\begin{itemize}
  \item Duże eksporty Messenger (50\,000+ wiadomości): 200--400\,ms blokowania UI
  \item Rozwiązanie: \tsfunc{new Worker(new URL('./parser.worker.ts', import.meta.url))}
  \item Użytkownik widzi responsywny UI z~progress barem zamiast zamrożonej strony
\end{itemize}

\subsubsection{IndexedDB quota management}

Brak zarządzania limitami przechowywania:

\begin{itemize}
  \item Każda analiza to $\sim$50--200\,KB w~IndexedDB (dane ilościowe + jakościowe + obrazy)
  \item 50 analiz = $\sim$5--10\,MB --- bez ostrzeżeń ani auto-cleanup
  \item Rozwiązanie: \tsfunc{navigator.storage.estimate()} przed zapisem, ostrzeżenie przy >80\% quota
\end{itemize}

\subsubsection{Wirtualizacja ShareCardGallery}

20+ kart share renderowanych jednocześnie:

\begin{itemize}
  \item Każda karta to komponent z~canvas rendering (html2canvas)
  \item Rozwiązanie: intersection-based lazy rendering lub \tstype{react-window}
\end{itemize}



% ============================================================
% NOWE SEKCJE via \input
% ============================================================

% ============================================================
% Sekcja S1: Unit economics i koszty AI
% ============================================================

\section{Unit economics i~koszty AI}
\label{sec:unit-economics}

Każde wywołanie analizy AI w~\podtekst generuje realne koszty po stronie Google Gemini API. Niniejsza sekcja dokumentuje konfigurację modelu, szacunki tokenów, koszty per analiza w~PLN oraz margines na tier w~kontekście proponowanego modelu cenowego. Wszystkie kalkulacje używają oficjalnego cennika Google Gemini z~lutego 2026 przeliczonego na PLN.

\subsection{Konfiguracja modelu}

\begin{table}[H]
\centering
\caption{Konfiguracja modelu Gemini w~\podtekst}
\label{tab:ue-model-config}
\begin{tabularx}{\textwidth}{L{4.5cm}L{4.5cm}X}
\toprule
\textbf{Parametr} & \textbf{Wartość} & \textbf{Plik} \\
\midrule
Model & \texttt{gemini-3-flash-preview} & \filepath{gemini.ts:67} \\
Temperature & 0.3 & \filepath{gemini.ts:70} \\
Max output tokens (domyślny) & 8\,192 & \filepath{gemini.ts:69} \\
Max output tokens (CPS/Subtext) & 16\,384 & per batch call \\
Response MIME type & \texttt{application/json} & \filepath{gemini.ts:71} \\
Safety settings & Wszystkie \texttt{BLOCK\_NONE} & \filepath{gemini.ts:49--54} \\
Retry policy & 3$\times$ exponential backoff (1s, 2s, 4s) & \filepath{gemini.ts:56--105} \\
Model obrazów & \texttt{gemini-3-pro-image-preview} & osobny endpoint \\
\bottomrule
\end{tabularx}
\end{table}

\begin{warningbox}[title={\textbf{Brak prompt cachingu i~response cachingu}}]
Na dzień audytu \podtekst nie implementuje żadnej formy cachowania:
\begin{itemize}
  \item \textbf{Prompt caching:} brak --- ten sam system prompt (np.\ \tstype{PASS\_1\_SYSTEM}) jest wysyłany jako pełny tekst przy każdym wywołaniu, nawet jeśli 10 użytkowników uruchomi analizę w~ciągu minuty.
  \item \textbf{Response caching:} brak --- ponowna analiza tej samej konwersacji generuje pełny koszt API, mimo identycznych danych wejściowych.
\end{itemize}
\end{warningbox}


\subsection{Tokeny promptów systemowych}

\begin{table}[H]
\centering
\caption{Szacunkowa liczba tokenów promptów systemowych}
\label{tab:ue-prompt-tokens}
\begin{tabularx}{\textwidth}{L{5cm}R{2cm}X}
\toprule
\textbf{Prompt} & \textbf{$\sim$Tokeny} & \textbf{Lokalizacja} \\
\midrule
PASS\_1\_SYSTEM (Overview) & $\sim$450 & \filepath{prompts.ts:16--57} \\
PASS\_2\_SYSTEM (Dynamics) & $\sim$1\,100 & \filepath{prompts.ts:63--145} \\
PASS\_3\_SYSTEM (Individual Profiles) & $\sim$1\,400 & \filepath{prompts.ts:151--296} \\
PASS\_4\_SYSTEM (Synthesis) & $\sim$900 & \filepath{prompts.ts:302--369} \\
ROAST\_SYSTEM & $\sim$550 & \filepath{prompts.ts:375--410} \\
ENHANCED\_ROAST\_SYSTEM & $\sim$750 & \filepath{prompts.ts:416--458} \\
STANDUP\_ROAST\_SYSTEM & $\sim$900 & \filepath{prompts.ts:464--516} \\
SUBTEXT\_SYSTEM & $\sim$1\,200 & \filepath{prompts.ts:574--620} \\
CPS\_BATCH\_PROMPT & $\sim$600--800 & \filepath{prompts.ts:526--568} \\
\midrule
\textbf{Suma (pełna analiza)} & \textbf{$\sim$7\,850} & \\
\bottomrule
\end{tabularx}
\end{table}


\subsection{Wywołania API per scenariusz}

Liczba wywołań Gemini API zależy od zakresu analizy:

\begin{table}[H]
\centering
\caption{Wywołania API Gemini per scenariusz analizy}
\label{tab:ue-api-calls}
\begin{tabularx}{\textwidth}{L{5.5cm}R{1.5cm}X}
\toprule
\textbf{Scenariusz} & \textbf{Wywołania} & \textbf{Szczegóły} \\
\midrule
Podstawowa analiza (standard) & 5 & Pass 1--4 + Roast \\
Z 2 uczestnikami (Pass 3 per osoba) & 6 & Pass 1--4 + Roast + dodatkowy Pass~3 \\
+ Enhanced Roast & +1 & Roast z~pełnym kontekstem psychologicznym \\
+ StandUp Comedy & +1 & 7 aktów w~jednym wywołaniu \\
+ CPS (3 batche) & +3 & 21 pytań per batch \\
+ Subtext (3 batche) & +3 & 8 okien kontekstowych per batch \\
+ Court Trial & +1 & Wykorzystuje wyniki Pass 1, 2, 4 \\
+ Dating Profile & +1 & Profil randkowy z~wzorców \\
+ Reply Simulator ($\times$5) & +5 & Maks. 5 wymian per sesja \\
+ Image Generation & +1 & Gemini Pro (droższy model) \\
\midrule
\textbf{Pełny zestaw (max)} & \textbf{$\sim$22} & Przy 2 uczestnikach \\
\bottomrule
\end{tabularx}
\end{table}


\subsection{Szacunek tokenów per wywołanie}

\begin{metricbox}
\textbf{Założenia szacunkowe:}
\begin{itemize}
  \item System prompt: $\sim$1\,000 tokenów (średnia z~tabeli \ref{tab:ue-prompt-tokens})
  \item Próbka wiadomości: 200--500 wiadomości $\times$ $\sim$30 tokenów/wiadomość $\approx$ 6\,000--15\,000 tokenów
  \item Kontekst ilościowy: $\sim$500 tokenów
  \item \textbf{Łącznie input per wywołanie: $\sim$16\,000 tokenów}
  \item Typowy output per wywołanie: $\sim$4\,000 tokenów (JSON z~analizą)
  \item Maksymalna długość wiadomości: 2\,000 znaków (\tsfunc{sanitizeForPrompt()})
\end{itemize}
\end{metricbox}

\subsection{Cennik Gemini API (PLN)}

\begin{table}[H]
\centering
\caption{Cennik Gemini API w~\podtekst (PLN, luty 2026)}
\label{tab:ue-gemini-pricing}
\begin{tabularx}{\textwidth}{L{5.5cm}R{3cm}R{3cm}}
\toprule
\textbf{Kategoria} & \textbf{Flash (tekst/obraz/video)} & \textbf{Pro (obraz)} \\
\midrule
Input (per 1M tokenów) & \textbf{0,25~PLN} & wyższy \\
Output z~thinking (per 1M tokenów) & \textbf{4,50~PLN} & wyższy \\
Cached input (per 1M tokenów) & \textbf{0,20~PLN} & --- \\
Batch API (50\% taniej) & 0,125 / 2,25~PLN & --- \\
\bottomrule
\end{tabularx}
\end{table}

\begin{warningbox}[title={\danger{Uwaga: thinking tokens w~cenie output}}]
Cena output \textbf{4,50~PLN/1M} obejmuje \textbf{thinking tokens} generowane wewnętrznie przez model. Jeśli model generuje 2--3$\times$ więcej thinking tokenów niż widoczny output, rzeczywisty koszt output może być 2--3$\times$ wyższy od wartości podanych w~tabeli \ref{tab:ue-cost-per-analysis}. Zalecamy wdrożenie monitoringu thinking tokenów (patrz tabela \ref{tab:ue-priorities}, pozycja~3).
\end{warningbox}


\subsection{Koszt per analiza (PLN)}

\begin{table}[H]
\centering
\caption{Koszt API per analiza --- scenariusze (PLN)}
\label{tab:ue-cost-per-analysis}
\renewcommand{\arraystretch}{1.3}
\begin{tabularx}{\textwidth}{L{4cm}R{1.5cm}R{2.5cm}R{2.5cm}R{2.5cm}}
\toprule
\textbf{Scenariusz} & \textbf{Wywoł.} & \textbf{Input (tokens)} & \textbf{Output (tokens)} & \textbf{Koszt} \\
\midrule
\rowcolor{PodSuccess!5}
Podstawowa (5 calls) & 5 & 80\,000 & 20\,000 & \textbf{0,11~PLN} \\
Rozszerzona (10 calls) & 10 & 160\,000 & 40\,000 & \textbf{0,22~PLN} \\
\rowcolor{PodWarning!5}
Pełna (20 calls) & 20 & 320\,000 & 80\,000 & \textbf{0,44~PLN} \\
\rowcolor{PodWarning!5}
Pełna + obraz & 21 & 340\,000 & 85\,000 & \textbf{$\sim$0,50~PLN} \\
\bottomrule
\end{tabularx}
\end{table}

\begin{infobox}[title={\textbf{Wzór kalkulacji (PLN)}}]
$$
\text{Koszt} = \frac{\text{Input tokens} \times 0{,}25}{1\,000\,000} + \frac{\text{Output tokens} \times 4{,}50}{1\,000\,000}
$$
Przykład --- podstawowa analiza (5 wywołań):
$$
\frac{80\,000 \times 0{,}25}{1\,000\,000} + \frac{20\,000 \times 4{,}50}{1\,000\,000} = 0{,}020 + 0{,}090 = 0{,}11~\text{PLN}
$$
\end{infobox}

\begin{metricbox}
\textbf{Kluczowa obserwacja:} Koszt output (\textbf{4,50~PLN/1M}) dominuje w~strukturze kosztów --- stanowi 82\% kosztu podstawowej analizy (0,090 z~0,110~PLN). Input (0,25~PLN/1M) jest marginalny. To oznacza, że optymalizacja input cachingu ma \textbf{ograniczony wpływ} na całkowity koszt.
\end{metricbox}


\subsection{Margines per tier (PLN)}

Poniższa tabela porównuje przychód z~subskrypcji z~kosztem AI dla różnych scenariuszy użycia. Ceny tierów w~PLN:

\begin{itemize}
  \item \textbf{Pro:} \textbf{29,99~PLN/mies.}
  \item \textbf{Unlimited:} \textbf{49,99~PLN/mies.}
\end{itemize}

\begin{table}[H]
\centering
\caption{Analiza marginu per tier i~scenariusz użycia (PLN)}
\label{tab:ue-margin-analysis}
\renewcommand{\arraystretch}{1.3}
\begin{tabularx}{\textwidth}{L{3.5cm}R{1.5cm}R{2cm}R{1.5cm}R{1.5cm}R{2cm}}
\toprule
\textbf{Scenariusz} & \textbf{Tier} & \textbf{Cena} & \textbf{Analizy/m} & \textbf{Koszt AI} & \textbf{Margines} \\
\midrule
\rowcolor{PodSuccess!5}
Pro, 3 basic & Pro & 29,99~PLN & 3 & 0,33~PLN & \score{99\%} \\
\rowcolor{PodSuccess!5}
Pro, 15 basic & Pro & 29,99~PLN & 15 & 1,65~PLN & \score{95\%} \\
\rowcolor{PodSuccess!5}
Pro, 5 full & Pro & 29,99~PLN & 5 & 2,20~PLN & \score{93\%} \\
\rowcolor{PodSuccess!5}
Pro, 15 full & Pro & 29,99~PLN & 15 & 6,60~PLN & \score{78\%} \\
\midrule
\rowcolor{PodSuccess!5}
Unlimited, 10 basic & Unlimited & 49,99~PLN & 10 & 1,10~PLN & \score{98\%} \\
\rowcolor{PodSuccess!5}
Unlimited, 30 basic & Unlimited & 49,99~PLN & 30 & 3,30~PLN & \score{93\%} \\
\rowcolor{PodSuccess!5}
Unlimited, 30 full & Unlimited & 49,99~PLN & 30 & 13,20~PLN & \score{74\%} \\
\bottomrule
\end{tabularx}
\end{table}

\begin{infobox}[title={\textbf{Marże vs poprzednia analiza USD --- z~kontekstem}}]
Marże w~PLN wyglądają zdrowiej niż w~analizie USD (Pro 15~full: 78\% vs 14\%, Unlimited 30~full: 74\% vs 8\%). Ale porównanie wymaga kontekstu:

\begin{itemize}
  \item \textbf{Arbitraż walutowy:} koszty API denominowane w~USD ($\sim$4,05~PLN). Przy deprecjacji PLN (np.\ do 4,50~PLN/USD) marże spadają o~$\sim$5~pp.
  \item \textbf{Niższa gotowość do płacenia:} polski rynek consumer SaaS akceptuje niższe ceny niż rynek USD (patrz tabela~\ref{tab:ue-polish-market}).
  \item \textbf{Marża na AI $\neq$ rentowność biznesu:} koszty CAC (pozyskania użytkownika), infrastruktury, developmentu nie są uwzględnione.
\end{itemize}
\end{infobox}

\begin{table}[H]
\centering
\caption{Kontekst cenowy --- polski rynek consumer SaaS}
\label{tab:ue-polish-market}
\begin{tabularx}{\textwidth}{L{3.5cm}R{2.5cm}C{2.5cm}X}
\toprule
\textbf{Produkt} & \textbf{Cena/mies.} & \textbf{Częstotliwość} & \textbf{Porównanie z~\podtekst} \\
\midrule
Spotify Premium & 23,99~PLN & Codziennie & Pro 29,99 = \warn{droższa} \\
Netflix Basic & 33~PLN & Codziennie & Pro 29,99 = tańsza \\
Tinder Gold & $\sim$50~PLN & Codziennie & Unlimited 49,99 $\approx$ Tinder \\
YouTube Premium & 26,99~PLN & Codziennie & Pro 29,99 $\approx$ porównywalny \\
\bottomrule
\end{tabularx}
\end{table}

\begin{warningbox}[title={\textbf{Problem: cena vs częstotliwość użycia}}]
\podtekst w~cenie 29,99~PLN/mies.\ kosztuje więcej niż Spotify Premium, a~jest używana \textbf{okazjonalnie} (1--5$\times$ w~miesiącu vs codziennie). Gotowość polskich użytkowników do płacenia za narzędzie okazjonalne jest istotnie niższa niż za usługi codzienne.

\textbf{Rekomendacja:} rozważyć model \textbf{pay-per-analysis} (np.\ 4,99~PLN/analiza) obok subskrypcji, lub obniżyć Pro do \textbf{14,99--19,99~PLN/mies.} Alternatywnie: zaoferować roczny plan z~rabatem ($\sim$199~PLN/rok = 16,60~PLN/mies.).
\end{warningbox}


\subsection{Analiza break-even}

\subsubsection{Ile analiz może wykonać użytkownik, zanim stanie się nierentowny?}

\begin{table}[H]
\centering
\caption{Break-even: maksymalna liczba analiz per tier przy 0\% marginu (PLN)}
\label{tab:ue-breakeven}
\begin{tabularx}{\textwidth}{L{3cm}R{2cm}R{2.5cm}R{2.5cm}R{2.5cm}}
\toprule
\textbf{Tier} & \textbf{Cena} & \textbf{Break-even (basic)} & \textbf{Break-even (full)} & \textbf{Limit w~planie} \\
\midrule
Pro & 29,99~PLN & $\sim$273 analiz & $\sim$68 analiz & 15 analiz \\
Unlimited & 49,99~PLN & $\sim$454 analizy & $\sim$114 analiz & Soft cap 50/mies. \\
\bottomrule
\end{tabularx}
\end{table}

\begin{metricbox}
\textbf{Wniosek:} Przy cenach w~PLN break-even jest \textbf{wielokrotnie wyższy} niż realistyczne użycie. Tier Pro z~limitem 15 analiz/miesiąc jest bezpieczny nawet przy \textbf{pełnych} analizach (break-even = 68, limit = 15 --- 4,5$\times$ zapas). Tier Unlimited z~soft capem 50/miesiąc również ma duży margines bezpieczeństwa (break-even = 114 full, soft cap = 50 --- 2,3$\times$ zapas).
\end{metricbox}


\subsection{Wpływ cachingu na koszty}

\begin{table}[H]
\centering
\caption{Wpływ prompt cachingu na koszt podstawowej analizy (PLN)}
\label{tab:ue-caching-impact}
\begin{tabularx}{\textwidth}{L{4cm}R{3cm}R{3cm}R{3cm}}
\toprule
\textbf{Metryka} & \textbf{Bez cachingu} & \textbf{Z cachingiem} & \textbf{Oszczędność} \\
\midrule
Input cost (5 calls) & 0,020~PLN & 0,016~PLN & 20\% inputu \\
Output cost (5 calls) & 0,090~PLN & 0,090~PLN & 0\% \\
\textbf{Suma} & \textbf{0,110~PLN} & \textbf{0,106~PLN} & \textbf{$\sim$4\%} \\
\bottomrule
\end{tabularx}
\end{table}

\begin{warningbox}[title={\textbf{Ograniczony wpływ cachingu}}]
Prompt caching ma \textbf{bardzo ograniczony wpływ} na koszt całkowity analizy, ponieważ:
\begin{itemize}
  \item Koszt output (\textbf{4,50~PLN/1M}) stanowi 82\% kosztu i~\textbf{nie podlega cachowaniu}
  \item Koszt input (\textbf{0,25~PLN/1M}) jest marginalny --- nawet redukcja o~80\% (caching z~0,25 do 0,05~PLN) oszczędza zaledwie \textbf{$\sim$4\%} na podstawowej analizie (z~0,110 na 0,106~PLN)
  \item Prompt systemowy ($\sim$1\,000 tokenów) stanowi tylko 6\% całego inputu --- reszta to unikalne dane konwersacji
\end{itemize}
W~porównaniu z~poprzednią analizą (14--36\% oszczędności), rzeczywisty wpływ cachingu jest \textbf{minimalny}.
\end{warningbox}


\subsection{Propozycje optymalizacji kosztów}

\subsubsection{1. Response caching (hash)}

Ponowna analiza tej samej konwersacji powinna korzystać z~wcześniej wygenerowanych wyników:

\begin{itemize}
  \item Hash wiadomości (SHA-256 z~treści i~timestampów) jako klucz cache'u
  \item Jeśli hash się zgadza --- zwróć wynik z~IndexedDB zamiast wywoływać API
  \item \textbf{Oszczędność:} 100\% kosztów re-analiz (szacunkowo 10--20\% wszystkich wywołań)
  \item \textbf{Trudność:} Łatwa --- wyłącznie po stronie klienta
\end{itemize}

\subsubsection{2. Batch API dla funkcji non-real-time}

Funkcje, które nie wymagają natychmiastowego wyniku, mogą korzystać z~Batch API (50\% taniej):

\begin{itemize}
  \item \textbf{StandUp Comedy PDF} --- użytkownik i~tak musi czekać na generowanie PDF
  \item \textbf{CPS (3 batche)} --- wynik nie jest interaktywny
  \item \textbf{Szacowana redukcja:} 50\% kosztów tych wywołań ($\sim$4 z~20 wywołań)
  \item Batch API: 0,125~PLN input / 2,25~PLN output per 1M tokenów
\end{itemize}

\subsubsection{3. Monitoring thinking tokenów}

\begin{itemize}
  \item Wdrożyć logowanie liczby thinking tokenów per wywołanie
  \item Jeśli model generuje $>$2$\times$ thinking tokens vs widoczny output --- rozważyć zmianę temperature lub struktury promptu
  \item \textbf{Cel:} Widoczność kosztowa --- bez monitoringu nie wiemy, czy output kosztuje 4,50~PLN czy efektywnie 9--13~PLN per 1M tokenów
  \item \textbf{Trudność:} Łatwa --- dane dostępne w~response metadata Gemini
\end{itemize}

\subsubsection{4. Prompt caching (Gemini Context Caching)}

\begin{itemize}
  \item Prompty systemowe (\tstype{PASS\_1\_SYSTEM}, etc.) identyczne dla każdego użytkownika
  \item Cached input: 0,20~PLN/1M zamiast 0,25~PLN/1M (20\% taniej)
  \item \textbf{Rzeczywista oszczędność: $\sim$4\%} na całkowitym koszcie analizy --- ograniczona, bo output dominuje
  \item \textbf{Trudność:} Średnia --- wymaga zarządzania cache TTL
\end{itemize}

\subsubsection{5. Korekta limitów per tier}

Przy zdrowych marżach w~PLN korekta nie wymaga agresywnych zmian:

\begin{table}[H]
\centering
\caption{Rekomendowane limity tierów (PLN)}
\label{tab:ue-tier-correction}
\begin{tabularx}{\textwidth}{L{3.5cm}C{3cm}C{3cm}X}
\toprule
\textbf{Metryka} & \textbf{Obecny plan} & \textbf{Rekomendowany} & \textbf{Uzasadnienie} \\
\midrule
Pro --- analizy/mies. & 15 & 15 & Bezpieczne (break-even = 68 full) \\
Pro --- cena & --- & 29,99~PLN & Rynek polski, przystępna cena \\
Unlimited --- cena & --- & 49,99~PLN & Zdrowy margines do 114 full \\
Unlimited --- soft cap & brak & 50/mies. & Fair use, 2,3$\times$ zapas do break-even \\
Funkcje rozrywkowe & w~Pro & Opcjonalnie add-on & 9,99~PLN/mies.\ (opcjonalny) \\
\bottomrule
\end{tabularx}
\end{table}


\subsection{Skorygowana projekcja przychodów (PLN)}

\begin{warningbox}[title={\danger{Zastrzeżenie: scenariusz modelowy}}]
Poniższa projekcja opiera się na tych samych \textbf{niewalidowanych} założeniach MAU co tabela~\ref{tab:audyt-projekcja} (500 / 2\,000 / 5\,000 użytkowników). Brak strategii akwizycji, budżetu marketingowego i~danych historycznych o~ruchu. Celem tej tabeli jest wyłącznie ilustracja struktury kosztów AI w~modelu subskrypcyjnym --- nie jest prognozą biznesową.
\end{warningbox}

\begin{table}[H]
\centering
\caption{Projekcja przychodów --- scenariusz modelowy, niewalidowany (PLN)}
\label{tab:ue-corrected-projection}
\renewcommand{\arraystretch}{1.3}
\begin{tabularx}{\textwidth}{L{4cm}R{2.5cm}R{2.5cm}R{2.5cm}}
\toprule
\textbf{Metryka} & \textbf{Miesiąc 3} & \textbf{Miesiąc 6} & \textbf{Miesiąc 12} \\
\midrule
Użytkownicy Pro & 15 & 100 & 250 \\
Użytkownicy Unlimited & 2 & 15 & 50 \\
\midrule
MRR (przychód) & 550~PLN & 3\,749~PLN & 9\,997~PLN \\
Koszt AI & $\sim$48~PLN & $\sim$333~PLN & $\sim$925~PLN \\
\midrule
\rowcolor{PodSuccess!5}
Margines brutto & \score{91\%} & \score{91\%} & \score{91\%} \\
\midrule
Zysk brutto/mies. & 502~PLN & 3\,416~PLN & 9\,072~PLN \\
\midrule
\textbf{ARR (Miesiąc 12)} & \multicolumn{3}{r}{\textbf{$\sim$120\,000~PLN}} \\
\bottomrule
\end{tabularx}
\end{table}

\begin{metricbox}
\textbf{Kluczowy wniosek (z~zastrzeżeniami):} Przy cenach w~PLN model subskrypcyjny jest \textbf{strukturalnie rentowny pod względem kosztów AI} --- marża brutto $\sim$91\%. Koszty AI (0,11--0,50~PLN per analiza) są niskie w~stosunku do cen subskrypcji (29,99 / 49,99~PLN). Jednak marża brutto na koszcie AI nie oznacza rentowności biznesu --- koszty CAC, infrastruktury i~marketingu nie są tu uwzględnione. Pełna prognoza z~uwzględnieniem strategii akwizycji, budżetów marketingowych i~kosztów CAC: patrz sekcja~\ref{sec:strategia-akwizycji}.
\end{metricbox}


\subsection{Podsumowanie rekomendacji}

\begin{table}[H]
\centering
\caption{Priorytety optymalizacji kosztów AI}
\label{tab:ue-priorities}
\begin{tabularx}{\textwidth}{C{1cm}L{4cm}R{2.5cm}C{2cm}X}
\toprule
\textbf{Nr} & \textbf{Optymalizacja} & \textbf{Oszczędność} & \textbf{Trudność} & \textbf{Priorytet} \\
\midrule
1 & Response caching (hash) & 10--20\% total & Łatwa & P0 \\
2 & Batch API (StandUp, CPS) & 50\% na 4 callach & Średnia & P1 \\
3 & Thinking token monitoring & Widoczność kosztowa & Łatwa & P0 \\
4 & Prompt caching (Gemini) & $\sim$4\% total (ograniczony) & Średnia & P2 \\
5 & Korekta limitów tierów & Redukcja ryzyka & Brak kodu & P1 \\
\bottomrule
\end{tabularx}
\end{table}

% ============================================================
% Sekcja S2: Bezpieczeństwo i prywatność
% ============================================================

\section{Bezpieczeństwo i~prywatność}
\label{sec:bezpieczenstwo}

Niniejsza sekcja stanowi audyt bezpieczeństwa \podtekst na dzień lutego 2026 (po Fazie~22). Analizuje przepływ danych, walidację API, nagłówki HTTP, rate limiting w~środowisku serverless, zgodność z~RODO, znane podatności oraz ryzyka specyficzne dla integracji Discord.


\subsection{Przepływ danych}

\begin{figure}[H]
\centering
\begin{tikzpicture}[
  node distance=1.2cm and 2.5cm,
  every node/.style={font=\small},
  >=Stealth,
]

% === Client layer ===
\node[podbox blue, minimum width=4.5cm, minimum height=1.5cm] (browser) {
  \begin{minipage}{4cm}
  \centering
  \textbf{Przeglądarka}\\
  {\scriptsize Parsowanie, analiza ilościowa}\\
  {\scriptsize IndexedDB, localStorage}
  \end{minipage}
};

% Upload arrow
\node[font=\scriptsize\color{PodTextMuted}, above=0.3cm of browser] {Plik eksportu (JSON/TXT)};

% === Data minimization ===
\node[podbox amber, minimum width=4.5cm, minimum height=1.2cm, below=1.5cm of browser] (sample) {
  \begin{minipage}{4cm}
  \centering
  \textbf{Próbka}\\
  {\scriptsize 200--500 wiadomości ($<$1\%)}\\
  {\scriptsize max 2\,000 znaków/msg}
  \end{minipage}
};

\draw[dataarrow, color=PodDanger] (browser) -- (sample)
  node[midway, right=10pt, font=\small\bfseries\color{PodDanger}] {Redukcja $>$99\%};

% === Server layer ===
\node[podbox purple, minimum width=4.5cm, minimum height=1.5cm, right=3.5cm of sample] (server) {
  \begin{minipage}{4cm}
  \centering
  \textbf{Next.js API Routes}\\
  {\scriptsize \texttt{import 'server-only'}}\\
  {\scriptsize Zod walidacja, sanityzacja}
  \end{minipage}
};

\draw[dataarrow, color=PodPurple] (sample) -- (server)
  node[midway, above, font=\scriptsize\color{PodTextSecondary}] {POST /api/analyze/*}
  node[midway, below, font=\scriptsize\color{PodTextSecondary}] {JSON over HTTPS};

% === Gemini API ===
\node[podbox green, minimum width=4.5cm, minimum height=1.2cm, below=1.5cm of server] (gemini) {
  \begin{minipage}{4cm}
  \centering
  \textbf{Gemini API}\\
  {\scriptsize gemini-3-flash-preview}\\
  {\scriptsize Przetwarzanie, brak retencji}
  \end{minipage}
};

\draw[dataarrow, color=PodBlue] (server) -- (gemini)
  node[midway, right=10pt, font=\scriptsize\color{PodTextSecondary}] {API key (server-only)};

% === Response back ===
\node[podbox green, minimum width=4.5cm, minimum height=1cm, below=1.5cm of sample] (result) {
  \begin{minipage}{4cm}
  \centering
  \textbf{Wynik AI (JSON)}\\
  {\scriptsize Bez cytatów wiadomości}
  \end{minipage}
};

\draw[dataarrow, color=PodSuccess] (gemini) -- (result)
  node[midway, below, font=\scriptsize\color{PodTextSecondary}] {SSE stream};

\draw[dataarrow, color=PodSuccess] (result) -- (browser)
  node[midway, left=10pt, font=\scriptsize\color{PodTextSecondary}] {IndexedDB};

% === Annotations ===
\node[font=\scriptsize\itshape\color{PodSuccess!80!black}, below=0.2cm of browser, xshift=-3cm] {\score{Dane nigdy nie opuszczają przeglądarki}};
\node[font=\scriptsize\itshape\color{PodDanger!80!black}, right=0.2cm of server, xshift=1cm] {\danger{Brak retencji}};
\node[font=\scriptsize\itshape\color{PodDanger!80!black}, right=0.2cm of gemini, xshift=1cm] {\danger{Brak retencji}};

% === Discord path (alternative) ===
\node[podbox blue, minimum width=3cm, minimum height=1cm, below=1.5cm of result] (discord) {
  \begin{minipage}{2.8cm}
  \centering
  \textbf{Discord API}\\
  {\scriptsize Bot Token w~body}
  \end{minipage}
};

\draw[podarrow dashed] (browser) -- ++(0,-5.5) -| (discord)
  node[near start, left=10pt, font=\scriptsize\color{PodWarning}] {\warn{Token z~klienta}};
\draw[dataarrow, color=PodBlue] (discord) -| (server)
  node[near end, right=5pt, font=\scriptsize\color{PodTextSecondary}] {/api/discord/fetch-messages};

\end{tikzpicture}
\caption{Przepływ danych w~\podtekst --- od uploadu do wyniku. Czerwone adnotacje oznaczają brak trwałego przechowywania.}
\label{fig:sec-data-flow}
\end{figure}

\subsubsection{Kluczowe gwarancje}

\begin{enumerate}[label=\textcolor{PodBlue}{\arabic*.}]
  \item \textbf{Surowe wiadomości} --- przetwarzane wyłącznie w~przeglądarce (parsery + analiza ilościowa). Nigdy nie wysyłane na serwer w~pełnej postaci.
  \item \textbf{Próbka} --- 200--500 wiadomości (z~50\,000+), każda obcięta do 2\,000 znaków. Znaki kontrolne usunięte (\filepath{gemini.ts:147--153}).
  \item \textbf{Serwer} --- przetwarza próbkę w~pamięci, wysyła do Gemini API, streamuje wynik przez SSE. Żadne dane nie są zapisywane na dysk ani w~bazie.
  \item \textbf{Gemini API} --- Google deklaruje brak retencji danych przesłanych przez API (API Terms of Service).
\end{enumerate}


\subsection{Walidacja API --- przegląd schematów Zod}

\begin{table}[H]
\centering
\caption{Inwentarz walidacji Zod per endpoint --- problemy}
\label{tab:sec-zod-audit}
\renewcommand{\arraystretch}{1.3}
\begin{tabularx}{\textwidth}{L{3cm}L{3.5cm}L{2.5cm}X}
\toprule
\textbf{Endpoint} & \textbf{Schema} & \textbf{Status} & \textbf{Uwagi / Problemy} \\
\midrule
\texttt{/api/analyze} & \tstype{analyzeRequestSchema} & \warn{Częściowa} & \tstype{samplesSchema} = \texttt{z.object(\{\}).passthrough()} --- brak głębokiej walidacji struktury próbek \\
\texttt{/api/analyze/cps} & \tstype{cpsRequestSchema} & \score{OK} & \tstype{participantName} min 1 char \\
\texttt{/api/analyze/subtext} & \tstype{subtextRequestSchema} & \score{OK} & Min 100 wiadomości, pełny schemat \tstype{SimplifiedMsg} \\
\texttt{/api/analyze/image} & \tstype{imageRequestSchema} & \score{OK} & Min 1 excerpt, typowane pola \\
\texttt{/api/analyze/court} & \tstype{courtRequestSchema} & \warn{Częściowa} & \tstype{existingAnalysis} pola to \tstype{z.unknown()} --- brak walidacji typów \\
\texttt{/api/analyze/dating} & Zod & \warn{Niezweryfikowana} & Schemat nie wyeksportowany do \filepath{schemas.ts} \\
\texttt{/api/analyze/simulate} & Zod & \warn{Niezweryfikowana} & Schemat nie wyeksportowany do \filepath{schemas.ts} \\
\texttt{/api/analyze/standup} & \tstype{standUpRequestSchema} & \score{OK} & \tstype{quantitativeContext} wymagany \\
\texttt{/api/analyze/enhanced-roast} & \tstype{enhancedRoastRequestSchema} & \score{OK} & Wymaga 4 passów w~\tstype{qualitative} \\
\texttt{/api/discord/fetch} & brak Zod & \danger{Brak} & Ręczna walidacja \tstype{botToken.length >= 50} \\
\bottomrule
\end{tabularx}
\end{table}

\begin{warningbox}[title={\textbf{Problem: \texttt{passthrough()} w~samplesSchema}}]
Schemat \tstype{samplesSchema = z.object(\{\}).passthrough()} przyjmuje \textbf{dowolny obiekt} --- nie waliduje, czy zawiera wymagane klucze (\texttt{overview}, \texttt{dynamics}, \texttt{perPerson}), ani typów wartości wewnątrz. Złośliwy aktor może przesłać obiekt z~dowolną strukturą, co może spowodować runtime error w~\tsfunc{formatMessagesForAnalysis()} lub --- co gorsze --- wstrzyknięcie nieoczekiwanych danych do promptu Gemini.

\textbf{Naprawa:} Zdefiniować pełny schemat \tstype{AnalysisSamples} z~walidacją zagnieżdżonych tablic \tstype{SimplifiedMessage[]}.
\end{warningbox}


\subsection{Nagłówki bezpieczeństwa HTTP}

\begin{table}[H]
\centering
\caption{Nagłówki bezpieczeństwa --- obecne vs brakujące}
\label{tab:sec-headers-audit}
\renewcommand{\arraystretch}{1.3}
\begin{tabularx}{\textwidth}{L{4.5cm}L{4cm}C{2cm}X}
\toprule
\textbf{Nagłówek} & \textbf{Wartość} & \textbf{Status} & \textbf{Źródło} \\
\midrule
\rowcolor{PodSuccess!5}
\texttt{X-Content-Type-Options} & \texttt{nosniff} & \score{Obecny} & \filepath{next.config.ts:10} \\
\rowcolor{PodSuccess!5}
\texttt{X-Frame-Options} & \texttt{DENY} & \score{Obecny} & \filepath{next.config.ts:11} \\
\rowcolor{PodSuccess!5}
\texttt{Referrer-Policy} & \texttt{strict-origin-when-cross-origin} & \score{Obecny} & \filepath{next.config.ts:12} \\
\rowcolor{PodSuccess!5}
\texttt{Permissions-Policy} & \texttt{camera=(), microphone=(), geolocation=()} & \score{Obecny} & \filepath{next.config.ts:13} \\
\midrule
\rowcolor{PodDanger!5}
\texttt{Content-Security-Policy} & --- & \danger{Brak} & Wymaga konfiguracji nonce \\
\rowcolor{PodDanger!5}
\texttt{Strict-Transport-Security} & --- & \danger{Brak} & HSTS dla HTTPS enforcement \\
\rowcolor{PodWarning!5}
\texttt{X-XSS-Protection} & --- & \warn{Brak} & Przestarzały, ale wciąż rekomendowany \\
\bottomrule
\end{tabularx}
\end{table}

\subsubsection{Content-Security-Policy --- rekomendacja}

CSP jest najbardziej brakującym nagłówkiem. Konfiguracja dla \podtekst wymaga uwzględnienia:

\begin{lstlisting}[style=podcode, caption={Proponowana konfiguracja CSP}, label={lst:sec-csp-proposal}]
Content-Security-Policy:
  default-src 'self';
  script-src 'self' 'nonce-{GENERATED}';
  style-src 'self' 'unsafe-inline';   // Tailwind
  connect-src 'self'
    https://generativelanguage.googleapis.com
    https://discord.com/api/
    https://www.google-analytics.com;
  img-src 'self' data: blob:;         // Generated images
  font-src 'self';
  frame-src https://app.spline.design; // Spline 3D
  object-src 'none';
  base-uri 'self';
\end{lstlisting}

\begin{infobox}[title={\textbf{Trudność wdrożenia CSP}}]
Konfiguracja CSP w~Next.js z~Tailwind CSS (\texttt{style-src 'unsafe-inline'}) i~Spline (\texttt{frame-src}) wymaga starannego testowania. Alternatywa: CSP w~trybie \texttt{report-only} na 2--4 tygodnie, aby zidentyfikować naruszenia przed włączeniem enforcing mode.
\end{infobox}

\subsubsection{Strict-Transport-Security --- rekomendacja}

\begin{lstlisting}[style=podcode, caption={Proponowany nagłówek HSTS}, label={lst:sec-hsts-proposal}]
Strict-Transport-Security: max-age=31536000; includeSubDomains
\end{lstlisting}

Cloud Run obsługuje HTTPS natywnie, ale HSTS zapobiega downgrade'owi na HTTP w~przypadku przekierowania.


\subsection{Rate limiting w~środowisku serverless}
\label{sec:sec-rate-limit-serverless}

\subsubsection{Obecna implementacja}

Rate limiter \podtekst (\filepath{src/lib/rate-limit.ts}) używa \tstype{Map<string, \{count, resetTime\}>} w~pamięci procesu:

\begin{lstlisting}[style=podcode, caption={Rate limiter --- stan aktualny (wyłączony)}, label={lst:sec-rate-limit-current}]
const rateLimitMap = new Map<string, {
  count: number; resetTime: number;
}>();

export function rateLimit(_limit: number, _windowMs: number) {
  // TODO: re-enable rate limiting before production
  return function checkRateLimit(_ip: string) {
    return { allowed: true };  // ZAWSZE true
  };
}
\end{lstlisting}

\subsubsection{Dwa problemy}

\begin{description}
  \item[\danger{Problem 1: wyłączony}] Funkcja \tsfunc{rateLimit()} ignoruje parametry \tstype{\_limit} i~\tstype{\_windowMs} (prefiks \texttt{\_} = nieużywane) i~zawsze zwraca \tstype{\{allowed: true\}}. Komentarz TODO sugeruje, że jest to tymczasowe wyłączenie na czas developmentu --- ale kod działa w~produkcji (Cloud Run).

  \item[\danger{Problem 2: in-memory Map w~serverless}] Nawet po przywróceniu logiki, \tstype{Map} in-memory \textbf{nie działa} w~środowisku Cloud Run:
\end{description}

\begin{figure}[H]
\centering
\begin{tikzpicture}[
  node distance=0.8cm and 1.5cm,
  every node/.style={font=\small},
  >=Stealth,
]

% Container instances
\node[podbox blue, minimum width=3cm] (c1) {Kontener A\\{\scriptsize Map: 0 req}};
\node[podbox blue, minimum width=3cm, right=2cm of c1] (c2) {Kontener B\\{\scriptsize Map: 0 req}};
\node[podbox blue, minimum width=3cm, right=2cm of c2] (c3) {Kontener C\\{\scriptsize Map: 0 req}};

% Load balancer
\node[podbox purple, minimum width=10cm, above=1.5cm of c2] (lb) {Cloud Run Load Balancer};

% User
\node[podbox amber, minimum width=3cm, above=1.5cm of lb] (user) {Atakujący\\{\scriptsize 15 requestów}};

% Arrows
\draw[dataarrow] (user) -- (lb);
\draw[podarrow] (lb) -- (c1) node[midway, left, font=\scriptsize] {5 req};
\draw[podarrow] (lb) -- (c2) node[midway, left, font=\scriptsize] {5 req};
\draw[podarrow] (lb) -- (c3) node[midway, right, font=\scriptsize] {5 req};

% Annotation
\node[font=\scriptsize\color{PodDanger}, below=0.3cm of c2] {\danger{Każdy kontener widzi tylko 5 req --- limit 5/10min = nigdy nieprzekroczony}};

\end{tikzpicture}
\caption{Problem rate limitingu in-memory w~środowisku Cloud Run. Trzy instancje = trzykrotny efektywny limit.}
\label{fig:sec-rate-limit-serverless}
\end{figure}

\subsubsection{Mechanizm problemu}

\begin{enumerate}
  \item Cloud Run uruchamia \textbf{stateless kontenery} --- każdy cold start tworzy nową instancję z~pustą \tstype{Map}.
  \item Przy wielu jednoczesnych żądaniach Cloud Run skaluje się do N~kontenerów --- każdy z~własnym, niezależnym licznikiem.
  \item Atakujący z~jednego IP wysyła 15 żądań. Load balancer rozkłada je na 3 kontenery po 5. Każdy kontener widzi 5 requestów --- poniżej limitu 5/10min. \textbf{Wszystkie 15 przechodzi.}
  \item Nawet w~jednym kontenerze --- po minutach bezczynności Cloud Run może go zamknąć. Nowy cold start = \tstype{Map} = pusta.
\end{enumerate}

\subsubsection{Rekomendacja: Upstash Redis}

\begin{lstlisting}[style=podcode, caption={Proponowany rate limiter z~Upstash Redis}, label={lst:sec-upstash-rate-limit}]
import { Ratelimit } from '@upstash/ratelimit';
import { Redis } from '@upstash/redis';

const redis = new Redis({
  url: process.env.UPSTASH_REDIS_REST_URL!,
  token: process.env.UPSTASH_REDIS_REST_TOKEN!,
});

export const analyzeRateLimit = new Ratelimit({
  redis,
  limiter: Ratelimit.slidingWindow(5, '10 m'),
  analytics: true,
  prefix: 'podtekst:rate-limit',
});
\end{lstlisting}

\begin{itemize}
  \item \textbf{Upstash Redis} --- serverless Redis z~REST API, darmowy tier (10\,000 req/dzień)
  \item \textbf{Sliding window} --- algorytm okna przesuwnego zamiast fixed window (bardziej sprawiedliwy)
  \item \textbf{Współdzielony state} --- wszystkie instancje Cloud Run korzystają z~tego samego Redis
  \item \textbf{Wbudowana analityka} --- dashboard z~rate-limit events
\end{itemize}


\subsection{Checklist RODO}

\begin{table}[H]
\centering
\caption{Checklist zgodności z~RODO --- stan na luty 2026}
\label{tab:sec-gdpr-checklist}
\renewcommand{\arraystretch}{1.4}
\begin{tabularx}{\textwidth}{C{0.8cm}L{4.5cm}C{2cm}X}
\toprule
\textbf{Nr} & \textbf{Wymóg RODO} & \textbf{Status} & \textbf{Szczegóły} \\
\midrule
\rowcolor{PodSuccess!5}
1 & Minimalizacja danych & \score{Spełniony} & $<$1\% wiadomości wysyłanych na serwer \\
\rowcolor{PodSuccess!5}
2 & Brak retencji na serwerze & \score{Spełniony} & Dane w~pamięci tylko podczas przetwarzania \\
\rowcolor{PodSuccess!5}
3 & Cookie consent & \score{Spełniony} & \tstype{CookieConsent.tsx}, GA4 warunkowo \\
\rowcolor{PodSuccess!5}
4 & Prawo do usunięcia (Art.~17) & \score{Spełniony} & \tsfunc{deleteAnalysis()} --- atomowe usunięcie \\
\rowcolor{PodSuccess!5}
5 & Prawo do przenoszenia (Art.~20) & \score{Spełniony} & Eksport PDF \\
\rowcolor{PodSuccess!5}
6 & Przetwarzanie lokalne & \score{Spełniony} & IndexedDB, brak server-side storage \\
\midrule
\rowcolor{PodDanger!5}
7 & Polityka prywatności & \danger{Brak} & Brak strony \texttt{/privacy} \\
\rowcolor{PodDanger!5}
8 & Endpoint usunięcia danych & \danger{Brak} & Brak API \texttt{DELETE /api/user-data} \\
\rowcolor{PodDanger!5}
9 & Eksport danych (maszynowy) & \danger{Brak} & PDF nie jest formatem maszynowym (brak JSON export) \\
\rowcolor{PodWarning!5}
10 & Informacja o~przetwarzaniu AI & \warn{Częściowy} & Brak jawnej informacji, że wiadomości trafiają do Google Gemini \\
\rowcolor{PodWarning!5}
11 & Okres retencji cookies & \warn{Nieokreślony} & \tstype{CookieConsent} nie informuje o~okresie ważności GA4 cookies \\
\rowcolor{PodWarning!5}
12 & DPO (Data Protection Officer) & \warn{Nie dotyczy} & Wymagany przy przetwarzaniu na dużą skalę (przyszłość) \\
\bottomrule
\end{tabularx}
\end{table}


\subsection{Raport podatności}

\subsubsection{jsPDF --- 3 podatności HIGH}

Pakiet \texttt{jspdf} w~wersji \texttt{\^{}4.1.0} (plik \filepath{package.json:23}) posiada trzy znane podatności o~poziomie \textbf{HIGH}:

\begin{table}[H]
\centering
\caption{Podatności jsPDF 4.1.0}
\label{tab:sec-jspdf-vulns}
\renewcommand{\arraystretch}{1.3}
\begin{tabularx}{\textwidth}{L{3cm}L{3.5cm}C{1.5cm}X}
\toprule
\textbf{CVE / GHSA} & \textbf{Typ} & \textbf{Severity} & \textbf{Opis} \\
\midrule
\rowcolor{PodDanger!5}
GHSA-p5xg-68wr-hm3m & PDF Injection & \danger{HIGH} & Wstrzyknięcie obiektów PDF przez AcroForm RadioButton \\
\rowcolor{PodDanger!5}
GHSA-9vjf-qc39-jprp & Object Injection & \danger{HIGH} & Wstrzyknięcie obiektów PDF via \tsfunc{addJS()} \\
\rowcolor{PodDanger!5}
GHSA-67pg-wm7f-q7fj & DoS & \danger{HIGH} & Denial of Service przez spreparowane wymiary GIF \\
\bottomrule
\end{tabularx}
\end{table}

\begin{infobox}[title={\textbf{Naprawa}}]
Aktualizacja do \texttt{jspdf@4.2.0+} rozwiązuje wszystkie trzy podatności. Wymagane zmiany w~kodzie: brak --- API jest kompatybilne.

\begin{lstlisting}[style=podcodeBash]
pnpm update jspdf@latest
\end{lstlisting}
\end{infobox}

\subsubsection{Brak \texttt{dangerouslySetInnerHTML}}

Przeszukanie całego katalogu \filepath{src/} potwierdza \textbf{zero} wystąpień \tsfunc{dangerouslySetInnerHTML}. React auto-escaping chroni przed XSS w~warstwie renderingu.


\subsection{Ryzyko: token Discord w~body requestu}
\label{sec:sec-discord-token-risk}

Endpoint \texttt{/api/discord/fetch-messages} przyjmuje \texttt{botToken} w~body HTTP requestu:

\begin{lstlisting}[style=podcode, caption={Akceptacja botToken z~body requestu}, label={lst:sec-discord-token}]
// src/app/api/discord/fetch-messages/route.ts:43-58
let body: {
  botToken?: string;
  channelId?: string;
  messageLimit?: number;
};
// ...
const botToken = (body.botToken && body.botToken.length >= 50)
  ? body.botToken
  : process.env.DISCORD_BOT_TOKEN;
\end{lstlisting}

\subsubsection{Wektor ataku}

\begin{enumerate}
  \item Użytkownik wkleja swój Discord Bot Token w~UI (\tstype{DiscordImport.tsx}).
  \item Token jest przechowywany w~React state (pamięć przeglądarki).
  \item Token jest wysyłany w~body POST do \texttt{/api/discord/fetch-messages}.
  \item Jeśli strona jest podatna na XSS (np.\ przez wstrzyknięty skrypt w~extension'ie przeglądarki):
  \begin{itemize}
    \item Atakujący może odczytać token z~React state lub przechwycić request
    \item Token umożliwia dostęp do \textbf{wszystkich kanałów} serwera Discord
  \end{itemize}
\end{enumerate}

\subsubsection{Ocena ryzyka}

\begin{description}
  \item[Prawdopodobieństwo:] \warn{Niskie} --- wymaga XSS na stronie \podtekst (brak \tsfunc{dangerouslySetInnerHTML}, CSP planowany)
  \item[Wpływ:] \danger{Wysoki} --- skradziony bot token daje pełny dostęp do API Discorda
  \item[Mitigacja obecna:] Token nie jest zapisywany w~localStorage ani IndexedDB --- istnieje tylko w~pamięci React state podczas sesji
\end{description}

\subsubsection{Rekomendacja}

\begin{enumerate}
  \item \textbf{Preferować server-side token:} Używać \tstype{process.env.DISCORD\_BOT\_TOKEN} zamiast przyjmowania tokenu od klienta. Endpoint już obsługuje fallback do env var.
  \item \textbf{Szyfrowanie w~transit:} Jeśli konieczne jest przyjęcie tokenu od użytkownika --- szyfrowanie asymetryczne (public key na kliencie, private key na serwerze).
  \item \textbf{Jednorazowy token:} Po pobraniu wiadomości --- natychmiast wyczyścić token z~React state.
\end{enumerate}


\subsection{Brak autentykacji}

\begin{warningbox}[title={\danger{Brak jakiejkolwiek warstwy autentykacji}}]
Na dzień audytu \podtekst nie posiada:
\begin{itemize}
  \item Systemu użytkowników (brak rejestracji, logowania)
  \item Sesji (brak cookies sesyjnych, brak JWT)
  \item Uprawnień (brak ról, tierów, permisji)
  \item Identyfikacji użytkownika (rate limit per IP, nie per user)
\end{itemize}
Każdy, kto zna URL endpointu API, może wysyłać żądania bez ograniczeń (rate limit wyłączony). W~kontekście monetyzacji (patrz \secref{sec:model-freemium}) oznacza to, że paywall oparty na \tstype{TierContext} (localStorage) można obejść bezpośrednim wywołaniem API.
\end{warningbox}


\subsection{Priorytety napraw bezpieczeństwa}

\begin{table}[H]
\centering
\caption{Priorytety napraw bezpieczeństwa}
\label{tab:sec-fix-priorities}
\renewcommand{\arraystretch}{1.3}
\begin{tabularx}{\textwidth}{C{0.8cm}L{4.5cm}C{1.5cm}C{2cm}X}
\toprule
\textbf{Nr} & \textbf{Problem} & \textbf{Severity} & \textbf{Trudność} & \textbf{Priorytet} \\
\midrule
\rowcolor{PodDanger!5}
1 & Przywrócić rate limiting & \danger{CRITICAL} & Łatwa & P0 \\
\rowcolor{PodDanger!5}
2 & Migracja rate limit na Redis & \danger{HIGH} & Średnia & P0 \\
\rowcolor{PodDanger!5}
3 & Aktualizacja jsPDF do 4.2.0+ & \danger{HIGH} & Łatwa & P0 \\
\rowcolor{PodDanger!5}
4 & Dodanie polityki prywatności & \danger{HIGH} & Średnia & P0 \\
\midrule
\rowcolor{PodWarning!5}
5 & Content-Security-Policy & \warn{MEDIUM} & Trudna & P1 \\
\rowcolor{PodWarning!5}
6 & Strict-Transport-Security & \warn{MEDIUM} & Łatwa & P1 \\
\rowcolor{PodWarning!5}
7 & Deep validation \tstype{samplesSchema} & \warn{MEDIUM} & Średnia & P1 \\
\rowcolor{PodWarning!5}
8 & Informacja o~przetwarzaniu AI & \warn{MEDIUM} & Łatwa & P1 \\
\midrule
9 & Discord token risk mitigation & LOW & Średnia & P2 \\
10 & JSON export (Art.~20 RODO) & LOW & Średnia & P2 \\
11 & Centralizacja schematów Zod & LOW & Łatwa & P2 \\
\bottomrule
\end{tabularx}
\end{table}

\begin{featurebox}
\textbf{Podsumowanie bezpieczeństwa:} Architektura client-first \podtekst zapewnia silne domyślne zabezpieczenia --- surowe wiadomości nigdy nie opuszczają przeglądarki, brak \tsfunc{dangerouslySetInnerHTML}, brak retencji danych na serwerze. Jednak \textbf{trzy krytyczne braki} (wyłączony rate limit, brak CSP, brak autentykacji) oraz \textbf{trzy podatności HIGH} w~jsPDF wymagają natychmiastowej uwagi przed skalowaniem produktu.
\end{featurebox}   % Unit economics kosztów AI + Audyt bezpieczeństwa

% ============================================================
\section{Audyt Mobile UX}
\label{sec:mobile-ux}
% ============================================================

Mobilne doswiadczenie uzytkownika stanowi krytyczny wektor wzrostu --- ponad 60\% ruchu na aplikacjach SaaS B2C pochodzi z~urzadzen mobilnych. Niniejsza sekcja analizuje responsywnosc, ergonomie dotykowa i~wydajnosc \podtekst na urzadzeniach z~ekranami 375--430\,px.


% ────────────────────────────────────────────────────────────
\subsection{Analiza breakpointow}
\label{sec:mobile-breakpoints}

\podtekst wykorzystuje standardowe breakpointy Tailwind CSS~v4 bez customowych rozszerzen:

\begin{table}[H]
\centering
\caption{Breakpointy Tailwind v4 --- wykorzystanie w~projekcie}
\label{tab:mobile-breakpoints}
\begin{tabularx}{\textwidth}{C{2cm}C{2cm}C{2.5cm}X}
\toprule
\textbf{Prefix} & \textbf{Szerokosc} & \textbf{Urzadzenia} & \textbf{Zastosowanie w~\podtekst} \\
\midrule
--- & $<640$\,px & Telefony & Layout bazowy, hero diagonal, bottom bar nav \\
\texttt{sm:} & $\geq640$\,px & Duze telefony & Scroll indicator visibility, minor spacing \\
\texttt{md:} & $\geq768$\,px & Tablety & Przelaczenie: sidebar nav, desktop hero, particle bg \\
\texttt{lg:} & $\geq1024$\,px & Laptopy & Dashboard sidebar, wieksze karty share \\
\texttt{xl:} & $\geq1280$\,px & Desktopy & Max-width containery, pelny layout \\
\bottomrule
\end{tabularx}
\end{table}

\begin{warningbox}[title={\textbf{Brak breakpointa pomiedzy 640--768\,px}}]
Przelaczenie z~mobilnego bottom bar na desktopowy sidebar nastepuje na \texttt{md:768px}. Na tabletach w~orientacji portrait (768\,px) uzytkownik widzi desktop layout, ktory moze byc ciasny. Brak posredniego breakpointa dla tabletow (np.~\texttt{tab:900px}) oznacza, ze interfejs ,,skacze'' z~mobile na desktop bez plynnego przejscia.
\end{warningbox}


% ────────────────────────────────────────────────────────────
\subsection{Analiza paska nawigacji mobilnej}
\label{sec:mobile-tab-bar}

Komponent \tstype{SectionNavigator} (\filepath{src/components/analysis/SectionNavigator.tsx}, linie~118--145) renderuje na mobile dolny pasek nawigacyjny z~horizontalnym scrollem.

\subsubsection{Obliczenie dopasowania na 375\,px}

\begin{table}[H]
\centering
\caption{Dopasowanie tabow na ekranie 375\,px}
\label{tab:mobile-tab-fit}
\begin{tabularx}{\textwidth}{L{3.5cm}C{2.5cm}C{2cm}C{2cm}X}
\toprule
\textbf{Widok} & \textbf{Liczba tabow} & \textbf{Szer. taba} & \textbf{Suma} & \textbf{Miesci sie?} \\
\midrule
Rozmowa 2-os. & 6 & $\sim$70\,px & 420\,px & \warn{Wymaga scrollu} \\
Rozmowa grupowa & 6 & $\sim$70\,px & 420\,px & \warn{Wymaga scrollu} \\
Server view (5+) & 9 & $\sim$70\,px & 630\,px & \danger{68\% ukryte} \\
Idealny (5 tabow) & 5 & $\sim$70\,px & 350\,px & \score{Tak (25\,px zapas)} \\
\bottomrule
\end{tabularx}
\end{table}

\begin{metricbox}
\textbf{Kalkulacja:} Dostepna szerokosc = 375\,px $-$ 2$\times$12\,px (padding \texttt{px-3}) = 351\,px. Pojedynczy tab: ikona~16\,px + gap~4\,px + tekst~$\sim$40\,px + padding~$2\times12$\,px = $\sim$84\,px. W~praktyce krotkie polskie etykiety (,,Przeg.'', ,,AI'', ,,Share'') pozwalaja zmiescic 4--5 tabow bez scrollu.
\end{metricbox}

Aktualna implementacja poprawnie obsluguje overflow:
\begin{itemize}
  \item \texttt{overflow-x-auto} z~\texttt{scrollbar-none} --- scroll bez widocznego scrollbara
  \item Gradienty na krawedziach (\texttt{bg-gradient-to-r/l}) sygnalizuja mozliwosc scrollowania
  \item Bezpieczny margines dolny: \texttt{pb-[max(0.375rem,env(safe-area-inset-bottom))]} --- poprawne zachowanie na iPhone'ach z~notchem
  \item Dotykowe sprzezenie zwrotne: \texttt{active:scale-95 active:opacity-80}
\end{itemize}

\begin{infobox}[title={\textbf{Rekomendacja}}]
Dla server view (9~tabow) rozwazyc \textbf{ikony bez etykiet} na mobile --- redukuje szerokosc taba do $\sim$40\,px, co pozwala zmiescic 8~tabow na 375\,px bez scrollu.
\end{infobox}


% ────────────────────────────────────────────────────────────
\subsection{Audyt celow dotykowych}
\label{sec:mobile-touch-targets}

Wytyczne WCAG~2.1 (Success Criterion 2.5.8) wymagaja minimalnego rozmiaru celu dotykowego \textbf{44$\times$44\,px}. Google Material Design zaleca \textbf{48$\times$48\,px}.

\begin{table}[H]
\centering
\caption{Audyt celow dotykowych --- zgodnosc z~WCAG~2.5.8}
\label{tab:mobile-touch-targets}
\begin{tabularx}{\textwidth}{L{4.5cm}C{2cm}C{2cm}C{1.5cm}X}
\toprule
\textbf{Element} & \textbf{Rozmiar} & \textbf{Minimum} & \textbf{Wynik} & \textbf{Uwagi} \\
\midrule
Nawigacja mobilna (tab) & $\sim$84$\times$36\,px & 44$\times$44\,px & \warn{Szer. OK} & Wysokosc 36\,px $<$ 44\,px \\
Desktop sidebar buttons & 32$\times$32\,px & 44$\times$44\,px & \score{N/D} & Desktop only (\texttt{md:flex}) \\
Przycisk ,,wróć na gore'' & 40$\times$40\,px & 44$\times$44\,px & \warn{Bliski} & 4\,px ponizej minimum \\
Hamburger (Topbar) & 18$\times$18\,px & 44$\times$44\,px & \danger{FAIL} & Brak padding area \\
Standardowe buttony & $\sim$48$\times$48\,px & 44$\times$44\,px & \score{PASS} & \texttt{p-3} lub \texttt{p-4} \\
Share card download & $\sim$44$\times$44\,px & 44$\times$44\,px & \score{PASS} & Minimalne \\
CTA ,,Inicjuj analize'' & $\sim$200$\times$52\,px & 44$\times$44\,px & \score{PASS} & Pelna szerokosc mobile \\
Checkbox AI consent & $\sim$20$\times$20\,px & 44$\times$44\,px & \danger{FAIL} & Bez touch padding \\
Tab switcher (Upload) & $\sim$120$\times$40\,px & 44$\times$44\,px & \warn{Bliski} & Wys. 40\,px \\
\bottomrule
\end{tabularx}
\end{table}

\paragraph{Krytyczne naruszenia:}
\begin{enumerate}[label=\textcolor{PodDanger}{\arabic*.}]
  \item \textbf{Hamburger 18$\times$18\,px} (\filepath{src/components/shared/Topbar.tsx}) --- ikona renderowana jako \texttt{<Menu size=\{18\} />} bez dodatkowego paddingu. Uzytkownik musi trafic w~kwadrat 18\,px, co jest 6$\times$ mniejsze niz minimum WCAG. Naprawa: dodac \texttt{p-3} do otaczajacego \texttt{<button>}.
  \item \textbf{Checkbox AI consent} --- natywny \texttt{<input type="checkbox">} bez customowego hit area. Naprawa: opakowac w~\texttt{<label>} z~odpowiednim paddingiem.
\end{enumerate}


% ────────────────────────────────────────────────────────────
\subsection{Wplyw Framer Motion na mobile}
\label{sec:mobile-framer-impact}

Animacje \tsfunc{whileInView} z~biblioteki Framer Motion tworza osobne instancje \tstype{IntersectionObserver} per animowany element.

\begin{table}[H]
\centering
\caption{IntersectionObserver --- rozklad na stronie analizy (wszystkie zrodla)}
\label{tab:mobile-io-breakdown}
\begin{tabularx}{\textwidth}{L{5cm}C{2.5cm}C{2.5cm}X}
\toprule
\textbf{Zrodlo} & \textbf{Instancje} & \textbf{RAM (est.)} & \textbf{Plik} \\
\midrule
\tsfunc{whileInView} w~page.tsx & 37 & 7.4--18.5\,MB & page.tsx \\
EmojiReactions & 1 & 0.2--0.5\,MB & EmojiReactions.tsx:54 \\
IntimacyChart & 2 & 0.4--1.0\,MB & IntimacyChart.tsx:86,181 \\
LongitudinalDelta & 1 & 0.2--0.5\,MB & LongitudinalDelta.tsx:178 \\
SentimentChart & 1 & 0.2--0.5\,MB & SentimentChart.tsx:76 \\
TimelineChart & 1 & 0.2--0.5\,MB & TimelineChart.tsx:116 \\
TopWordsCard & 1 & 0.2--0.5\,MB & TopWordsCard.tsx:85 \\
WeekdayWeekendCard & 1 & 0.2--0.5\,MB & WeekdayWeekendCard.tsx:72 \\
SectionNavigator & 6--9 & 1.2--4.5\,MB & SectionNavigator.tsx \\
\midrule
\textbf{Razem} & \textbf{51--54} & \textbf{10.2--27\,MB} & --- \\
\bottomrule
\end{tabularx}
\end{table}

\begin{warningbox}[title={\textbf{Wplyw na urzadzenia mobilne}}]
Typowy smartfon z~4\,GB RAM dysponuje $\sim$1.5--2\,GB dla przegladarki. Przy 27\,MB samych IntersectionObserver + DOM strony analizy ($\sim$50\,MB) + Recharts SVG ($\sim$15\,MB), laczne zuzycie pamieci moze osiagnac \textbf{90--120\,MB} --- blisko progu, przy ktorym iOS Safari rozpoczyna agresywne odzyskiwanie pamieci (\emph{tab reloading}).

Na Androidzie z~Chrome problem jest mniej ostry (wiekszy limit per tab), ale uzytkownicy moga odczuwac \textbf{jank} (stuttering) podczas scrollowania przez animowane sekcje.
\end{warningbox}

\paragraph{Strategia naprawy:}
\begin{enumerate}
  \item \textbf{Architektura tabowa} (\secref{sec:architektura-tabowa}) --- redukcja do $\sim$8--10 obserwatorow per aktywny tab
  \item \textbf{Shared IntersectionObserver} --- jeden observer z~\texttt{threshold: [0, 0.5, 1]} zamiast osobnych per element
  \item \textbf{CSS-only animations na mobile} --- \texttt{@media (prefers-reduced-motion: no-preference)} z~CSS \texttt{animation-timeline: view()} zamiast JS-based Framer Motion
  \item \textbf{Wylaczenie animacji na low-end} --- detekcja via \tsfunc{navigator.deviceMemory} (Chrome) lub \texttt{navigator.hardwareConcurrency $\leq$ 4}
\end{enumerate}


% ────────────────────────────────────────────────────────────
\subsection{Problemy html2canvas na Safari/iOS}
\label{sec:mobile-html2canvas}

Projekt uzywa \texttt{html2canvas-pro} v1.6.7 w~komponentach \tstype{StoryShareCard} i~\tsfunc{useCardDownload} do generowania PNG z~kart share. Analiza kodu zrodlowego wykazala \textbf{brak jakichkolwiek workaroundow} dla znanych problemow Safari:

\begin{table}[H]
\centering
\caption{Znane problemy html2canvas na Safari/iOS}
\label{tab:mobile-html2canvas-issues}
\begin{tabularx}{\textwidth}{L{3.5cm}C{2cm}X}
\toprule
\textbf{Problem} & \textbf{Dotkliwosc} & \textbf{Opis i~naprawa} \\
\midrule
CORS proxy images & \danger{Wysoka} & Safari blokuje rasteryzacje obrazow z~innych domen. Brak opcji \texttt{useCORS: true} ani \texttt{allowTaint: true} w~konfiguracji. \\
SVG rendering & \warn{Srednia} & Ikony Lucide renderowane jako \texttt{<svg>} moga nie byc przechwycone poprawnie. Naprawa: \texttt{foreignObjectRendering: true}. \\
Retina scaling & \warn{Srednia} & Brak \texttt{scale: window.devicePixelRatio} --- karty moga byc rozmyte na Retina (@2x/@3x). \\
Canvas memory limit & \danger{Wysoka} & iOS Safari limituje canvas do $\sim$16\,MP. Duze karty (np.~ReceiptCard z~50+ liniami) moga przekroczyc limit. \\
Font rendering & \warn{Niska} & Custom fonty (Syne, Geist) moga nie byc zaladowane w~momencie rasteryzacji --- \texttt{onclone} callback z~\tsfunc{document.fonts.ready}. \\
\bottomrule
\end{tabularx}
\end{table}

\begin{infobox}[title={\textbf{Rekomendacja: migracja na Satori + \texttt{@vercel/og}}}]
Alternatywa server-side: generowanie kart jako SVG (Satori) z~konwersja do PNG (Sharp). Eliminuje wszystkie problemy Safari, daje deterministyczne wyniki i~odciaża przegldarke. Wymaga nowych API routes per typ karty, ale jest znacznie bardziej niezawodna.
\end{infobox}


% ────────────────────────────────────────────────────────────
\subsection{Flow uploadu na mobile}
\label{sec:mobile-upload}

Komponent \tstype{DropZone} (\filepath{src/components/upload/DropZone.tsx}) obsluguje upload plikow konwersacji.

\begin{table}[H]
\centering
\caption{Kompatybilnosc uploadu na platformach mobilnych}
\label{tab:mobile-upload-compat}
\begin{tabularx}{\textwidth}{L{3.5cm}C{2.5cm}C{2.5cm}X}
\toprule
\textbf{Mechanizm} & \textbf{iOS Safari} & \textbf{Android Chrome} & \textbf{Uwagi} \\
\midrule
\texttt{<input type="file">} & \score{Tak} & \score{Tak} & \texttt{accept=".json,.txt"} \\
Drag-and-drop & \warn{Czesciowo} & \score{Tak} & iOS 13+ via \texttt{webkitGetAsEntry} \\
Folder picker & \danger{Nie} & \warn{Czesciowo} & \texttt{webkitdirectory} nieobslugiwany na iOS \\
Multiple files & \score{Tak} & \score{Tak} & Atrybut \texttt{multiple} \\
Rozmiar area (min-h) & \score{200\,px} & \score{200\,px} & Wystarczajacy cel dotykowy \\
\bottomrule
\end{tabularx}
\end{table}

\begin{warningbox}[title={\textbf{Krytyczny problem: folder upload na iOS}}]
Eksport Messenger to folder z~wieloma plikami JSON. Na desktopie uzytkownik moze przeciagnac caly folder. Na iOS \textbf{nie ma takiej mozliwosci} --- \texttt{webkitdirectory} nie jest obslugiwany. Uzytkownik musi recznie wybrac wszystkie pliki JSON z~folderu --- UX jest znaczaco gorszy.

\textbf{Naprawa:} Dodac instrukcje krok-po-kroku dla iOS z~GIF-em pokazujacym jak wybrac wiele plikow w~aplikacji Pliki. Alternatywnie: wspierac upload ZIP z~automatyczna ekstrakcja (via \texttt{JSZip}).
\end{warningbox}


% ────────────────────────────────────────────────────────────
\subsection{Nawigacja mobilna}
\label{sec:mobile-navigation}

\begin{table}[H]
\centering
\caption{Ocena nawigacji mobilnej}
\label{tab:mobile-nav-assessment}
\begin{tabularx}{\textwidth}{L{4cm}C{2cm}X}
\toprule
\textbf{Element} & \textbf{Ocena} & \textbf{Szczegoly} \\
\midrule
Drawer (szuflada) & \score{Dobra} & 280\,px szerokosc, \texttt{z-50}, backdrop blur, safe-area-bottom \\
Topbar & \score{Dobra} & Sticky \texttt{top-0 z-50}, responsywne logo \\
SectionNavigator (mobile) & \warn{Srednia} & Bottom bar z~scrollem, ale 6--9 tabow to duzo \\
Back-to-top & \warn{Srednia} & \texttt{size-10} (40\,px) --- 4\,px ponizej WCAG \\
Pozycja CTA (landing) & \score{Dobra} & \texttt{absolute bottom-[6vh]} --- zawsze widoczne \\
Particle background & \score{OK} & Ukryte na mobile (\texttt{hidden md:block}) \\
Scroll indicator & \score{OK} & Ukryty na malych ekranach (\texttt{hidden sm:block}) \\
\bottomrule
\end{tabularx}
\end{table}

\subsubsection{Landing hero mobile}

Mobilny hero (\filepath{src/components/landing/LandingHero.tsx}, linia~133) uzywa osobnego layoutu \texttt{md:hidden}:

\begin{itemize}
  \item \textbf{Typografia diagonalna} --- \texttt{transform: rotate(-2deg)} z~responsywnym \texttt{clamp(2.2rem, 8vw, 3.5rem)}
  \item \textbf{Animacja per-word} --- \texttt{heroFadeSlideLeft} z~kaskadowym opoznieniem (\texttt{0.05 + i * 0.04}s)
  \item \textbf{CTA przypiete do dolu} --- \texttt{absolute bottom-[6vh] left-6 right-6}
  \item \textbf{Brak ciezkich efektow} --- particle background i~Spline 3D ukryte na mobile
\end{itemize}

\begin{featurebox}
\textbf{Pozytywne aspekty mobile UX:} Diagnostycznie, mobilny hero jest dobrze zrobiony --- osobny layout unika problemow z~responsywnoscia desktopowego hero, a CTA jest zawsze widoczne. Glowne problemy koncentruja sie na stronie analizy (IntersectionObserver, touch targets) i~upload flow (brak folder picker na iOS).
\end{featurebox}

\subsubsection{Podsumowanie Mobile UX}

\begin{figure}[H]
\centering
\begin{tikzpicture}[
  catbox/.style={draw=PodBorder, fill=white, rounded corners=4pt,
    minimum width=3.5cm, minimum height=0.8cm, align=center, font=\small},
  score/.style={font=\small\bfseries},
  >=Stealth
]

% Categories
\node[catbox, fill=PodSuccess!10, draw=PodSuccess!50] (c1) at (0,0) {Landing mobile};
\node[catbox, fill=PodSuccess!10, draw=PodSuccess!50] (c2) at (4.5,0) {Nawigacja};
\node[catbox, fill=PodWarning!10, draw=PodWarning!50] (c3) at (9,0) {Touch targets};
\node[catbox, fill=PodDanger!10, draw=PodDanger!50] (c4) at (0,-1.5) {IntersectionObserver};
\node[catbox, fill=PodDanger!10, draw=PodDanger!50] (c5) at (4.5,-1.5) {Upload iOS};
\node[catbox, fill=PodWarning!10, draw=PodWarning!50] (c6) at (9,-1.5) {html2canvas};

% Scores
\node[score, text=PodSuccess] at (0,-0.7) {8/10};
\node[score, text=PodSuccess] at (4.5,-0.7) {7/10};
\node[score, text=PodWarning] at (9,-0.7) {5/10};
\node[score, text=PodDanger] at (0,-2.2) {3/10};
\node[score, text=PodDanger] at (4.5,-2.2) {3/10};
\node[score, text=PodWarning] at (9,-2.2) {4/10};

\end{tikzpicture}
\caption{Podsumowanie ocen Mobile UX per kategoria}
\label{fig:mobile-ux-summary}
\end{figure}


% ============================================================
\section{Onboarding i~retencja}
\label{sec:onboard-audyt}
% ============================================================

Najlepsza aplikacja analityczna nie ma wartosci, jesli uzytkownik nie przejdzie od wejscia na strone do pierwszego ,,wow moment''. Niniejsza sekcja analizuje sciezke onboardingu, mechanizmy retencji i~punkty tarcia (friction points) w~\podtekst.


% ────────────────────────────────────────────────────────────
\subsection{Flow uzytkownika: od landing do wynikow}
\label{sec:onboard-flow}

\begin{figure}[H]
\centering
\begin{tikzpicture}[
  step/.style={draw=PodBlue!60, fill=PodBlue!8, rounded corners=6pt,
    minimum height=1.2cm, minimum width=2.8cm, align=center, font=\small\bfseries,
    text=PodBlueDark},
  optional/.style={step, draw=PodPurple!50, fill=PodPurple!6, text=PodPurpleDark},
  dropoff/.style={font=\tiny\color{PodDanger}, align=center},
  time/.style={font=\tiny\color{PodSuccess}, align=center},
  >=Stealth
]

% Steps
\node[step] (s1) at (0,0) {Landing\\page.tsx};
\node[step] (s2) at (3.5,0) {CTA click\\,,Inicjuj''};
\node[step] (s3) at (7,0) {Upload\\DropZone};
\node[step] (s4) at (10.5,0) {Parsing\\(client)};
\node[step] (s5) at (3.5,-3) {Wyniki\\ilosciowe};
\node[optional] (s6) at (7,-3) {AI Analysis\\(opcjonalne)};
\node[optional] (s7) at (10.5,-3) {Pelne\\wyniki};

% Arrows
\draw[dataarrow] (s1) -- (s2);
\draw[dataarrow] (s2) -- (s3);
\draw[dataarrow] (s3) -- (s4);
\draw[dataarrow] (s4) -- ++(0,-1.2) -| (s5);
\draw[dataarrow] (s5) -- (s6);
\draw[dataarrow] (s6) -- (s7);

% Drop-off estimates (heuristic, no data)
\node[dropoff] at (1.75,0.8) {$\sim$40\%*};
\node[dropoff] at (5.25,0.8) {$\sim$25\%*};
\node[dropoff] at (8.75,0.8) {$\sim$15\%*};
\node[dropoff] at (5.25,-2.2) {$\sim$30\%*};

% Time annotations
\node[time] at (0,-0.9) {$t=0$s};
\node[time] at (3.5,-0.9) {$t \approx 5$s};
\node[time] at (7,-0.9) {$t \approx 15$s};
\node[time] at (10.5,-0.9) {$t \approx 16$s};
\node[time] at (3.5,-3.9) {$t \approx 17$s};
\node[time] at (7,-3.9) {$t \approx 60$s};
\node[time] at (10.5,-3.9) {$t \approx 120$s};

% Curtain annotation
\node[font=\tiny\itshape\color{PodTextMuted}, anchor=north] at (0,-1.2) {CurtainReveal\\$\sim$3$s$ animacja};

% Interaction count
\node[font=\scriptsize\color{PodBlue}, anchor=south, fill=PodBlue!5, rounded corners=2pt, inner sep=3pt]
  at (5.25,-4.5) {Calkowite interakcje do pierwszych wynikow: \textbf{4} (CTA $\rightarrow$ upload $\rightarrow$ parse $\rightarrow$ view)};

\end{tikzpicture}
\caption{Flow uzytkownika z~heurystycznymi szacunkami drop-off (*bez danych --- patrz zastrzezenie ponizej)}
\label{fig:onboard-flow-diagram}
\end{figure}

\begin{warningbox}[title={\danger{*Zastrzezenie: szacunki bez danych}}]
Wszystkie wartosci drop-off na powyzszym diagramie to \textbf{heurystyki benchmarkowe} oparte na typowych lejkach SaaS B2C, \textbf{nie na danych z~\podtekst}. GA4 jest zaimplementowane, ale nie sledzi lejka konwersji --- brak event trackingu na: upload rozpoczety, upload zakonczony, AI trigger, AI complete. Ponizsze wartosci nalezy traktowac jako \textbf{hipotezy do walidacji}, nie fakty.
\end{warningbox}

\begin{metricbox}
\textbf{Hipotetyczny lejek konwersji (do walidacji):} Z~1000 uzytkownikow wchodzacych na landing, $\sim$600 klika CTA, $\sim$450 uploaduje plik, $\sim$383 widzi wyniki ilosciowe, $\sim$268 uruchamia AI. Hipotetyczna konwersja landing $\rightarrow$ AI: \textbf{$\sim$27\%}. Glowne punkty tarcia (hipotezy): brak pliku eksportu i~oczekiwanie na AI ($\sim$60$s$). \textbf{Wymagana walidacja:} wdrozenie GA4 funnel events przed podjęciem decyzji optymalizacyjnych.
\end{metricbox}


% ────────────────────────────────────────────────────────────
\subsection{Analiza danych demo}
\label{sec:onboard-demo}

Plik \filepath{src/components/landing/demo-card-data.tsx} (682~LOC) zawiera kompletny zestaw danych demo:

\begin{table}[H]
\centering
\caption{Dane demo --- zawartosc i~wykorzystanie}
\label{tab:onboard-demo-data}
\begin{tabularx}{\textwidth}{L{4cm}C{3cm}X}
\toprule
\textbf{Element} & \textbf{Wartosc} & \textbf{Wykorzystanie} \\
\midrule
Fikcyjna para & ,,Ania'' \& ,,Kuba'' & Realistyczne polskie imiona \\
Liczba wiadomosci & 12\,847 & 11~mies. (III.2024 -- II.2025) \\
QuantitativeAnalysis & Kompletna & Wszystkie 60+ metryk \\
QualitativeAnalysis & Kompletna & Wszystkie 4~passy + roast \\
CPS, Subtext, Court & Kompletne & Pelne dane entertainment \\
Dating Profile & Kompletny & Profile obu osob \\
Delusion Quiz & Kompletny & Wyniki z~Delusion Index \\
Share cards & 20+ typow & Interaktywne w~LandingDemo \\
\bottomrule
\end{tabularx}
\end{table}

\subsubsection{Co dziala dobrze}

\begin{itemize}
  \item \textbf{Natychmiastowa gratyfikacja} --- uzytkownik widzi pelne, realistyczne wyniki bez uploadu
  \item \textbf{Interaktywnosc} --- 20+ kart share mozna klikac, pobierac, udostepniac
  \item \textbf{Wiarygodnosc danych} --- 12\,847 wiadomosci z~realistycznymi wzorcami czasowymi
  \item \textbf{Pelnosc} --- pokrywa wszystkie funkcje aplikacji, wlacznie z~AI i~entertainment
\end{itemize}

\subsubsection{Czego brakuje}

\begin{warningbox}[title={\textbf{Brak konwersji demo $\rightarrow$ wlasna analiza}}]
Po interakcji z~demo kartami uzytkownik nie widzi \textbf{zadnego CTA} kierujacego do uploadu wlasnej rozmowy. Demo pokazuje koncowy rezultat, ale nie prowadzi uzytkownika dalej. Brakuje:
\begin{itemize}
  \item Przycisku ,,Analizuj swoja rozmowe'' pod/obok demo kart
  \item Porownania ,,Ania \& Kuba vs Twoja rozmowa'' zachecajacego do uploadu
  \item Notki ,,To dane demo. Twoje wyniki beda unikalne!''
  \item Animacji przejscia z~demo do uploadu (np.~morph kart demo w~puste karty z~,,?'')
\end{itemize}
\end{warningbox}


% ────────────────────────────────────────────────────────────
\subsection{Audyt stanow ladowania}
\label{sec:onboard-loading-states}

Stany ladowania podczas analizy AI (\filepath{src/components/analysis/AIAnalysisButton.tsx}, linie 26--531):

\begin{table}[H]
\centering
\caption{Stany ladowania analizy AI}
\label{tab:onboard-loading-states}
\begin{tabularx}{\textwidth}{C{1cm}L{3.5cm}L{4.5cm}C{2cm}C{1.5cm}}
\toprule
\textbf{Krok} & \textbf{Etykieta PL} & \textbf{Opis dzialania} & \textbf{Stan wizualny} & \textbf{Ocena} \\
\midrule
1 & ,,Czytam miedzy wierszami...'' & Pass 1: ton i~styl & Spinner + tekst & \score{Dobry} \\
2 & ,,Mapuje dynamike konwersacji...'' & Pass 2: dynamika relacji & Progress step & \score{Dobry} \\
3 & ,,Profiluje osobowosci...'' & Pass 3: profile indywidualne & Progress step & \score{Dobry} \\
4 & ,,Wyciagam wnioski. Przygotuj sie.'' & Pass 4: synteza i~health score & Progress step & \score{Dobry} \\
\midrule
Blad & Czerwony alert & Komunikat bledu + retry & Alert box & \score{Dobry} \\
Heartbeat & (niewidoczny) & SSE heartbeat co 15$s$ & --- & \score{OK} \\
Consent & Checkbox + opis & Zgoda AI (persisted) & Inline & \warn{Sredni} \\
\bottomrule
\end{tabularx}
\end{table}

\begin{featurebox}
\textbf{Pozytywne:} Etykiety ladowania sa kreatywne i~zgodne z~brandem (,,Czytam miedzy wierszami'' nawiazuje do tagline). Kazdy krok ma wizualny progres (pending $\rightarrow$ running $\rightarrow$ complete). Hook \tsfunc{useCPSAnalysis} implementuje plynna interpolacje co 150\,ms.

\textbf{Do poprawy:} Brak \textbf{szacowanego czasu} (,,ok. 45$s$'') i~\textbf{paska postepu z~procentami}. Uzytkownik nie wie, ile musi czekac. CPS consent checkbox jest maly i~latwo przeoczyc.
\end{featurebox}


% ────────────────────────────────────────────────────────────
\subsection{Brakujace elementy onboardingu}
\label{sec:onboard-missing}

Analiza kodu zrodlowego wykazala \textbf{calkowity brak} standardowych elementow onboardingu:

\begin{table}[H]
\centering
\caption{Brakujace elementy onboardingu --- checklist}
\label{tab:onboard-missing-checklist}
\begin{tabularx}{\textwidth}{L{4.5cm}C{2cm}C{2cm}X}
\toprule
\textbf{Element} & \textbf{Stan} & \textbf{Priorytet} & \textbf{Wplyw} \\
\midrule
Welcome modal (1-sze wejscie) & \danger{Brak} & P1 & Orientacja nowego uzytkownika \\
Guided tour / step-by-step & \danger{Brak} & P1 & Redukcja drop-off na upload \\
Tooltips (dismissible) & \danger{Brak} & P2 & Zrozumienie metryk \\
Video tutorial / GIF & \danger{Brak} & P2 & Jak wyeksportowac rozmowe \\
Progress indicator (onboarding) & \danger{Brak} & P2 & Motywacja do ukonczenia \\
Empty state (dashboard) & \warn{Minimalny} & P1 & Dashboard bez analiz jest pusty \\
,,Analizuj kolejna'' CTA & \danger{Brak} & P1 & Retencja po 1-szej analizie \\
Sukces celebration & \warn{Czesciowy} & P3 & \texttt{sessionStorage} per analiza \\
Platform-specific export guide & \danger{Brak} & P0 & \textbf{Najwazniejszy} --- bez pliku nie ma analizy \\
\bottomrule
\end{tabularx}
\end{table}

\begin{warningbox}[title={\danger{Najwazniejszy brak: przewodnik eksportu rozmowy}}]
Uzytkownik musi \textbf{samodzielnie} wiedziec, jak wyeksportowac rozmowe z~Messengera, WhatsAppa, Instagrama lub Telegrama. Nie ma zadnego przewodnika, screenshotow ani instrukcji. To \textbf{najwiekszy punkt tarcia} calego flow --- uzytkownik, ktory nie ma pliku, nie moze uzyc aplikacji.

\textbf{Rekomendacja:} Komponent \tstype{ExportGuide} z~tabami per platforme, krokami i~screencastami. Wyswietlany na stronie upload (\filepath{/analysis/new}) oraz jako link w~sekcji FAQ landing page.
\end{warningbox}


% ────────────────────────────────────────────────────────────
\subsection{Analiza petli retencji}
\label{sec:onboard-retention}

\subsubsection{Obecne mechanizmy retencji}

\begin{table}[H]
\centering
\caption{Mechanizmy retencji --- stan obecny vs idealny}
\label{tab:onboard-retention-current-ideal}
\begin{tabularx}{\textwidth}{L{4cm}C{2.5cm}C{2.5cm}X}
\toprule
\textbf{Mechanizm} & \textbf{Obecny} & \textbf{Idealny} & \textbf{Wplyw na retencje} \\
\midrule
Dashboard z~historia & \score{Tak} & \score{Tak} & Sredni --- powrot do wynikow \\
Share cards (viral loop) & \score{20+ typow} & 20+ z~watermarkiem & \textbf{Wysoki} --- nowi uzytkownicy \\
PDF export & \score{2 typy} & 2+ typy & Niski --- offline, brak powrotu \\
Story/Wrapped mode & \score{Tak} & Tak & Sredni --- efekt ,,wow'' \\
GA4 analytics & \score{Tak} & Tak & --- (tracking, nie retencja) \\
Email notifications & \danger{Brak} & Reminders & \textbf{Wysoki} --- reaktywacja \\
Push notifications & \danger{Brak} & Web push & \textbf{Wysoki} --- reaktywacja \\
Newsletter & \danger{Brak} & Cotygodniowy digest & Sredni --- zaangazowanie \\
,,Analizuj kolejna'' CTA & \danger{Brak} & Post-analiza CTA & \textbf{Wysoki} --- natychmiastowy powrot \\
Porownanie z~czasem & \warn{Czesciowy} & Auto-reminder co miesiac & Wysoki --- longitudinalny \\
Gamifikacja & \warn{Odznaki} & Odznaki + streak + level & Sredni --- zaangazowanie \\
Sharing incentives & \danger{Brak} & Odblokuj feature za share & \textbf{Wysoki} --- viral + retencja \\
\bottomrule
\end{tabularx}
\end{table}

\subsubsection{Diagram petli retencji}

\begin{figure}[H]
\centering
\begin{tikzpicture}[
  loop/.style={draw=PodBlue!60, fill=PodBlue!8, rounded corners=6pt,
    minimum height=1cm, minimum width=2.5cm, align=center, font=\small\bfseries,
    text=PodBlueDark},
  missing/.style={loop, draw=PodDanger!50, fill=PodDanger!5, text=PodDanger!80!black,
    dashed},
  viral/.style={loop, draw=PodPurple!60, fill=PodPurple!8, text=PodPurpleDark},
  >=Stealth
]

% Current loop (inner)
\node[loop] (upload) at (0,0) {Upload};
\node[loop] (analyze) at (3.5,0) {Analiza};
\node[loop] (results) at (7,0) {Wyniki};
\node[viral] (share) at (7,-2.5) {Share Cards};
\node[loop] (dashboard) at (0,-2.5) {Dashboard};

\draw[dataarrow] (upload) -- (analyze);
\draw[dataarrow] (analyze) -- (results);
\draw[dataarrow] (results) -- (share);
\draw[dataarrow] (share) -- (dashboard);
\draw[dataarrow] (dashboard) -- (upload) node[midpoint, font=\tiny\color{PodTextMuted}, above, yshift=2pt] {kolejna rozmowa};

% Viral loop (outward)
\draw[podarrow purple, thick] (share.east) -- ++(1.5,0) node[right, font=\scriptsize\color{PodPurple}] {Social media} -- ++(0,1.5) -- (results.east |- 0,1) node[above, font=\tiny\color{PodPurple}] {nowy uzytkownik};

% Missing elements
\node[missing] (remind) at (3.5,-2.5) {\scriptsize Reminder};
\node[missing] (cta) at (10,-1.25) {\scriptsize ,,Analizuj\\kolejna''};

\draw[podarrow dashed] (results) -- (cta);
\draw[podarrow dashed] (remind) -- (upload);
\draw[podarrow dashed] (dashboard) -- (remind);

% Legend
\node[font=\tiny\color{PodTextMuted}] at (3.5,-4) {--- linia ciagla = zaimplementowane\quad --- linia przerywana = brak};

\end{tikzpicture}
\caption{Petla retencji --- obecna (ciagla) vs brakujace elementy (przerywana)}
\label{fig:onboard-retention-loop}
\end{figure}


% ────────────────────────────────────────────────────────────
\subsection{Punkty tarcia --- ranking wplywu}
\label{sec:onboard-friction}

\begin{table}[H]
\centering
\caption{Punkty tarcia uszeregowane wg wplywu na konwersje}
\label{tab:onboard-friction-ranking}
\begin{tabularx}{\textwidth}{C{1cm}L{4cm}C{2cm}C{2cm}X}
\toprule
\textbf{\#} & \textbf{Punkt tarcia} & \textbf{Etap} & \textbf{Wplyw} & \textbf{Naprawa} \\
\midrule
1 & Brak instrukcji eksportu rozmowy & Pre-upload & \danger{Krytyczny} & \tstype{ExportGuide} z~tab per platforme \\
2 & Brak CTA ,,Analizuj kolejna'' po wynikach & Post-analiza & \danger{Wysoki} & Sticky CTA na dole wynikow \\
3 & Demo nie prowadzi do uploadu & Landing & \danger{Wysoki} & CTA pod demo kartami \\
4 & AI analiza trwa $\sim$60$s$ bez progress \% & Analiza & \warn{Sredni} & Progress bar z~ETA \\
5 & Brak welcome modal na 1-szym wejsciu & Landing & \warn{Sredni} & Krotki 3-step wizard \\
6 & Folder upload nie dziala na iOS & Upload & \warn{Sredni} & Instrukcja + ZIP support \\
7 & Consent checkbox latwy do przeoczenia & AI trigger & \warn{Niski} & Wiekszy, bardziej widoczny \\
8 & Brak tooltipow przy metrykach & Wyniki & \warn{Niski} & Inline \texttt{<Tooltip>} na KPI cards \\
9 & Pusty dashboard (0~analiz) & Dashboard & \warn{Niski} & Empty state z~CTA \\
10 & Brak celebracji 1-szej analizy & Post-analiza & \warn{Niski} & Confetti + ,,Udostepnij wynik!'' \\
\bottomrule
\end{tabularx}
\end{table}

\begin{featurebox}
\textbf{Podsumowanie:} Onboarding \podtekst jest \textbf{minimalny, ale funkcjonalny}. Flow od landing do wynikow wymaga zaledwie 4~interakcji ($\sim$17$s$) --- co jest doskonale. Glowne luki to:
\begin{enumerate}
  \item \textbf{Pre-upload friction} --- uzytkownik musi samodzielnie wiedziec, jak wyeksportowac rozmowe (brak instrukcji)
  \item \textbf{Post-analiza dead end} --- brak CTA do kolejnej analizy, brak zachet do udostepniania
  \item \textbf{Demo $\rightarrow$ upload gap} --- demo pokazuje koncowy rezultat, ale nie konwertuje widzow w~uzytkownikow
  \item \textbf{Zero retencji zewnetrznej} --- brak emaili, push, reminders --- jedyny powrot to share cards (viral) i~pamiec uzytkownika
\end{enumerate}
Implementacja punktow 1--3 z~tabeli tarcia powinna poprawic konwersje landing $\rightarrow$ AI --- skala poprawy nieznana bez baseline'u z~GA4 funnel events.
\end{featurebox}

   % Audyt mobile UX + Audyt onboardingu
% ============================================================
% Sekcja S5: Jakość kodu i Developer Experience
% ============================================================

\section{Jakość kodu i Developer Experience}
\label{sec:jakosc-kodu}

\begin{center}
\Large\itshape\color{PodBlue}
,,Kod bez testów to jak rozmowa bez kontekstu --- niby działa, ale nie wiesz dlaczego.''
\end{center}

\vspace{8pt}

Niniejsza sekcja ocenia jakość techniczną kodu źródłowego \podtekst pod kątem: konfiguracji TypeScript, pokrycia testami, bezpieczeństwa zależności, wzorców obsługi błędów, architektury parserów oraz higieny bazy kodu (TODO/FIXME/HACK).

\subsection{Metryki jakości kodu --- podsumowanie}
\label{sec:code-metryki}

\begin{table}[H]
\centering
\caption{Scorecard jakości kodu \podtekst}
\label{tab:code-scorecard}
\begin{tabularx}{\textwidth}{L{4cm}C{3cm}C{2.5cm}X}
\toprule
\textbf{Metryka} & \textbf{Wartość} & \textbf{Ocena} & \textbf{Komentarz} \\
\midrule
Typy \tstype{any} w~\filepath{src/} & 0 & \score{Doskonale} & Zero unsafe-any w~całym projekcie \\
\tstype{strict: true} w~tsconfig & Tak & \score{Doskonale} & Pełna rygorystyczność kompilatora \\
Pliki testowe & 3 & \danger{Krytycznie} & Tylko parsery przetestowane \\
Pokrycie testami (szac.) & $<$5\% & \danger{Krytycznie} & Brak testów API, komponentów, hooków \\
Zależności produkcyjne & 17 & \score{Dobrze} & Niski footprint \\
Zależności deweloperskie & 10 & \score{Dobrze} & Minimalistyczne \\
Luki bezpieczeństwa (HIGH) & 4 & \danger{Poważne} & jsPDF + minimatch \\
Luki bezpieczeństwa (MODERATE) & 1 & \warn{Umiarkowane} & ajv (ESLint transitive) \\
Luki bezpieczeństwa (LOW) & 1 & \score{Niskie} & hono (shadcn transitive) \\
Bloki \tstype{catch} & 63 & --- & W~35 plikach \\
\tstype{console.error} & 28 & \warn{Do poprawy} & Brak strukturalnego logowania \\
TODO/FIXME/HACK & 1 & \score{Dobrze} & Ale krytyczny: wyłączony rate limiter \\
\bottomrule
\end{tabularx}
\end{table}

\subsubsection{Konfiguracja TypeScript}

Plik \filepath{tsconfig.json} jest skonfigurowany wzorcowo:

\begin{lstlisting}[style=podcode, caption={tsconfig.json --- kluczowe opcje}, label={lst:code-tsconfig}]
{
  "compilerOptions": {
    "target": "ES2017",
    "strict": true,
    "noEmit": true,
    "isolatedModules": true,
    "moduleResolution": "bundler",
    "jsx": "react-jsx",
    "incremental": true,
    "paths": { "@/*": ["./src/*"] }
  }
}
\end{lstlisting}

\begin{itemize}
  \item \tstype{strict: true} --- włącza \tstype{strictNullChecks}, \tstype{noImplicitAny}, \tstype{strictFunctionTypes} i~pozostałe flagi rygorystyczności
  \item \tstype{isolatedModules: true} --- kompatybilność z~bundlerami (SWC/Turbopack)
  \item \tstype{incremental: true} --- przyrostowa kompilacja, szybsze rebuildy
  \item Alias \tstype{@/*} $\rightarrow$ \filepath{./src/*} --- czyste importy bez relatywnych ścieżek
\end{itemize}

\begin{metricbox}
\textbf{Wynik:} Konfiguracja TypeScript zasługuje na ocenę \score{10/10}. Zero kompromisów typowania, zero \tstype{any}, pełna rygorystyczność.
\end{metricbox}


\subsection{Analiza pokrycia testami}
\label{sec:code-testy}

\subsubsection{Stan obecny}

Framework testowy: \textbf{Vitest v4.0.18} z~konfiguracją w~\filepath{vitest.config.ts}:

\begin{lstlisting}[style=podcode, caption={vitest.config.ts}, label={lst:code-vitest}]
import { defineConfig } from 'vitest/config';
import path from 'path';

export default defineConfig({
  test: {
    globals: true,
    environment: 'node',
  },
  resolve: {
    alias: { '@': path.resolve(__dirname, './src') },
  },
});
\end{lstlisting}

Istniejące pliki testowe --- \textbf{tylko 3}:

\begin{table}[H]
\centering
\caption{Istniejące pliki testowe}
\label{tab:code-testy-istniejace}
\begin{tabularx}{\textwidth}{L{7.5cm}C{2cm}X}
\toprule
\textbf{Plik} & \textbf{Moduł} & \textbf{Zakres} \\
\midrule
\filepath{src/lib/parsers/\_\_tests\_\_/detect.test.ts} & Parsery & Auto-detekcja platformy \\
\filepath{src/lib/parsers/\_\_tests\_\_/messenger.test.ts} & Parsery & Parser Messenger JSON \\
\filepath{src/lib/parsers/\_\_tests\_\_/quantitative.test.ts} & Analiza & Metryki ilościowe \\
\bottomrule
\end{tabularx}
\end{table}

\subsubsection{Mapa luk w~pokryciu testami}

\begin{warningbox}[title={\danger{Krytyczne braki testowe}}]
Poniższa tabela przedstawia moduły \textbf{całkowicie pozbawione testów} --- uporządkowane według ryzyka biznesowego:

\begin{table}[H]
\centering
\caption{Moduły bez testów --- analiza luk}
\label{tab:code-brak-testow}
\begin{tabularx}{\textwidth}{L{4cm}C{1.5cm}C{2cm}X}
\toprule
\textbf{Moduł} & \textbf{Pliki} & \textbf{Ryzyko} & \textbf{Opis braku} \\
\midrule
API routes (\filepath{api/}) & 10 & \danger{Krytyczne} & Zero testów SSE, walidacji, rate limitingu \\
Gemini integration & 1 & \danger{Krytyczne} & Brak mock/stub Gemini API \\
Parsery: WA, IG, TG, Discord & 4 & \danger{Wysokie} & Tylko Messenger ma testy \\
Analiza jakościowa & 3 & \warn{Wysokie} & Sampling, context building \\
Moduły \filepath{quant/} & 6 & \warn{Wysokie} & bursts, trends, reciprocity, sentiment \\
Komponenty React & 145+ & \warn{Średnie} & Zero testów komponentów \\
Share cards & 20+ & \warn{Średnie} & Zero testów renderingu kart \\
Hooki (\filepath{hooks/}) & 3 & \warn{Średnie} & useShareCard, useCPSAnalysis, useSubtext \\
PDF export & 2 & \warn{Średnie} & Krytyczna funkcja bez testów \\
Utils i~walidacja & 3 & \score{Niskie} & schemas.ts, utils.ts, encode.ts \\
\bottomrule
\end{tabularx}
\end{table}
\end{warningbox}

\subsubsection{Rekomendacje testowe --- priorytetyzacja}

\begin{enumerate}[label=\textcolor{PodBlue}{\textbf{P\arabic*.}}]
  \item \textbf{Parsery WhatsApp, Instagram, Telegram, Discord} --- parsowanie to fundament aplikacji. Każdy parser powinien mieć minimum 10 test case'ów: poprawne dane, edge case'y (puste wiadomości, emoji, wielojęzyczne), błędne formaty.
  \item \textbf{API routes} --- testy integracyjne z~mockami Gemini. Sprawdzenie: walidacji wejścia (Zod), kodów HTTP (400, 413, 429), SSE event format, abort signal.
  \item \textbf{Moduły \filepath{quant/}} --- czyste funkcje (zero side-effects), idealne do testów jednostkowych. \tsfunc{detectBursts()}, \tsfunc{computeTrends()}, \tsfunc{computeReciprocityIndex()}, \tsfunc{computeSentimentScore()}.
  \item \textbf{Utils i~walidacja} --- \filepath{schemas.ts} (Zod), \filepath{encode.ts} (share link encoding), \filepath{utils.ts}.
  \item \textbf{Komponenty} --- snapshot testy dla share cards (krytyczne dla viralności), smoke testy dla głównych widoków.
\end{enumerate}

\begin{metricbox}
\textbf{Cel:} Pokrycie testami $>$60\% modułów \filepath{lib/} w~ciągu 2 sprintów. Dodanie pre-commit hooka z~\texttt{vitest run} przed każdym commitem. Szacowany nakład: 3--5 dni pracy deweloperskiej.
\end{metricbox}


\subsection{Raport podatności bezpieczeństwa}
\label{sec:code-vulnerabilities}

Wyniki \texttt{pnpm audit} (stan: luty 2026):

\begin{table}[H]
\centering
\caption{Podatności bezpieczeństwa --- \texttt{pnpm audit}}
\label{tab:code-vulns}
\begin{tabularx}{\textwidth}{L{3.8cm}L{2.3cm}C{1.8cm}X}
\toprule
\textbf{CVE / GHSA} & \textbf{Pakiet} & \textbf{Severity} & \textbf{Problem} \\
\midrule
GHSA-p5xg-68wr-hm3m & jspdf $<$4.2.0 & \danger{HIGH} & PDF Injection --- AcroForm RadioButton \\
GHSA-9vjf-qc39-jprp & jspdf $<$4.2.0 & \danger{HIGH} & PDF Object Injection via \tsfunc{addJS()} \\
GHSA-67pg-wm7f-q7fj & jspdf $<$4.2.0 & \danger{HIGH} & DoS via Malicious GIF Dimensions \\
(dodatkowe CVE) & jspdf $<$4.2.0 & \danger{HIGH} & Dodatkowa luka jsPDF (4. CVE) \\
GHSA-3ppc-4f35-3m26 & minimatch $<$10.2.1 & \danger{HIGH} & ReDoS --- powtórzone wildcardy \\
\midrule
GHSA-2g4f-4pwh-qvx6 & ajv $<$6.14.0 & \warn{MODERATE} & ReDoS z~opcją \tstype{\$data} \\
\midrule
GHSA-gq3j-xvxp-8hrf & hono $<$4.11.10 & \score{LOW} & Timing comparison \\
\bottomrule
\end{tabularx}
\end{table}

\subsubsection{Analiza i~plan naprawczy}

\begin{description}
  \item[jsPDF (4 luki HIGH)] Aktualna wersja: 4.1.0. Wszystkie 4 CVE naprawione w~wersji \textbf{4.2.0+}. \textbf{Naprawa:} \texttt{pnpm update jspdf} --- zero breaking changes między 4.1.0 a~4.2.x. Ryzyko braku aktualizacji: \danger{wysokie} --- PDF export to core feature, a~aplikacja generuje PDF-y po stronie klienta (jsPDF). Atakujący mógłby przygotować złośliwe dane wejściowe powodujące injection do generowanego PDF-a.

  \item[minimatch (HIGH)] Zależność tranzytywna ESLinta. Nie wpływa na runtime produkcyjny (ESLint = devDependency). \textbf{Naprawa:} \texttt{pnpm update eslint} lub override w~\filepath{package.json}. Ryzyko: \score{niskie} w~produkcji, \warn{średnie} w~CI/CD (ReDoS przy lintingu).

  \item[ajv (MODERATE)] Zależność tranzytywna ESLinta (JSON Schema validation). \textbf{Naprawa:} analogicznie jak minimatch --- update ESLint. Ryzyko produkcyjne: \score{zerowe}.

  \item[hono (LOW)] Zależność tranzytywna shadcn CLI. Nie jest bundlowana do produkcji. \textbf{Naprawa:} \texttt{pnpm update shadcn}. Ryzyko: \score{zerowe}.
\end{description}

\begin{featurebox}
\textbf{Priorytet naprawy:}
\begin{enumerate}
  \item \danger{Natychmiast:} \texttt{pnpm update jspdf} $\rightarrow$ 4.2.x (eliminuje 4 HIGH CVE)
  \item \warn{W~następnym sprincie:} override lub update ESLint (eliminuje minimatch + ajv)
  \item \score{Przy okazji:} \texttt{pnpm update shadcn} (eliminuje hono)
\end{enumerate}
\end{featurebox}


\subsection{Wzorce obsługi błędów}
\label{sec:code-error-handling}

Analiza 63 bloków \tstype{try-catch} w~35 plikach ujawnia 4 wzorce:

\begin{table}[H]
\centering
\caption{Wzorce obsługi błędów w~\podtekst}
\label{tab:code-error-patterns}
\begin{tabularx}{\textwidth}{C{0.6cm}L{3.5cm}C{1.5cm}C{1.5cm}X}
\toprule
\textbf{Nr} & \textbf{Wzorzec} & \textbf{Ilość} & \textbf{Ocena} & \textbf{Przykład} \\
\midrule
1 & Try-catch + HTTP response & 22 & \score{Dobrze} & API routes: \tstype{Response.json(\{error\}, \{status: 500\})} \\
2 & Try-catch + silent fallback & 18 & \warn{OK*} & \tsfunc{decodeFBString()}: zwraca oryginalny string przy błędzie dekodowania \\
3 & Try-catch + \tstype{console.error} & 16 & \warn{Do poprawy} & \filepath{gemini.ts}, \filepath{ExportPDFButton.tsx}: logowanie bez struktury \\
4 & Pusty catch (cleanup) & 7 & \score{OK} & Heartbeat SSE: \tstype{catch \{\}} przy zamknięciu strumienia \\
\bottomrule
\end{tabularx}
\end{table}

\textit{*Wzorzec 2 jest akceptowalny dla funkcji dekoderów, gdzie fallback na oryginalną wartość jest pożądanym zachowaniem.}

\subsubsection{Pliki z~\tstype{console.error} (28 wystąpień)}

\begin{table}[H]
\centering
\caption{Rozkład \tstype{console.error} po plikach}
\label{tab:code-console-errors}
\begin{tabularx}{\textwidth}{L{6cm}C{2cm}X}
\toprule
\textbf{Plik} & \textbf{Wystąpienia} & \textbf{Kontekst} \\
\midrule
\filepath{src/lib/analysis/gemini.ts} & 6 & Błędy Gemini API, retry logic \\
\filepath{src/components/analysis/ExportPDFButton.tsx} & 3 & Generowanie PDF \\
\filepath{src/components/analysis/StandUpPDFButton.tsx} & 3 & Generowanie PDF comedy \\
\filepath{src/components/share-cards/ShareCardGallery.tsx} & 3 & Canvas rendering, Web Share \\
\filepath{src/app/error.tsx} & 2 & Error boundary globalny \\
Discord lib files (4 pliki) & 8 & Bot interactions, webhook \\
Pozostałe & 3 & Różne \\
\bottomrule
\end{tabularx}
\end{table}

\begin{warningbox}[title={\textbf{Problem: brak strukturalnego logowania}}]
Wszystkie błędy logowane przez \tstype{console.error} bez:
\begin{itemize}
  \item Ustrukturyzowanego formatu (np. JSON z~timestamp, severity, context)
  \item Integracji z~usługą monitoringu (Sentry, LogRocket, Datadog)
  \item Korelacji błędów (request ID, session ID)
  \item Alertów przy anomaliach (np. spike błędów Gemini API)
\end{itemize}
\textbf{Rekomendacja:} Wprowadzić \filepath{src/lib/logger.ts} z~ustrukturyzowanym formatem, level severity i~opcjonalną integracją z~Google Cloud Logging (skoro deployment jest na Cloud Run).
\end{warningbox}


\subsection{Architektura parserów --- porównanie}
\label{sec:code-parsery}

\begin{table}[H]
\centering
\caption{Porównanie parserów --- architektura i~metryki}
\label{tab:code-parsery-porownanie}
\begin{tabularx}{\textwidth}{L{2.5cm}C{1.2cm}L{2.5cm}L{2.5cm}X}
\toprule
\textbf{Parser} & \textbf{LOC} & \textbf{Format wejścia} & \textbf{Specyfika} & \textbf{Współdzielony kod} \\
\midrule
\filepath{messenger.ts} & 163 & JSON & FB unicode (\tsfunc{decodeFBString}) & Eksportuje \tsfunc{decodeFBString} \\
\filepath{whatsapp.ts} & 232 & TXT & Regex, detekcja 12h/24h & Niezależny \\
\filepath{instagram.ts} & 116 & JSON & Format zbliżony do Messengera & Importuje \tsfunc{decodeFBString} z~messenger.ts \\
\filepath{telegram.ts} & 155 & JSON & Mieszane tablice tekstowe & Niezależny \\
\filepath{discord.ts} & --- & API & HTTP Bot, nie plik & Niezależny \\
\filepath{detect.ts} & 41 & Dowolny & Auto-detekcja po strukturze & Importuje typy \\
\bottomrule
\end{tabularx}
\end{table}

\subsubsection{Ocena architektury}

\begin{itemize}
  \item \score{Niski poziom duplikacji} --- każdy parser ma unikatową logikę specyficzną dla formatu platformy. Jedyny współdzielony kod (\tsfunc{decodeFBString}) jest poprawnie wyeksportowany i~zaimportowany.
  \item \score{Zunifikowany typ wyjściowy} --- wszystkie parsery produkują \tstype{ParsedConversation} zdefiniowany w~\filepath{types.ts}, co zapewnia jednolity interfejs dla dalszego pipeline'u analizy.
  \item \warn{Brak walidacji wejścia} --- parsery zakładają poprawność struktury pliku. Błędny format powoduje \tstype{TypeError} zamiast czytelnego komunikatu.
  \item \warn{Brak limitów rozmiaru} --- parser przetworzy plik 500\,MB tak samo jak 500\,KB, bez ostrzeżeń ani chunked processing.
\end{itemize}

\begin{metricbox}
\textbf{Rekomendacja:} Dodać walidację wejścia (Zod schema per parser) i~limit rozmiaru pliku (np.\ 50\,MB) z~czytelnym komunikatem błędu po polsku.
\end{metricbox}


\subsection{Konwencje nazewnicze --- audyt}
\label{sec:code-naming}

\begin{table}[H]
\centering
\caption{Konwencje nazewnicze --- audyt spójności}
\label{tab:code-naming}
\begin{tabularx}{\textwidth}{L{3.5cm}L{3cm}C{2cm}X}
\toprule
\textbf{Kategoria} & \textbf{Konwencja} & \textbf{Spójność} & \textbf{Przykłady} \\
\midrule
Komponenty React & PascalCase & \score{100\%} & AnalysisHeader.tsx, ShareCardGallery.tsx \\
Moduły / utils & kebab-case & \score{100\%} & viral-scores.ts, rate-limit.ts \\
Pliki testowe & \_\_tests\_\_/kebab.test.ts & \score{100\%} & detect.test.ts, messenger.test.ts \\
Pliki typów & types.ts, schemas.ts & \score{100\%} & parsers/types.ts, analysis/types.ts \\
Katalogi & kebab-case & \score{100\%} & share-cards/, analysis/ \\
Hooki React & camelCase (use*) & \score{100\%} & useShareCard.ts, useCPSAnalysis.ts \\
Zmienne CSS & kebab-case (--font-*) & \score{100\%} & --font-geist-sans, --font-syne \\
\bottomrule
\end{tabularx}
\end{table}

\begin{metricbox}
\textbf{Wynik:} \score{100\% spójności} we wszystkich kategoriach nazewniczych. Brak inkonsystencji, brak mieszanych konwencji. Wzorcowa higiena nazewnicza.
\end{metricbox}


\subsection{Skan TODO / FIXME / HACK}
\label{sec:code-todo-scan}

Przeszukanie całego katalogu \filepath{src/} ujawniło \textbf{dokładnie 1} komentarz typu TODO:

\begin{lstlisting}[style=podcode, caption={Jedyny TODO w~codebase}, label={lst:code-todo}]
// src/lib/rate-limit.ts, linia 16:
// TODO: re-enable rate limiting before production deployment
\end{lstlisting}

\begin{warningbox}[title={\danger{Krytyczny TODO w~produkcji}}]
Ten jedyny TODO jest jednocześnie \textbf{najpoważniejszym} problemem bezpieczeństwa w~całej aplikacji. Rate limiting jest \danger{wyłączony} w~kodzie produkcyjnym działającym na Google Cloud Run. Plik zawiera poprawną implementację \tstype{rateLimitMap} (linie 1--13), ale funkcja \tsfunc{rateLimit()} na linii~15 nigdy z~niej nie korzysta --- zawsze zwraca \tstype{\{allowed: true\}}.

\textbf{Konsekwencje braku rate limitingu:}
\begin{itemize}
  \item Nieograniczone requesty do Google Gemini API $\rightarrow$ \danger{niekontrolowane koszty}
  \item Brak ochrony przed DDoS/abuse na endpointach AI
  \item Naruszenie SLA Gemini API (rate limits Google'a zamiast własnych)
\end{itemize}

\textbf{Naprawa:} Przywrócić logikę rate limitingu w~\tsfunc{checkRateLimit()} --- implementacja \tstype{Map<string, \{count, resetTime\}>} już istnieje w~tym samym pliku.
\end{warningbox}

\begin{itemize}
  \item \textbf{FIXME:} 0 wystąpień --- \score{czysto}
  \item \textbf{HACK:} 0 wystąpień --- \score{czysto}
  \item \textbf{XXX:} 0 wystąpień --- \score{czysto}
  \item \textbf{@deprecated:} 0 wystąpień --- \score{czysto}
\end{itemize}

\begin{metricbox}
\textbf{Wynik skanu:} Baza kodu jest wyjątkowo czysta pod względem zadłużenia technicznego (technical debt markers). Jedyny TODO ma jednak \danger{krytyczne} konsekwencje bezpieczeństwa i~kosztowe.
\end{metricbox}


% ============================================================
% Sekcja S6: SEO i web performance
% ============================================================

\section{SEO i web performance}
\label{sec:seo-performance}

\begin{center}
\Large\itshape\color{PodBlue}
,,Najlepsze SEO to takie, którego użytkownik nie zauważa --- a~robot Google'a tak.''
\end{center}

\vspace{8pt}

Sekcja analizuje gotowość \podtekst do indeksowania przez wyszukiwarki, optymalizację zasobów webowych oraz wydajność ładowania strony.

\subsection{Kompletność metadanych}
\label{sec:seo-metadata}

\begin{table}[H]
\centering
\caption{Audyt metadanych SEO --- \filepath{layout.tsx}}
\label{tab:seo-metadata}
\begin{tabularx}{\textwidth}{L{4cm}C{2cm}X}
\toprule
\textbf{Element} & \textbf{Status} & \textbf{Szczegóły} \\
\midrule
\tstype{title} & \score{PASS} & ,,PodTeksT --- odkryj to, co kryje się między wierszami'' \\
\tstype{description} & \score{PASS} & 164 znaki, język polski, z~CTA \\
\tstype{metadataBase} & \score{PASS} & \texttt{https://podtekst.app} \\
\tstype{openGraph.title} & \score{PASS} & Zgodny z~\tstype{title} \\
\tstype{openGraph.description} & \score{PASS} & Skrócona wersja opisu \\
\tstype{openGraph.locale} & \score{PASS} & \texttt{pl\_PL} \\
\tstype{openGraph.type} & \score{PASS} & \texttt{website} \\
\tstype{openGraph.images} & \score{PASS} & \texttt{/og/podtekst-og.png} 1200$\times$630 \\
\tstype{twitter.card} & \score{PASS} & \texttt{summary\_large\_image} \\
\tstype{twitter.title} & \score{PASS} & Zgodny z~OG title \\
\tstype{twitter.images} & \score{PASS} & Zgodny z~OG images \\
\tstype{viewport} & \score{PASS} & \texttt{device-width}, \texttt{initialScale: 1}, \texttt{viewportFit: cover} \\
\midrule
\tstype{robots} (meta tag) & \warn{BRAK} & Brak jawnej deklaracji w~metadanych \\
\tstype{alternates} (hreflang) & \warn{N/A} & Brak i18n --- jedynie polski \\
\tstype{canonical} & \warn{BRAK} & Brak jawnego canonical URL \\
\bottomrule
\end{tabularx}
\end{table}

\begin{metricbox}
\textbf{Wynik:} 12/12 kluczowych pól metadanych wypełnionych poprawnie. Brakuje \tstype{canonical} i~jawnego \tstype{robots} meta tagu, ale nie są krytyczne przy aktualnej skali.
\end{metricbox}


\subsection{Problemy w~robots.txt}
\label{sec:seo-robots}

\begin{warningbox}[title={\danger{Błędna domena w~robots.txt}}]
Plik \filepath{public/robots.txt}:

\begin{lstlisting}[style=podcodeBash, caption={robots.txt --- stan aktualny}, label={lst:seo-robots}]
User-agent: *
Allow: /
Disallow: /api/
Disallow: /dashboard
Disallow: /analysis/

Sitemap: https://chatscope.app/sitemap.xml
\end{lstlisting}

\textbf{Błąd:} Linia \texttt{Sitemap} wskazuje na \texttt{chatscope.app} --- starą domenę sprzed rebrandingu. Poprawna wartość: \texttt{https://podtekst.app/sitemap.xml}.

\textbf{Dodatkowe uwagi:}
\begin{itemize}
  \item \tstype{Disallow: /analysis/} blokuje indeksowanie stron analiz --- \score{poprawne} (prywatność użytkowników)
  \item \tstype{Disallow: /dashboard} blokuje panel --- \score{poprawne}
  \item \tstype{Disallow: /api/} blokuje endpointy API --- \score{poprawne}
  \item Brak \tstype{Disallow: /share/} --- strony share są indeksowalne (potencjalnie pożądane dla wiralności, ale ryzyko prywatności)
\end{itemize}
\end{warningbox}

\paragraph{Naprawa:} Zmiana jednej linii w~\filepath{public/robots.txt}:
\begin{lstlisting}[style=podcodeBash, caption={robots.txt --- poprawka}, label={lst:seo-robots-fix}]
Sitemap: https://podtekst.app/sitemap.xml
\end{lstlisting}

\subsubsection{Sitemap}

Plik \filepath{src/app/sitemap.ts} generuje dynamiczną sitemap z~2 stronami:

\begin{table}[H]
\centering
\caption{Strony w~sitemap.xml}
\label{tab:seo-sitemap}
\begin{tabularx}{\textwidth}{L{5.5cm}C{2.5cm}C{2cm}X}
\toprule
\textbf{URL} & \textbf{changeFrequency} & \textbf{priority} & \textbf{Uwagi} \\
\midrule
\texttt{https://podtekst.app} & weekly & 1.0 & Strona główna \\
\texttt{https://podtekst.app/analysis/new} & monthly & 0.8 & Strona uploadu \\
\bottomrule
\end{tabularx}
\end{table}

Dynamiczne strony \texttt{/analysis/[id]} celowo pominięte --- prywatność danych użytkowników (dane w~IndexedDB, nie na serwerze).


\subsection{Analiza stosu fontów}
\label{sec:seo-fonts}

\begin{table}[H]
\centering
\caption{Fonty ładowane w~\filepath{layout.tsx} --- analiza rozmiaru}
\label{tab:seo-fonts}
\begin{tabularx}{\textwidth}{L{3cm}C{1.5cm}L{3.5cm}C{2cm}X}
\toprule
\textbf{Rodzina} & \textbf{Warianty} & \textbf{Wagi} & \textbf{Szac.\ rozmiar} & \textbf{Użycie} \\
\midrule
Geist Sans & 1 & auto (variable) & $\sim$25\,KB & Body text \\
Geist Mono & 1 & auto (variable) & $\sim$20\,KB & Code, dane \\
JetBrains Mono & 5 & 400, 500, 600, 700, 800 & $\sim$80\,KB & Monospace (duplikacja?) \\
Syne & 4 & 400, 600, 700, 800 & $\sim$55\,KB & Brand, nagłówki \\
Space Grotesk & 5 & 300, 400, 500, 600, 700 & $\sim$65\,KB & Story mode body \\
\midrule
\multicolumn{3}{r}{\textbf{Szacowany łączny rozmiar:}} & $\sim$\textbf{245\,KB} & \\
\bottomrule
\end{tabularx}
\end{table}

\subsubsection{Konfiguracja fontów}

\begin{itemize}
  \item \score{\tstype{display: 'swap'}} --- wszystkie 5 rodzin używa \tstype{swap}, co eliminuje FOIT (Flash of Invisible Text)
  \item \score{\tstype{subsets: ['latin']}} --- tylko łacińskie glify, bez CJK czy cyrylicy
  \item \score{Next.js font optimization} --- \tstype{next/font/google} automatycznie self-hostuje fonty i~eliminuje external requests do Google Fonts
\end{itemize}

\subsubsection{Problemy i~rekomendacje}

\begin{warningbox}[title={\warn{Duplikacja fontów monospace}}]
Projekt ładuje \textbf{dwa} fonty monospace: \textbf{Geist Mono} (body code) i~\textbf{JetBrains Mono} (5 wag). JetBrains Mono jest używany głównie w~trybie Story --- ale ładowany na \textbf{każdej stronie}.

\textbf{Rekomendacje:}
\begin{enumerate}
  \item Zredukować JetBrains Mono do 2 wag (400, 700) --- oszczędność $\sim$48\,KB
  \item Rozważyć usunięcie JetBrains Mono i~użycie Geist Mono wszędzie
  \item Załadować Space Grotesk i~JetBrains Mono dynamicznie (\tsfunc{next/dynamic}) tylko na stronach Story/Wrapped --- oszczędność $\sim$145\,KB na stronie głównej
\end{enumerate}
\end{warningbox}

\begin{metricbox}
\textbf{Potencjalna oszczędność:} ładowanie warunkowe Space Grotesk + JetBrains Mono (tylko Story) zmniejszyłoby initial font payload o~\textbf{$\sim$145\,KB} ($\sim$60\% redukcji).
\end{metricbox}


\subsection{Audyt optymalizacji obrazów}
\label{sec:seo-images}

\begin{table}[H]
\centering
\caption{Audyt użycia obrazów --- \tstype{next/image} vs \tstype{<img>}}
\label{tab:seo-images}
\begin{tabularx}{\textwidth}{L{4.5cm}C{2cm}L{3cm}X}
\toprule
\textbf{Plik} & \textbf{Element} & \textbf{Źródło} & \textbf{Problem} \\
\midrule
\filepath{DiscordImport.tsx:400} & \tstype{<img>} & External (guild icon) & Brak lazy loading, brak WebP fallback \\
\filepath{AnalysisImageCard.tsx:193} & \tstype{<img>} & Generated (screenshot) & Brak optymalizacji rozmiaru \\
\filepath{ParticipantPhotoUpload.tsx:77} & \tstype{<img>} & User upload (photo) & Brak resize, brak compression \\
\bottomrule
\end{tabularx}
\end{table}

\begin{warningbox}[title={\warn{Zero użyć \tstype{next/image}}}]
Projekt \textbf{nie korzysta} z~komponentu \tstype{next/image} --- kluczowego narzędzia Next.js do:
\begin{itemize}
  \item Automatycznej konwersji do WebP/AVIF
  \item Responsive \tstype{srcSet} z~wieloma rozdzielczościami
  \item Native lazy loading z~blur placeholder
  \item Zapobiegania CLS (Cumulative Layout Shift) przez wymuszenie wymiarów
\end{itemize}
Choć aplikacja ma niewiele obrazów (3 instancje \tstype{<img>}), brakuje konfiguracji w~\filepath{next.config.ts}:
\begin{lstlisting}[style=podcode, caption={Brakująca konfiguracja obrazów}, label={lst:seo-image-config}]
// next.config.ts - BRAK:
images: {
  formats: ['image/avif', 'image/webp'],
  remotePatterns: [
    { protocol: 'https', hostname: 'cdn.discordapp.com' },
  ],
},
\end{lstlisting}
\end{warningbox}

\begin{metricbox}
\textbf{Priorytet:} \score{Niski} --- tylko 3 obrazy w~aplikacji. Warto dodać konfigurację \tstype{images} do \filepath{next.config.ts} przygotowawczo, ale nie jest to blocker.
\end{metricbox}


\subsection{Analiza proporcji Client vs Server Components}
\label{sec:seo-client-server}

\begin{table}[H]
\centering
\caption{Proporcja Client vs Server Components}
\label{tab:seo-client-server}
\begin{tabularx}{\textwidth}{L{4cm}C{2.5cm}C{2.5cm}X}
\toprule
\textbf{Typ} & \textbf{Ilość} & \textbf{Udział} & \textbf{Uwagi} \\
\midrule
Client Components (\tstype{'use client'}) & $\sim$150 & $\sim$95\% & Interaktywność, animacje, localStorage \\
Server Components & $\sim$8 & $\sim$5\% & Page routes, layouts, API routes \\
\bottomrule
\end{tabularx}
\end{table}

\subsubsection{Uzasadnienie proporcji}

Wysoki udział Client Components ($\sim$95\%) jest \textbf{uzasadniony} specyfiką aplikacji:

\begin{enumerate}[label=\textcolor{PodBlue}{\arabic*.}]
  \item \textbf{Interaktywne wykresy} --- Recharts wymaga DOM i~event handlerów (Client)
  \item \textbf{Animacje} --- Framer Motion \tsfunc{motion.div}, \tsfunc{whileInView} (Client)
  \item \textbf{localStorage / IndexedDB} --- cały pipeline danych to API przeglądarki (Client)
  \item \textbf{SSE streaming} --- hooki \tsfunc{useEffect} + \tstype{EventSource} (Client)
  \item \textbf{Formularze i~stan} --- quiz, simulator, uploader (Client)
  \item \textbf{Canvas rendering} --- html2canvas dla share cards (Client)
\end{enumerate}

\begin{metricbox}
\textbf{Ocena:} Proporcja jest \score{akceptowalna} dla aplikacji typu SPA z~heavy interactivity. Próba forsowania Server Components na siłę (np.\ dla wykresów) byłaby antyproduktywna. Jedyna optymalizacja: wydzielenie statycznych sekcji landing page'a (FAQ, Footer, SocialProof) jako Server Components --- szacowana oszczędność: $\sim$15\,KB JS.
\end{metricbox}


\subsection{Przegląd optymalizacji bundla}
\label{sec:seo-bundle}

\subsubsection{Obecne optymalizacje}

\begin{table}[H]
\centering
\caption{Optymalizacje bundla --- stan obecny}
\label{tab:seo-bundle-current}
\begin{tabularx}{\textwidth}{L{5cm}C{2cm}X}
\toprule
\textbf{Optymalizacja} & \textbf{Status} & \textbf{Szczegóły} \\
\midrule
\tstype{optimizePackageImports} & \score{Tak} & framer-motion, lucide-react, recharts \\
Dynamic imports (\tsfunc{next/dynamic}) & \score{Tak} & 20+ komponentów (Spline, story, analysis) \\
\tstype{React.lazy} & \score{Tak} & $\sim$5 komponentów (Wrapped, share cards) \\
Tree shaking & \score{Tak} & Automatyczne przez Turbopack/SWC \\
\tstype{output: 'standalone'} & \score{Tak} & Minimalny Docker image \\
\midrule
Bundle analyzer & \danger{Brak} & Brak \tstype{@next/bundle-analyzer} \\
\tstype{images.formats} & \danger{Brak} & Brak WebP/AVIF konfiguracji \\
\tstype{images.remotePatterns} & \danger{Brak} & Brak whitelist domen obrazów \\
CSS code splitting & \warn{Brak} & Jeden \filepath{globals.css} dla całej aplikacji \\
\bottomrule
\end{tabularx}
\end{table}

\subsubsection{Dynamic imports --- analiza}

Projekt używa 28 dynamicznych importów, ale wiele z~nich ładuje się \textbf{eagerly} na stronie analizy, ponieważ leżą w~jednym drzewie renderowania (\filepath{page.tsx} importuje wszystko):

\begin{itemize}
  \item \tsfunc{next/dynamic} z~\tstype{ssr: false} --- poprawne dla komponentów Spline 3D, canvas-heavy kart share
  \item Problem: brak code splitting \textbf{między sekcjami} strony analizy --- wszystkie 50+ komponentów ładowane jednocześnie
  \item Rozwiązanie: architektura tabowa (\secref{sec:architektura-tabowa}) naturalnie wprowadza code splitting per tab
\end{itemize}

\subsubsection{Brakujące optymalizacje}

\begin{warningbox}[title={\warn{Brak \tstype{@next/bundle-analyzer}}}]
Bez analizatora bundla nie ma danych o:
\begin{itemize}
  \item Rozmiarze poszczególnych chunk'ów JS
  \item Które biblioteki dominują w~bundlu (podejrzane: Recharts, Framer Motion, jsPDF)
  \item Duplikacji kodu między chunk'ami
  \item Efektywności tree shakingu
\end{itemize}

\textbf{Instalacja:} \texttt{pnpm add -D @next/bundle-analyzer} + wrapper w~\filepath{next.config.ts}:
\begin{lstlisting}[style=podcode, caption={Konfiguracja bundle analyzer}, label={lst:seo-bundle-analyzer}]
const withBundleAnalyzer = require('@next/bundle-analyzer')({
  enabled: process.env.ANALYZE === 'true',
});
module.exports = withBundleAnalyzer(nextConfig);
\end{lstlisting}
Uruchomienie: \texttt{ANALYZE=true pnpm build}.
\end{warningbox}


\subsection{Rekomendacje wydajnościowe --- priorytetyzacja}
\label{sec:seo-recommendations}

\begin{table}[H]
\centering
\caption{Rekomendacje SEO i~performance --- uporządkowane wg wpływu}
\label{tab:seo-recommendations}
\begin{tabularx}{\textwidth}{C{0.6cm}C{1.5cm}L{4cm}C{2cm}X}
\toprule
\textbf{Nr} & \textbf{Priorytet} & \textbf{Rekomendacja} & \textbf{Szac.\ wpływ} & \textbf{Nakład} \\
\midrule
1 & \danger{P0} & Naprawić domenę w~robots.txt & SEO: krytyczny & 1 minuta \\
2 & \warn{P1} & Zainstalować bundle analyzer & Diagnostyka & 15 minut \\
3 & \warn{P1} & Warunkowe ładowanie fontów Story (JetBrains Mono + Space Grotesk) & $-$145\,KB initial & 1--2h \\
4 & \warn{P1} & Dodać \tstype{images.formats} + \tstype{remotePatterns} do next.config.ts & Gotowość na obrazy & 15 minut \\
5 & \warn{P1} & Server Components dla statycznych sekcji landing page'a & $-$15\,KB JS & 2--3h \\
6 & \score{P2} & Zamienić \tstype{<img>} na \tstype{next/image} (3 instancje) & Lazy load, WebP & 30 minut \\
7 & \score{P2} & Dodać \tstype{canonical} URL do metadata & SEO: deduplikacja & 5 minut \\
8 & \score{P2} & CSS code splitting (wydzielić story.css, wrapped.css) & $-$20--30\,KB CSS & 1--2h \\
9 & \score{P3} & Dodać structured data (JSON-LD: WebApplication) & SEO: rich snippets & 30 minut \\
10 & \score{P3} & Rozważyć usunięcie \filepath{/share/} z~indeksowania & Prywatność & 5 minut \\
\bottomrule
\end{tabularx}
\end{table}

\begin{featurebox}
\textbf{Podsumowanie SEO:}
\begin{itemize}
  \item \textbf{Metadane:} \score{doskonałe} --- 12/12 pól OG + Twitter poprawnie skonfigurowanych
  \item \textbf{robots.txt:} \danger{1 krytyczny błąd} --- stara domena \texttt{chatscope.app} w~Sitemap
  \item \textbf{Fonty:} \warn{ciężkie} --- 5 rodzin / 245\,KB, z~czego 145\,KB ładowane na stronach, które ich nie potrzebują
  \item \textbf{Obrazy:} \warn{brak optymalizacji} --- zero \tstype{next/image}, brak konfiguracji formatów
  \item \textbf{Bundle:} \warn{brak diagnostyki} --- bez bundle analyzer trudno ocenić realny rozmiar
  \item \textbf{Client/Server:} \score{uzasadnione} --- 95\% Client Components wynika z~natury aplikacji (SPA)
\end{itemize}
\end{featurebox}
   % Audyt jakości kodu + Audyt SEO


% ============================================================
\section{Metryki docelowe --- wszystkie osie audytu}
\label{sec:audyt-metryki-docelowe}
% ============================================================

Poniższa tabela konsoliduje metryki ze wszystkich dziewięciu osi audytu w~jednym miejscu, z~priorytetami od P0 (krytyczne) do P2 (do optymalizacji w~kolejnych sprintach).

\begin{table}[H]
\centering
\caption{Skonsolidowane metryki docelowe --- wszystkie 9 osi audytu}
\label{tab:audyt-metryki-docelowe}
\begin{tabularx}{\textwidth}{L{4.5cm}C{2.8cm}C{2.8cm}C{1.5cm}}
\toprule
\textbf{Metryka} & \textbf{Stan obecny} & \textbf{Cel} & \textbf{Priorytet} \
\midrule
\multicolumn{4}{l}{\textit{\textbf{Wydajność (Performance)}}} \
LCP (Largest Contentful Paint) & 2--4\,s & $<2{,}5$\,s & P1 \
INP (Interaction to Next Paint) & 200--800\,ms & $<200$\,ms & P1 \
IntersectionObserver instances & 51--54 & $<$10 per tab & P0 \
Initial JS bundle (analysis) & 100\% & 40--60\% & P1 \
DOM nodes (jednocześnie) & 50+ & $\sim$10 per tab & P0 \
Memory (strona analizy) & 90--120\,MB & $<$50\,MB & P2 \
Parser blocking time & 200--400\,ms & $<$50\,ms (Worker) & P2 \
Font payload & 245\,KB & $\sim$100\,KB & P1 \
\midrule
\multicolumn{4}{l}{\textit{\textbf{Bezpieczeństwo (Security)}}} \
Rate limit compliance & \danger{0\% (wyłączony)} & \score{100\%} & P0 \
Security vulns (HIGH) & 5 & 0 & P0 \
\midrule
\multicolumn{4}{l}{\textit{\textbf{Jakość kodu (Code Quality)}}} \
Test coverage (\filepath{lib/}) & $<$5\% & $>$60\% & P1 \
\midrule
\multicolumn{4}{l}{\textit{\textbf{Mobile UX}}} \
Touch target compliance & $\sim$70\% & 100\% & P1 \
\midrule
\multicolumn{4}{l}{\textit{\textbf{SEO}}} \
\filepath{robots.txt} domain & chatscope.app & podtekst.app & P0 \
\midrule
\multicolumn{4}{l}{\textit{\textbf{Unit Economics}}} \
AI cost per basic analysis & 0,11~PLN & 0,07~PLN (z~cache) & P2 \
Margin brutto (avg) & $\sim$91\% & $>$85\% & monitor \
\midrule
\multicolumn{4}{l}{\textit{\textbf{Onboarding}}} \
Onboarding drop-off (landing$\rightarrow$AI) & $\sim$73\%$^{\dagger}$ & $<$50\% & P1 \
\bottomrule
\end{tabularx}
\end{table}

\begin{metricbox}
Metryki P0 wymagają natychmiastowej naprawy --- każda z~nich stanowi albo ryzyko bezpieczeństwa (rate limit, CVE), albo blokuje skalowalność (DOM overload, IntersectionObserver), albo jest błędem konfiguracji (\filepath{robots.txt}). Metryki P1 wpływają bezpośrednio na konwersję i~retencję użytkowników. Metryki P2 to optymalizacje, które stają się istotne przy skali $>$1000 MAU.

$^{\dagger}$~Wartość 73\% to \textbf{heurystyka benchmarkowa} (typowy SaaS B2C), nie pomiar z~\podtekst. GA4 jest zaimplementowane, ale nie śledzi lejka konwersji. Wymagana walidacja przez GA4 funnel events.
\end{metricbox}


% ============================================================
\section{Skonsolidowana roadmapa}
\label{sec:audyt-roadmapa}
% ============================================================

Poniższa roadmapa integruje rekomendacje ze wszystkich dziewięciu osi audytu w~cztery sprinty, uporządkowane według priorytetu i~zależności implementacyjnych.

\subsection{Sprint 0 --- naprawy bezpieczeństwa (1--2 dni)}

\begin{table}[H]
\centering
\caption{Sprint 0 --- naprawy bezpieczeństwa (P0, bez zmian architektonicznych)}
\label{tab:audyt-sprint0}
\begin{tabularx}{\textwidth}{L{5.5cm}C{2cm}X}
\toprule
\textbf{Zadanie} & \textbf{Oś audytu} & \textbf{Pliki / uwagi} \
\midrule
Naprawić \filepath{rate-limit.ts} + migracja do Upstash Redis & Bezpieczeństwo & \filepath{src/lib/rate-limit.ts} \
Aktualizacja jsPDF do 4.2.0+ & Bezpieczeństwo & \filepath{package.json} (4 HIGH CVE) \
Naprawić domenę w~\filepath{robots.txt} & SEO & chatscope.app $\rightarrow$ podtekst.app \
Dodać monitoring thinking tokens & Unit Economics & logi API routes \
\bottomrule
\end{tabularx}
\end{table}

\begin{metricbox}
Sprint~0 to wyłącznie naprawy bezpieczeństwa i~konfiguracji --- żadne zmiany architektoniczne, żadne nowe pliki komponentów. Cel: zamknąć w~1--2 dni robocze, zdeployować niezależnie od dalszych prac.
\end{metricbox}

\subsection{Sprint 1 --- architektura tabowa (3--5 dni)}

\begin{table}[H]
\centering
\caption{Sprint 1 --- przebudowa architektury strony analizy}
\label{tab:audyt-sprint1}
\begin{tabularx}{\textwidth}{L{5.5cm}C{2cm}X}
\toprule
\textbf{Zadanie} & \textbf{Oś audytu} & \textbf{Pliki / uwagi} \
\midrule
Stworzyć \tstype{AnalysisTabs} + 5~tab komponentów & Design & 6 nowych plików \
Rozbić \filepath{page.tsx} na taby & Design & page.tsx 1322$\rightarrow\sim$200~LOC \
URL hash sync (\texttt{\#overview}, \texttt{\#ai}, ...) & Design & deep-linking \
Przenieść handlery \tsfunc{useCallback} do tabów & Optymalizacja & redukcja re-renderów \
\bottomrule
\end{tabularx}
\end{table}

\subsection{Sprint 2 --- P1: monetyzacja i~bezpieczeństwo}

\begin{table}[H]
\centering
\caption{Sprint 2 --- zadania P1 (ważne)}
\label{tab:audyt-sprint2}
\begin{tabularx}{\textwidth}{L{5.5cm}C{2cm}X}
\toprule
\textbf{Zadanie} & \textbf{Oś audytu} & \textbf{Pliki / uwagi} \
\midrule
\tstype{TierContext} + \tstype{PaywallGate} & Monetyzacja & 2 nowe pliki \
Strona polityki prywatności & Bezpieczeństwo & \filepath{/privacy}, GDPR compliance \
\tstype{Content-Security-Policy} + HSTS headers & Bezpieczeństwo & \filepath{next.config.ts} \
Deep validation \tstype{samplesSchema} & Bezpieczeństwo & \filepath{src/lib/validation/schemas.ts} \
Export guide per platform (onboarding) & Onboarding & ikony + instrukcje w~\tstype{DropZone} \
Welcome modal + guided tour & Onboarding & nowy komponent \
Conditional font loading (oszczędność 145\,KB) & SEO, Performance & \filepath{layout.tsx}, font subsetting \
\bottomrule
\end{tabularx}
\end{table}

\subsection{Sprint 3 --- P1/P2: wydajność i~jakość}

\begin{table}[H]
\centering
\caption{Sprint 3 --- zadania P1/P2 (wydajność i~jakość kodu)}
\label{tab:audyt-sprint3}
\begin{tabularx}{\textwidth}{L{5.5cm}C{2cm}X}
\toprule
\textbf{Zadanie} & \textbf{Oś audytu} & \textbf{Pliki / uwagi} \
\midrule
\tsfunc{React.lazy()} per tab & Optymalizacja & \tstype{AnalysisTabs.tsx} \
\tsfunc{React.memo()} na chartach & Optymalizacja & 6 komponentów wykresów \
Callback refactor per tab & Optymalizacja & przeniesienie handlerów do tabów \
Server Components dla statycznych sekcji landing & Optymalizacja & \filepath{LandingFAQ}, \filepath{LandingFooter} \
Bundle analyzer & Optymalizacja & \filepath{next.config.ts} \
Test coverage $>$60\% dla \filepath{lib/} & Jakość kodu & Vitest, nowe pliki testów \
Touch target fixes (hamburger, checkbox) & Mobile UX & min 44$\times$44\,px \
\bottomrule
\end{tabularx}
\end{table}

\subsection{Sprint 4 --- P2: optymalizacja zaawansowana}

\begin{table}[H]
\centering
\caption{Sprint 4 --- zadania P2 (optymalizacja zaawansowana)}
\label{tab:audyt-sprint4}
\begin{tabularx}{\textwidth}{L{5.5cm}C{2cm}X}
\toprule
\textbf{Zadanie} & \textbf{Oś audytu} & \textbf{Pliki / uwagi} \
\midrule
Web Worker parser & Optymalizacja & \filepath{parser.worker.ts} (nowy) \
IndexedDB quota management & Optymalizacja & \filepath{src/lib/utils.ts} \
Virtualized \tstype{ShareCardGallery} & Optymalizacja & \tstype{react-window} \
Response caching (hash-based) & Unit Economics & cache Gemini responses \
Batch API dla StandUp/CPS & Unit Economics & redukcja round-trips \
,,Przeanalizuj kolejną'' CTA po analizie & Onboarding & post-analysis retention \
Email/push notification system & Onboarding & powiadomienia o~nowych funkcjach \
Migracja Satori dla share cards & Performance & SSR zamiast html2canvas \
\bottomrule
\end{tabularx}
\end{table}


% ============================================================
\section{Podsumowanie}
\label{sec:audyt-podsumowanie}
% ============================================================

Audyt wieloagentowy w~dziewięciu osiach zidentyfikował \textbf{6 krytycznych (P0)}, \textbf{11 ważnych (P1)} i~\textbf{8 dodatkowych (P2)} problemów w~architekturze \podtekst. Poniżej synteza kluczowych wniosków z~każdej osi:

\begin{description}
  \item[1. Design (UX strony analizy)] Monolityczna strona analizy (1322~LOC, 50+ komponentów) wymaga rozbicia na architekturę tabową z~5~zakładkami. AI Analysis --- najcenniejsza treść --- jest pogrzebana pod 3000\,px danych ilościowych. Proponowane 5 tabów z~lazy-loadingiem redukują DOM o~80\% i~przesuwają AI na pozycję~3.

  \item[2. Monetyzacja] Brak jakiejkolwiek infrastruktury płatności. Proponowany model 3-tier (Free / Pro 29,99~PLN / Unlimited 49,99~PLN) z~paywallem w~momencie największego zaangażowania i~share cards jako viral loop. Strategia akwizycji (sekcja~\ref{sec:strategia-akwizycji}) oparta na 7 kanałach wzrostu: viral share cards, SEO, TikTok content, mikro/makro-influencerzy, Meta Ads, Google Ads. Realistyczna prognoza (scenariusz Lean Startup, $\sim$3\,500~PLN/mies budżetu): $\sim$3\,000~MAU i~$\sim$55\,000~PLN ARR w~miesiącu~12, break-even $\sim$miesiąc 15--18. Wymagana inwestycja do break-even: $\sim$30\,000~PLN.

  \item[3. Optymalizacja wydajności] Rate limiting wyłączony w~produkcji (ryzyko kosztowe), 51+ IntersectionObserver instances (pamięć), brak lazy-loadingu treści (initial bundle), brak \tsfunc{React.memo()} na ciężkich komponentach (re-rendery). Architektura tabowa rozwiązuje większość problemów naturalnie.

  \item[4. Unit economics] Koszty Gemini API są bardzo niskie w~przeliczeniu na PLN --- podstawowa analiza kosztuje $\sim$0,11~PLN, pełna $\sim$0,36~PLN. Przy cenach 29,99/49,99~PLN marża brutto wynosi $\sim$91\%. Główne ryzyko: thinking tokens w~nowych modelach Gemini mogą zwiększyć koszty 3--5$\times$ bez monitoringu.

  \item[5. Bezpieczeństwo] Rate limit \danger{wyłączony} w~produkcji, brak nagłówków \tstype{Content-Security-Policy} i~HSTS, 5 podatności HIGH w~zależnościach (jsPDF CVE), brak walidacji głębokiej w~\tstype{samplesSchema}, brak strony polityki prywatności (GDPR). Wszystkie 5 HIGH CVE wymagają natychmiastowej naprawy.

  \item[6. Mobile UX] Landing page jest dobrze zoptymalizowany (diagonal hero, pinned CTA). Strona analizy ma problemy z~nadmiarem IntersectionObserver (51+ na mobile = jank), touch targets $<$44\,px (hamburger menu, checkboxy), brak optymalizacji uploadu dla iOS (brak instrukcji ,,Pliki'' app).

  \item[7. Onboarding] Minimalny, ale funkcjonalny flow w~4 krokach (landing $\rightarrow$ upload $\rightarrow$ parse $\rightarrow$ AI). Krytyczny brak: instrukcja eksportu per platforma --- użytkownicy nie wiedzą \emph{jak} wyeksportować chat z~Messengera/WhatsAppa. Heurystyczny szacunek drop-off landing$\rightarrow$AI: $\sim$73\% (benchmark SaaS B2C, \textbf{bez danych własnych} --- GA4 nie śledzi lejka konwersji). Hipoteza: brak export guide odpowiada za znaczną część utraty.

  \item[8. Jakość kodu] Doskonała konfiguracja TypeScript (strict mode, 0 użyć \tstype{any}), ale zaledwie 3 pliki testów (pokrycie $<$5\% dla \filepath{lib/}). 5 podatności HIGH w~\tstype{npm audit}. Kod jest dobrze zorganizowany modularnie (\filepath{quant/} submoduły), ale brakuje testów regresji dla parserów i~analizy ilościowej.

  \item[9. SEO] Doskonałe metadane (Open Graph, Twitter Cards, structured data). Krytyczny błąd: \filepath{robots.txt} wskazuje na \texttt{chatscope.app} zamiast \texttt{podtekst.app} (rebranding niedokończony). Ciężkie fonty: 245\,KB (Geist Sans 140\,KB + Geist Mono 65\,KB + Syne 40\,KB) --- conditional loading może zaoszczędzić $\sim$145\,KB.
\end{description}

\begin{featurebox}
\textbf{Szacowany wpływ implementacji wszystkich 4 sprintów:}
\begin{itemize}
  \item \textbf{UX:} AI Analysis dostępne w~2 kliknięciach zamiast 3000\,px scrollowania; DOM $-$80\%
  \item \textbf{Performance:} $-$60\% initial JS, $-$80\% IntersectionObserver, INP $<$200\,ms, fonty $-$145\,KB
  \item \textbf{Revenue:} od 0~PLN do $\sim$55\,000~PLN ARR (scenariusz Lean Startup, budżet $\sim$3\,500~PLN/mies, break-even $\sim$miesiąc 15--18)
  \item \textbf{Security:} rate limiting przywrócony, CSP + HSTS, 0 podatności HIGH, GDPR compliance
  \item \textbf{Costs:} marża brutto $>$85\% nawet przy agresywnym użyciu AI; cache redukuje koszty o~35\%
  \item \textbf{Mobile:} 100\% touch target compliance, optymalizacja upload flow dla iOS
  \item \textbf{Onboarding:} drop-off landing$\rightarrow$AI: redukcja dzięki export guide + welcome modal (skala nieznana bez baseline'u z~GA4)
  \item \textbf{Quality:} test coverage \filepath{lib/} z~$<$5\% do $>$60\%, 0 podatności w~\tstype{npm audit}
  \item \textbf{SEO:} poprawny \filepath{robots.txt}, fonty $-$60\%, ranking stabilny
  \item \textbf{DX:} \filepath{page.tsx} z~1322 do $\sim$200~LOC, modularny kod, CI/CD z~testami
\end{itemize}
\end{featurebox}
