% ============================================================
% PodTeksT — Wstęp (Preface)
% ============================================================

\chapter*{Wstęp}
\addcontentsline{toc}{chapter}{Wstęp}
\markboth{Wstęp}{Wstęp}

\begin{center}
\Large\itshape\color{PodBlue}
,,Twoje rozmowy mówią więcej niż myślisz.''
\end{center}

\vspace{12pt}

\podtekst to aplikacja SaaS nowej generacji, która przekształca surowe eksporty rozmów z~komunikatorów w~głęboką analizę psychologiczną i~komunikacyjną. Użytkownik wrzuca pliki z~Messengera (JSON), WhatsAppa (TXT), Instagrama (JSON), Telegrama (JSON) lub bezpośrednio z~Discorda (Bot API) --- a~w~zamian otrzymuje ponad 60~metryk ilościowych, profil osobowości oparty na Big~Five i~MBTI, analizę stylu przywiązania, mapę dynamiki relacji i~konkretne, oparte na danych wskazówki.

\section*{Dla kogo jest ten dokument}

Ta dokumentacja jest kompletnym kompendium produktu \podtekst --- od wizji marki, przez każdy algorytm i~interfejs TypeScript, aż po konfigurację deploymentu.

\begin{description}
  \item[Inwestorzy i partnerzy biznesowi] znajdą tu pełną propozycję wartości, model cenowy, pozycjonowanie marki i~mapę rozwoju produktu (Rozdziały 1--2, 13).

  \item[Deweloperzy i architekci] znajdą kompletną specyfikację techniczną: stos technologiczny, architekturę systemu, definicje wszystkich typów danych, dokumentację każdej funkcji silnika analitycznego oraz diagramy przepływu danych (Rozdziały 3--12).

  \item[Projektanci UX/UI] znajdą system projektowy, paletę kolorów, typografię, inwentarz 40+ komponentów interfejsu oraz opis wszystkich animacji i~interakcji (Rozdziały 1, 8).

  \item[Specjaliści ds.\ bezpieczeństwa] znajdą opis zabezpieczeń, politykę prywatności, mechanizmy rate limitingu i~ochronę przed prompt injection (Rozdział 11).
\end{description}

\section*{Struktura dokumentu}

\begin{itemize}
  \item \textbf{Część I: Marka} (Rozdziały 1--2) --- Tożsamość marki, przegląd produktu, ścieżka użytkownika, model cenowy.
  \item \textbf{Część II: Architektura} (Rozdziały 3--7) --- Stos technologiczny, parsery, silnik analizy ilościowej, silnik AI, wynik zdrowia relacji.
  \item \textbf{Część III: Interfejs i API} (Rozdziały 8--9) --- System projektowy, komponenty UI, endpointy API.
  \item \textbf{Część IV: Operacje} (Rozdziały 10--11) --- Docker, Firebase, bezpieczeństwo, prywatność, RODO.
  \item \textbf{Część V: Referencja} (Rozdziały 12--13) --- Kompletna referencja typów danych, mapa rozwoju.
\end{itemize}

\section*{Konwencje}

W~całym dokumencie stosujemy następujące konwencje:

\begin{itemize}
  \item \tstype{NazwaTypu} --- typ lub interfejs TypeScript
  \item \tsfunc{nazwaFunkcji()} --- funkcja lub metoda
  \item \tskey{słowo kluczowe} --- słowo kluczowe języka
  \item \filepath{ścieżka/do/pliku.ts} --- ścieżka pliku w~projekcie
  \item \personA{Osoba A} / \personB{Osoba B} --- uczestnicy rozmowy (kolor niebieski / fioletowy)
  \item Wartości \score{pozytywne}, \warn{neutralne/ostrzegawcze} i~\danger{negatywne} są kolorowane kontekstowo
\end{itemize}

\begin{infobox}[title=Nota informacyjna]
Niebieskie ramki zawierają dodatkowe wyjaśnienia, kontekst lub ciekawostki techniczne.
\end{infobox}

\begin{warningbox}[title=Ważne zastrzeżenie]
Żółte ramki zawierają ostrzeżenia, ograniczenia techniczne lub istotne zastrzeżenia dotyczące interpretacji wyników.
\end{warningbox}

\vfill

\begin{center}
\small\color{PodTextMuted}
Dokument wygenerowany z~kodu źródłowego projektu \podtekst\\
Stan na: luty 2026 • Next.js 16 • React 19 • TypeScript 5
\end{center}
